\startsection[title=Théorie ZFC]

\startsubsection[title=La théorie de Zermelo, reference=sec:Zermelo]

La théorie de \words{Zermelo}, aussi dite « théorie Z » est une axiomatisation, dans le cadre de la logique du premier ordre avec égalité, de la théorie des ensembles. 
Elle fait intervenir des objets, appelés \emph{ensembles}\index{Ensemble}\footnote{Un ensemble est parfois appelé \emph{\words{espace}} ; mais ce terme est en général utilisé seulement en présence d'une structure additionnelle.}, et leurs relations, notamment des relations binaires. 
Une de ces relations est l'\emph{appartenance}, désignée par le symbole $\in$. 
Si $x$ et $y$ sont deux ensembles, alors $x \in y$ est une proposition bien formée (il s'agit d'un terme). 
Si elle est vraie, on dira que \emph{$x$ est un \words{élément} de $y$}, que \emph{$x$ appartient à $y$}, que \emph{$x$ est dans $y$}, que \emph{$y$ contient $x$}, ou que \emph{$y$ possède $x$}.
On définit aussi la relation $\ni$ par : $x \ni y$ est équivalente à $y \in x$. 
On a donc : $\forall x \, \forall y \, (x \ni y) \Leftrightarrow (y \in x)$ et la relation $\notin$ par $\forall x \, \forall y \, (x \notin y) \Leftrightarrow \neg (x \in y)$. 
Pour l'évaluation d'une formule, les relations $\in$ et $\ni$ sont (comme toute autre relation binaire) prioritaires par rapport à l'égalité, mais pas par rapport à $\neg$.

On définit la relation d'\words{inclusion} $\symbols{\subset}$ par : $a \subset b$ est équivalent à $\forall x \, (x \in a) \Rightarrow (x \in b)$, autrement dit,
\startformula
    \forall a \, \forall b \, (
        (a \subset b) \Leftrightarrow (\forall x \, (x \in a) \Rightarrow (x \in b))
    ). 
\stopformula
Si $a \subset b$, on dira que \emph{$a$ est un sous-ensemble de $b$}, que \emph{$a$ est inclus dans $b$}, ou que \emph{$a$ est une partie de $b$}.
Notons que, pour tout ensemble $a$, $a \subset a$ est vrai.%
\footnote{En effet, soit $x$ un ensemble, $x \in a$ a toujours la même valeur de vérité que lui-même, donc $(x \in a) \Rightarrow (x \in a)$ est vrai.}
On définit aussi la relation $\symbols{\supset}$ par :
\startformula
    \forall a \, \forall b \, (
        (a \supset b) \Leftrightarrow (\forall x \, (x \in a) \Leftarrow (x \in b))
    ). 
\stopformula

\blank[medium]

\noindent\bold{Lemme :} Soit $\forall a \, \forall b \, (a = b) \Leftrightarrow ((a \subset b) \wedge (b \subset a))$.

\blank[medium]

\noindent\bold{Démonstration :} 
    La formule $(a \subset b) \wedge (b \subset a)$ est équivalente à : $(\forall x (x \in a) \Rightarrow (x \in b)) \wedge (\forall y (y \in a) \Rightarrow (y \in b))$, et donc à $\forall x \, ((x \in a) \Rightarrow (x \in b)) \wedge ((x \in b) \Rightarrow (x \in a))$.
    Si $f$ et $g$ sont deux formules, $(f \Rightarrow g) \wedge (g \Rightarrow a)$ est équivalente à $f \Leftrightarrow g$.
    Donc, $(a \subset b) \wedge (b \subset a)$ est équivalente à $\forall x \, (x \in a) \Leftrightarrow (x \in b)$, et donc à $a = b$.
    Donc, $((a \subset b) \wedge (b \subset a)) \Leftrightarrow (a = b)$ est équivalente à $\mathss{V}$.
    Donc, $\forall a \, \forall b \, ((a \subset b) \wedge (b \subset a)) \Leftrightarrow (a = b)$ est vraie.

    \done

\blank[medium]

La théorie Z comporte six axiomes (l'axiome d'extensionnalité et les cinq axiomes de construction) ainsi qu'un schéma d'axiomes, correspondant à un axiome par formule à un paramètre libre. 

\blank[medium]

\noindent\emph{\bold{Axiome d'extensionnalité :} Si deux ensembles possèdent les mêmes éléments, alors ils sont égaux.}
\startformula
    \forall a \, \forall b \, (
        \forall x \, ((x \in a) \Leftrightarrow (x \in b)) \Rightarrow (a = b)
    ). 
\stopformula
La réciproque est une conséquence directe des propriétés de l'égalité en logique du premier ordre.%
\footnote{ En effet, soit deux ensembles $a$ et $b$ tels que $a = b$, et soit $x$ un ensemble, et $P$ le prédicat à un paramètre libre définit par $P y: x \in y$, puisque $a = b$, on doit avoir $P(a) \Leftrightarrow P(b)$, et donc $(x \in a) \Leftrightarrow (x \in b)$.}
On définit la relation $\neq$ par : $\forall a \, \forall b \, (a \neq b) \Leftrightarrow \neg (a = b)$.

\blank[medium]

\noindent\bold{Lemme :} 
    On définit la relation $R$ sur les ensembles par : soit $a$ et $b$ deux ensembles $(a \mathrel{R} b)$ a la même valeur de vérité que $(\forall x \, (x \in a) \Leftrightarrow (x \in b))$. 
    Alors, les trois prédicats suivants sont vrais :
    \startitemize[nowhite]
      \item $\forall x \, (x \mathrel{R} x)$ (réciprocité)
      \item $\forall x \, \forall y \, (x \mathrel{R} y) \Rightarrow (y \mathrel{R} x)$ (réflexivité)
      \item $\forall x \, \forall y \, \forall z \, ((x \mathrel{R} y) \wedge (y \mathrel{R} z)) \Rightarrow (x \mathrel{R} z)$.
    \stopitemize
    Cela suggère que l'axiome d'extensionalité est compatible avec la définition de l'égalité en logique du premier ordre (même s'il manque le schéma d'axiomes de Leibniz pour assurer la cohérence).

\blank[medium]

\noindent\bold{Démonstration :} 
\startitemize[nowhite]
  \item Soit $x$ un ensemble. Pour tout $y$, $y \in x$ a la même valeur de vérité que $y \in x$ (trivialement, puisqu'il s'agit de la même formule).
      Donc, $\forall y \, (y \in x) \Leftrightarrow (y \in x)$. 
      Donc, $x \mathrel{R} x$.
  \item Soit $x$ et $y$ deux ensembles tels que $x \mathrel{R} y$. 
      Puisque $x = y$, on a : $\forall z \, z \in x \Leftrightarrow z \in y$. 
      Puisque le connecteur $\Leftrightarrow$ est symétrique, on a donc : $\forall z \, z \in y \Leftrightarrow z \in y$.
      Donc, $y \mathrel{R} x$.
  \item Soit $x$, $y$ et $z$ trois ensembles tels que $x \mathrel{R} y$ et $y \mathrel{R} z$. 
      Pour tout ensemble $a$, on a $a \in x \Leftrightarrow a \in y$ et $a \in y \Leftrightarrow a \in z$. 
      Donc, par transitivité du connecteur $\Leftrightarrow$, $a \in x \Leftrightarrow a \in z$. 
      Cela étant valable pour tout ensemble $a$, on en déduit que $x \mathrel{R} z$.
\stopitemize
\done

\blank[medium]

\noindent\bold{Démonstration bis :} À titre d'exercice, re-faisons ces courtes démonstrations de manière plus formelle.
\startitemize[nowhite]
  \item Soit $f$ la formule à deux paramètres libres $x$ et $y$ donnée par : $f: y \in x$. 
      Puisque $f \Leftrightarrow f$ est équivalente à $\mathss{V}$, la formule $\forall x \, \forall y \, (f \Leftrightarrow f)$ est vraie. 
      Donc, $\forall x \, \forall y \, (y \in x) \Leftrightarrow (y \in x)$ est vraie.
      Donc, $\forall x \, x \mathrel{R} x$ est vraie.
  \item Soit $f$ la formule à deux paramètres libres $a$ et $x$ donnée par : $f: a \in x$, et $g$ la formule à deux paramètres libres $a$ et $y$ donnée par : $g: a \in y$. 
      Les deux formules $f \Leftrightarrow g$ et $g \Leftrightarrow f$ sont équivalentes (elles sont toutes deux vraies si $f$ et $g$ ont la même valeur de vérité et fausses sinon). 
      Donc, les formules $\forall a \, (f \Leftrightarrow g)$ et $\forall a \, (g \Leftrightarrow f)$ sont équivalentes. 
      Puisque $\forall a \, (f \Leftrightarrow g)$ est équivalente à $x \mathrel{R} y$ et $\forall a \, (g \Leftrightarrow f)$ à $y \mathrel{R} x$, on en déduit que $x \mathrel{R} y$ et $y \mathrel{R} x$ sont équivalentes. 
      Donc, $(x \mathrel{R} y) \Rightarrow (y \mathrel{R} x)$ est équivalente à $h \Rightarrow h$, où $h$ est la formule donnée par $h: x \mathrel{R} y$.
      Puisque $h \Rightarrow h$ est vraie que $h$ soit vraie ou fausse, elle est équivalente à $\mathss{V}$. 
      Donc, $\forall x \forall y (h \Rightarrow h)$ est vraie.
      Donc, $\forall x \forall y (x \mathrel{R} y) \Rightarrow (y \mathrel{R} x)$ est vraie.
  \item Soit $f$ la formule à deux paramètres libres $a$ et $x$ donnée par : $f: a \in x$, $g$ la formule à deux paramètres libres $a$ et $y$ donnée par : $g: a \in y$, et $h$ la formule à deux paramètres libres $a$ et $z$ donnée par : $h: a \in z$. 
      Alors, $((f \Leftrightarrow g) \wedge (g \Leftrightarrow h)) \Rightarrow (f \Leftrightarrow h)$ est vraie quelles que soient les valeurs de vérité de $f$, $g$ et $h$. 
      Donc, si $\forall a \, ((f \Leftrightarrow g) \wedge (g \Leftrightarrow h))$ est vraie, alors $\forall a \, (f \Leftrightarrow h)$ est vraie.
      Donc, si $\forall a \, (f \Leftrightarrow g)$ et $\forall a \, (g \Leftrightarrow h)$ sont vraies, alors $\forall a \, (f \Leftrightarrow h)$ est vraie.
      Puisque $\forall a \, (f \Leftrightarrow g)$ est équivalente à $x \mathrel{R} y$, $\forall a \, (g \Leftrightarrow h)$ est équivalente à $y \mathrel{R} z$, et $\forall a \, (f \Leftrightarrow h)$ est équivalente à $x \mathrel{R} z$, on en déduit que $((x \mathrel{R} y) \wedge (y \mathrel{R} z)) \Rightarrow (x \mathrel{R} z)$ est toujours vraie.
      Donc, $\forall x \, \forall y \, \forall z \, ((x \mathrel{R} y) \wedge (y \mathrel{R} z)) \Rightarrow (x \mathrel{R} z)$ est vraie .
\stopitemize

\done

\blank[medium]

\noindent \bold{Lemme :} La relation $\subset$ satisfait les trois propriétés suivantes : 
\blank[samepage]
\startitemize[nowhite]
  \item \emph{Réflexivité :} $\forall x \, x \subset x$.
  \item \emph{Antisymétrie :} $\forall x \, \forall y \, (x \subset y) \wedge (y \subset x) \Rightarrow (x = y)$.
  \item \emph{Transitivité :} $\forall x \forall y \forall z \, (x \subset y) \wedge (y \subset z) \Rightarrow (x \subset z)$.
\stopitemize

\blank[medium]

\noindent \bold{Démonstration :} 
\blank[samepage]
\startitemize[nowhite]
  \item Soit $x$ un ensemble. Pour tout élément $e$ de $x$, on a (par définition), $e \in x$. 
      Donc, le prédicat $\forall e \, (e \in x) \Rightarrow (e \in x)$ est vrai.
      Donc, $x \subset x$.
  \item Soit $x$ et $y$ deux ensembles tels que $x \subset y$ et $y \subset x$.
      Soit $e$ un ensemble. 
      Si $e \in x$ est vrai, alors $e \in y$ est vrai aussi puisque $x \subset y$.
      Si $e \in x$ est faux, alors $e \in y$ est faux aussi, sans quoi on aurait $e \in y$ et donc $e \in x$ puisque $y \subset x$.
      Cela montre que $\forall e \, (e \in x) \Leftrightarrow (e \in y)$ est vrai. 
      Donc, d'après l'axiome d'extensionnalité, $x = y$ est vrai.
  \item Soit $x$, $y$ et $z$ trois ensembles tels que $x \subset y$ et $y \subset z$. 
      Soit $e$ un ensemble. 
      Si $e \in x$, alors $e \in y$ puisque $x \subset y$, et donc $e \in z$ puisque $y \subset z$. 
      Cela montre que le prédicat $\forall e \, (e \in x) \Rightarrow (e \in z)$ est vrai.
      Donc, $x \subset z$.
\stopitemize

\done

\blank[medium]

\noindent \bold{Démonstration bis :} 
\startitemize[nowhite]
  \item Soit $f$ la formule $f: e \in x$. 
      La formule $f \Rightarrow f$ est vraie que $f$ soit vraie ou fausse, donc elle est équivalente à $\mathss{V}$. 
      Donc, $\forall e \, (f \Rightarrow f)$ est équivalente à $\mathss{V}$.
      Donc, $\forall e \, ((e \in x) \Rightarrow (e \in x))$ est équivalente à $\mathss{V}$. 
      Donc, $x \subset x$ est équivalente à $\mathss{V}$. 
      Donc, $\forall x \, x \subset x$ est vraie. 
  \item La formule $(x \subset y) \wedge (y \subset x)$ est équivalente à : $(\forall e \, (e \in x \Rightarrow e \in y)) \wedge (\forall f \, (f \in y \Rightarrow f \in x))$, et donc à $\forall e \, ((e \in x \Rightarrow e \in y) \wedge (e \in y \Rightarrow e \in x))$. 
      Puisque $(e \in x \Rightarrow e \in y) \wedge (e \in y \Rightarrow e \in x)$ est équivalente à $(e \in x \Leftrightarrow e \in y)$, la formule $(x \subset y) \wedge (y \subset x)$ est équivalente à $x = y$.
      Donc, $((x \subset y) \wedge (y \subset x)) \Rightarrow (x = y)$ est équivalente à $\mathss{V}$. 
      Donc, $\forall x \, \forall y \, ((x \subset y) \wedge (y \subset x)) \Rightarrow (x = y)$ est vraie. 
  \item La formule $(x \subset y) \wedge (y \subset z)$ est équivalente à $(\forall e \, e \in x \Rightarrow e \in y) \wedge (\forall f \, f \in y \Rightarrow f \in z)$, donc à $\forall e \, ((e \in x \Rightarrow e \in y) \wedge (e \in y \Rightarrow e \in z))$. 
      Soit $f$, $g$ et $h$ trois formules, $((f \Rightarrow g) \wedge (g \Rightarrow h))$ est équivalente à $(f \Rightarrow h) \wedge (f \Rightarrow g)$.
      Donc, la formule $(x \subset y) \wedge (y \subset z)$ est équivalente à $\forall e \, ((e \in x \Rightarrow e \in z) \wedge (e \in x \Rightarrow e \in y))$, et donc à $(\forall e \, (e \in x \Rightarrow e \in z)) \wedge (\forall f \, (f \in x \Rightarrow f \in y))$, et donc à $(x \subset z) \wedge (x \subset y)$. 
      Puisque, si $g$ et $h$ sont deux formules, $g \wedge h \Rightarrow g$ est toujours vraie, $((x \subset z) \wedge (x \subset y)) \Rightarrow (x \subset z)$ est équivalente à $\mathss{V}$, donc on en déduit que $((x \subset y) \wedge (y \subset z)) \Rightarrow (x \subset z)$ est équivalente à $\mathss{V}$, donc $\forall x \, \forall y \, ((x \subset y) \wedge (y \subset z)) \Rightarrow (x \subset z)$ est vraie.
\stopitemize

\done

\blank[medium]

\noindent\bold{Lemme :} La proposition $\forall a \, \forall b \, (a = b) \Leftrightarrow [(a \subset b) \wedge (b \subset a)]$ est vraie.
    Autrement dit, pour tous ensembles $a$ et $b$, la formule $a = b$ est équivalente à $(a \subset b) \wedge (b \subset a)$.

\blank[medium]

\noindent\bold{Démonstration :} 
    Soit $a$ et $b$ deux ensembles. 
    \startitemize[nowhite]
      \item Supposons d'abord que $a = b$.
          Soit $x$ tel que $x \in a$. 
          Puisque $a = b$, on a $x \in b$. 
          Donc, $\forall x \, (x \in a) \Rightarrow (x \in b)$.
          Donc, $a \subset b$. 
          Puisque l'égalité est symétrique, on montre de même en échangeant les rôles de $a$ et $b$ que $b \subset a$.
          Donc, $(a \subset b) \wedge (b \subset a)$. 
      \item Supposons maintenant que $(a \subset b) \wedge (b \subset a)$. 
          Soit $x$ un ensemble. 
          Si $x \in a$, et puisque $a \subset b$, alors $x \in b$.
          De même, si $x \in b$, et puisque $b \subset a$, alors $x \in a$.
          Donc, $\forall x \, (x \in a) \Leftrightarrow (x \in b)$. 
          Donc, $a = b$. 
    \stopitemize
    On a donc montré que les formules $a = b$ et $(a \subset b) \wedge (b \subset a)$ son équivalentes, au sens où chacune est vraie qi l'autre l'est (et donc, également, fausse si l'autre l'est).

   \done 

\blank[medium]

\noindent\bold{Démonstration bis :} 
    La formule $(a \subset b) \wedge (b \subset a)$ est équivalente à : $(\forall x \, (x \in a \Rightarrow x \in b)) \wedge (\forall y \, (y \in b \Rightarrow y \in a))$, et donc à $\forall x \, ((x \in a \Rightarrow x \in b) \wedge (x \in b \Rightarrow x \in a))$. 
    Si $f$ et $g$ sont deux formules, $(f \Rightarrow g) \wedge (g \Rightarrow f)$ est équivalente à $f \Leftrightarrow g$ (toutes deux sont vraies si $f$ et $g$ sont toutes deux vraies ou toutes deux fausses, fausses si l'une est vraie et l'autre est fausse, et (en présence de la valeur de vérité $\mathss{I}$) indéfinies si $f$ ou $g$ l'est).
    Donc, $(x \in a \Rightarrow x \in b) \wedge (x \in b \Rightarrow x \in a)$ est équivalente à $x \in a \Leftrightarrow x \in b$.
    Donc, la formule $(a \subset b) \wedge (b \subset a)$ est équivalente à $\forall x \, (x \in a \Leftrightarrow x \in b)$, et donc à $a = b$.

    Donc, la formule $((a \subset b) \wedge (b \subset a)) \Leftrightarrow (a = b)$ est équivalente à $(a = b) \Leftrightarrow (a = b)$, et donc toujours vraie, et donc équivalente à $\mathss{V}$. 
    Donc, la formule $\forall a \, \forall b \, ((a \subset b) \wedge (b \subset a)) \Leftrightarrow (a = b)$ est vraie. 

    \done

\blank[medium]

\noindent\emph{\bold{Axiome de la paire :} La paire formée par deux ensembles est un ensemble :}
\startformula
    \forall a \, \forall b \, \exists c \, \forall x \, (
        (x \in c) \Leftrightarrow ((x = a) \vee (x = b))
    ).
\stopformula
Si $a$ et $b$ sont deux ensembles, on note $\lbrace a, b \rbrace$ leur paire. 
Il s'agit de l'ensemble contenant $a$ et $b$ mais aucun autre (au sens de « non égal à $a$ ni à $b$ ») ensemble. 
Cet ensemble est unique d'après l'axiome d'extensionnalité.
Si de plus $b = a$, alors $\lbrace a, b \rbrace$ ne contient qu'un seul élément. 
Il peut alors être abrégé en $\lbrace a \rbrace$. 
Puisque, pour tout $x$, la formule $(x = a) \vee (x=a)$ est équivalente à $x = a$, on a : 
\startformula
    \forall x \, (x \in \lbrace a \rbrace) \Leftrightarrow (x = a). 
\stopformula

\blank[medium]

\noindent\emph{\bold{Axiome de la réunion :} Pour tout ensemble $a$, il existe un ensemble qui est l'union des éléments de $a$ :}
\startformula
    \forall a \, \exists b \, \forall x \, (
        (x \in b) \Leftrightarrow (\exists y \, ((y \in a) \wedge (x \in y)))
    ).
\stopformula
La réunion d'un ensemble $a$ (noté $b$ dans la formule ci-dessus) est notée $\symbols{\cup} a$.
Cet ensemble est unique d'après l'axiome d'extensionnalité.
Si $a$ et $b$ sont deux ensembles, $\lbrace a, b \rbrace$ est aussi un ensemble d'après l'axiome de paire. 
La réunion de cet ensemble est notée $a \cup b$, et appelé \emph{\words{union}} de $a$ et $b$. 
Soit $a$, $b$ et $c$ trois ensembles. 
On note $\lb a, b, c \rb$ l'ensemble $\lbrace a, b \rbrace \cup \lbrace c \rbrace$. 

\blank[medium]

\noindent\bold{Lemme :} Soit $a$, $b$ et $x$ trois ensembles. 
    Le prédicat $x \in a \cup b$ est équivalent à $(x \in a) \vee (x \in b)$.

\blank[medium]

\noindent\bold{Démonstration :} 
    Le prédicat $x \in a \cup b$ est équivalent à $\exists y \, y \in \lbrace a, b \rbrace \wedge x \in y$, donc à $\exists y \, (y = a \vee y = b) \wedge x \in y$, donc à $\exists y \, ((y = a \wedge x \in y) \vee (y = b \vee x \in y))$, donc à $(\exists y \, y = a \wedge x \in y) \vee (\exists z \, z = b \wedge x \in z))$.
    Si $f$ est ue formule dépendant de deux paramètres libres $x$ et $y$ et si $a$ est un ensemble, alors $\exists (y = a) \wedge f(x,y)$ est équivalente à $f(x,a)$.
    En effet, si $f(x,a)$ est fausse, alors $(y = a) \wedge f(x,y)$ est fausse pour toute valeur de $y$ et, si elle est vraie, alors elle est vraie pour une valeur de $y$ (et cette valeur est $a$). 
    Donc, $x \in a \cup b$ est équivalente à $(x \in a) \vee (x \in b)$.

    \done

\blank[medium]

\noindent\emph{\bold{Axiome de l'ensemble des parties :} La collection des parties d'un ensemble est un ensemble :} 
\startformula
    \forall a \, \exists b \, \forall x \, (
        (x \in b) \Leftrightarrow (x \subset a)
    ).
\stopformula
Cet ensemble est unique d'après l'axiome d'extensionnalité. 
L'ensemble des parties (ou enseble des sous-ensembles) d'un ensemble $x$ est aussi appelé \emph{\words{ensemble puissance}} de $x$ et noté $\mathcal{P}(x)$.

\blank[medium]

\noindent\emph{\bold{Schéma d'axiomes de compréhension :} 
Pour tout prédicat $P$ à une variable libre $x$ et chaque ensemble $a$, il existe un ensemble qui a pour éléments l'ensemble des éléments de $a$ vérifiant la propriété $P$, c'est-à-dire :}
\startformula
\forall a \, \exists b \, \forall x \, [ (x \in b) \Leftrightarrow ((x \in a) \wedge P x)]. 
\stopformula

Avec les mêmes notations, cet ensemble est noté $\lbrace x \in a \vert P x \rbrace$. 
Il est unique d'après l'axiome d'extensionnalité. 
(En effet, si deux ensembles satisfont l'énoncé de l'axiome obtenu pour un même ensemble et une même propriété, alors tout élément de l'un appartient à l'autre.)
Ce schéma d'axiomes implique qu'il existe un ensemble vide, noté $\emptyset$, pourvu qu'au moins un ensemble $a$ existe—ce qui est nécessairement le cas puisque, en logique du premier ordre, les domaines d'interprétation des variables d'objets de base, ici les ensembles, sont non vides. 
On peut en effet le définir par : $\emptyset = \lbrace x \in a \vert x \neq x \rbrace$. 
Puisque tout ensemble $x$ satisfait $x = x$, il n'existe aucun $x$ tel que $x \in \emptyset$ ; autrement dit, la formule suivante est vraie : $\forall x \, x \notin \emptyset$. 
Cet ensemble est unique d'après l'axiome d'extensionnalité.

Notons que, puisque $\forall x \, x \notin \emptyset$ est vraie, $\exists x \, x \in \emptyset$ est fausse et $x \notin \emptyset$ est équivalente à $\mathss{V}$ et $x \in \emptyset$ à $\mathss{F}$.

\blank[medium]

\noindent\bold{Lemme :} Le prédicat suivant est vrai : $\forall x \, \emptyset \subset x$.

\blank[medium]

\noindent\bold{Démonstration :} Soit $x$ un ensemble. 
    La formule $\emptyset \subset x$ est équivalente à : $\forall e \, (e \in \emptyset) \Rightarrow (e \in x)$. 
    Or, pour tout ensemble $e$, $e \in \emptyset$ est faux, donc $(e \in \emptyset) \Rightarrow (e \in x)$ est vrai.
    Donc, $\forall e \, (e \in \emptyset) \Rightarrow (e \in x)$ est vrai. 
    Donc, $\emptyset \subset x$ est vrai.

\done

\blank[medium]

\noindent\bold{Démonstration bis :} 
    On veut montrer que le prédicat $P: \forall x \, \forall e \, (e \in \emptyset \Rightarrow e \in x)$ est vrai. 
    $P$ est équivalent à : $\forall x \, \forall e \, ((e \in x) \vee \neg (e \in \emptyset))$, c'est-à-dire, à : $\forall x \, \forall e \, ((e \in x) \vee (e \notin \emptyset))$.
    Puisque $\forall e \, e \notin \emptyset$ est vrai, $e \notin \emptyset$ est équivalent à $\mathss{V}$, donc $\forall e \, ((e \in x) \vee (e \notin \emptyset))$ est équivalent à $\forall e \, ((e \in x) \vee \mathss{V})$, donc à $\forall e \, \mathss{V}$, et donc à $\mathss{V}$.
    Donc, $P$ est vrai. 

    \done

\blank[medium]

\noindent\bold{Lemme :} Le prédicat suivant est vrai : $\forall x \, x \subset \emptyset \Rightarrow x = \emptyset$.

\blank[medium]

\noindent\bold{Démonstration :} Soit $x$ un ensemble satisfaisant $x \subset \emptyset$.
    Pour tout ensemble $y$, on a $y \notin \emptyset$, donc $y \notin x$.
    
\done

\blank[medium]

\noindent\bold{Démonstration bis :} 
    On veux montrer le prédicat $P: \forall x \, (x \subset \emptyset) \Rightarrow (x = \emptyset)$.
    Il est équivalent à : $\forall x \, (x \subset \emptyset) \Rightarrow ((x \subset \emptyset) \wedge (\emptyset \subset x))$, donc à $\forall x \, \neg (x \subset \emptyset) \vee ((x \subset \emptyset) \wedge (\emptyset \subset x))$, donc à $\forall x \, (\neg (x \subset \emptyset) \vee (x \subset \emptyset)) \wedge (\neg (x \subset \emptyset) \vee (\emptyset \subset x))$.
    Puisque $\neg (x \subset \emptyset) \vee (x \subset \emptyset)$ est toujours vrai (soit $f$ la formule $x \subset \emptyset$, il s'agit de $\neg f \vee f$, qui est vrai que $f$ soit vraie ou fausse), $P$ est équivalent à $\forall x \, (\neg (x \subset \emptyset) \vee (\emptyset \subset x))$.
    On a vu que $\forall x \, \emptyset \subset x$ est vrai.
    Donc, $\emptyset \subset x$ est équivalente à $\mathss{V}$.
    Donc, $P$ est équivalente à $\forall x \, (\neg (x \subset \emptyset) \vee \mathss{V})$, donc à $\forall x \, \mathss{V}$, et donc à $\mathss{V}$.
    Donc, $P$ est vrai.

    \done

\blank[medium]

L'axiome de compréhension peut aussi être utilisé pour définir la différence de deux ensembles. 
Soit $A$ et $B$ deux ensembles. 
On note $A \setminus B$ l'ensemble $\lbrace x \in A \vert x \notin B \rbrace$.

\blank[medium]

Notons qu'il s'agit bien d'un schéma d'axiomes, c'est-à-dire une méthode permettant de construire des axiomes, et non d'un seul axiome : puisqu'on ne peut pas quantifier les prédicats en logique du premier ordre, ce schéma définit un axiome pour chaque prédicat à un paramètre libre. 
En théorie Z, on considère le prédicat obtenu à partir de tout prédicat $P$ à une variable libre comme vrai.

Ce schéma peut être reformulé en notant que, si $P$ est un prédicat à une variable libre $x$ et d'autres variables libres éventuelles $a_1 \cdots a_p$, et si $\alpha 1 \dots \alpha_p$ est une collection d'ensembles pouvant remplacer $a_1 \dots a_p$, alors le prédicat $Q$ défini par $Q: P x \alpha1 \cdots \alpha_p$ a une unique variable libre $x$. 
Le schéma d'axiomes de compréhension peut ainsi être reformulé de la manière suivante : 
\emph{Pour tout prédicat $P$ à une variable libre $x$ et d'éventuels autres variables libres collectivement notées $a_1 \dots a_p$, pour chaque valeur des variables $a_1 \cdots a_p$ et chaque ensemble $b$, il existe un ensemble qui a pour éléments l'ensemble des éléments de $b$ vérifiant la propriété $P x a_1 \dots a_p$, c'est-à-dire :}
\startformula
\forall a_1 \dots a_p \, \forall b \, \exists c \, \forall x \, [ (x \in c) \Leftrightarrow ((x \in b) \wedge P x a_1 \dots a_p)]. 
\stopformula
(Dans cette formule, il est entendu que le premier quantificateur est absent si $P$ n'a qu'une seule variable libre.) 

\blank[medium]

\noindent\bold{Lemme :} Soit $A$ et $B$ deux ensembles.
    Alors, $(A \setminus B) \cup B = A \cup B$.

\blank[medium]

\noindent\bold{Démonstration :}
    Soit $x$ un élément de $A \cup B$. 
    Si $x \in B$, alors $x \in (A \setminus B) \cup B$.
    Sinon, $x \in A$, donc $x \in A \setminus B$, donc $x \in (A \setminus B) \cup B$.
    Donc, dans tous les cas, $x \in (A \setminus B) \cup B$.

    Soit $x$ un élément de $(A \setminus B) \cup B$.
    Alors, $x \in A \setminus B$ ou $x \in B$. 
    Si $x \in A \setminus B$, alors $x \in A$ puisque $A \setminus B \subset A$, donc $x \in A \cup B$.
    Si $x \in B$, alors $x \in A \cup B$.
    Donc, dans tous les cas, $x \in A \cup B$.

    On a donc montré que : $\forall x \, (x \in A \cup B) \Leftrightarrow (x \in (A \setminus B) \cup B)$, et donc que $(A \setminus B) \cup B = A \cup B$.

    \done

\blank[medium]

\noindent\bold{Démonstration bis :} 
    Le prédicat $x \in A \cup B$ est équivalent à $(x \in A) \vee (x \in B)$.
    Le prédicat $x \in (A \setminus B) \cup B$ est équivalent à $(x \in A \setminus B) \vee (x \in B)$, et donc à $((x \in A) \wedge (x \notin B)) \vee (x \in B)$.
    Ce dernier est équivalent à $((x \in A) \vee (x \in B)) \wedge ((x \notin B) \vee (x \in B))$.
    Pour toute formule $f$, $(\neg f) \vee f$ est vrai que $f$ soit vraie ou fausse, donc équivalent à $\mathss{V}$.
    Donc, $x \in (A \setminus B) \cup B$ est équivalent à $((x \in A) \vee (x \in B)) \wedge \mathss{V}$, donc à $(x \in A) \vee (x \in B)$, et donc à $x \in A \cup B$.
    Donc, $(x \in A \cup B) \Leftrightarrow (x \in (A \setminus B) \cup B)$ est équivalent à $(x \in A \cup B) \Leftrightarrow (x \in A \cup B)$.
    Pour toute formule $f$, $f \Leftrightarrow f$ est vrai que $f$ soit vraie ou fausse, et donc équivalent à $\mathss{V}$.
    Donc, $\forall x \, (x \in A \cup B) \Leftrightarrow (x \in (A \setminus B) \cup B)$ est vrai.

\done

\blank[medium]

\noindent\bold{Lemme :} Soit $A$ et $B$ deux ensembles tels que $B \subset A$.
    Alors, $A \cup B = A$.

\blank[medium]

\noindent\bold{Démonstration :}
    Soit $x$ un élément de $A \cup B$. 
    Alors, $x \in A$ ou $x \in B$.
    Si $x \in B$, et puisque $B \subset A$, $x \in A$.
    Donc, $x \in A$.

    Soit $x$ un élément de $A$, on a $x \in A \cup B$.

    Ainsi, $A \cup B = A$.

    \done

\blank[medium]

\noindent\bold{Démonstration bis :} 
    Puisque $B \subset A$, le prédicat $\forall x \, (x \in B) \Rightarrow (x \in A)$ est vrai.
    Donc, le prédicat $(x \in B) \Rightarrow (x \in A)$ est équivalent à $\mathss{V}$.
    Donc, le prédicat $(x \in A) \vee (x \notin B)$ est équivalent à $\mathss{V}$.
    
    Le prédicat $x \in A \cup B$ est équivalent à $(x \in A) \vee (x \in B)$.
    Puisque, pour tout prédicat $P$, $P \wedge \mathss{V}$ est équivalent à $P$, $x \in A \cup B$ est équivalent à $((x \in A) \vee (x \in B)) \wedge ((x \in A) \vee (x \notin B))$, et donc à $(x \in A) \vee ((x \in B) \wedge (x \notin B))$.
    Puisque, pour tout prédicat $P$, $P \wedge \neg P$ est équivalent à $\mathss{F}$, cela est équivalent à $(x \in A) \vee \mathss{F}$, et donc à $x \in A$.
    Donc, $x \in A \cup B$ est équivalent à $x \in A$.
    Donc, $\forall x \, x \in A \cup B \Leftrightarrow x \in A$ est vrai.
    Donc, $A \cup B = A$.

\done

\blank[medium]

\noindent\emph{\bold{Axiome de l'infini :} Il existe un ensemble contenant l'ensemble vide et clos par application du successeur $x \mapsto x \cup \lbrace x \rbrace$.} Formellement, cet axiome s'écrit : 
\startformula
    \exists Y \, (\emptyset \in Y) \wedge (\forall y \, ((y \in Y) \Rightarrow (y \cup \lbrace y \rbrace \in Y))). 
\stopformula

\blank[medium]

L'ensemble ainsi défini contient $\emptyset$, $\lbrace \emptyset \rbrace$, $\lbrace \emptyset, \lbrace \emptyset \rbrace \rbrace$, $\lbrace \emptyset, \lbrace \emptyset \rbrace, \lbrace \emptyset, \lbrace \emptyset \rbrace \rbrace \rbrace$, ...

\blank[medium]

\noindent\bold{Notation :} Soit $E$ un ensemble et $P$ un prédicat dépendant des variables $x$, $a$, ..., $b$.
    On peut noter par 
    \startitemize[nowhite]
      \item $\forall x \in E \, P(x, a, \dots, b)$ le prédicat $\forall x \, x \in E \Rightarrow P(x, a, \dots, b)$,
      \item $\exists x \in E \, P(x, a, \dots, b)$ le prédicat $\exists x \, x \in E \wedge P(x, a, \dots, b)$.
    \stopitemize

\stopsubsection

\stopsection
