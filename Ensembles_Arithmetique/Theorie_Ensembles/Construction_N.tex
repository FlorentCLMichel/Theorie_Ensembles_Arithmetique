\startsection[title=Construction de $\mathbb{N}$, reference=sec:constN]

\startsubsection[title=Définition]

L'ensemble des entiers naturels, noté \symbols{$\mathbb{N}$}, est défini de la manière suivante. 
Notons $\mathrm{Cl}$ le prédicat à un paramètre libre défini par :
\startformula
  \mathrm{Cl}(A): (\emptyset \in A) \wedge (\forall a \, (a \in A \Rightarrow a \cup \lbrace a \rbrace \in A)). 
\stopformula
D'après l'axiome de l'infini, il existe un ensemble $A$ tel que $\mathrm{Cl}(A)$ est vrai.
Soit $\mathrm{Ent}$ le prédicat à un paramètre libre défini par : 
\startformula
  \mathrm{Ent}(x): \forall A \, (\mathrm{Cl}(A) \Rightarrow x \in A). 
\stopformula
Soit $I$ un ensemble tel que $\mathrm{Cl}(I)$ est vrai. 
L'ensemble $\mathbb{N}$ est défini par : 
\startformula
  \mathbb{N} = \lbrace x \in I \vert \mathrm{Ent}(x) \rbrace. 
\stopformula
Notons que cette définition ne dépend pas du choix de $I$. 
Notons aussi que $\emptyset \in \mathbb{N}$ et $\forall n \, n \in \mathbb{N} \Rightarrow n \cup \lbrace n \rbrace \in \mathbb{N}$.

\blank[medium]

\noindent\bold{Démonstration :}

\startitemize[nowhite]
  \item Montrons d'abord que $\emptyset \in \mathbb{N}$. 
    Puisque $\mathrm{Cl}(I)$ est vrai, $\emptyset \in I$.
    Soit $A$ un ensemble tel que $\mathrm{Cl}(A)$ est vrai.
    Alors, $\emptyset \in A$.
    Donc, $\mathrm{Ent}(\emptyset)$ est vrai. 
    On a donc $\emptyset \in I \wedge \mathrm{Ent}(\emptyset)$. 
    Donc, $\emptyset \in \mathbb{N}$.
  \item Soit $n$ un élément de $\mathbb{N}$. 
    Alors, $n \in I$.
    Puisque $\mathrm{Cl}(I)$ est vrai, on en déduit que $n \cup \lbrace n \rbrace \in I$.
    Soit $A$ un ensemble tel que $\mathrm{Cl}(A)$ est vrai.
    Puisque $\mathrm{Ent}(n)$ est vrai, $n \in A$.
    Alors, puisque $\mathrm{Cl}(A)$ est vrai, $n  \cup \lbrace n \rbrace \in A$.
    On en déduit que $\mathrm{Ent}(n  \cup \lbrace n \rbrace)$ et vrai. 
    On a donc $n  \cup \lbrace n \rbrace \in I \wedge \mathrm{Ent}(n  \cup \lbrace n \rbrace)$. 
    Donc, $n  \cup \lbrace n \rbrace \in \mathbb{N}$.
  \item Motrons finalement que la définition de $\mathbb{N}$ ne dépends pas du choix de $I$. 
    Soit $J$ un ensemble tel que $\mathrm{Cl}(J)$ est vrai.
    Soit $\mathbb{M}$ l'ensemble défini par : $\mathbb{M} = \lbrace x \in J \vert \mathrm{Ent}(x) \rbrace$. 
    Il s'agit de montrer que $\mathbb{M} = \mathbb{N}$. \\
    Soit $x$ un élément de $\mathbb{N}$. 
    Puisque $\mathrm{Cl(J)}$ et $\mathrm{Ent(x)}$ sont vrais, $x \in J$ est vrai aussi. 
    Donc, $x \in J \wedge \mathrm{Ent}(x)$. 
    Donc, $x \in \mathbb{M}$. 
    Cela montre que $\mathbb{N} \subset \mathbb{M}$. \\
    Soit $y$ un élément de $\mathbb{M}$. 
    Puisque $\mathrm{Cl(I)}$ et $\mathrm{Ent(y)}$ sont vrais, $y \in I$ est vrai aussi. 
    Donc, $y \in I \wedge \mathrm{Ent}(x)$. 
    Donc, $y \in \mathbb{N}$. 
    Cela montre que $\mathbb{M} \subset \mathbb{N}$. \\
    On a donc $\mathbb{M} = \mathbb{N}$.
\stopitemize

\done

\blank[medium]

On notera souvent $0$ l'ensemble $\emptyset$. 
Pour tout élément $n$ de $\mathbb{N}$, on notera $n+1$ l'ensemble $n \cup \lbrace n \rbrace$, appelé \emph{\words{successeur}} de $n$. 
Cela définit une application $\mathrm{Suc}$ de $\mathbb{N}$ vers lui-même, qui à un élément $n$ associe $n+1$. 
Notons que, pour tout entier naturel $n$, on a $n \subset n+1$.
Les premiers entiers sont notés de la manière suivante en base $10$ (voir section~\ref{sub:base} pour une définition générale de la base) : 

\startplacetable[location=force,number=no]
  \starttabulate[|c|c|][rulethickness=0.8pt,distance={0pt,0pt}]
    \NC $n$ \NC $n+1$ \NC\NR\TB[-2ex]
    \NC\NC\NC\NR
    \HL
    \NC\NC\NC\NR\TB[-2ex]
    \NC $0$ \NC $1$ \NC\NR
    \NC $1$ \NC $2$ \NC\NR
    \NC $2$ \NC $3$ \NC\NR
    \NC $3$ \NC $4$ \NC\NR
    \NC $4$ \NC $5$ \NC\NR
    \NC $5$ \NC $6$ \NC\NR
    \NC $6$ \NC $7$ \NC\NR
    \NC $7$ \NC $8$ \NC\NR
    \NC $8$ \NC $9$ \NC\NR
    \NC $9$ \NC $10$ \NC\NR
  \stoptabulate
\stopplacetable

\noindent\bold{NB:} Notons que $\mathrm{Cl}(\mathbb{N})$ est vraie et, si $E$ est un ensemble tel que $\mathrm{Cl}(E)$ est vraie, alors $\mathbb{N} \subset E$. 
En ce sens, $\mathbb{N}$ est le plus petit ensemble satisfaisant $\mathrm{Cl}$.

\blank[medium]

\noindent\bold{Démonstration :}

Tout d'abord, on a vu ci-dessus que $\emptyset \in \mathbb{N}$ et $\forall n \, n \in \mathbb{N} \Rightarrow n \cup \lbrace n \rbrace \in \mathbb{N}$. 
Donc, $\mathrm{Cl}(\mathbb{N})$ est vrai. 

Soit $E$ un ensemble tel que $\mathrm{Cl}(E)$ est vrai. 
Soit $x$ un élément de $\mathbb{N}$. 
Alors, $\mathrm{Ent}(x)$ est vrai, donc $x \in E$. 
Ainsi, $\mathrm{N} \subset E$. 

\done

\blank[medium]

Un élément de $\mathbb{N}$ est dit \emph{\words{entier naturel}} (ou parfois simplement \emph{entier} quand il n'y a pas de confusion possible avec d'autres définitions). 
Il est dit \emph{non nul} s'il est différent de $0$.

\blank[medium]

\noindent\bold{Lemme :} Soit $n$ un élément de $\mathbb{N}$ et $m$ un ensemble tel que $m \subset n+1$. 
    Alors $n \in m$ ou $m \subset n$.

\blank[medium]

\noindent\bold{Démonstration :}
    Si $n \in m$, le résultat est vrai. 
    Supposons que $n \notin m$. 
    Soit $x$ un élément de $m$. 
    Puisque $m \subset n+1$, on a $x \in n+1$, et donc $x \in n$ ou $x \in \lbrace n \rbrace$. 
    La seconde option est impossible puisqu'elle impliquerait $x = n$, et donc $n \in m$, en contradiction avec notre hypothèse. 
    Donc, $x \in n$. 
    Cela étant vrai pour tout élément $x$ de $m$, on en déduit $m \subset n$.

   \done 

\blank[medium]

\noindent\bold{Définition :} 
    On note \symbols{$\mathbb{N}^*$} l'ensemble $\mathbb{N} \setminus \lbrace 0 \rbrace$.

\stopsubsection

\startsubsection[title=Relation d'ordre : définition]

On définit une relation binaire, notée $\leq$, sur $\mathbb{N}$ par : pour tous éléments $n$ et $m$ de $\mathbb{N}$, 
\startformula
  n \leq m \Leftrightarrow n \subset m .
\stopformula
Il sagit d'une relation d'ordre puisque la relation $\subset$ est réflexive, antisymétrique et transitive.
On définit la relation d'ordre strict $<$ par pour tous éléments $n$ et $m$ de $\mathbb{N}$, 
\startformula
  n < m \Leftrightarrow (n \leq m \wedge m \neq n).
\stopformula
Notons que, pour tout élément $n$ de $\mathbb{N}$, $0 \leq n$ (puisque l'ensemble vide est un sous-ensemble de tout ensemble) et $n < n+1$ (en effet, on a $n \subset n+1$, donc $n \leq n+1$, et $n \neq n+1$ ; nous démontrerons ce point dans la Section~\ref{sub:relOrdreProps}—pour le moment, nous avons seulement montré que $n \leq n+1$). 
Par antisymétrie, le premier point implique que le seul élément $n$ de $\mathbb{N}$ satisfaisant $n \leq 0$ est $0$ lui-même.

On définit aussi la relation d'ordre $\geq$ et la relation d'ordre strict $>$ sur $\mathbb{N}$ par : pour tous éléments $n$ et $m$ de $\mathbb{N}$, $n \geq m \Leftrightarrow m \leq n$ et $n > m \Leftrightarrow m < n$.

\stopsubsection

\startsubsection[title=Principe de récurrence, reference=sub:recurrence]

\noindent\bold{Lemme (principe de \words{récurrence}) :} Soit $P$ une formule à un paramètre libre. 
    On suppose que $P(0)$ est vraie et que, pour tout élément $n$ de $\mathbb{N}$, $P(n) \Rightarrow P(n+1)$ est vraie. 
    Alors, $P(n)$ est vraie pour tout $n \in \mathbb{N}$.
    \index{Récurrence}

On dira parfois, dans ce contexte, que $P$ est vraie « au rang $n$ » pour signifier que $P(n)$ est vrais. 
Ainsi, démontrer $P$ parrécurrence revient à montrer que $P(0)$ est vraie et que, pour tout entier naturel $n$, « si $P$ est vraie au rang $n$, alors $P$ est vraie au rang $n+1$ ».

\blank[medium]

\noindent\bold{Démonstration :} Soit $E$ l'ensemble défini par $E = \lbrace n \in \mathbb{N} \vert P(n) \rbrace$. 
    Puisque $P(0)$ et vraie, on a $0 \in E$. 
    En outre, pour tout $n \in E$, $P(n)$ est vrai, donc $P(n+1)$ est vrai aussi, et donc $n+1 \in E$. 
    Donc, $\mathrm{Cl}(E)$ est vraie. 
    Donc, $\mathbb{N} \subset E$. 
    Soit $n \in \mathbb{N}$, on a donc $n \in E$, et donc $P(n)$ est vraie.

   \done 

\blank[medium]

\noindent\bold{Lemme (\words{Récurrence finie}) :} Soit $E$ un sous-ensemble non vide de $\mathbb{N}$ tel que : $\forall n \in E, \, \forall m \in \mathbb{N}, \, m \leq n \Rightarrow m \in E$. 
    Soit $P$ une formule à un paramètre libre. 
    On suppose que $P(0)$ est vraie et que, pour tout $n \in E$ tel que $n+1 \in E$, $P(n) \Rightarrow P(n+1)$ est vraie. 
    Alors, $P(n)$ est vraie pour tout $n \in E$.
    \index{Récurrence finie}

\blank[medium]

\noindent\bold{Démonstration :} Notons que $0 \in E$. Définissons la formule $Q$ à un paramètre libre par $Q(n): P(n) \vee (n \notin E)$.
    Alors $Q(0)$ est vraie. 
    En outre, soit $n \in \mathbb{N}$ tel que $Q(n)$ est vraie, soit $n+1 \in E$, donc $n \in E$, donc $P(n)$ est vraie, donc $P(n+1)$ est vraie, et donc $Q(n+1)$ est vraie, soit $n+1 \notin E$ et donc $Q(n+1)$ est vraie.
    Par récurrence, $Q(n)$ est vraie pour tout $n \in \mathbb{N}$.
    Soit $n \in E$. 
    Puisque $Q(n)$ est vraie et que $n \notin E$ ne peut être vraie, on en déduit que $P(n)$ est vraie.

   \done 

\blank[medium]

\noindent
Donnons un exemple facile de démonstration par récurrence. 

\blank[medium]

\noindent\bold{Lemme :} Soit $n$ un entier naturel. Alors $n=0$ ou il existe un entier naturel $m$ tel que $n=m+1$.

\blank[medium]

\noindent\bold{Remarque :} On montre Section~\ref{subsub:relOrdreProps} que deux entiers naturels $a$ et $b$ satisfaisant $a+1=b+1$ sont égaux. Donc, l'entier naturel $m$ défini par l'énoncé du lemme est unique. 

\blank[medium]

\noindent\bold{Démonstration :} Soit $P$ le prédicat à un paramètre défini par : $P(n): n=0 \vee (\exists m \in \mathbb{N}, \, n=m+1)$. 
    Tout d'abord, $P(0)$ est vrai puisque $0=0$ est vrai.
    Soit $n$ un élément de $\mathbb{N}$. 
    Alors $P(n+1)$ est vrai puisqu'il existe un entier naturel $m$ tel que $n+1 = m+1$—il suffit de prendre $m=n$. 
    Par récurrence, $P(n)$ est donc vrai pour tout élément $n$ de $\mathbb{N}$.

   \done 

\blank[medium]

\noindent\bold{Définition :} Soit $n$ un entier naturel tel que $n \neq 0$. L'entier naturel $m$ tel que $n=m+1$ est noté $n-1$.
    Notons que, pour tout entier naturel $n$, $(n+1)-1 = n$ et, si $n \neq 0$, $(n-1)+1 = n$.

\blank[medium] 

Cet exemple est conceptuellement très simple car la seconde étape du raisonnement ne fait pas appel au fait que le prédicat est vrai au rang précédent. 
Donnons maintenant un exemple légèrement moins aisé, et plus proche de la manière dont la démonstration par récurrence fonctionne la plupart du temps. 
On admet momentanément que, pour tout entier naturel $n$, $n+1 \neq n$, et donc $n \notin n$. 
(Cela sera démontré, sans utiliser le lemme suivant, section~\ref{subsub:relOrdreProps}.)

\blank[medium]

\noindent\bold{Lemme :} Soit $n$ et $m$ deux entiers naturels. S'il existe une bijection de $n$ vers $m$, alors $n=m$.

\blank[medium]

\noindent\bold{Démonstration :} Considérons le prédicat suivant, dépendant d'un paramètre libre $n$ : \emph{Pour tout entier naturel $m$, s'il existe une bijection de $n$ vers $m$, alors $n=m$.} 

    Pour $n=0$, le résultat est aisé : la seule fonction de $0$ vers un ensemble est $\emptyset$, dont l'image est $\emptyset$. 
    Si $E$ est un ensemble et s'il existe une biection de $0$ vers $E$, alors $E = \emptyset = 0$.

    Soit $n$ un entier naturel pour lequel le prédicat est vrai. 
    Soit $m$ un entier naturel et $f$ une bijection de $n+1$ vers $m$. 
    Puisqu'une telle bijection existe et $n+1$ est non vide (il contient au moins $n$), $m$ ne peut être égal à $0$ (il contient au moins les images des éléments de $n+1$). 
    Donc, d'après le lemme précédent, on peut choisir un entier naturel $k$ tel que $m = k+1$. 
    Montrons qu'il existe une bijection de $n$ vers $k$. 
    On aura alors $n=k$, donc $n+1=k+1$, et donc $n+1=m$, et le lemme sera montré par récurrence.

    On a : $m = k \cup \lbrace k \rbrace$. 
    Soit $g$ la fonction de $m$ vers $m$ définie par $g(k) = f(n)$, $g(f(n)) = k$ (notons que cela est toujours possible car ces deux conditions sont équivalentes si $f(n)=k$) et $g(x)=x$ pour tout élément $x$ de $m$ tel que $x \notin \lbrace k, f(n) \rbrace$. 
    Supposons avoir montré que $g$ est une bijection de $m$ vers $m$. 
    Alors, $g \circ f$ est une bijection de $n+1$ vers $m$ et $(g \circ f)(n) = k$. 
    Soit $h$ la fonction de $n$ vers $k$ définie par : pour tout élément $x$ de $n$, $h(x) = (g \circ f)(x)$. 
    (Son image est bien incluse dans $k$ puisque, pour tout élément $x$ de $n$, $(g \circ f)(x) \in m$ et $(g \circ f)(x) \neq k$ puisque $x \neq n$ (car $x \in n$ et $n \notin n$).)
    Montrons que $h$ est une bijection. 
    Soit $x$ et $y$ deux éléments de $n$ tels que $h(x)=h(y)$.
    Alors, $(g \circ f)(x) = (g \circ f)(y)$.
    Puisque  $g \circ f$ est une bijection, cela implique $x=y$.
    Donc, $h$ est injective. 
    Soit $y$ un élément de $k$. 
    Puisque $g \circ f$ est bijective, on peut choisir un élément $x$ de $n+1$ tel que $(g \circ f)(x) = y$. 
    En outre, $y \in k$, donc $y \neq k$. 
    Puisque $(g \circ f)(n)=k$, cela implique $x \neq n$, et donc $x \in n$. 
    On a donc $h(x) = y$. 
    Donc, $h$ est surjective.
    La fonction $h$ est ainsi une bijection de $n$ vers $k$.

    Il nous reste à montrer que la fonction $g$ est bijective. 
    Montrons d'abord qu'elle est injective. 
    Soit $x$ et $y$ deux éléments de $m$ tels que $g(x) = g(y)$.
    Si ni $x$ ni $y$ ne sont dans $\lbrace f(n), k \rbrace$, alors $g(x) = x$ et $g(y) = y$, donc $x=y$. 
    Si $x \in \lbrace f(n), k \rbrace$, alors $y \in \lbrace f(n), k \rbrace$ (sans quoi on aurait $g(x) \in \lbrace f(n), k \rbrace$ et $g(y) \notin \lbrace f(n), k \rbrace$). 
    De même, si $y \in \lbrace f(n), k \rbrace$, alors $x \in \lbrace f(n), k \rbrace$. 
    Supposons $x = k$. 
    Alors, $g(x) = f(n)$, donc $g(y) = f(n)$. 
    Si $y = f(n)$, on a $g(y) = k$, ce qui contredit $g(x) = g(y)$ sauf si $f(n) = k$. 
    Donc, $y = k$ ou $y = f(n)$ et $f(n) = k$. 
    Dans les deux cas, on a $y = k$, et donc $y = x$. 
    Enfin, supposons $x = f(n)$. 
    Alors, $g(x) = k$, donc $g(y) = k$. 
    Si $y = k$, on a $g(y) = f(n)$, ce qui contredit $g(x) = g(y)$ sauf si $k = f(n)$. 
    Donc, $y = f(n)$ ou $y = k$ et $k = f(n)$. 
    Dans les deux cas, on a $y = f(n)$, et donc $y = x$. 
    Ainsi, $g$ est bien injective. 

    Montrons qu'elle est surjective. 
    Soit $y$ un élément de $m$. 
    Si $y \notin \lbrace k, f(n) \rbrace$, on a $g(y) = y$. 
    Si $y \in \lbrace k, f(n) \rbrace$, on a $y = k$ ou $y = f(n)$. 
    Dans le premier cas, $g(f(n)) = y$. 
    Dans le second cas, $g(k) = y$. 
    Dans tous les cas, il existe donc un élément $x$ de $m$ tel que $g(x) = y$. 
    Ainsi, $g$ est surjective. 
    Il s'agit donc bien d'une bijection. 

   \done 

\blank[medium]

Pour prouver par récurrence qu'un prédicat $P$ dépendant d'une variable libre $n$ est vraie pour tout entier naturel $n$ supérieur ou égal à un entier naturel $n_0$, on pourra procéder comme suit : 
\startitemize[nowhite]
  \item Montrer que $P(n_0)$ est vrai.
  \item Pour tout entier naturel $n$ tel que $n \geq n_0$ et $P(n)$ est vrai, montrer que $P(n+1)$ est vrai.
\stopitemize
Dans ce schéma de raisonnement, $n$ pourra être appelé \emph{rang}, et le prédicat $P$ dit \emph{vraie au rang $n$} si $P(n)$ est vrai.

\stopsubsection

\startsubsection[title={Relation d'ordre : propriétés}, reference=sub:relOrdreProps]

\noindent\bold{Lemme :} Pour tout élément $n$ de $\mathbb{N}$, $0 \leq n$.

\blank[medium]

\noindent\bold{Démonstration :} Évident car $\emptyset \subset E$ pour tout ensemble $E$. 

\done 

\blank[medium]

\noindent
\bold{Lemme :} Pour tout élément $n$ de $\mathbb{N}$, $n \leq 0 \Rightarrow n = 0$.

\blank[medium]

\noindent\bold{Corolaire :} Il n'existe aucun élément $n$ de $\mathbb{N}$ tel que $n < 0$.

\blank[medium]

\noindent\bold{Démonstration :} Conséquence directe du lemme précédent et de l'antisymétrie de $\leq$. 

   \done 

\blank[medium]

\noindent\bold{Lemme :} Pour tout élément $n$ de $\mathbb{N}$, on a $n \neq 0 \Rightarrow 0 \in n$.

\blank[medium]

\noindent\bold{Démonstration :} 
    On procède par récurrence. 
    Soit $P: n \neq 0 \Rightarrow 0 \in n$.
    Pour $n = 0$, le résultat est évident car $n \neq 0$ est fausse, donc $P(0)$ est vraie. 
    Soit $n$ un élément de $\mathbb{N}$ tel que $P(n)$ est vraie. 
    Si $n = 0$, $n+1 = \lbrace \emptyset \rbrace$, donc $0 \in n+1$, donc $P(n+1)$ est vraie.
    Si $n \neq 0$, $0 \in n$, donc, puisque $n \subset n+1$, $0 \in n+1$, donc $P(n+1)$ est vraie.
    Par récurrence, on en déduit que $P(n)$ est vraie pour tout élément $n$ de $\mathbb{N}$.

   \done 

\blank[medium]

\noindent\bold{Lemme :} 
    Soit $n$ un élément de $\mathbb{N}$. 
    Pour tout élément $m$ de $\mathbb{N}$ tel que $n \leq m$, on a $m \notin n$.
    En particulier, pour tout élément $n$ de $\mathbb{N}$, on a $n+1 \neq n$. 
    Puisque $n \subset n+1$, $n \leq n+1$, donc cela implique $n < n+1$. 

\blank[medium]

\noindent\bold{Démonstration :} 
    Montrons d'abord que la première partie du lemme implique bien le cas particulier. 
    Soit $n$ un élément de $\mathbb{N}$. 
    On a $n \in n+1$. 
    Si la première partie du lemme est vraie, on a aussi $n \notin n$, d'où $n+1 \neq n$.

    Montrons maintenant la première partie du lemme. 
    On procède par récurrence. 
    La propriété attendue est évidente pour $0$ puisqu'il s'agit de l'ensemble vide. 

    Soit $n$ un élément de $\mathbb{N}$ et supposons que, pour tout élément $m$ de $\mathbb{N}$ tel que $n \leq m$, $m \notin n$. 
    Soit $m$ un élément de $\mathbb{N}$ tel que $n+1 \leq m$. 
    Puisque $n \subset n+1$ et $n+1 \subset m$, on a $n \subset m$, et donc $n \leq m$.
    Donc, $m \notin n$. 
    En outre, $n \in n+1$ et $n \notin n$ (puisque $n \leq n$), donc $n+1 \subset n$ ne peut être vrai, donc $m \neq n$, donc $m \notin \lbrace n \rbrace$. 
    Puisque $n+1 = n \cup \lbrace n \rbrace$, on en déduit $m \notin n+1$. 
    La propriété attendue est donc vraie pour $n+1$.

    Par récurrence, la propriété est vraie pour tout élément $n$ de $\mathbb{N}$.

   \done 

\blank[medium]

\noindent\bold{Lemme :} 
    Soit $n$ un élément de $\mathbb{N}$. 
    Pour tout élément $m$ de $\mathbb{N}$ tel que $m \in n$, on a $n > m$.

\blank[medium]

\noindent\bold{Démonstration :} 
    On procède par récurrence sur $n$. 
    Pour $n = 0$, le résultat est évident puisqu'aucun élément $m$ de $\mathbb{N}$ ne satisfait $m \in 0$.
    Soit $n$ un élément de $\mathbb{N}$ satisfaisant la propriété énoncée dans le lemme. 
    Soit $m$ un élément de $\mathbb{N}$ tel que $m \in n+1$. 
    Alors, $m \in n$ ou $m = n$.
    \startitemize[nowhite]
        \item Si $m \in n$, on a $n > m$.
            En outre, d'après le lemme précédent, on a $n+1 > n$. 
            Donc, $n+1 > m$.
        \item Si $m = n$, on a $n+1 > m$ d'après le lemme précédent.
    \stopitemize
    Ainsi, $n+1$ satisfait également la propriété énoncée dans le lemme. 
    Par récurrence, on en déduit qu'elle est vraie pour tout élément $n$ de $\mathbb{N}$.
    
   \done 

\blank[medium]

\noindent\bold{Lemme :} 
    Soit $n$ un élément de $\mathbb{N}$. 
    Pour tout élément $m$ de $\mathbb{N}$ tel que $m < n$, on a $m \in n$.

\blank[medium]

\noindent\bold{Démonstration :} 
    On procède par récurrence. 
    Pour $n=0$, le résultat est évident puisqu'il n'existe aucun élément $m$ de $\mathbb{N}$ tel que $m < 0$.
    Soit $n$ un élément de $\mathbb{N}$ tel que, pour tout élément $m$ de $\mathbb{N}$ tel que $m < n$, $m \in n$. 
    Soit $m$ un élément de $\mathbb{N}$ tel que $m < n+1$. 
    Montrons d'abord que $n \notin m$. 
    Si on avait $n \in m$, alors on aurait $m > n$ d'après le lemme précédent, d'où $n \subset m$ et (puisque $n \in m$) $n+1 \subset m$, en contradiction avec $m < n+1$. 
    Donc, $n \notin m$. 
    Donc, puisque $m \subset n+1$, $m \subset n$. 
    (En effet, soit $x$ un élément de $m$, on a $x \in n+1$, donc $x \in n$ ou $x \in \lbrace n \rbrace$ ; la seconde option est impossible car $n \notin m$, donc $m \in n$.)
    Donc, $m \leq n$. 
    Si $m = n$, on a $m \in n+1$. 
    Sinon, $m < n$, donc $m \in n$, et donc $m \in n+1$. 
    Dans les deux cas, $m \in n+1$. 
    Ainsi, la propriété énoncée dans le lemme est vraie pour $n+1$. 
    Par récurrence, on en déduit qu'elle l'est pour tout élément $n$ de $\mathbb{N}$.

   \done 

\blank[medium]

\noindent\bold{Corolaire :} 
    Soit $n$ et $m$ deux éléments de $\mathbb{N}$. 
    D'après les deux lemmes précédents, les propositions $m \in n$ et $m < n$ sont équivalentes.

\blank[medium]

\noindent\bold{Lemme :} 
    Soit $n$ un élément de $\mathbb{N}$. 
    Pour tout élément $m$ de $\mathbb{N}$ tel que $m \notin n$, on a $n \leq m$.

\blank[medium]

\noindent\bold{Démonstration :} 
    On procède par récurrence sur $n$. 
    Pour $n=0$, le résultat est évident car $0 \leq m$ pour tout élément $m$ de $\mathbb{N}$.
    Soit $n$ un élément de $\mathbb{N}$ tel que, pour tout élément $m$ de $\mathbb{N}$ tel que $m \notin n$, $n \leq m$. 
    Soit $m$ un élément de $\mathbb{N}$ tel que $m \notin n+1$. 
    Alors, $m \notin n$ (donc $n \leq m$) et $m \neq n$, donc $n < m$. 
    D'après le lemme précédent, cela implique $n \in m$. 
    Puisque $n < m$, on a en outre $n \subset m$. 
    Donc, $n+1 \subset m$. 
    Donc, $n+1 \leq m$. 
    On en déduit que le résultat est vrai pour $n+1$.
    Par récurrence, il est vrai pour tout élément $n$ de $\mathbb{N}$. 

   \done 

\blank[medium]

\noindent\bold{Corolaire :} 
    Soit $n$ et $m$ deux élément de $\mathbb{N}$. 
    Les formules $m \notin n$ et $n \leq m$ sont équivalentes.

\blank[medium]

\noindent\bold{Démonstration :} 
    Soit $n$ et $m$ deux éléments de $\mathbb{N}$. 
    Si $n \leq m$, alors $n \subset m$. 
    Puisque $m \notin m$, cela implique $m \notin n$. 
    Donc, $(n \leq m) \Rightarrow (m \notin n)$.
    Le lemme précédent montre en outre que $(n \leq m) \Leftarrow (m \notin n)$.
    Donc, $(n \leq m) \Leftrightarrow (m \notin n)$.
    
   \done 

\blank[medium]

\noindent\bold{Corolaire :} Soit $n$ et $m$ deux éléments de $\mathbb{N}$ tels que $n \notin m$ et $m \notin n$. 
    Alors $m \leq n$ et $n \leq m$, et donc $n = m$.

\blank[medium]

\noindent\bold{Lemme :} La relation d'ordre $\leq$ sur $\mathbb{N}$ est une relation d'ordre total.

\blank[medium]

\noindent\bold{Démonstration :} 
    Soit $n$ et $m$ deux éléments de $\mathbb{N}$. 
    Alors, $m \in n$ ou $m \notin n$. 
    Dans le premier cas, $n > m$, donc $m < n$, et donc $m \leq n$.
    Dans le second cas, $n \leq m$.

   \done 

\blank[medium]

\noindent\bold{Corolaire :} Soit $n$ et $m$ deux éléments de $\mathbb{N}$. 
    Si $n \leq m$ est fausse, alors $m \leq n$ est vraie (d'après le lemme précédent) et $n \neq m$ est vraie (car $n \leq n$), donc $m < n$ est vraie. 
    Donc, $\neg (n \leq m) \Rightarrow (m < n)$. 
    Par ailleurs, si $m < n$, alors $n \leq m$ est fausse (sans quoi on aurait $m \subset n$ et $n \subset m$, et donc $m = n$).
    Ainsi, $\neg (n \leq m)$ est équivalente à $m < n$, et donc à $n > m$.
    De même, $\neg (n \geq m)$ est équivalente à $m > n$, et donc à $n < m$.

\blank[medium]

\noindent\bold{Corolaire :} Soit $n$ et $m$ deux éléments de $\mathbb{N}$. 
    Puisque $\leq$ est une relation d'ordre totale, on a $n \leq m$ ou $n \geq m$. 
    Donc, $n < m$ ou $n = m$ ou $n > m$. 

\blank[medium]

Notons que, si deux éléments $n$ et $m$ de $\mathbb{N}$ satisfont $n+1 = m+1$, on a soit $n = m$ soit $n \in m$ et $m \in n$.%
\footnote{
    En effet, puisque $n \in n+1$ et $m \in m+1$, la formule $n+1 = m+1$ implique $(n \in m+1) \wedge (m \in n+1)$, d'où $((n = m) \vee (n \in m)) \wedge ((m = n) \vee (m \in n))$. 
    En utilisant deux fois la distributivité de $\wedge$ sur $\vee$ ainsi que sa symétrie, cette formule se récris $((n = m) \wedge (m = n)) \vee ((n = m) \wedge (m \in n)) \vee ((n \in m) \wedge (m = n)) \vee ((n \in m) \wedge (m \in n))$. 
    Puisque $(n = m) \wedge (m \in n)$ et $(n \in m) \wedge (m = n)$ ne peuvet être vraies, et par symétrie de l'égalité, cette formule est équivalente à $(n = m) \vee (n \in m) \wedge (m \in n)$.
}
La seconde possibilité implique $m < n$ et $n < m$, qui ne peuvent être stisfaites simultanément (car cela impliquerait $m \leq n$ et $n \leq m$, d'où $n = m$, ce qui est incompatible avec $m < n$). 
On en déduit le lemme suivant :

\blank[medium]

\noindent\bold{Lemme :} 
    Soit $n$ et $m$ deux éléments de $\mathbb{N}$. 
    Si $n+1 = m+1$, alors $n=m$.

\blank[medium]

\noindent\bold{Lemme :} 
    Soit $n$ un élément de $\mathbb{N}$. 
    Pour tout élément $m$ de $\mathbb{N}$ tel que $m < n+1$, on a $m \leq n$. 
    La réciproque est évidente puisque $n < n+1$ : pour tout élément $m$ de $\mathbb{N}$, si $m \leq n$, $m < n+1$. 
    Donc, pour tout élément $m$ de $\mathbb{N}$, on a $m < n+1 \Leftrightarrow m \leq n$. 

\blank[medium]

\noindent\bold{Corolaire :} 
    En prenant la négation de la formule de chaque côté du connecteur $\Leftrightarrow$, on obtient, pour tout élément $m$ de $\mathbb{N}$ : $m \geq n+1 \Leftrightarrow m > n$.

\blank[medium]

\noindent\bold{Démonstration :} 
    Soit $m$ un élément de $\mathbb{N}$ tel que $m < n+1$. 
    Alors $m \subset n \cup \lbrace n \rbrace$. 
    Si $n \in m$, on a $m > n$, donc $n \subset m$, et donc $n+1 \subset m$ et donc $n+1 \leq m$, ce qui est impossible par hypothèse. 
    On en déduit que $n \notin m$, donc que $m \subset n$, et donc que $m \leq n$. 
    Ainsi, $\forall m \in \mathbb{N}\, m < n+1 \Rightarrow m \leq n$. 
    Cela montre la première partie du lemme, de laquelle le reste découle. 
    
   \done 

\blank[medium]

\noindent\bold{Lemme :} 
    Pour tout entier naturel $n$, on a : $n = \lbrace m \in \mathbb{N} \vert m < n \rbrace$. 

\blank[medium]

\noindent\bold{Démonstration :} 
    On procède par récurrence. 
    Tout d'abord, il n'existe aucun entier naturel $m$ tel que $m < 0$. 
    Donc, $\lbrace m \in \mathbb{N} \vert m < 0 \rbrace = \emptyset = 0$. 
    Soit $n$ un entier naturel tel que $n = \lbrace m \in \mathbb{N} \vert m < n \rbrace$. 
    Puisque $n+1 = n \cup \lbrace n \rbrace$, on a : $n + 1 = \lbrace m \in \mathbb{N} \vert m < n \vee m = n \rbrace$. 
    Cela peut se récrire : $n + 1 = \lbrace m \in \mathbb{N} \vert m \leq n \rbrace$. 
    D'après le lemme précédent, cela est équivalent à : $n + 1 = \lbrace m \in \mathbb{N} \vert m < n + 1 \rbrace$. 
    Par récurrence, le résultat attendu est donc vrai pour tout entier naturel.
    
   \done 

\blank[medium]

\noindent\bold{Lemme (récurrence en partant d'un rang non nul) :} 
    Soit $n$ un entier naturel et $P$ un prédicat à un paramètre. 
    On suppose que $P(n)$ est vrai et que, pour tout entier naturel $m$ tel que $m \geq n$, $P(m) \Rightarrow P(m+1)$. 
    Alors, $\forall m \in \mathbb{N}, \, m \geq n \Rightarrow P(n)$.

\blank[medium]

\noindent\bold{Démonstration :} On procède par récurrence. 
    Soit $Q$ le prédicat à un paramètre libre défini par $Q(m): m \geq n \Rightarrow P(n)$.
    Si $n = 0$, $P(0)$ est vrai, donc $Q(0)$ l'est aussi. 
    Si $n \neq 0$, $n > 0$, donc $0 \geq n$ est fausse et $Q(0)$ est vraie. 
    Dans  tous les cas, $Q(0)$ est vraie. 

    Soit $m$ un entier naturel tel que $Q(m)$ est vrai. 
    Alors, 
    \startitemize[nowhite]
        \item Si $m+1 < n$, $n \geq m+1$ est fausse, donc $Q(m+1)$ est vrai.
        \item Si $m+1 = n$, $P(m+1)$ est vrai, donc $Q(m+1)$ est vrai.
        \item Si $m+1 > n$, $m \geq n$, donc $P(m)$ est vrai (puisque $Q(m)$ l'est), donc $P(m+1)$ est vrai, donc $Q(m+1)$ est vrai.
    \stopitemize
    On a donc montré que, pour tout entier naturel $m$, $Q(m) \Rightarrow Q(m+1)$. 
    Par récurrence, $Q(m)$ est donc vrai pour tout entier naturel $m$.

   \done 

\blank[medium]

\noindent\bold{Définition :} Soit $a$ et $b$ deux entiers naturels. 
    On définit l'ensemble $[\![a, b]\!]$ par : 
    \startformula
        
        [\![a, b]\!] = \lbrace n \in \mathbb{N} \vert (n \geq a) \wedge (n \leq b) \rbrace.
    \stopformula
    Notons que $[\![a, b]\!] = \emptyset$ si $a > b$. 
    En effet, dans ce cas, tout élément $x$ de $\mathbb{N}$ satisfaisant $x \geq a$ satisfait $x > b$, et donc ne satistfait pas $x \geq b$.

\blank[medium]

\noindent\bold{Lemme :} Soit $n$ un entier naturel. 
    On a : $[\![0, n-1]\!] = n$.

\blank[medium]

\noindent\bold{Démonstration :} 
\startitemize[nowhite]
    \item Soit $x$ un élément de $[\![0, n-1]\!]$. 
        Puisque $[\![0, n-1]\!]$ est un sous-ensemble de $\mathbb{N}$, $x \in \mathbb{N}$. 
        En outre, $x \leq n-1$.
        Puisque $(n-1)+1 = n$, $n-1 < n$, et donc $x < n$. 
        Donc, $x \in n$.
    \item Soit $x$ un élément de $n$. 
        Puisque $n$ est un sous-ensemble de $\mathbb{N}$, $x \in \mathbb{N}$. 
        Donc, $x \geq 0$.
        En outre, $x < n$.
        Donc, $x \leq n-1$. 
        Donc, $x \in [\![1,n-1]\!]$.
\stopitemize

\done

\stopsubsection

\startsubsection[title=Récurrence forte]

\noindent\bold{Lemme (principe de récurrence forte) :} Soit $P$ une formule à un paramètre libre. 
    On suppose que $P(0)$ est vraie et que, pour tout élément $n$ de $\mathbb{N}$, la formule $(\forall m \in \mathbb{N} \, m \leq n \Rightarrow P(m)) \Rightarrow P(n+1)$ est vrai. 
    Alors, pour tout élément $n$ de $\mathbb{N}$, $P(n)$ est vraie. 
    \index{Récurrence forte}

\blank[medium]

\noindent\bold{Démonstration :} Considérons la formule à un paramètre libre $Q$ définie par $Q(n): \forall m \in \mathbb{N} \, m \leq n \Rightarrow P(m)$. 
    Notons que, d'après la seconde hypothèse faite sur $P$, pour tout élément $n$ de $\mathbb{N}$ $Q(n) \Rightarrow P(n+1)$.
    Montrons que $Q(n)$ est vraie pour tout élément $n$ de $\mathbb{N}$. 
    Tout d'abord, $Q(0)$ est équivalente à $P(0)$ (car le seul élément $m$ de $\mathbb{N}$ tel que $m \leq 0$ est $0$). 
    Donc, $Q(0)$ est vraie. 
    Soit $n \in \mathbb{N}$ tel que $Q(n)$ est vraie. 
    Soit $m \in \mathbb{N}$ tel que $m \leq n+1$. 
    Alors, $m \leq n$ ou $m = n+1$. 
    Si $m \leq n$, $P(m)$ est vraie car $Q(n)$ est vraie.
    Si, $m = n+1$, $P(m)$ est vraie puisque $Q(n)$ est vraie et $Q(n) \Rightarrow P(n+1)$. 
    Donc, $Q(n+1)$ est vraie.
    Par récurrence, on en déduit que $Q(n)$ est vraie pour tout élément $n$ de $\mathbb{N}$.

    Montrons que cela implique le lemme. 
    Soit $n$ un élément de $\mathbb{N}$. 
    On a vu que $Q(n)$ est vraie. 
    Donc, pour tout élément $m$ de $\mathbb{N}$ tel que $m \leq n$, $P(m)$ est vraie. 
    Puisque $n \leq n$ par reflexivité de la relation d'ordre, $P(n)$ est vraie.

   \done 

\stopsubsection


\startsubsection[title={Suites ; définition par récurrence}, reference=sub:suites]

\noindent\bold{Définition :} Soit $E$ un ensemble non vide. 
    Une \emph{\words{suite}} $u$ d'éléments de $E$ est une fonction de $\mathbb{N}$ vers $E$. 
    Si $u$ est une suite d'éléments de $E$ et $n$ un élément de $\mathbb{N}$, l'élément $u(n)$ de $E$ est parfois noté $u_n$. 
    Si $f$ est une formule dépendant d'un paramètre libre telle que, pour tout élément $n$ de $\mathbb{N}$, $f(n) = u(n)$, la suite $u$ est parfois notée $\left( f(n) \right)_{n \in \mathbb{N}}$.
    Si $u$ est une suite et $n$ une variable, dans la formule $u_n$, $n$ est parfois appelé \emph{\words{indice}}.

\blank[medium]

\noindent\bold{Lemme (définition par récurrence) :} Soit $E$ un ensemble non vide et $f$ une fonction de $\mathbb{N} \times E$ vers $E$.
    Soit $e_0$ un élément de $E$. 
    Il existe une unique fonction $u$ de $\mathbb{N}$ vers $E$ telle que $u(0) = e_0$ et, pour tout $n \in \mathbb{N}$, $u(n+1) = f(n, u(n))$.

\blank[medium]

\noindent Ce lemme permet notamment de \emph{définir} une suite par récurrence, étant donnés son image de $0$ et une fonction $f$ donnant son image de $n+1$ connaissant celle de $n$.

\blank[medium]

\noindent\bold{Démonstration :}
\emph{Unicité :} Soit $u$ et $v$ deux fonctions satisfaisant les propriétés de l'énoncé. 
       Tout d'abord, on a $u(0) = e_0$ et $v(0) = e_0$ par hypothèse, et donc $u(0) = v(0)$.
       Soit $n$ un élément de $\mathbb{N}$ et supposons $u(n) = v(n)$. Alors, $u(n+1) = f(n, u(n))$ donne $u(n+1) = f(n, v(n))$, d'où $u(n+1) = v(n+1)$. 
       Par récurrence, on a donc $u(n) = v(n)$ pour tout élément $n$ de $\mathbb{N}$.

\emph{Éxistence :} Une fonction $v$ d'un sous-ensemble non vide de $\mathbb{N}$ dans $E$ est dite \emph{$f$-inductive} si elle satisfait les trois propriétés suivantes : 
    \startitemize[nowhite]
        \item son domaine $D$ satisfait $\forall x \in D, \, \forall n \in \mathbb{N}, \, n \leq x \Rightarrow n \in D$, 
        \item si $0$ est dans son domaine, alors $v(0) = e_0$, 
        \item si $n$ est un élément de $\mathbb{N}$ tel que $n$ et $n+1$ sont tous deux dans son domaine, alors $v(n+1) = f(n, v(n))$. 
    \stopitemize
   Chacune de ces fonctions est un sous-ensemble de $\mathbb{N} \times E$. 
   
   Soit $v$ une fonction $f$-injective. 
   Puisque son domaine est non nul, on peux choisir un élément $x$ de $D$. 
   Puisque $D$ est un sous-ensemble de $\mathbb{N}$, on a $x \in \mathbb{N}$. 
   Donc, $0  \leq x$, et donc $0 \in D$.
   Cela montre que $0$ appartient au domaine de définition de toute fonction $f$-inductive. 

   Soit $u$ l'union de toutes les fonctions $f$-inductives. 
   (Cet ensemble existe d'après l'axiome de compréhension obtenu avec l'ensemble des parties de $\mathbb{N} \times E$ et la conjonction des trois propriétés définissant une fonction $f$-inductive.)
   Montrons que $u$ est une fonction de $\mathbb{N}$ dans $E$ satisfaisant les propriétés de l'énoncé. 

   Tout d'abord, $\lbrace (0,e_0) \rbrace$ (vu comme une fonction de $\lbrace 0 \rbrace$ vers $E$) est $f$-inductive, donc $(0,e_0) \in u$, et donc $0$ appartient au domaine de $u$.
   Soit $n \in \mathbb{N}$ tel que $n$ appartient au domaine de $u$. 
   Soit $v$ une fonction $f$-inductive dont le domaine contient $n$ et $v' = v \cup \lbrace n+1, f(n, f(v(n))) \rbrace$. 
   On vérifie facilement que $v'$ est une fonction $f$-inductive avec pour domaine $D \cup \lbrace n+1 \rbrace$, où $D$ est le domaine de $v$. 
   (Il s'agit bien d'une fonction car $v$ en est une et, si $n+1$ est aussi dans le domaine de $v$, on a $v(n+1) = f(n, f(v(n)))$ ; elle satisfait la première propriété car un entier $m$ satisfaisant $m \leq n+1$ est égal à $n+1$ s'il contient $n$ ou satisfait $m \leq n$ (et est donc dans le domaine de $v$) sinon, la seconde car l'image de $0$ est égale à $v(0)$, donc à $e_0$, la troisième pour tout entier $m$ satisfaisant $m \neq n$ car $v$ est $f$-inductive (si $m$ et $m+1$ sont dans son domaines, alors ils sont aussi dans celui de $v$, d'où le résultat), et la troisième pour l'entier $n$ car $v'(n+1) = f(n,v(n))$ et $v(n) = v'(n)$.)
   Donc, $n+1$ appartient au domaine de $u$. 
   Cela montre (par récurrence) que le domaine de $u$ est $\mathbb{N}$. 

   Soit $n \in \mathbb{N}$ et $v$ et $v'$ deux fonctions $f$-inductives dont les domaines contiennent $n$. 
   On montre facilement par récurrence finie que $v(n) = v'(n)$. 
   (Cela est vrai pour $n=0$ car $v(0)$ et $v'(0)$ sont tous deux égaux à $e_0$ et, si un entier $m$ est tel que $m+1$ appartienne à leurs domaine de définition, alors $m$ y appartient également (puisque $m < m+1$) ; si de plus $v'(m) = v(m)$, alors $v'(m+1) = f(m,v'(m)) = f(m,v(m)) = v(m+1)$, donc $v'(m+1) = v(m+1)$.)
   Donc, $u$ est bien une fonction.

   Par ailleurs, on a $u(0) = e_0$. 
   Soit $n \in \mathbb{N}$, $n+1$ appartient à $\mathbb{N}$, donc au domaine de $u$, donc on peut choisir une fonction $v$ $f$-inductive telle que $n+1$ appartienne au domaine de $v$. 
   Puisque $n < n+1$, $n$ est aussi dans le domaine de $v$. 
   On a donc $v(n+1) = f(n, v(n)) = f(n,u(n))$, et donc $u(n+1) = f(n, u(n))$.

  \done 

\blank[medium] 

Ce résultat étant particulièrement important pour la suite, nous en donnons ci-dessous une démonstration formulée un brin différemment, et un peu plus détaillée. 
On reprend les notations du lemme. 

Montrons tout d'abord que, si une fonction de $\mathbb{N}$ dans $E$ satisfaisant les deux propriétés de l'énoncé existe, alors elle est unique. 
On suppose avoir deux telles fonctions, notées $u$ et $v$. 
Montrons qu'elles sont nécéssairement égales. 
Pour ce faire, il suffit de montrer que, pour tout élément $n$ de $\mathbb{N}$, $u(n) = v(n)$. 
On procède par récurrence. 
D'après la première propriété de l'énoncé, on a $u(0) = e_0$ et $v(0) = e_0$. 
Donc, $u(0) = v(0)$. 
Considérons maintenant un élément $n$ de $\mathbb{N}$ tel que $u(n) = v(n)$. 
On a $f(n, u(n)) = f(n, v(n))$. 
Or, on a aussi, d'après la deuxième propriété de l'énoncé : $f(n, u(n)) = u(n+1)$ et $f(n, v(n)) = v(n+1)$. 
Donc, $u(n+1) = v(n+1)$. 
Cela étant vrai pour tout élément $n$ de $\mathbb{N}$ tel que $u(n) = v(n)$, et puisque $u(0) = v(0)$, on en déduit par récurrence que, pour tout élément $n$ de $\mathbb{N}$, $u(n) = v(n)$, et donc que $u = v$. 
Ainsi, il existe au plus une fonction satisfaisant les conditions de l'énoncé.

Montrons maintenant qu'une telle fonction existe bien. 
Pour ce faire, définissons d'abord la notion de fonction $f$-injective%
~\footnote{
    Cette définition est un peu bancale puisqu'elle dépend de $f$ mais aussi de $e_0$. 
    Une appellation plus approppriée serait « $f$-injective avec élément initial $e_0$ ». 
    Pour simplifier, et puisque cette notion n'est utilisée que dans cette preuve où $e_0$ est fixé, nous la raccourcissons en « $f$-injective », l'élément initial étant implicite.
}
de la manière suivante. 
Une fonction $f$-injective est une fonction, notée $v$ dans la suite de cette définition, d'un sous-ensemble non vide $D$ de $\mathbb{N}$ vers $E$ telle que les conditions suivantes sont satisfaites :
\startitemize[nowhite]
    \item Pour tout élément $x$ de $D$, pour tout élément $n$ de $\mathbb{N}$ tel que $n \leq x$, $n \in D$. 
        (C'est-à-dire : $\forall x \in D, \, \forall n \in \mathbb{N}, \, n \leq x \Rightarrow x \in D$ ; dans la suite, on note $P_1$ le prédicat obtenu en remplaçant $D$ par la formule $\lbrace x \in \mathbb{N} \vert \exists y \in \mathbb{N} \, (x,y) \in v \rbrace$.)
    \item Si $0 \in D$, $v(0) = e_0$.
        (C'est-à-dire : $0 \in D \Rightarrow v(0) = e_0$ ; dans la suite, on note $P_2$ le prédicat obtenu en remplaçant $D$ par la formule $\lbrace x \in \mathbb{N} \vert \exists y \in \mathbb{N} \, (x,y) \in v \rbrace$.)
.)
    \item Si $n$ est un élément de $D$ tel que $n+1 \in D$, alors $v(n+1) = f(n, v(n))$. 
        (C'est-à-dire : $\forall n \in D, \, n+1 \in D \Rightarrow v(n+1) = f(n, v(n))$ ; dans la suite, on note $P_3$ le prédicat obtenu en remplaçant $D$ par la formule $\lbrace x \in \mathbb{N} \vert \exists y \in \mathbb{N} \, (x,y) \in v \rbrace$.)
)
\stopitemize
Notons que la première condition impose $0 \in D$. 
En effet, $D$ doit être non vide et, soit $x$ un élément de $D$ (un tel élément existe donc), on a $x \in \mathbb{N}$, donc $0 \leq x$, et donc $0 \in D$. 
La seconde condition peut donc être simplifiée en $v(0) = e_0$. 

Toute fonctions $f$-injective est un sous-ensemble de $\mathbb{N} \times E$. 
En effet, si $v$ est une telle fonction et $z$ un élément de $v$, on peut choisir un élément $x$ du domaine $D$ de $v$ et un élément $y$ de $E$ tels que $z = (x,y)$.
Puisque $D$ est un sous-ensemble de $\mathbb{N}$, on a $x \in \mathbb{N}$, et donc $z \in \mathbb{N} \times E$.

En appliquant l'axiome de compréhension avec l'ensemble des parties de $\mathbb{N} \times E$ et la propriété $P: P_1 \wedge P_2 \wedge P_3$, on montre que l'ensemble des fonctions $f$-inductives existe. 
Notons qu'il existe au moins une fonction $f$-injective : $\lbrace (0, e_0) \rbrace$. 
Il s'agit d'une fonction de $\lbrace 0 \rbrace$ vers $E$ (en effet, son seul élément est dans $\lbrace 0 \rbrace \times E$, l'unique élément de $\lbrace 0 \rbrace$ a une image $e_0$, et, si $x$ est un élément de $\lbrace 0 \rbrace$, et $y$ et $y'$ deux images de $x$, alors $y = e_0$ et $y' = e_0$, donc $y = y'$) ; son domaine est $\lbrace 0 \rbrace$, qui est bien un sous-ensemble de $\mathbb{N}$ ; le seul élément $n$ de $\mathbb{N}$ satisfaisant $n \leq 0$ est $0$ lui-même, qui est bien dans $D$ ; on a $v(0) = e_0$ ; il n'existe aucun élément $n$ de $D$ tel que $n+1 \in D$ puisque $0+1 \neq 0$. 
Notons $u$ l'union de tous les éléments de l'ensemble des fonctions $f$-injectives. 
(L'ensemble $u$ existe d'après l'axiome de la réunion.) 
Nous nous proposons de montrer que $u$ est une fonction de $\mathbb{N}$ vers $E$ puis qu'elle satisfait les deux propriétés du lemme. 

En tant qu'union de sous-ensembles de $\mathbb{N} \times E$, $u$ en est un également.%
\footnote{
    En effet, soit $z \in u$, il existe un élément $v$ de l'ensemble des fonctions $f$-injectives tel que $z \in v$. 
    Soit $D$ son domaine. 
    On a $z \in D \times E$
    Puisque $D$ est un sous-ensemble de $\mathbb{N}$, $D \times E$ est un sous-ensemble de $\mathbb{N} \times E$, donc $z \in \mathbb{N} \times E$.)
}
Pour montrer que $u$ est une fonction de $\mathbb{N}$ vers $E$, il suffit donc de montrer que, pour tout élément $n$ de $\mathbb{N}$, il existe un unique élément $e$ de $E$ tel que $(n,e) \in u$. 
On procède par récurrence. 
Pour $n = 0$, le résultat est facile à démontrer : $\lbrace (0, e_0) \rbrace$ est une fonction $f$-inductive, donc $(0, e_0) \in u$. 
En outre, soit $e$ un élément de $e$ tel que $(0,e) \in E$, il existe une fonction $f$-inductive $v$ telle que $(0,e) \in v$. 
La première propriété du lemme donne alors $e = e_0$. 
Ainsi, il existe un unique élément $e$ de $E$ ($e_0$) tel que $(0, e_0) \in u$. 

Soit $n$ un élément de $\mathbb{N}$ et supposons qu'il existe un unique élément de $E$, noté $e$ dans la suite de ce paragraphe tel que $(n,e) \in u$. 
Soit $e_1$ et $e_2$ deux éléments de $E$ tels que $(n+1, e_1) \in u$ et $(n+1, e_2) \in u$. 
On peut trouver deux fonctions $f$-injectives $v_1$ et $v_2$ dont les domaines contiennent $n+1$ et telles que $v_1(n+1) = e_1$ et $v_2(n+1) = e_1$. 
Puisque $n \leq n+1$, $n$ appartient aussi à leurs domaines de définition. 
Puisque $(n, v_1(n)) \in u$ et $(n, v_2(n)) \in u$, on a $v_1(n) = e$ et $v_2(n) = e$. 
Donc, d'après le troisième critère de définition d'une fonction $f$-infductive, $v_1(n+1) = f(n,e)$ et $v_2(n+1) = f(n,e)$.
Donc, $e_1 = f(n,e)$ et $e_2 = f(n,e)$. 
Donc, $e_1 = e_2$. 
Il existe donc au plus un élément $e'$ de $E$ tel que $(n+1, e') \in u$. 
Montrons qu'il existe bien. 
Soit $v$ une fonction $f$-inductive dont le domaine de définition contient $n$. 
Montrons que $v \cup \lbrace (n+1, f(n, v(n))) \rbrace$ est une fonction $f$-inductive. 
Cela montrera que $(n+1, f(n,v(n))) \in u$. 
Par récurrence, nous aurons alors montré que $u$ est bien une fonction, et que l'image par $u$ d'un élément $n$ de $\mathbb{N}$ est $v(n)$, où $v$ est une fonction $f$-injective (quelconque) dont le domaine contient $n$. 

Notons $v'$ l'ensemble $v \cup \lbrace (n+1, f(n, v(n))) \rbrace$ et $D$ le domaine de $v$. 
Montrons que $v'$ est une fonction de $D \cup \lbrace n+1 \rbrace$ dans $E$. 
Soit $m \in D$. 
Puisque $v$ est une fonction de $D$ vers $E$, on peut choisir un unique élément $e$ de $E$ tel que $(m,e) \in v$. 
On a alors $(m,e) \in v'$.
Si $m \neq n+1$, il n'existe pas d'autre élément de $v'$ dont la première composante soit $n$ (car le seul élément de $v'$ qui ne soit pas dans $v$ a $n+1$ pour première composante ; un élément de $v'$ dont la première composante est $m$ doit donc être un élément de $v$, et sa deuxième composante ne peut alors être que $e$ puisque $v$ est une fonction). 
Si $m = n+1$, on a $v(n+1) = f(n, v(n))$ car $v$ est $f$-injective. 
Donc, $(n+1, f(n,v(n))) \in v$ et $v' = v$, donc $v'$ et une fonction et n'a pas plus d'un élément avec $n+1$ comme première composante.
Par ailleurs, si $n+1$ n'est pas un élément de $D$, alors le seul élément de $v'$ dont la première composante est $n+1$ est $(n+1,f(n,v(n)))$ (tout autre élément de $v'$ appartient à $v$, et a donc sa première composante dans $D$).
Ainsi, dans les deux cas (que $n+1$ soit ou non un élément de $D$) $v'$ est une fonction. 

Montrons qu'elle est $f$-injective. 
Son domaine de définition est celui de $v$, auquel on ajoute éventuellement $n+1$. 
Pour tout élément $m$ de ce domaine distinct de $n+1$, $m$ est dans le domaine de $v$, donc pour tout élément $k$ de $\mathbb{N}$ tel que $k \leq m$, $k$ est dans le domaine de $v$ et donc dans celui de $v'$. 
Soit $m$ un élément de $\mathbb{N}$ tel que $m \leq n+1$. 
On a $m < n + 1$ ou $m = n + 1$.
Dans le premier cas, on a $m \leq n$. 
Puisque $n$ est dans le domaine de $v$ et car $v$ est $f$-injective, $m$ y est également, et est donc dans celui de $v'$.
Dans le second cas $m$ est bien dans le domaine de $v'$ puisque $(n+1, f(n,v(n))) \in v'$. 
Ainsi, la fonction $v'$ satisfait $P_1$.

On a $v'(0) = v(0)$, donc, puisque $v$ est $f$-injective, $v'(0) = e_0$. 
La fonction $v'$ satisfait donc $P_2$. 

Enfin, soit $m$ un élément du domaine de $v'$, 
\startitemize[nowhite]
    \item Si $m \neq n$ et, et si $m+1$ est dans le domaine de $v'$, alors $m+1$ est dans le domaine de $v$ (en effet, si $m \neq n$, $m+1 \neq n+1$). 
        Donc, puisque $m \leq m+1$, $m$ est dans le domaine de $v$. 
        Puisque $v$ est $f$-injective, $v(m+1) = f(m, v(m))$. 
        Puisque $v'(m) = v(m)$ et $v'(m+1) = v(m+1)$, on en déduit $v'(m+1) = f(m, v'(m))$.
    \item Si $m = n$, on a $v'(m+1) = f(n, v(n))$. Puisque $v'(n) = v(n)$, on en déduit $v'(m+1) = f(m,v'(m))$.
\stopitemize
Ainsi, $v'$ satisfait $P_3$. 
Cette fonction est donc bien $f$-injective.

Il ne reste plus qu'à montrer que $u$ satisfait les deux propriétés de l'énoncé. 
Nous avons vu plus haut que $u(0) = e_0$. 
Soit $n$ un élément de $\mathbb{N}$. 
Alors, $n+1$ appartient à $\mathbb{N}$ et donc au domaine de $u$, donc il on peut choisir une fonction $f$-injective $v$ dont le domaine contient $n+1$. 
Puisque $n \leq n+1$, $n$ appartient aussi au domaine de $v$. 
On a donc $v(n+1) = f(n, v(n))$. 
Puisque $u(n) = v(n)$ et $u(n+1) = v(n+1)$, on en déduit $u(n+1) = f(n, u(n))$.

\done

\blank[medium]

\noindent\bold{Définition :}
    Soit $E$ un ensemble et $\leq$ une relation d'ordre sur $E$.
    Soit $u$ une suite d'éléments de $E$. 
    On dit que $u$ est \emph{majorée} s'il existe un élément $e$ de $E$ tel que : $\forall n \in \mathbb{N} \, u_n \leq e$.
    On dit que $u$ est \emph{minorée} s'il existe un élément $e$ de $E$ tel que : $\forall n \in \mathbb{N} \, e \leq u_n$.
    Un élément $e$ de $E$ satisfaisant $\forall n \in \mathbb{N} \, u_n \leq e$ est dit \emph{majorant} de $u$.
    Un élément $e$ de $E$ satisfaisant $\forall n \in \mathbb{N} \, e \leq u_n$ est dit \emph{minorant} de $u$.
    Une suite à la fois majorée et minorée est dite \emph{bornée} ; son majorant et son minorant sont aussi appelés \emph{bornes}. 
    \index{Suite majorée}
    \index{Suite minorée}
    \index{Majorant}
    \index{Minorant}
    \index{Suite bornée}
    \index{Bornes}

\blank[medium]

\noindent\bold{Définition :}
    Soit $E$ un ensemble et $\leq$ une relation d'ordre sur $E$.
    Soit $u$ une suite d'éléments de $E$. 
    On dit que $u$ est \emph{croissante} si, pour tous entiers naturels $n$ et $m$, $n \leq m \Rightarrow u_n \leq u_m$.
    On dit que $u$ est \emph{décroissante} si, pour tous entiers naturels $n$ et $m$, $n \leq m \Rightarrow u_m \leq u_n$.
    \index{Suite croissante}
    \index{Suite décroissante}

\blank[medium]

\noindent\bold{Lemme :}
    Soit $E$ un ensemble et $\leq$ une relation d'ordre sur $E$.
    Soit $u$ une suite d'éléments de $E$. 
    On suppose que $u_{n+1} \geq u_n$ pour tout entier naturel $n$. 
    Alors, $u$ est croissante. 

\blank[medium]

\noindent\bold{Démonstration :} Soit $n$ un entier naturel. 
    On veut montrer que, pour tout entier naturel $m$ supérieur ou égal à $n$, $u_n \leq u_m$.
    On procède par récurrence sur $m - n$, noté $k$.
    Si $k = 0$, alors $m = n$, donc $u_n = u_m$ et $u_n \leq u_m$.

    Supposons le résultat attendu pour un entier naturel $k$, \emph{i.e.}, $u_n \leq u_{n+k}$. 
    Puisque $u$ est croissante, $u_{n+k} \leq u_{(n+k)+1}$.
    Donc, $u_n \leq u_{(n+k)+1}$.
    Donc, $u_n \leq u_{n+(k+1)}$.
    Le résultat attendu est vrai au rang $k+1$.
    Par récurrence, il l'est pour tout entier naturel $k$.

    \done

\blank[medium]

\noindent\bold{Lemme :}
    Soit $E$ un ensemble et $\leq$ une relation d'ordre sur $E$.
    Soit $u$ une suite d'éléments de $E$. 
    On suppose que $u_n \geq u_{n+1}$ pour tout entier naturel $n$. 
    Alors, $u$ est décroissante. 

\blank[medium]

\noindent\bold{Démonstration :} Soit $n$ un entier naturel. 
    On veut montrer que, pour tout entier naturel $m$ supérieur ou égal à $n$, $u_m \leq u_n$.
    On procède par récurrence sur $m - n$, noté $k$.
    Si $k = 0$, alors $m = n$, donc $u_m = u_n$ et $u_m \leq u_n$.

    Supposons le résultat attendu pour un entier naturel $k$, \emph{i.e.}, $u_{n+k} \leq u_n$. 
    Puisque $u$ est décroissante, $u_{(n+k)+1} \leq u_{n+k}$.
    Donc, $u_{(n+k)+1} \leq u_n$.
    Donc, $u_{n+(k+1)} \leq u_n$.
    Le résultat attendu est vrai au rang $k+1$.
    Par récurrence, il l'est pour tout entier naturel $k$.

    \done

\stopsubsection

\stopsection
