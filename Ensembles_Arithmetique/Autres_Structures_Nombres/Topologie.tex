\startsection[title=Éléments de Topologie Générale]

\noindent\bold{Définition :} Soit $E$ un ensemble.
    Un ensemble $T$ est une \emph{\words{topologie} sur $E$} si les conditions suivantes sont satisfaites ; le couple $(E, T)$ est alors dit \emph{\words{espace topologique}} et les éléments de $T$ sont ses \emph{ouverts}\index{ouvert} :
    \startitemize[nowhite]
        \item $T$ est un ensemble de parties de $E$ : $\forall x \in T \, x \subset E$,
        \item l'ensemble vide appartient à $T$ : $\emptyset \in T$,
        \item $E$ appartient à $T$ : $E \in T$,
        \item toute réunion d'ouverts est un ouvert : $\forall I \, I \subset T \Rightarrow \cup I \in T$,
        \item toute intersection finie d'ouverts est un ouvert : pour tout entier naturel strictement positif $n$, $\forall O_1 \in T \, \forall O_2 \in T \cdots \forall O_n \in T \, O_1 \cap O_2 \cap \cdots \cap O_n \in T$.
    \stopitemize

\blank[medium]

\noindent\bold{Définition :} Soit $(E, T)$ un espace topologique.
    Un sous-ensemble $F$ de $E$ est dit \emph{fermé} si $E \setminus F$ et un ouvert, \emph{i.e.}, si $E \setminus F \in T$.

\blank[medium]

\noindent\bold{Définition :} Soit $(E, T)$ et $(F, U)$ deux espaces topologiques et $f$ une fonction de $E$ vers $F$.
    On dit que $f$ est \emph{continue} (pour les topologies $T$ et $U$) si, pour tout élément $F'$ de $U$, l'image inverse de $F'$ par $f$ (\emph{i.e.} l'ensemble des éléments $e$ de $E$ tels que $f(e) \in F'$) est un élément de $T$ ; autrement dit, une fonction est continue si l'image inverse de tout ouvert est un ouvert.

\stopsection
