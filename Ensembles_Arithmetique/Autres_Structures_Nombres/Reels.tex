\startsection[title=Nombres réels]

\startsubsection[title=Suites de Cauchy, reference=sub:Cauchy]

\noindent\bold{Définition :} Soit $u$ une suite de nombres rationnels. 
    Elle est dite \emph{de Cauchy}\index{Suite de Cauchy} si la propriété suivante est satisfaite : 
    \startformula
        \forall \epsilon \in \mathbb{Q} \, 
        \exists n \in \mathbb{N} \,
        \forall m \in \mathbb{N} \,
        \forall k \in \mathbb{N} \,
        (m \geq n) \wedge (k \geq n)
        \Rightarrow \abs{u_n - u_k} \leq \epsilon
        .
    \stopformula

\blank[medium]

Intuitivement, une suite est de Cauchy si la distance maximale entre un de ses éléments et les suivants tend vers $0$. 
On pourrait penser qu'une telle suite doit être convergente. 
Ce n'est en fait pas le cas sur $\mathbb{Q}$, mais c'est là l'intuition menant aux nombres réels, qui peuvent être formellement définis comme limites des suites de Cauchy de rationnels. 
(On montrera aussi que toute suite de Cauchy de nombres réels converge.)

\blank[medium]

\noindent\bold{Lemme :} Toute suite de nombres rationnels majorée et croissante est de Cauchy. 

\blank[medium]

\noindent\bold{Démonstration :} Soit $u$ une suite de nombres rationnels majorée et croissante. 
    Soit $M$ un majorant de $u$.
    On suppose par l'absurde que la suite $u$ n'est pas de Cauchy. 
    On peut donc choisir un nombre rationnel $\epsilon$ strictement positif tel que, pour tout entier naturel $n$, il existe deux entiers naturels $m$ et $k$ supérieurs ou égaux à $n$ tels que $\abs{u_m - u_k} > \epsilon$. 
    Montrons par récurrence sur $l$ que, pour tout entier naturel $l$, on peut choisir un entier naturel $n$ tel que $u_n \geq u_0 + l \times \epsilon$. 
    Cela montrera que $u$ tend vers $+\infty$. 
    En effet, pour tout nombre rationnel $q$, on peut choisir un entier naturel $l$ tel que $u_0 + l \times \epsilon \geq q$~\footnote{
        On peut montrer cela de la manière suivante. 
        Soit $v$, $w$ et $x$ les trois suites définies par : pour tout entier naturel $l$, $v_l = l$, $w_l = v_l \times \epsilon$, et $x_l = u_0 + w_l$.
        Alors, $v$ tend vers $+\infty$ (voir ci-dessus), $w = \epsilon \times v$, et $x = u_0 + w$.
        Puisque $\epsilon > 0$, on en déduit que $w$ tend vers $+\infty$, et donc que $x$ tend vers $+\infty$. 
        On peut donc choisir un entier naturel $l$ tel que $x_k \geq q$ pour tout entier naturel $k$ supérieur ou égal à $l$, et donc, en particulier, $x_l \geq q$, et donc $u_0 + l \times \epsilon \geq q$.
    }, donc $u_n \geq q$ avec le $n$ définit ci-dessus, et donc (puisque $u$ est croissante) $u_m \geq q$ pour tout ntier naturel $m$ supérieur ou égal à $n$.
    Ainsi, $u$ tendra vers $+\infty$, ce qui est impossible puisque $u$ est majorée.
    On en déduira que $u$ est de Cauchy.

    Pour $l = 0$, le résultat attendu est évident puisque $u_0 = u_0$, donc $u_0 \geq u_0 + 0 \times \epsilon$, donc $u_n \geq u_0 + l \times \epsilon$ pour $n = 0$.

    Soit $l$ un entuer naturel. 
    On suppose avoir un netier naturel $n$ tel que $u_n \geq u_0 + l \times \epsilon$.
    Par hypothèse, on peut choisir deux entiers naturels $a$ et $b$ supérieurs ou égaux à $n$ tels que $\abs{u_a - u_b} > \epsilon$.
    Sans perte de généralité, on suppose $a \leq b$ (si ce n'est pas le cas, on s'y ramène en échangeant les rôles de $a$ et $b$, ce qui ne change pas le phrase précédente puisque $\abs{u_b - u_a} = \abs{u_a - u_b}$).
    Puisque $u$ est croissante, $u_a \leq u_b$, donc $\abs{u_a - u_b} = u_b - u_a$.
    Donc, $u_b - u_a \geq \epsilon$, et donc $u_b \geq u_a + \epsilon$.
    Puisque $a \geq n$ et puisque $u$ est croissante, $u_a \geq u_n$, donc $u_b \geq u_n + \epsilon$, donc $u_b \geq (l+1) \times \epsilon$.
    Le résultat attendu est donc vrai au rang $l+1$. 

    Par récurrence, il est vrai pour tout entier naturel $l$; ce qui conclut la preuve.

    \done

\blank[medium]

\noindent\bold{Lemme :} Toute suite de nombres rationnels convergente est de Cauchy. 

\blank[medium]

\noindent\bold{Démonstration :} Soit $u$ uns suite de nombres rationnels.
    On suppose que $u$ est convergente. 
    Soit $l$ sa limite. 
    Soit $\epsilon$ un nombre rationnel strictement positif. 
    Alors, $\epsilon \divslash 2$ est strictement positif.
    Puisque $u$ tend vers $l$, on peut choisir un entier naturel $n$ tel que, pour tout entier naturel $m$ supérieur ou égal à $n$, $\abs{u_m - l} \leq \epsilon \divslash 2$. 
    Soit $m$ et $p$ deux entiers naturels supérieurs ou égaux à $n$. 
    On a : $\abs{u_n - u_m} = \abs{(u_n - l) + (l - u_m)} \leq \abs{u_n - l} + \abs{l - u_m} = \epsilon \divslash 2 + \epsilon \divslash 2 = \epsilon$. 
    La suite $u$ satisfait donc la définition d'une suite de Cauchy. 

    \done

\blank[medium]

\startNoteBox
\noindent
\bold{Remarque :} La réciproque est fausse.
    Pour montrer cela, considérons la suite $u$ définie par récurrence de la manière suivante : 
    \startitemize[nowhite]
        \item $u_0 = 1$
        \item pour tout entier naturel $n$, $u_{n+1} = u_n + \left(2 - u_n^2 \right) \divslash 4$ .
    \stopitemize
   
    Supposons que la suite $u$ admette une limite $l$. 
    Définissons les suites $v$ et $w$ par : pour tout entier naturel $n$, $v_n = u_{n+1}$ et $w_n = u_n + \left(2 - u_n^2 \right) \divslash 2$.
    Alors, $v$ tend vers $l$ et $w$ tend vers $l + (2 - l \times l) \divslash 4$. 
    Puisque $v_n = u_n$ pour tout entier naturel $n$, on a donc $l = l + (2 - l \times l) \divslash 4$, donc $(2 - l \times l) \divslash 4 = 0$, donc $2 - l \times l = 0$, et donc $l \times l = 2$.

    Montrons que cela est impossible. 
    On procède par l'absurde. 
    Supposons qu'il existe un nombre rationnel $q$ tel que $q^2 = 2$. 
    Sans perte de généralité, on peut supposer $q \geq 0$ (si ce n'est pas le cas, il suffit de remplacer $q$ par $-q$, qui est alors positif et de carré égal à $2$.).
    Soit $a$ et $b$ deux entiers tels que $(a, b)$ est la représentation irréductible de $q$. 
    Puisque $q$ est positif, $a$ et $b$ sont tous deux positifs.
    En outre, $q \neq 0$ (sas quoi on aurait $q^2 = 0$, et donc $q^2 \neq 2$). 
    Donc, $q \neq 0$. 
    Donc, $a$ et $b$ sont premiers entre eux.
    
    Par ailleurs, on a : $b^2 \times q^2 = a^2$, donc $2 \, b^2 = a^2$. 
    Cela montre que $2$ divise $a^2$. 
    Puisque $2$ est premier, on en conclut que $2$ divise $a$. 
    On peut donc choisir un entier naturel $c$ tel que $a = 2 \, c$.
    Donc, $a^2 = 4 \, c^2$, et donc $b^2 = 2 \, c$.
    Donc, $2$ divise $b^2$, et donc $b$.
    Ainsi, $2$ est un diviseur comun à $a$ et $b$, ce qui est impossible puisque $a$ et $b$ sont premiers entre eux.
    On en déduit qu'un tel nombre rationnel $q$ ne peut exister.

    \blank[medium]

    Montrons maintenant que la suite $u$ est de Cauchy. 
    Pour ce faire, on montre d'abord par récurrence que $u_n^2 \leq 2$ pour tout entier naturel $n$. 
    Cette suite est donc majorée—en effet, pour tout entier naturel $n$, on aura $u_n \leq 2$ (sans quoi on aurait $u_n^2 > 4 > 2$). 
    Pour tout entier naturel $n$, on aura alors $\left(2 - u_n^2 \right) \divslash 4 \geq 0$, montrant que la suite $u$ est croissante, et donc (car majorée et croissante) de Cauchy.

    Plus préciément, on montre par récurrence sur $n$ que, pour tout entier naturel $n$, $0 \leq u_n$ et $u_n^2 \leq 2$.
    
    \blank[medium]

    Pour $n = 0$, on a $u_n = 1$, donc $u_n \geq 0$ et $u_n^2 = 1$, donc $u_n^2 \leq 2$. 

    Soit $n$ un entier naturel et supposons que $u_n \geq 0$ et $u_n^2 \leq 2$. 
    Alors, $2 - u_n^2 \geq 0$, donc $u_{n+1} \geq u_n$, donc $u_{n+1} \geq 0$.
    
    Par ailleurs, on a : 
    \startformula \startmathalignment
        \NC u_{n+1} \times u_{n+1} 
        \NC = u_n^2 + \left(2 - u_n^2 \right) \times u_n \divslash 2 + \left(2 - u_n^2 \right)^2 \divslash 16 \NR
        \NC\NC = \frac{1}{4} + u_n - \frac{u_n^2}{2} \times u_n + \frac{u_n^4}{16} . \NR
    \stopmathalignment\stopformula
    Puisque $u_n^2 \leq 2$, $u_n^4 \leq 4$.
    En outre, $u_n \leq 3 \divslash 2$ (sans quoi on aurait $u_n^2 > 9 \divslash 4 = 2 + 1 \divslash 4 > 2$).
    Donc, 
    \startformula
        u_{n+1} \times u_{n+1} \leq 2 - \frac{u_n^2}{2} \times u_n.
    \stopformula
    Enfin, $u_n \geq 0$, cela donne $u_{n+1} \leq 2$.
    Le résultat attendu est donc vrai au rang $n+1$.
    Par récurrence, il l'est pour tout entier naturel $n$, ce qui conclut la preuve.
\stopNoteBox

\blank[medium]

\noindent\bold{Lemme :} Toute suite de nombres rationnels de Cauchy est bornée. 

\blank[medium]

\noindent\bold{Démonstration :} 
    Soit $u$ une suite de nombres rationnels. 
    On suppose que $u$ est de Cauchy. 
    On peut choisir un entier naturel $n_0$ tel que $\abs{u_n - u_m} \leq 1$ pour tous entiers naturels $n$ et $m$ supérieurs ou égaux à $n_0$. 
    Soit $E$ l'ensemble $\left\lbrace \abs{u_0}, \abs{u_1}, \dots, \abs{u_{n_0 - 1}}, \abs{u_{n_0}} + 1 \right\rbrace$. 
    L'ensemble $E$ est fini et non vide (son cardinal est $n_0$). Donc (puisque $\leq$ est une relation d'ordre total sur $\mathbb{Q}$), il admet un maximum, noté $M$ dans la suite de cette démonstration.
    Soit $n$ un entier naturel quelconque. 
    Si $n < n_0$, alors $\abs{u_n} \in E$, donc $\abs{u_n} \leq M$.
    Sinon, on a $\abs{u_n - u_{n_0}} \leq 1$, donc, d'après l'inégalité triangulaire, $\abs{u_n} - \abs{u_{n_0}} \leq 1$, donc $\abs{u_n} \leq \abs{u_{n_0}} + 1$.
    Puisque $\abs{u_{n_0}} + 1 \in E$, $\abs{u_{n_0}} + 1 \leq M$, donc $u_n \leq M$. 

    Pour tout entier naturel $n$, on a donc $\abs{u_n} \leq M$, et donc $-M \leq u_n \leq M$.
    Cela montre que la suite $u$ est bornée. 

    \done

\blank[medium]

\emph{À écrire...}

\stopsubsection

\startsubsection[title=Les nombres réels comme limites de suites de Cauchy]

\noindent\bold{Définition :} Soit $\mathcal{U}$ l'ensemble des suites de Cauchy de nombres rationnels.
    On définit la relation d'équivalence $R$ sur $\mathcal{U}$ par : soit deux élément $u$ et $v$ de $\mathcal{U}$, $u \mathrel{R} v$ si et seulement si $u - v$ tend vers $0$.
    Un \emph{nombre réel} est une classe d'équivalence de $R$.
    L'\emph{ensemble des nombres réels}, noté $\symbols{\mathbb{R}}$, est l'ensemble des classes d'équivalences de $R$. 

\blank[medium]

\noindent\bold{Preuve qu'il s'agit bien d'une relation d'équivalence :}
\startitemize[nowhite]
    \item \emph{Réflexivité :} Soit $u$ une suite de Cauchy.
        Alors, $u - u$ est la suite constante égale à $0$, donc tend vers $0$.
        Donc, $u \mathrel{R} u$.
    \item \emph{Symétrie :} Soit $u$ et $v$ deux suites de Cauchy telles que $u \mathrel{R} v$.
        Alors, $u - v$ tend vers $0$.
        Donc, $v - u$ tend vers $0$.
        Donc, $v \mathrel{R} u$.
    \item \emph{Transitivité :} Soit $u$, $v$. et $w$ trois suites de Cauchy telles que $u \mathrel{R} v$ et $v \mathrel{R} w$.
        Alors, $u - v$ tend vers $0$ et $v - w$ tend vers $0$.
        Puisque $u - w = (u - v) + (v - w)$~\footnote{
            En effet, pour tout entier naturel $n$, on a : $((u - v) + (v - w))_n = (u - v)_n + (v - w)_n = (u_n - v_n) + (v_n - w_n) = (u_n + (-v_n)) + (v_n - w_n) = u_n + ((-v_n) + (v_n - w_n)) = u_n + (- w_n) = u_n - w_n = (u - w)_n$.
        }, on en déduit que $u - w$ tend vers $0$, et donc que $u \mathrel{R} w$.
\stopitemize

\done

\blank[medium]

\emph{À écrire...}

\stopsubsection

\startsubsection[title=Relation d'ordre]

\emph{À écrire...}

\stopsubsection

\startsubsection[title=Structure de corps]

\emph{À écrire...}

\stopsubsection

\startsubsection[title=Puissances]

\noindent\bold{Définition :} On définit par récurrence les \emph{puissances entières positives} d'un nombre réel $x$ par :
    \startitemize[nowhite]
        \item $x^0 = 1$,
        \item pour tout entier naturel $n$, $x^{n+1} = x \times x^n$.
    \stopitemize
    Notons que $x^1 = x$ et $x^n \neq 0$ pour tout entier naturel $n$ (se montre facilement par récurrence).
    \index{Puissance}
    Pour tout entier naturel non nul $n$, on définit la \emph{puissance entière négative} $x^{-n}$ comme égale à $1 \divslash z^x$.
    \index{Puissance négative}

\blank[medium]

\noindent\bold{Définition :} Soit $x$ et $y$ deux nombres réels et $n$ un entier naturel non nul.
    On dit que $y$ est une \emph{racine $n$e} de $x$ si $x^n = y$.
    \index{Racine}

\blank[medium]

\noindent\bold{Lemme :} Soit $x$ un nombre réel et $n$ et $m$ deux entiers. 
    On suppose $q > 0$ ou $n \geq 0$ et $m \geq 0$.
    Alors, 
    \startitemize[nowhite]
        \item $x^n \times x^m = x^{n+m}$,
        \item $(x^n)^m = x^{n m}$.
    \stopitemize

\blank[medium]

\noindent\bold{Démonstration :} Voir démonstration du théorème équivalent pour les nombres rationnels \at{page}{}[demo:rel_puissances_q] en remplaçant $q$ par $x$.

\blank[medium]

\noindent\bold{Définition :} Soit $n$ un entier naturel non nul. 
    Tout nombre réel positif $x$ a exactement une racine $n$e positive, notée $x^{1 \divslash n}$ ou $\sqrt[n]{x}$.
    (Pour $n = 2$, le nombre $2$ pourra être omis dans cette seconde notation.)

\blank[medium]

\noindent\bold{Démonstration :} ***

\blank[medium]

\emph{À écrire...}

\stopsubsection

\startsubsection[title=Nombres entiers et rationnels comme sous-ensembles des nombres réels]
\index{Nombre rationnel} \index{Rationnel}
\index{Nombre entier} \index{Entier}

On dira, quand il n'y a pas d'ambiguité, qu'un nombre réel $x$ est \emph{rationnel} si au moins un de ses éléments converge dans $\mathbb{Q}$.
On dira que $x$ est \emph{entier} si la limite de cet élément est entière, \emph{i.e.} si le dénominateur de sa représentation irréductible est $1$.

\blank[medium]

\noindent\bold{Lemme :} Soit $x$ un nombre réel. Si $x$ est rationnel selon la définition ci-dessus, alors tous ses éléments convergent dans $\mathbb{Q}$ et leurs limites sont identiques.

\blank[medium]

\noindent\bold{Démonstration :} ***

\stopsubsection

\startsubsection[title=Le tore]

On définit le tore $\symbols{\mathbb{T}}$ par l'ensemble des classe d'équivalences pour la relation $R$ sur $\mathbb{R}$ définie par : pour tous nombres réels $x$ et $y$, $a \mathrel{R} y$ si et seulement si $x - y$ est entier. 

\blank[medium]

\noindent\bold{Preuve qu'il s'agit bien d'une relation d'équivalence :}
\startitemize[nowhite]
    \item \emph{Réflexivité :} Soit $x$ un nombre réel.
        Par définition, $x - x$ a pour représentant la suite constante égale à $0$, qui a pour limite $0$.
        Donc, $x - x$ est entier, donc $x \mathrel{R} x$.
    \item \emph{Symétrie :} Soit $x$ et $y$ deux ombres réels.
        On suppose $x \mathrel{R} y$. 
        Soit $u$ un représentant de $x$ et $v$ un représentant de $y$.
        Alors, $u - v$ conerge et sa limite est un entier $n$.
        Donc, $v - u$ tend vers $-n$, qui est aussi entier.
        Donc, $y \mathrel{R} x$.
    \item \emph{Transitivité :} Soit $x$, $y$ et $z$ tels que $x \mathrel{R} y$ et $y \mathrel{R} z$. 
        Soit $u$ un représentant de $x$, $v$ un représentant de $y$ et $w$ un représentant de $z$.
        Alors, $u - v$ tend vers un entier $n$ et $v - w$ tend vers un entier $m$.
        Donc, $u - w$ (égale à $(u - v) + (v - w)$) tend vers $n + m$, qui est entier.
        Donc, $x \mathrel{R} z$.
\stopitemize

\done
 
\stopsection
