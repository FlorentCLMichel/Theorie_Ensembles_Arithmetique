\startmode[export] 
\setupbackend[export=yes,hyphen=yes]
\setupexport[
    cssfile=Ensembles_Arithmetique/Resources/style.css,
    width=55em, 
    firstpage=Ensembles_Arithmetique/Resources/firstpage.png, 
    author={Florent Michel}]
\definereferenceformat[Content][left=,right=,type=title]
\define[2]\startchapterRef{\startchapter[title=#1,reference=#2,list=\Content[#2]]}
\define\startformula{\par$$}
\define\stopformula{$$\par}
\define\startimageifexport\startimage
\define\stopimageifexport\stopimage
\define[1]\imagebuffer{
  \startimage
    \getbuffer[#1]
  \stopimage}
  \define\bimage{\starttextdisplay\startimage}
  \define\eimage{\stopimage\stoptextdisplay}
\stopmode
\startnotmode[export] 
\define[2]\startchapterRef{\startchapter[title=#1,reference=#2]}
\define\startimageifexport{}
\define\stopimageifexport{}
\define[1]\imagebuffer{
  \startimage
    \typesetbuffer[#1]
  \stopimage}
  \define\bimage{\startimage}
  \define\eimage{\stopimage}
\stopnotmode

% Variables
\setvariables[metadata][
    title=Théorie des Ensembles et Arithmétique,
    author=Florent Michel
]


% Load resources
\loadmarkfile{Ensembles_Arithmetique/Resources/buff-imp-cpp}


% Set the document language
\mainlanguage[fr]


% Load TiKZ and PGF
\usemodule[tikz]
\usetikzlibrary{positioning, automata, calc, arrows.meta}
\usemodule[pgfplots]
% To parse points using macros in PGFPlots
\unprotect
\let\orig@pgfplots@foreach@plot@coord@NEXT\pgfplots@foreach@plot@coord@NEXT
\def\pgfplots@foreach@plot@coord@NEXT{%
    \expandafter\orig@pgfplots@foreach@plot@coord@NEXT\expandafter
}
\protect


% Document layout
\setuppapersize[letter,portrait]
\setuplayout
  [backspace=2cm,
   topspace=1cm,
   header=1cm,
   footer=1cm,
   top=1cm,
   bottom=1cm,
   margin=2cm,
   margindistance=0cm,
   edge=0cm,
   edgedistance=0cm,
   width=fit,
   height=fit]
\setuppagenumbering[location=bottom]


% Interactions
\setupcolors[state=start]
\definecolor[linkcolor][r=.1,g=.3,b=.9]
\setupinteraction[state=start, focus=standard, color=linkcolor, contrastcolor=linkcolor, style=]

% Bookmarks
\placebookmarks [chapter,section,subject,subsection]
                [chapter,section,subject]

% Font Setup
\definefontfamily [myfamily] [serif] [XITS]
\definefontfamily [myfamily] [math] [XITS Math]
\definefontfamily [myfamily] [sans] [Lato] [scale=0.9]
\definefontfamily [myfamily] [mono] [Fira Code] [scale=0.88]
\setupbodyfont [myfamily, 10pt]


% TOC format
\setupcombinedlist[content][
    interaction=all,
    list={chapter,section}, 
    alternative=c,
]
\setuplist[chapter][
    label=yes,
    width=fit,
    distance=0.5em,
    style=bold,
    before=\blank[0.5em],
    numbercommand=\ss,
    textcommand=\ss,
    pagecommand=\ss,
]
\setuplist[section][
    width=fit,
    margin=1em,
    distance=0.5em,
    before=\blank[0.em],
    numbercommand=\ss,
    textcommand=\ss,
    pagecommand=\ss,
]
\startmode[export]
  \setuplist[chapter][pagenumber=no]
  \setuplist[section][pagenumber=no]
  \setuplist[subsection][pagenumber=no]
\stopmode


% Define indices
\startnotmode[export]
  \defineregister[wordsRegister][
    pagestyle=,
    alternative=A,
    pageleft=\dotfill,
    n=3,
    method={pm,mc,uc}]
  \defineregister[symbolsRegister][
    pagestyle=,
    indicator=no,
    before=,
    pageleft=\dotfill,
    n=4]
  \define[1]\index{\lua{s = "#1"; context("\\wordsRegister{" .. s:sub(1,1):upper() .. s:sub(2) .. "}")}}
  \define[1]\words{\index{#1}#1}
  \define[1]\symbols{$\symbolsRegister{$#1$}$#1}
\stopnotmode
\startmode[export]
  \define[1]\index{}
  \define[1]\words{#1}
  \define[1]\symbols{#1}
  \define[1]\symbolsRegister{}
\stopmode


% Move footnotes to end of chapters in export mode
\startmode[export] 
  \setupnotes[location=none] % Don't show footnotes on each page
  \define[1]\footnotenumberstyle{\high{\color[black]{#1}}}
  \setupnotation[footnote][
      numbercommand=,
      distance=0.4em,
      margin=-0.2em,
      way=bychapter,
      alternative=left, 
      hang=1,
  ]
  \setuphead[chapter][
    aftersection=\ifcase\rawcountervalue[footnote]\relax\else\blank[4ex]\vfill\hrule\pagebreak[no]\placefootnotes\fi
  ]
\stopmode


% Style for headers
\setuphead[title][align=center]
\setuphead[chapter][
    page=yes,
    prefix=Chapitre,
    numberstyle=\ss\tfc\bf,
    textstyle=\ss\tfc\bf]
\setuphead[section][
    before=\blank[4ex],
    numberstyle=\ss\tfb\bf,
    textstyle=\ss\tfb\bf]
\setuphead[subsection][
    before=\blank[2ex],
    numberstyle=\tfa\ss\bf,
    textstyle=\tfa\ss\bf]


% Default style for enumerations
\setupitemize[
    before=\setupwhitespace[0pt]\blank[.5ex],
    inbetween=\blank[.5ex],
    after=\blank[nowhite]\blank[.5ex],
    leftmargin=1em,
    indentnext=auto,
]


% Styles for page numbers
\definestructureconversionset [frontpart:pagenumber] [] [Romannumerals]
\definestructureconversionset [bodypart:pagenumber] [] [numbers]
\definestructureconversionset [backpart:pagenumber] [] [numbers]


% Mathematics
\setupmathematics[
    autopunctuation=yes,
    differentiald=upright,
]
\setmathspacing \mathordcode \mathclosecode \allmathstyles 0mu
\setmathspacing \mathopencode \mathordcode \allmathstyles 0mu
\redefine{\sum}{\utfchar{"2211}}
\redefine{\prod}{\utfchar{"220f}}
\redefine{\ii}{\mathrm{i}}        % root of -1
\define{\lii}{\utfchar{"27e6}}    % integer interval (left)
\define{\rii}{\utfchar{"27e7}}    % integer interval (right)
\define{\ds}{\utfchar{"2215}}     % division slash


% Special styles for the frontmatter
\startsectionblockenvironment[frontmatter]
  \definestructureconversionset [romannumerals]
\stopsectionblockenvironment


% Format for the TOC at the start of each chapter
\startluacode
interfaces.implement {
  name      = "chaptertoc",
  public    = true,
  arguments = { "hash", "argument" },
  actions   = function(hash, content)
    hash = hash or {}
    local bottomruleoffset = hash.bottomruleoffset or "2pt"
    context(
      "\\determinelistcharacteristics[section]"
      .. "\\doifmode{*list}{\\vskip1ex\\hrule\\startsimplecolumns[n=2]\\placecontent\\stopsimplecolumns"
      .. "\\vskip" .. bottomruleoffset
      .. "\\hrule}"
    )
  end
}
\stopluacode
%\define\chaptertoc{%
%  \determinelistcharacteristics[section]%
%  \doifmode{*list}{\vskip1ex\hrule\startsimplecolumns[n=2]\placecontent\stopsimplecolumns\blank[2pt]\hrule}%
%}


% Special styles for the main text
\define[1]\chapnumber{Chapitre~#1~:}
\startsectionblockenvironment[bodypart]
  \resetuserpagenumber
  \setupcombinedlist[content][
      list={section,subsection}, 
  ]
  \setuplist[section][
      label=yes,
      width=fit,
      margin=0em,
      distance=0.5em,
      style=bold,
      before=\blank[0.5em],
      numbercommand=\ss,
      textcommand=\ss,
      pagecommand=\ss,
  ]
  \setuplist[subsection][
      width=fit,
      margin=1em,
      distance=0.5em,
      before=\blank[0em],
      numbercommand=\ss,
      textcommand=\ss,
      pagecommand=\ss,
  ]
  \setuphead[chapter][after=\blank[big],numbercommand=\chapnumber]
\stopsectionblockenvironment


% Special styles for the appendices
\define[1]\appchapnumber{Appendice~#1~:}
\startsectionblockenvironment[appendix]
  \setupcombinedlist[content][
      list={section,subsection}, 
  ]
  \setuplist[section][
      label=yes,
      width=fit,
      margin=0em,
      distance=0.5em,
      style=bold,
      before=\blank[0.5em],
      numbercommand=\ss,
      textcommand=\ss,
      pagecommand=\ss,
  ]
  \setuplist[subsection][
      width=fit,
      margin=1em,
      distance=0.5em,
      before=\blank[0em],
      numbercommand=\ss,
      textcommand=\ss,
      pagecommand=\ss,
  ]
  \setuphead[chapter][after=\blank[big],numbercommand=\appchapnumber]
\stopsectionblockenvironment


% Excursus style
\definetextbackground
  [leftbartext]
  [
    location=paragraph,
    mp=mpos:region:leftbar,
    width=broad,
    frame=off,
    rulethickness=0.3em,
    leftoffset=0.5em,
    rightoffset=0.5em,
    topoffset=0ex,
    bottomoffset=0ex,
  ]
\startuseMPgraphic{mpos:region:leftbar}
  draw_multi_pars;
  draw_multi_side;
\stopuseMPgraphic
\definetextbackground
  [blocktext]
  [leftbartext]
  [
    framecolor=black,
    backgroundcolor=white,
  ]
\definedescription
  [excursus]
  [
    margin=0.5em,
    width=broad,
    before={\startblocktext},
    after={\stopblocktext},
  ]


% Style for the abstract
\definenarrower[NarrowerAbstract][middle=1cm]
\definedescription
  [abstract]
  [
    text=Résumé,
    alternative=top,
    headstyle=\tfa\ss\bf,
    margin=0.5em,
    width=broad,
    before={\startNarrowerAbstract\startblocktext},
    after={\stopblocktext\stopNarrowerAbstract},
  ]

% text in a box
\startluacode
interfaces.implement {
  name      = "inbox",
  public    = true,
  arguments = { "hash", "argument" },
  actions   = function(hash, content)
    hash = hash or {}
    local col = hash.color or "black"
    context("\\starttikzpicture[baseline=-0.7ex, inner sep=0.5ex, color="
            .. col
            .. "] \\node[draw] {" 
            .. content 
            .. "\\vphantom{0123456789}}; \\stoptikzpicture")
  end
}
\stopluacode
%\define[1]\inbox{\starttikzpicture[baseline=-0.7ex, inner sep=0.5ex] \node[draw] {#1\vphantom{0123456789}}; \stoptikzpicture}

% Paragraph indentation
\setupindenting[yes,medium]


% colored box style

\startuseMPgraphic{mp:coloredframe}
  coloredFrameMargin := EmWidth;
  begingroup;
    for i=1 upto nofmultipars :
      if multikind[i] = "single" :
        path p;
        p := ( llcorner multipars[i] + (0, -coloredFrameMargin)
               -- lrcorner multipars[i] + (0, -coloredFrameMargin)
               .. lrcorner multipars[i] + (coloredFrameMargin/sqrt(2), -coloredFrameMargin/sqrt(2))
               .. lrcorner multipars[i] + (coloredFrameMargin, 0)
               -- urcorner multipars[i] + (coloredFrameMargin, 0)
               .. urcorner multipars[i] + (coloredFrameMargin/sqrt(2), coloredFrameMargin/sqrt(2))
               .. urcorner multipars[i] + (0, coloredFrameMargin)
               -- ulcorner multipars[i] + (0, coloredFrameMargin)
               .. ulcorner multipars[i] + (-coloredFrameMargin/sqrt(2), coloredFrameMargin/sqrt(2))
               .. ulcorner multipars[i] + (-coloredFrameMargin, 0)
               -- llcorner multipars[i] + (-coloredFrameMargin, 0)
               .. llcorner multipars[i] + (-coloredFrameMargin/sqrt(2), -coloredFrameMargin/sqrt(2))
               .. llcorner multipars[i] + (0, -coloredFrameMargin)
               -- cycle ) ;
        fill p withcolor boxfillcolor ;
      elseif  multikind[i] = "first"  :
        path p;
        p := ( llcorner multipars[i] + (0, -coloredFrameMargin)
               -- lrcorner multipars[i] + (coloredFrameMargin, -coloredFrameMargin)
               -- urcorner multipars[i] + (coloredFrameMargin, 0)
               .. urcorner multipars[i] + (coloredFrameMargin/sqrt(2), coloredFrameMargin/sqrt(2))
               .. urcorner multipars[i] + (0, coloredFrameMargin)
               -- ulcorner multipars[i] + (0, coloredFrameMargin)
               .. ulcorner multipars[i] + (-coloredFrameMargin/sqrt(2), coloredFrameMargin/sqrt(2))
               .. ulcorner multipars[i] + (-coloredFrameMargin, 0)
               -- llcorner multipars[i] + (-coloredFrameMargin, -coloredFrameMargin)
               -- cycle ) ;
        fill p withcolor boxfillcolor ;
      elseif  multikind[i] = "middle" :
        path p;
        p := ( llcorner multipars[i] + (0, -coloredFrameMargin)
               -- lrcorner multipars[i] + (coloredFrameMargin, -coloredFrameMargin)
               -- urcorner multipars[i] + (coloredFrameMargin, coloredFrameMargin)
               -- ulcorner multipars[i] + (-coloredFrameMargin, coloredFrameMargin)
               -- llcorner multipars[i] + (-coloredFrameMargin, -coloredFrameMargin)
               -- cycle ) ;
        fill p withcolor boxfillcolor ;
      elseif  multikind[i] = "last"   :
        path p;
        p := ( llcorner multipars[i] + (0, -coloredFrameMargin)
               -- lrcorner multipars[i] + (0, -coloredFrameMargin)
               .. lrcorner multipars[i] + (coloredFrameMargin/sqrt(2), -coloredFrameMargin/sqrt(2))
               .. lrcorner multipars[i] + (coloredFrameMargin, 0)
               -- urcorner multipars[i] + (coloredFrameMargin, coloredFrameMargin)
               -- ulcorner multipars[i] + (-coloredFrameMargin, coloredFrameMargin)
               -- llcorner multipars[i] + (-coloredFrameMargin, 0)
               .. llcorner multipars[i] + (-coloredFrameMargin/sqrt(2), -coloredFrameMargin/sqrt(2))
               .. llcorner multipars[i] + (0, -coloredFrameMargin)
               -- cycle ) ;
        fill p withcolor boxfillcolor ;
      fi;
    endfor ;
  setbounds currentpicture to OverlayBox ;
  endgroup;
\stopuseMPgraphic

\definecolor[coloredFrameColor][r=0.5,g=1,b=1]
\definetextbackground
  [coloredFrame]
  [mp=mp:coloredframe,
   location=paragraph,
   backgroundcolor=coloredFrameColor,
   framecolor=coloredFrameColor,
   before={},
   after={}]

\definestartstop
  [NoteBox]
  [before={\blank[2em]\startcoloredFrame},
   after={\stopcoloredFrame\blank[2em]}]


% A simple table style
\startsetups fTable
  \setupTABLE[frame=off,rulethickness=1pt,align={middle,lohi}, offset=0.5ex, strut=no]
  \setupTABLE[r][1][bottomframe=on]
  \setupTABLE[r][2][toffset=0.5ex+0.25pt]
  \setupTABLE[c][\getvariable{myvars}{fTableNLeftCols}][rightframe=on,roffset=0.5ex+1pt]
  \setupTABLE[c][\getvariable{myvars}{fTableNLeftCols}+1][loffset=0.5ex+1pt]
\stopsetups

% styles
\definehighlight [bold] [style=bold]
\definehighlight [emph] [style=italic]
\definehighlight [sans] [style=\ss]

% Get the height and width of content
\define[3]\getHWof
  {\bgroup
   \setbox\scratchbox\hbox{#1}%
   \expanded{\egroup\definemeasure[#2][\the\ht\scratchbox]
                    \definemeasure[#3][\the\wd\scratchbox]}}
\define[1]\Center{%
  \getHWof{#1}{tempHeight}{tempWidth}
  \offset[x=0.5*\hsize-0.5*\measured{tempWidth}]{#1}}

% Math commands
\define\lb{\left\lbrace}
\define\rb{\right\rbrace}
\define\divslash{∕}
\define\setminus{\mirror{\divslash}} % Required for XITS Math
\define\ep{\epsilon}

% Miscelanous commands
\define[2]\href{\goto{\color[linkcolor]{#1}}[url(#2)]}
\define[1]\lua{\ctxlua{#1}}
\define[1]\plua{\lua{context(#1)}}
\define[1]\done{\hfill$\square$}
