\section{Codes Haskell, C/C++ et Rust}

\subsection{Écriture d'un entier naturel en base \texorpdfstring{$b$}{b}}
\label{app:Haskell_baseb}

\begin{lstlisting}[language=Haskell]
-- read a non-negative integer (first argument) in base b
from_base_acc :: Integer -> [Integer] -> Integer -> Integer
from_base_acc acc [] b = acc
from_base_acc acc (x:l) b = from_base_acc (x + b*acc) l b 
from_base = from_base_acc 0

-- conversion of a non-negative integer n in base b
to_base_acc :: [Integer] -> Integer -> Integer -> [Integer]
to_base_acc acc x b | x == 0 = acc
                    | otherwise = let y = (mod x b) in 
                         to_base_acc (y:acc) (div (x-y) b) b
to_base = to_base_acc []
\end{lstlisting}

\subsection{Test de primalité}
\label{app:Haskell_primalité}

\noindent 
La fonction \texttt{isPrime} ci-dessous teste si un entier $p$ est un nombre premier. 
Sa complexité est $O \left( p^{1/2} \right)$.

\begin{lstlisting}[language=Haskell]
-- test if an integer is divisible by another
isDivisible :: Integer -> Integer -> Bool
isDivisible a b = b * (div a b) == a

-- check that p is not divisible by an integer between q and the square root of p
testNonDivisible :: Integer -> Integer -> Bool
testNonDivisible p q | q*q > p = True
                     | isDivisible p q = False
                     | otherwise = testNonDivisible p (q+1)

-- check if a non-negative integer p is prime
isPrime :: Integer -> Bool
isPrime p | p <= 1 = False
          |otherwise = testNonDivisible p 2
\end{lstlisting}

\noindent
La fonction \texttt{findNPrimes} ci-dessous prend un entier naturel $N$ en argument et retourne la liste des $N$ premiers nombres premiers. 
\lstinputlisting[language=Haskell]{Ensembles_Arithmétique/Code/FindNPrimes.hs}

\medskip

\noindent Version C du test de primalité: 
\lstinputlisting[language=C]{Ensembles_Arithmétique/Code/is_prime.c}

\medskip

\noindent Version Rust du test de primalité: 
\lstinputlisting[language=Rust]{Ensembles_Arithmétique/Code/is_prime.rs}

\subsection{PGCD}
\label{app:Haskell_pgcd}

\noindent 
La fonction \texttt{pgcd} ci-dessous prend comme arguments deux entiers naturels et donne leur pgcd.
\begin{lstlisting}[language=Haskell]
pgcd :: Integer -> Integer -> Integer
pgcd n 0 = n
pgcd n m | n < m = pgcd m n
         | otherwise = pgcd m (n - (m * (div n m)))
\end{lstlisting}

\medskip

\noindent Version C: 
\lstinputlisting[language=C]{Ensembles_Arithmétique/Code/pgcd.c}

\medskip

\noindent Version Rust: 
\lstinputlisting[language=Rust]{Ensembles_Arithmétique/Code/pgcd.rs}

\subsection{Crible d'Érastosthène}
\label{app:code_erastosthene}

Il s'agit d'un algorithme permettant de trouver tous les nombres premiers infèrieurs ou égaux à un entier naturel donné.

\bigskip

\noindent Version C++: 
\lstinputlisting[language=C++]{Ensembles_Arithmétique/Code/Érastosthène.cpp}

\bigskip

\noindent Version Rust\footnote{
    Cette version est sensiblement plus longue car le compilateur rustc 1.51.0, utilisé pour tester la version Rust, n'optimise pas les vecteurs de type \texttt{bool}—chaque entrée prend ainsi un octet en mémoire. 
    Nous définissons donc des fonctions auxiliaires afin de pouvoir représenter huit \texttt{bool}s pour chaque octet, afin de ne pas utiliser plus de mémoire que nécessaire.
    Le compilateur g++ 7.5.0 avec lequel la version C++ a été testée optimise les vecteurs de type \texttt{bool} pour que chaque entrée ne prenne (en moyenne et pour de grands vecteurs) qu'un bit de mémoire ; il n'est donc pas besoin, pour cette version, d'employer des fonctions auxiliaires. 
}: 
\lstinputlisting[language=Rust]{Ensembles_Arithmétique/Code/Érastosthène.rs}
