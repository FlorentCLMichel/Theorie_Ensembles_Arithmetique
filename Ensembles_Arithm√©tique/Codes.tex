\section{Codes Haskell, C/C++ et Rust}

\subsection{Écriture d'un entier naturel en base \texorpdfstring{$b$}{b}}
\label{app:Haskell_baseb}

\noindent
\textattachfile{Ensembles_Arithmétique/Code/base.hs}{\textbf{Base.hs}}
\vspace*{-1ex}
\inputminted{Haskell}{Ensembles_Arithmétique/Code/base.hs}

\subsection{Test de primalité}
\label{app:Haskell_primalité}

\noindent 
La fonction \texttt{isPrime} ci-dessous teste si un entier $p$ est un nombre premier. 
Sa complexité est $O \left( p^{1/2} \right)$.

\vspace*{1ex}

\noindent
\textattachfile{Ensembles_Arithmétique/Code/isPrime.hs}{\textbf{isPrime.hs}}
\vspace*{-1ex}
\inputminted{Haskell}{Ensembles_Arithmétique/Code/isPrime.hs}

\noindent
La fonction \texttt{findNPrimes} ci-dessous prend un entier naturel $N$ en argument et retourne la liste des $N$ premiers nombres premiers. 

\vspace*{1ex}

\noindent
\textattachfile{Ensembles_Arithmétique/Code/FindNPrimes.hs}{\textbf{FindNPrimes.hs}}
\vspace*{-1ex}
\inputminted{Haskell}{Ensembles_Arithmétique/Code/FindNPrimes.hs}

\medskip

\noindent Version C du test de primalité: 
\textattachfile{Ensembles_Arithmétique/Code/is_prime.c}{\textbf{is\_prime.c}}
\inputminted{C}{Ensembles_Arithmétique/Code/is_prime.c}

\medskip

\noindent Version Rust du test de primalité: 
\textattachfile{Ensembles_Arithmétique/Code/is_prime.rs}{\textbf{is\_prime.rs}}
\inputminted{Rust}{Ensembles_Arithmétique/Code/is_prime.rs}

\subsection{PGCD}
\label{app:Haskell_pgcd}

\noindent 
La fonction \texttt{pgcd} ci-dessous prend comme arguments deux entiers naturels et donne leur pgcd.

\vspace*{1ex}

\noindent
\textattachfile{Ensembles_Arithmétique/Code/Base.hs}{\textbf{pgcd.hs}}
\vspace*{-1ex}
\inputminted{Haskell}{Ensembles_Arithmétique/Code/pgcd.hs}

\medskip

\noindent Version C: 
\textattachfile{Ensembles_Arithmétique/Code/pgcd.c}{\textbf{pgcd.c}}
\inputminted{C}{Ensembles_Arithmétique/Code/pgcd.c}

\medskip

\noindent Version Rust: 
\textattachfile{Ensembles_Arithmétique/Code/pgcd.rs}{\textbf{pgcd.rs}}
\inputminted{Rust}{Ensembles_Arithmétique/Code/pgcd.rs}

\subsection{Crible d'Érastosthène}
\label{app:code_erastosthene}

Il s'agit d'un algorithme permettant de trouver tous les nombres premiers infèrieurs ou égaux à un entier naturel donné.

\bigskip

\noindent Version C++: 
\textattachfile{Ensembles_Arithmétique/Code/Érastosthène.cpp}{\textbf{Érastosthène.cpp}}
\inputminted{C++}{Ensembles_Arithmétique/Code/Érastosthène.cpp}

\bigskip

\noindent Version Rust\footnote{
    Cette version est sensiblement plus longue car le compilateur rustc 1.51.0, utilisé pour tester la version Rust, n'optimise pas les vecteurs de type \texttt{bool}—chaque entrée prend ainsi un octet en mémoire. 
    Nous définissons donc des fonctions auxiliaires afin de pouvoir représenter huit \texttt{bool}s pour chaque octet, afin de ne pas utiliser plus de mémoire que nécessaire.
    Le compilateur g++ 7.5.0 avec lequel la version C++ a été testée optimise les vecteurs de type \texttt{bool} pour que chaque entrée ne prenne (en moyenne et pour de grands vecteurs) qu'un bit de mémoire ; il n'est donc pas besoin, pour cette version, d'employer des fonctions auxiliaires. 
}: 
\textattachfile{Ensembles_Arithmétique/Code/Érastosthène.rs}{\textbf{Érastosthène.rs}}
\inputminted{Rust}{Ensembles_Arithmétique/Code/Érastosthène.rs}
