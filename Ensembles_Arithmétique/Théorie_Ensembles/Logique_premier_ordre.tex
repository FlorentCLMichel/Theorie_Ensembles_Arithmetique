\subsection{Logique du premier ordre}

La logique du premier ordre, aussi appelée \textit{logique des prédicats} ou \textit{calcul des prédicats du premier ordre}, est un cadre semi-formel%
\footnote{On adopte ici le point de vue que la logique du premier ordre ne repose pas sur une théorie vue comme plus fondamentale. Ses concepts fondamentaux sont ainsi définis intuitivement (puisque nous n'avons aucun concept plus fondamental qui permettrait de les définir formellement), d'où le qualificatif de « semi-formel », et non « formel ».} 
permettant de définir des théories. 
On peut la voir comme un langage, ou comme un ensemble d'éléments de langage.
Elle est utilisée tant en mathématiques qu'en philosophie, linguistique et informatique. 
Nous l'aborderons ici principalement d'un point de vue mathématique.

On considère ici une notion très basique du terme \textit{langage}, que l'on considère formé de deux éléments : 
\begin{itemize}[nosep]
    \item Un ensemble (au sens intuitif du terme) de \textit{symboles}.
    \item Des règles de formations de \textit{phrases} à partir des symboles.
\end{itemize}
Dans cette vision, les symboles constituent les fondations du langage, premettant de contruire les phrases, porteuses de sens.%
\footnote{Ce sens étant défini, \textit{in fine}, par un élément extérieur au langage, par exemple l'intuition de qui l'utilise.} 
On sépare parfois les symboles en deux catégories : \textit{fondamentaux} s'ils forment un ensemble unsécable, ou \textit{composites} s'ils sont formés d'autres symboles.

Intuitivement, la logique du premier ordre a pour symboles des variables (décrivant un domaine d'objets non logiques, c'est-à-dire non définis par la logique du premier ordre elle-même) quantifiées (par les quantificateurs « pour tout » et « il existe ») ou non, des symboles non logiques, ainsi que des connecteurs, utilisés pour construire des phrases, appelées \textit{formules}. 
Ces dernières sont aussi appelées \textit{propositions}, \textit{énoncés} ou \textit{prédicats}. 

Elle est une extension de la \textit{logique propositionelle}, qui exprime des énoncés, ou \textit{propositions}, aussi appelés \textit{prédicats}, auxquels on attribue une valeur dite de \textit{vérité} : vrai ou faux. 
Chaque proposition est soit vraie soit fausse, et ne peut être les deux simultanément. 
Ces énoncés peuvent être liés par conjonction, disjonction, implication, équivalence, ou modifés par négation. 
La logique du premier ordre contient, en outre, des variables et quantificateurs, ce qui la rend plus expressive. 
On peut dire qu'elle contient la logique propositionelle, au sens où cette dernière est équivalente à la logique du premier ordre élaguée des variables et quantificateurs.

Une théorie définie dans le cadre de la logique du premier ordre porte sur un domaine de discours spécifié que les variables quantifiées décrivent, permettant de définir des prédicats sur ce domaine, auxquels un ensemble d'axiomes tenus pour vrais permet d'associer une valeur de vérité. 
Un prédicat ne peut avoir pour arguments que des variables sur ce domaine, et seules les variables peuvent être quantifiés. 
Cela distingue la logique du premier ordre des logiques d'ordre supérieur, où un prédicat peut avoir un prédicat plus général comme argument ou des quantificateurs de prédicats peuvent être autorisés.

Plus formellement, une théorie définie dans le cadre de la logique du premier ordre se compose des éléments suivants : 
\begin{itemize}[nosep]
    \item Un \textit{alphabet}, c'est-à-dire un ensemble (au sens intuitif du terme) de symboles, dont certaines chaînes forment des \textit{termes}. 
          On divise généralement les symboles et deux catégories : les \textit{symboles logiques}, dont la signification est fixée, et les \textit{symboles non logiques}, dont le sens n'est pas univoquement défini par la théorie et doit être défini au cas par cas. 
          Certains de ces symboles sont définis par la logique du premier ordre ; d'autres peuvent être propres à la théorie. 
    \item Un \textit{domaine de discours} non vide que les variables décrivent (si $x$ désigne une variable, la formule $\exists x \mathsf{V}$ est toujours vraie (voir ci-dessous pour la signification de cette formule)).
    \item Des \textit{règles de formation}, exprimant comment construire les termes et formules. 
          Là encore, certaines sont définies par la logique du premier ordre et d'autres peuvent être propres à la théorie.
    \item Des \textit{formules} (aussi appelées \textit{propositions}) obtenues à partir de ces règles, exprimant des prédicats. 
        (Le terme \textit{prédicat} est aussi utilisé pour désigner une formule elle-même.)
        Une proposition est toujours vraie ou fausse%
        \footnote{À moins d'inclure la valeur de vérité indéfinie, voir section~\ref{subsub:Indéfinie}.}%
        , et ne peut être simultanément vraie et fausse. 
        Deux formules seront dites \textit{équivalentes} si elles prennent toujours la même valeur de vérité. 
        \begin{itemize}[nosep]
            \item Si $f$ et $g$ sont deux formules équivalentes, $g$ et $f$ sont équivalentes.
            \item Si $f$ et $g$ sont trois formules telles que $f$ et $g$ sont équivalentes et $g$ et $h$ sont équivalentes, alors $f$ et $h$ sont équivalentes.
        \end{itemize}
    \item Un ensemble d'\textit{axiomes}, ou propositions tenues pour vraies. 
          Ces axiomes permettent en général de déterminer la valeur de vérité d'autres prédicats.
\end{itemize}

\subsubsection{Symboles logiques}

Les symboles logiques incluent : 
\begin{itemize}
    \item Le symbole de quantification universelle $\forall$ (« pour tout »).
    \item Le symbole de quantification existentielle $\exists$ (« il existe »).
    \item Le connecteur de conjonction $\wedge$ (« et ») : si $P$ et $Q$ sont deux formules, $P \wedge Q$ est vraie si $P$ et $Q$ sont vraies et fausse sinon. 
    \item Le connecteur de disjonction $\vee$ (« ou ») : si $P$ et $Q$ sont deux formules, $P \vee Q$ est vraie si $P$ est vraie ou si $Q$ est vraie et fausse sinon. 
    \item Le connecteur de négation $\neg$ (« non ») : si $P$ est une formule, $\neg P$ est vraie si $P$ est fausse et fausse si $P$ est vraie. 
    \item Le connecteur d'implication $\Rightarrow$ (« implique ») : si $P$ et $Q$ sont deux formules, $P \Rightarrow Q$ est fausse si $P$ est vraie et $Q$ est fausse et vraie sinon. 
    La formule $P \Rightarrow Q$ est ainsi équivalente à $Q \vee \neg P$ (voir ci-dessous pour la signification des parenthèses et les règles d'évaluation).
    \item Le connecteur $\Leftarrow$ : si $P$ et $Q$ sont deux formules, $P \Leftarrow Q$ est fausse si $P$ est fausse et $Q$ est vraie et vraie sinon. 
    La formule $P \Leftarrow Q$ est ainsi équivalente à $P \vee \neg Q$.
    \item Le connecteur biconditionnel $\Leftrightarrow$ (« est équivalent à ») : si $P$ et $Q$ sont deux formules, $P \Leftrightarrow Q$ est vraie si $P$ et $Q$ sont soit toutes deux vraies soit toutes deux fausses, et fausse sinon. 
    La formule $P \Leftrightarrow Q$ est ainsi équivalente à $(P \wedge Q) \vee (\neg P \wedge \neg Q)$.
        Notons que, si $P$ et $Q$ sont deux prédicats, si $P \Leftrightarrow Q$ est vrai, alors $(\neg P) \Leftrightarrow (\neg Q)$ est vrai aussi.
    \item Un ensemble infini de \textit{variables}, souvent notées par des lettres grecques ou latines, éventuellement avec des indices ou exposants. 
    Les variables sont interprétées comme décrivant un domaine d'objets de base, qui ne peut être vide. 
    Elles sont aussi parfois appelées \textit{paramètres}. 
\end{itemize}

On définit également les constantes de vérité $\mathsf{V}$ pour « vraie » et $\mathsf{F}$ pour « fausse ». 
Elles sont deux formules, et $\mathsf{F}$ est équivalente à $\neg \mathsf{V}$.

Si $f$ est une formule, ces deux constantes de vérité sont équivalentes, respectivements, aux formules $f \vee (\neg f)$ et $f \wedge (\neg f)$. 
Enfin, on peut définir le connecteur (non standard) de vérité $\sharp$ : si $f$ est une formule, $\sharp f$ est vraie si $f$ est vraie et fausse sinon. 
(Avec ces notations, $\sharp f$ a toujours la même valeur de vérité que $f$. 
On introduit ce nouveau connecteur uniquement pour pouvoir exprimer la véracité d'une formule dans le cadre de la théorie ; il sera très peu employé dans la suite.)
Ce dernier connecteur ne rendant pas la théorie plus expressive, on l'omettra dans la suite sauf mention contraire.

Pour être plus formel, on peut ne définir dans un premiers temps que les variables et constantes de vérité, puis les symboles non logiques, les termes, et enfin les autres symboles logiques avec les formules qu'ils permettent de construire et l'égalité (voir ci-dessous). 
On adoptera ce point de vue dans la suite. 
Pour le moment, les symboles logiques (y compris l'égalité définie ci-dessous) ne sont donnés que comme une liste de symboles utilisés, qui prendront leur sens lorsque les formules et la sémantique seront définies.

Si $P$ est un prédicat à un ou plusieurs paramètres libres $a_1 a_2 \dots$ et si $b_1 b_2 \dots$ sont un même nombre de variables, on notera $P b_1 b_2 \dots$, ou $P(b_1, b_2, \dots)$ la formule obtenue en remplaçant dans $P$ les paramètres $a_1 a_2 \dots$ par $b_1 b_2 \dots$.

\subsubsection{Égalité} 

La \textit{logique du premier ordre avec égalité} inclut un autre symbole logique, $=$, définissant une relation binaire, dite \textit{égalité}, satisfaisant les axiomes suivants : 
\begin{itemize}
    \item Axiome de réciprocité : $\forall x \, (x = x)$. 
    \item Réflexivité : $\forall x \, \forall y \, [ (x=y) \Rightarrow (y=x) ]$.
    \item Transitivité : $\forall x \, \forall y \, \forall z \, [((x=y) \wedge (y=z)) \Rightarrow (x=z)]$.
    \item Schéma d'axiomes de Leibniz : Soit $P$ un prédicat à une variable. On a : $\forall x \, \forall y \, [ (x = y) \Rightarrow (P(x) \Leftrightarrow P(y))]$.
\end{itemize}

Deux objets $x$ et $y$ définis par une théorie sont dits \textit{égaux} si $x = y$. 
On considèrera alors qu'il s'agit du même objet. 
En particulier, changer l'un pour l'autre dans une formule ne modifie pas sa valeur de vérité.

Si $x$, $y$ et $z$ sont trois objets, on notera parfois par $x = y = z$ la formule $(x = y) \wedge (y = z)$. 

En présence de l'égalité, on définit aussi le symbole d'\textit{inégalité} $\neq$ définissant une relation binaire comme suit : la formule $x \neq y$ est équivalente à $\neg (x = y)$.

\subsubsection{Symboles non logiques}

Un symbole non logique est un symbole n'ayant pas de signification donnée par la logique du premier ordre. 
Il représentent généralement un prédicat, pouvant dépendre de variables placées à sa droite, éventuellement entre parenthèses.

\subsubsection{Termes}

Les termes sont définis comme suit : 
\begin{itemize}
    \item Toute variable est un terme. 
    \item Si $P$ est un prédicat ne dépendant d'aucune variable, alors $P$ est un terme.
    \item Si $P$ est un prédicat dépendant des variables $a_1 \dots a_N$, alors $P a_1 \dots a_N$, aussi noté $P (a_1 \dots a_N)$, est un terme. 
    \item En présence de l'égalité, si $x$ et $y$ sont deux variables, alors $x = y$ est un terme.
\end{itemize}

\subsubsection{Parenthèses, symboles $($, $)$, $[$, $]$}

Si $f$ est une formule, alors $(f)$ et $[f]$ sont deux formule équivalentes à $f$. 
Nous omettrons parfois les parenthèses lorsque qu'il n'y a pas d'ambiguité sur la manière dont elles peuvent être incluses, ou lorsque les différentes manières de les inclure donnent des formules équivalentes. 

L'écriture d'une formule en terme de sous-formules contient toujours des arenthèses implicites. 
Ainsi, si les symboles $f$ et $g$ désignent deux formules, si $\mathrm{C_u}$ est un connecteur unaire et $\mathrm{C_b}$ un connecteur binaire, alors la notation $\mathop{\mathrm{C_u}} f$ désigne $\mathop{\mathrm{C_u}} (f)$ et $f \mathop{\mathrm{C_b}} g$ désigne $(f) \mathop{\mathrm{C_b}} (g)$.

\subsubsection{Formules}

Les formules sont définies de la manière suivante : 
\begin{itemize}
    \item Tout terme est une formule. 
    \item Si $x$ est une variable et $f$ une formule dans laquelle $x$ n'est pas quantifiée, alors $\exists x \, (f)$ et $\forall x \, (f)$ sont des formules. 
        On les notera parfois respectivement $\exists x, \, f$ et $\forall x, \, f$ pour plus de lisibilité.
    \item D'autres formules sont construites à l'aide des autres symboles logiques : 
    \begin{itemize}
        \item Si $f$ est une formule, alors $\neg (f)$ (et $\sharp (f)$, si on l'admet dans la théorie) sont des formules.
        \item Si $f$ et $g$ sont deux formules telles qu'aucune variable quantifiée dans l'une n'apparaît dans l'autre, alors $(f) \vee (g)$, $(f) \wedge (g)$, $(f) \Rightarrow (g)$, $(f) \Leftarrow (g)$ et $(f) \Leftrightarrow (g)$ sont des formules.
    \end{itemize}
\end{itemize}

Une variable apparaissant dans une formule (aussi dite \textit{paramètre} de la formule) est dite \textit{liée} si elle est quantifiée (\textit{i.e.}, si l'une de ses occurrences est immédiatement précédée d'un quantificateur) et \textit{libre} si elle ne l'est pas.%
\footnote{
    Afin de simplifier les tournures de phrases, on parlera parfois, quand il n'y a pas de confusion possible, simplement de « variables » ou « paramètres » d'une formule pour désigner ses variables libres. 
}
%Cette distinction s'applique indépendamment à chaque occurrence de chaque variable. 
On impose parfois (et on le fera par la suite sauf mention contraire) qu'une même variable ne puisse être quantfiée plus d'une fois dans une même formule.
Si une formule $F$ contient des variables libres $a_1 a_2 \dots$, et si $\alpha_1 \alpha_2 \dots$ sont autant d'éléments définis par une théorie, on note parfois $F \alpha_1 \alpha_2 \dots$ ou $F (\alpha_1 \alpha_2 \dots)$ la formule obtenue à partir de $F$ en remplaçant $a_1 a_2 \dots$ par $\alpha_1 \alpha_2 \dots$.
Comme annoncé ci-dessus, à chaque formule correspond une unique valeur de vérité, vraie ou fausse. 
Ainsi, une formule non vraie est fausse, une formule vraie est non fausse, une formule fausse est non vraie et une formule non fausse est vraie.

Une formule peut être représentée par un symbole non logique. 
Ce lien peut être noté par le dit symbole suivi de « $:$ » puis de la dite formule ; on dira de ce lien qu'il \textit{définit} le symbole non logique, qui peut alors être employé comme un terme, avec la valeur de vérité associée à la formule qui lui est liée. 
Une formule ne peut contenir de symbole non logique qui ne soit précédemment défini. 

Parfois, une virgule « $,$ » est utilisée pour séparer deux parties d'une formule et la rendre plus lisible, sans en modifier le sens. 
Chaque partie d'une formule ainsi définie doit être une formule à part entière. 

Une formule faisant partie d'une autre formule est dite \textit{sous-formule}.

\medskip

\noindent\textbf{NB :} Un prédicat ne peut référer à un prédicat que si ce dernier est déjà défini. En particulier, il ne peut référer à lui-même, sans quoi on arrive vite à des paradoxes. (Par exemple, si on pouvait definir in prédicat $P$ par $P: \neg P$, alors il serait vrai s'il est faux et faux s'il est vrai.)

\subsubsection{Formule à nombre non spécifié de paramètres}

Il est parfois utile de considérer des formules avec un nombre non spécifié de variables. 
Celles-ci peuvent alors être collectivement désignés par une suite de symboles séparés de points de suspensions, par exemple $a_1 \cdots a_p$. 
Notons formellement $S$ cette séquence.
Les notations $\forall S$ et $\exists S$ désignent, respectivement, les séquences de quantification universelles et existentielles pour chacune des variables. 
Ainsi, 
\begin{itemize}[nosep]
    \item Si la séquence $S$ est vide, \textit{i.e.} ne contient aucune variable, alors $\forall S$ et $\exists S$ ne représentent rien : si $f$ est une formule, $\forall S \, f$ et $\exists \, f$ représentent simplement $f$.
    \item Si $S = a$ où $a$ est une variable, $\forall S$ représente $\forall a$ et $\exists S$ représente $\exists a$. 
    \item Si $S = a b$ où $a$ et $b$ sont deux variables, $\forall S$ représente $\forall a \forall b$ et $\exists S$ représente $\exists a \exists b$. 
    \item Si $S = a_1 a_2 \cdots a_p$ où $a_1$, $a_2$, ..., $a_p$ sont des variables, $\forall S$ représente $\forall a_1 \forall a_2 \dots \forall a_p$ et $\exists S$ représente $\exists a_1 \exists a_2 \dots \exists a_p$. 
\end{itemize}

\subsubsection{Quantificateur d'unicité}

En logique du premier ordre avec égalité, on définit le quantificateur $\exists !$ de la manière suivante : si $P$ est un prédicat à un paramètre libre $x$ et d'éventuels autres paramètres dénotés par $a_1 \dots a_p$, la formule $\exists ! x \, P x a_1 \dots a_p$ est équivalente à $(\exists x \, P x a_1 \dots a_p) \wedge (\forall x \forall y \, (P x a_1 \dots a_p \wedge P y a_1 \dots a_p) \Rightarrow (x=y))$.
    
Moins formellement, on définit l'unicité de la manière suivante : dans le cadre d'une théorie définie en logique du premier ordre avec égalité, si $P$ est un prédicat à un paramètre libre, on dira qu'\textit{il existe au plus un unique objet satisfaisant $P$} si et seulement si le prédicat suivant est vrai : 
\begin{equation*}
    \forall x \, \forall y \, (P(x) \wedge P(y)) \Rightarrow (x = y).
\end{equation*}
On dira qu'\textit{il existe exactement un objet satisfaisant $P$} si et seulement si le prédicat suivant est vrai :
\begin{equation*}
    (\forall x \, \forall y \, (P(x) \wedge P(y)) \Rightarrow (x = y)) \wedge (\exists x \, P(x)).
\end{equation*}
Ce dernier pourra être abrégé en : 
\begin{equation*}
    \exists! x \, P(x).
\end{equation*}

\subsubsection{Sémantique}

Les règles énoncées ci-dessus, complétées par des règles propres à chaque théorie, permettent (au moins dans certains cas) d'attribuer une \textit{valeur de vérité} à une formule. Les parenthèses $($ et $)$ (ou $[$ et $]$), indiquent que, pour évaluer la valeur d'une formule (vraie ou fausse), la formule délimitée par la premiere (à gauche) et la seconde (à droite) est évaluée en tant que formule indépendante. 
Si une formule est construite à partir d'autres formules, sa valeur peut dépendre des leurs, et peut être explicitée par une table de vérité (voir ci-dessous). 

Cinq autres règles sont :
\begin{itemize}
    \item Les variables n'ont pas de sens intrinsèque. 
    Ainsi, si $f$ est une formule faisant intervenir une variable $x$, et si $y$ est une variable n'apparaissant pas dans $f$, alors remplacer toutes les occurrences de $x$ par $y$ dans $f$ ne peut modifier sa valeur de vérité : la formule ainsi obtenue est équivalente à $f$.
    On considèrera parfois que la formule obtenue est la même (ou que les deux séquences de symboles représentent la même formule). 
    \item Si $f$ est une formule et $x$ et $y$ deux variables qui ne sont pas quantifiées dans $f$, alors les formules $\forall x \, \forall y \, f$ et $\forall y \, \forall x \, f$ sont équivalentes.
    \item La valeur de vérité d'une formule est inchangée par le remplacement d'une sous-formule par une formule équivalente.
    \item Si une formule peut s'écrire comme une séquence de sous-formules et de connecteurs telle qu'elle prend toujours la même valeur de vérité lorsque ces sous-formules sont remplacées indépendamment par $\mathsf{V}$ ou par $\mathsf{F}$, alors elle prend cette valeur de vérité, et est équivalente à $\mathsf{V}$ si vraie ou à $\mathsf{F}$ si fausse.
\end{itemize}

On omet parois les parenthèses dans une formule lorsque celles-ci ne modifie pas sa valeur de vérité ; l'ordre d'évaluation des differents termes d'une formule est alors déterminé par les règles suivantes : 
\begin{itemize}
    \item L'évaluation s'effectue de gauche à droite sauf si cela est contraire à une des règles ci-dessous.
    \item Les prédicats sont évalués en premier.
    \item Lorsqu'une parenthèse ouvrante est atteinte, la formule se trouvant entre elle et la parenthèse fermante correspondante est évaluée en priorité.
    \item Ordre d'évaluation des connecteurs et quantificateurs : d'abords les quantificateurs $\exists$ et $\forall$, puis $\neg$, puis (en présence de l'égalité) $=$, puis $\wedge$ et $\vee$ (avec la même priorité), puis $\Rightarrow$, $\Leftarrow$ et $\Leftrightarrow$ (avec la même priorité).
\end{itemize}

Un connecteur binaire $\mathrm{C}$ est dit \textit{transitif} si, pour toutes formules $f$, $g$ et $h$, les formules $(f \mathop{\mathrm{C}} g) \mathop{\mathrm{C}} h$ et $\mathop{\mathrm{C}} (g \mathop{\mathrm{C}} h)$ sont équivalentes.
Un connecteur binaire $\mathrm{C}$ est dit \textit{symmétrique} si, pour toutes formules $f$ et $g$, les formules $f \mathop{\mathrm{C}} g$ et $g \mathop{\mathrm{C}} f$ sont équivalentes.

Dans la suite, si $\mathrm{C}$ désigne un connecteur transitif et si $f$, $g$ et $h$ sont trois formules, on omettra parfois les parenthèses dans des formules de la forme $(f \mathop{\mathrm{C}} g) \mathop{\mathrm{C}} h$ ou $f \mathop{\mathrm{C}} (g \mathop{\mathrm{C}} h)$. 
Plus généralement, on omettra parfois les parenthèses lorsque toutes les manières d'ajouter des parenthèses pour obtenir une formule correctement formée donnent des formules équivalentes.

Si $f$ est une formule et $x$ une variable n'apparaissant pas comme variable liée dans $f$, la formule $\exists x f$ est vraie s'il existe au moins une valeur possible pour $x$ telle que la formule obtenue en remplaçant $x$ par cette valeur dans $f$ est vraie, et fausse si toutes les formules obtenues en remplaçant $x$ par chacune de ses valeurs possible sont fausses. 
Sous les mêmes conditions, la formule $\forall x f$ est fausse s'il existe au moins une valeur possible pour $x$ telle que la formule obtenue en remplaçant $x$ par cette valeur dans $f$ est fausse, et vraie si toutes les formules obtenues en remplaçant $x$ par chacune de ses valeurs possible sont vraies. 
On formalise cela par les règles suivantes : 
\begin{itemize}
    \item si $x$ est une variable et $f$ une formule dans laquelle $x$ n'apparait pas, $\forall x \, f$ est équivalente à $f$ ; 
    \item pour toute variable $x$ et toute formule $f$, la formule $\forall x \, f$ est équivalente à $\neg (\exists x \, \neg f)$ ;
    \item soit $f$ une formule admettant exactement $a_1 a_2 \dots a_n$ pour paramètres libres ; si $\forall a_1 \, \forall a_2 \cdots \forall a_n \, f$ est vraie, alors $f$ est équivalente à $\mathsf{V}$ ; 
    \item en présence de l'égalité, si $f(x)$ est une formule à un paramètre libre éventuel $x$ et $a$ un objet, alors $\exists x \, (x = a) \wedge f(x)$ est équivalente à $f(a)$.
\end{itemize}

Ainsi, par exemple, si $f$ est une formule et $x$ une variable, la formule $\forall x \, (f \Leftrightarrow f)$ est vraie. 
En effet, 
\begin{itemize}[nosep]
    \item la formule $f \Leftrightarrow f$ est vraie que $f$ soit vraie ou fausse, donc elle est équivalente à $\mathsf{V}$,
    \item la formule $\forall x \, (f \Leftrightarrow f)$ est donc éuivalente à $\forall x \, \mathsf{V}$, donc à $\mathsf{V}$, et donc vraie.
\end{itemize}

Quelques conséquences immédiates sont (en remplaçant $f$ par $\neg f$ et en notant que $\neg (\neg f))$ est équivalente à $f$ pour toute formule $f$) : 
\begin{itemize}
    \item Si $f$ est une formule et $x$ et $y$ deux variables qui ne sont pas quantifiées dans $f$, alors les formules $\exists x \, \exists y \, f$ et $\exists y \, \exists x \, f$ sont équivalentes.
    \item si $x$ est une variable, alors $\exists x \, \mathsf{F}$ est fausse (en effet, sa négation est $\forall x \, \mathsf{V}$, qui est vraie) et $\exists x \, \mathsf{V}$ est vraie (en effet, sa négation est $\forall x \, \mathsf{F}$, qui est fausse) ; 
    \item soit $f$ une formule admettant exactement $a_1 a_2 \dots a_n$ pour paramètres libres ; si $\exists a_1 \, \exists a_2 \cdots \exists a_n \, f$ est fausse, alors $f$ est équivalente à $\mathsf{F}$ ; 
    \item soit $f$ et $g$ deux formules à un paramètre libre ; les formules $(\forall x \, f(x)) \wedge (\forall y \, g(y))$ et $\forall x \, (f(x) \wedge g(x))$ sont équivalentes~\footnote{
            En effet, 
            \begin{itemize}[nosep]
                \item Si $(\forall x \, f(x)) \wedge (\forall y \, g(y))$ est vraie, alors $\forall x \, f(x)$ et $\forall y \, g(y)$ sont vraies, donc $f$ et $g$ sont équivalentes à $\mathsf{V}$, donc $f(x) \wedge g(x)$ également, donc $\forall x \, f(x) \wedge g(x)$ est vraie.
                \item Si $(\forall x \, f(x)) \wedge (\forall y \, g(y))$ est fausse, alors $\forall x \, f(x) \wedge g(x)$ doit être fausse. 
                    En effet, si elle était vraie, alors $f(x) \wedge g(x)$ serait équivalente à $\mathsf{V}$, donc $f$ et $g$ également, et donc $(\forall x \, f(x)) \wedge (\forall y \, g(y))$ serait vraie.
            \end{itemize}
        } ; 
    \item soit $f$ et $g$ deux formules à un paramètre libre ; si $\forall x \, f(x)$ est vraie, alors les formules $\forall x \, (f(x) \wedge g(x))$ et $\forall x \, g(x)$ sont équivalentes ; 
    \item soit $f$ et $g$ deux formules à un paramètre libre ; si $\exists x \, f(x)$ est fausse, alors les formules $\forall x \, (f(x) \vee g(x))$ et $\forall x \, g(x)$ sont équivalentes (en effet, $\forall x \, \neg f(x)$ est alors vraie, donc $f$ est équivalente à $\mathsf{F}$, et donc $f(x) \vee g(x)$ à $g(x)$) ; 
    \item soit $f$ et $g$ deux formules à un paramètre libre ; si $\exists x \, f(x)$ est fausse, alors la formule $\forall x \, (f(x) \wedge g(x))$ est fausse ; 
    \item soit $f$ et $g$ deux formules à un paramètre libre ; si $\forall x \, f(x)$ est vraie, alors la formule $\forall x \, (f(x) \vee g(x))$ est vraie ; 
    \item soit $f$ une formule à un paramètre libre ; si $\forall x \, f(x)$ est vraie, alors la formule $\exists x \, f(x)$ est vraie ; 
    \item si $x$ est une variable et $f$ une formule dans laquelle $x$ n'apparait pas, $\exists x \, f$ est équivalente à $f$ (en effet, $x$ n'apparait pas dans $f$, donc $\forall x \, \neg f$ est équivalente à $\neg f$, donc $\neg (\forall x \, \neg f)$ est équivalente à $f$, et donc $\exists x \, f$ à $f$) ; 
    \item pour toute variable $x$ et toute formule $f$ dans laquelle $x$ n'est pas une variable quantifiée, la formule $\exists x \, f$ est équivalente à $\neg (\forall x \, \neg f)$.
    \item soit $f$ et $g$ deux formules à un paramètre libre ; les formules $(\exists x \, f(x)) \vee (\exists y \, g(y))$ et $\exists x \, (f(x) \vee g(x))$ sont équivalentes ;
    \item soit $f$ et $g$ deux formules et $x$ une variable ; si $\forall x \, f$ et $\forall x \, (f \Rightarrow g)$ sont vraies, alors $\forall x \, g$ est vraie (puisqu'alors $\forall x \, (f \wedge (f \Rightarrow g))$ est vraie) ; 
    \item soit $x$ une variable et $f$ et $g$ deux formules (faisant ou non intervenir $x$) ; si $\forall x \, f$ et $\exists x \, (f \Rightarrow g)$ sont vraies, alors $\exists x \, g$ est vraie (en effet, $\forall x \, \neg (f \Rightarrow g)$ est fausse, donc $\forall x \, (f \wedge \neg g)$ est fausse, donc $(\forall y \, f) \wedge (\forall x \neg g)$ est fausse ; puisque $\forall y \, f$ est vraie, on en déduit que $\forall x \, \neg g$ est fausse, et donc que $\exists x \, g$ est vraie) ; 
    \item soit $x$ une variable et $f$ et $g$ deux formules (faisant ou non intervenir $x$) ; si $\exists x \, f$ et $\forall x \, (f \Rightarrow g)$ sont vraies, alors $\exists x \, g$ est vraie (en effet, $\forall x \, (g \vee \neg f)$ est vraie, donc, si $\exists x \, g$ était fausse, on aurait $\forall x \, ((g \vee \neg f) \wedge (\neg g))$, donc $\forall x \, \neg f$, ce qui n'est pas le cas puisque $\exists x \, f(x)$ est vraie). 
\end{itemize}

\textit{Stricto sensu}, il est donc possible de se passer d'un de ces deux quantificateurs, ou de voir l'un d'eux comme fondamental et l'autre comme dérivé. 
Par exemple, on peut voir le quantificateur $\exists$ comme le seul quantificateur fondamental, et définir $\forall$ \textit{via} l'équivalence de $\forall x \, f$ et $\neg \left( \exists x \, \neg f \right)$ pour toute variable $x$ et toute formule $f$. 

\medskip

\noindent\textbf{Attention :} 
    Une formule vraie (au sens où sa valeur de vérité est « vrai ») n'est pas nécessairement équivalente à $\mathsf{V}$.
    De même, une formule faussee (au sens où sa valeur de vérité est « faux ») n'est pas nécessairement équivalente à $\mathsf{F}$.
    Par contre, une formule équivalente à $\mathsf{V}$ est nécessairement vraie et une formule équivalente à $\mathsf{F}$ nécessairement fausse.

\subsubsection{Relations binaires} 

Une théorie définie dans le cadre de la logique du premier ordre peut inclure des relations binaires entre les objets de son domaine de discours, chacune étant représentée par un symbole. 
Si $x$ et $y$ sont deux variables, et $R$ le symbole dénotant une relation binaire, alors $x \mathrel{R} y$ est un terme. 
L'égalité est un exemple de reation binaire, avec pour symbole $=$. 

Soit $P$ un prédicat dépendant de deux variables. 
On peut définir une relation binaire $R$ par la formule 
\begin{equation*}
    \forall x \forall y \, ((x \mathrel{R} y) \Leftrightarrow P x y),
\end{equation*}
ignifiant que, pour chaque $x$ et chaque $y$, $x \mathrel{R} y$ est vrai si et seulement si $P x y$ est vrai. 
Autrement dit, cette formule signiie que les prédicats $P x y$ et $x \mathrel{R} y$ sont équivalents.

Lors de l'évaluation d'une formule, et sauf mention contraire, les relations binaires autres que l'égalité sont prioritaires sur cette dernière, mais pas sur le connecteur $\neq$.

\subsubsection{Réciproque}

Soit $f$ et $g$ deux formules n'ayant pas de quantificateur et $P: f \Rightarrow g$. 
On suppose que le connecteur reliant $f$ et $g$ peut être évalué en dernier.
La \textit{réciproque} de $P$ est la formule $g \Rightarrow f$. 

Plus généralement, on définit la réciproque d'une formule formée de variables quantifiées et d'une formule de cette forme par celle obtenue en prenant la contraposée de cette dernière : si $Q$ est une séquence de variables quantifiées (de la forme $\forall a_1 \dots \forall a_n \exists b_1 \dots \exists b_m \dots$, où les formules $\forall a_1 \dots \forall a_n$ et $\forall b_1 \dots \forall b_m$ sont comprises comme pouvant contenir chacune, et indépendamment, aucune, une seule, ou plusieurs variables quantifiées), la réciproque de la formule $Q \, f \rightarrow q$ est $Q \, g \Rightarrow f$. 

\subsubsection{Contraposée}

Soit $f$ et $g$ deux formules n'ayant pas de quantificateur et $P: f \Rightarrow g$. 
On suppose que le connecteur reliant $f$ et $g$ peut être évalué en dernier.
La \textit{contraposée} de $P$ est la formule $\neg g \Rightarrow \neg f$. 
La formule $P$ et sa contraposée ont toujours la même valeur de vérité (elles sont vraies si $f$ est fausse ou $g$ est vraie et fausses sinon). 

Plus généralement, on définit la contraposée d'une formule formée de variables quantifiées et d'une formule de cette forme par celle obtenue en prenant la contraposée de cette dernière : si $Q$ est une séquence de variables quantifiées (de la forme $\forall a_1 \dots \forall a_n \exists b_1 \dots \exists b_m \dots$, où les formules $\forall a_1 \dots \forall a_n$ et $\forall b_1 \dots \forall b_m$ sont comprises comme pouvant contenir chacune, et indépendamment, aucune, une seule, ou plusieurs variables quantifiées), la contraposée de la formule $Q f \rightarrow q$ est $Q (\neg g \Rightarrow \neg f)$. 
La contraposée d'une formule a toujours la même valeur de vérité que la formule initiale.

\subsubsection{NAND et NOR}

Notons que chacun des connecteurs peut être construit à l'aide d'un unique connecteur, que l'on  note ici $\circ$, appelé \textit{NAND}, définit de la manière suivante : si $f$ et $g$ sont deux formules, alors $f \circ g$ est une formule, vraie si et seulement si $f$ et $g$ ne sont pas toutes deux vraies. 
En effet, si $f$ et $g$ sont deux formules, et en considérant que deux formules sont équivalentes si elles prennent toujours la même valeur,
\begin{itemize}
    \item $\neg f$ est équivalente à $f \circ f$,
    \item $f \wedge g$ est équivalente à $\neg (f \circ g)$,
    \item $f \vee g$ est équivalente à $(\neg f) \circ (\neg g)$,
    \item $f \Rightarrow g$ est équivalente à $(\neg f) \vee g$,
    \item $f \Leftarrow g$ est équivalente à $f \vee (\neg g)$,
    \item $f \Leftrightarrow g$ est équivalente à $(f \wedge g) \vee ((\neg f) \wedge (\neg g))$.
\end{itemize}
Un tel connecteur, permettant de construire tous les autres, est dit \textit{universel}.

Il existe un autre connecteur universel, appelé \textit{NOR}, que l'on note dans ce paragraphe $\times$, défini par : si $f$ et $g$ sont deux formules, alors $f \circ g$ est une formule, vraie si et seulement si $f$ et $g$ sont toutes deux fausses. 
En effet, si $f$ et $g$ sont deux formules, $\neg f$ est équivalente à $f \times f$ et $f \wedge g$ à $(\neg f) \times (\neg g)$, donc $f \circ g$ est équivalente à $[(f \times f) \times (g \times g)] \times [(f \times f) \times (g \times g)]$. 
Puisque le connecteur $\circ$ est universel, le connecteur $\times$ l'est donc aussi.

\subsubsection{XOR} 
\label{subsub:XOR}

On définit le connecteur \textit{XOR}, noté $\oplus$, de la manière suivante : si $f$ et $g$ sont deux formules, alors $f \oplus g$ est une formule vraie si $f$ est vraie et $g$ est fausse ou si $f$ est fausse et $g$ est vraie, et fausse sinon. 
Si $f$ et $g$ sont deux formules, alors $f \oplus g$ est équivalente à $f \Leftrightarrow (\neg g)$. 

L'utilité du connecteur XOR découle des trois propriétés suivantes : 
\begin{itemize}[nosep]
    \item Il est \textit{symmétrique} : si $f$ et $g$ sont deux formules, $f \oplus g$ est équivalente à $g \oplus f$ (en effet, toutes deux sont vraies si une des formules $f$ et $g$ est vraie et l'autre est fausse, et fausse sinon).
    \item Il est \textit{transitif} : si $f$, $g$ et $h$ sont trois formules, $(f \oplus g) \oplus h$ est équivalente à $f \oplus (g \oplus h)$ (en effet, toutes deux sont vraies soit si les trois formules $f$, $g$ et $h$ sont vraies ou si une d'entre elles est vraie et les deux autres sont fausses, et fausses sinon).
    \item Soit $f$ une formule, $f \oplus f$ est toujours fausse.
\end{itemize}
Notons aussi que, si $f$ est une formule, $f \oplus \mathsf{F}$ est équivalente à $f$ et $f \oplus \mathsf{V}$ à $\neg f$.

\subsubsection{Tables de vérité}

Les valeurs de formules construites à partir d'autres formules peuvent être consignées dans des tableaux appelés \textit{tables de vérité}, contenant sur la première ligne plusieurs formules et sur les autres leurs valeurs (un tiret indiquant qu'elle peut prendre la valeur vraie ou fausse). 
En voici un exemple, pour deux formules $f$ et $g$ : 

\begin{center}
\begin{tabular}{c c | c c c c c c}
    $f$ & $g$ & $\neg f$ & $f \wedge g$ & $f \vee g$ & $f \Rightarrow g$ & $f \Leftarrow g$ & $f \Leftrightarrow g$ 
    \\ \hline 
    $\mathsf{F}$ & $\mathsf{F}$ & $\mathsf{V}$ & $\mathsf{F}$ & $\mathsf{F}$ & $\mathsf{V}$ & $\mathsf{V}$ & $\mathsf{V}$
    \\
    $\mathsf{F}$ & $\mathsf{V}$ & $\mathsf{V}$ & $\mathsf{F}$ & $\mathsf{V}$ & $\mathsf{V}$ & $\mathsf{F}$ & $\mathsf{F}$
    \\ 
    $\mathsf{V}$ & $\mathsf{F}$ & $\mathsf{F}$ & $\mathsf{F}$ & $\mathsf{V}$ & $\mathsf{F}$ & $\mathsf{V}$ & $\mathsf{F}$ 
    \\ 
    $\mathsf{V}$ & $\mathsf{V}$ & $\mathsf{F}$ & $\mathsf{V}$ & $\mathsf{V}$ & $\mathsf{V}$ & $\mathsf{V}$ & $\mathsf{V}$ 
\end{tabular}
\end{center}

On peut utiliser des tables de vérités pour montrer l'équivalence entre plusieurs formules. 
Montrons par exemple les trois propriétés énoncées section~\ref{subsub:XOR}. 
Pour trois formules $f$, $g$ et $h$, on a :
\begin{center}
\begin{tabular}{c c c | c c c c c}
    $f$ & $g$ & $h$ & $f \oplus g$ & $g \oplus f$ & $(f \oplus g) \oplus h$ & $f \oplus (g \oplus h)$ & $f \oplus f$ 
    \\ \hline 
    $\mathsf{F}$ & $\mathsf{F}$ & $\mathsf{F}$ & $\mathsf{F}$ & $\mathsf{F}$ & $\mathsf{F}$ & $\mathsf{F}$ & $\mathsf{F}$
    \\
    $\mathsf{F}$ & $\mathsf{F}$ & $\mathsf{V}$ & $\mathsf{F}$ & $\mathsf{F}$ & $\mathsf{V}$ & $\mathsf{V}$ & $\mathsf{F}$
    \\
    $\mathsf{F}$ & $\mathsf{V}$ & $\mathsf{F}$ & $\mathsf{V}$ & $\mathsf{V}$ & $\mathsf{V}$ & $\mathsf{V}$ & $\mathsf{F}$
    \\
    $\mathsf{F}$ & $\mathsf{V}$ & $\mathsf{V}$ & $\mathsf{V}$ & $\mathsf{V}$ & $\mathsf{F}$ & $\mathsf{F}$ & $\mathsf{F}$
    \\
    $\mathsf{V}$ & $\mathsf{F}$ & $\mathsf{F}$ & $\mathsf{V}$ & $\mathsf{V}$ & $\mathsf{V}$ & $\mathsf{V}$ & $\mathsf{F}$
    \\
    $\mathsf{V}$ & $\mathsf{F}$ & $\mathsf{V}$ & $\mathsf{V}$ & $\mathsf{V}$ & $\mathsf{F}$ & $\mathsf{F}$ & $\mathsf{F}$
    \\
    $\mathsf{V}$ & $\mathsf{V}$ & $\mathsf{F}$ & $\mathsf{F}$ & $\mathsf{F}$ & $\mathsf{F}$ & $\mathsf{F}$ & $\mathsf{F}$
    \\
    $\mathsf{V}$ & $\mathsf{V}$ & $\mathsf{V}$ & $\mathsf{F}$ & $\mathsf{F}$ & $\mathsf{V}$ & $\mathsf{V}$ & $\mathsf{F}$
    \\
\end{tabular}
\end{center}
On remarque, comme attendu, que
\begin{itemize}[nosep]
    \item Les formules $f \oplus g$ et $g \oplus f$ prennent toujours la même valeur.
    \item Les formules $(f \oplus g) \oplus h$ et $f \oplus (g \oplus h)$ prennent toujours la même valeur.
    \item La formule $f \oplus f$ est toujours fausse.
\end{itemize}

\subsubsection{Quelques propriétés}

Les propriétés suivantes peuvent être facilement démontrées en écrivant les tables de vérités correspondantes :
\begin{itemize}
    \item Soit $f$ une formule. La formule $f \wedge \mathsf{F}$ est toujours fausse et $f \vee \mathsf{V}$ est toujours vraie.
    \item Soit $f$ une formule. Les formule $f \wedge \mathsf{V}$, $f \vee \mathsf{F}$, $f \wedge f$, $f \vee f$ et $f \Leftrightarrow \mathsf{V}$ ont la même valeur de vérité que $f$.
    \item Le connecteur $\wedge$ est symmétrique : Soit $f$ et $g$ deux formules ; si $f \wedge g$ est vraie, alors $f$ et $g$ sont toutes deux vraies, donc $g \wedge f$ l'est également.
    \item Le connecteur $\wedge$ est transitif : Soit $f$, $g$ et $h$ trois formules, $f \wedge (g \wedge h)$ a la même valeur de vérité que $(f \wedge g) \wedge h$. En effet, toute deux sont vraies si et seulement si $f$, $g$ et $h$ sont toutes trois vraies.
    \item Soit $f$, $g$ et $h$ trois formules ; si $f \wedge g$ et $g \wedge h$ sont vraies, alors $f \wedge h$ l'est également.
    \item Le connecteur $\vee$ est symmétrique : Soit $f$ et $g$ deux formules ; si $f \vee g$ est vraie, alors au moins une des deux formules $f$ et $g$ est vraie, donc $g \vee f$ l'est également.
    \item Le connecteur $\vee$ est transitif : Soit $f$, $g$ et $h$ trois formules, $f \vee (g \vee h)$ a la même valeur de vérité que $(f \vee g) \vee h$. En effet, toutes deux son vraies si et seulement si au moins une des deux formules $f$, $g$ et $h$ est vraie.
    \item Le connecteur $\Leftrightarrow$ est symmétrique : Soit $f$ et $g$ deux formules ; si $f \Leftrightarrow g$ est vraie, alors $g \Leftrightarrow f$ l'est également.
    \item Le connecteur $\Leftrightarrow$ est transitif : Soit $f$, $g$ et $h$ trois formules, $f \Leftrightarrow (g \Leftrightarrow h)$ a la même valeur de vérité que $(f \Leftrightarrow g) \Leftrightarrow h$. En effet, toutes deux son vraies si et seulement si les trois formules $f$, $g$ et $h$ ont la même valeur de vérité.
    \item Soit $f$, $g$ et $h$ trois formules. 
        \begin{itemize}[nosep]
            \item Si $f \Leftrightarrow g$ et $g \Leftrightarrow h$ sont vraies, alors $f \Leftrightarrow h$ l'est également. 
            \item Si $f \Rightarrow g$ et $g \Rightarrow h$ sont vraies, alors $f \Rightarrow h$ l'est également. 
            \item Si $f \Leftarrow g$ et $g \Leftarrow h$ sont vraies, alors $f \Leftarrow h$ l'est également. 
        \end{itemize}
    \item Soit $f$ et $g$ deux formules. Alors, $\neg (f \wedge g)$ a la même valeur de vérité que $(\neg f) \vee (\neg g)$. 
        En effet, toutes deux sont vraies si au moins une des formules $f$ et $g$ est fausse, et fausses sinon.
    \item Soit $f$ et $g$ deux formules. Alors, $\neg (f \vee g)$ a la même valeur de vérité que $(\neg f) \wedge (\neg g)$. 
        En effet, toutes deux sont vraies si les deux formules $f$ et $g$ sont fausses, et fausses sinon.
    \item Soit $f$ et $g$ deux formules. Si $f \Leftrightarrow g$ est vraie, alors $\neg f \Leftrightarrow \neg g$ l'est aussi.
    \item Soit $f$ et $g$ deux formules ; la formule $f \Leftrightarrow g$ est équivalente à $(f \Rightarrow g) \wedge (g \rightarrow f)$. 
    \item Le connecteur $\wedge$ est distributif sur $\vee$ : si $f$, $g$ et $h$ sont trois formules, les deux formules $f \wedge (g \vee h)$ et $(f \wedge g) \vee (f \wedge h)$ ont la même valeur de vérité (toutes deux sont vraies si et seulement si $f$ ainsi qu'au moins une des deux formules $g$ et $h$ sont vraies). 
    \item Le connecteur $\vee$ est distributif sur $\wedge$ : si $f$, $g$ et $h$ sont trois formules, les deux formules $f \vee (g \wedge h)$ et $(f \vee g) \wedge (f \vee h)$ ont la même valeur de vérité (toutes deux sont vraies si $f$ est vraie ou si $g$ et $h$ sont toutes deux vraies et fausses sinon).
    \item Soit $f$ et $g$ deux formules. Si $f \Rightarrow g$, alors $f \wedge g$ est équivalente à $f$ et $f \vee g$ est équivalente à $g$.
    \item Soit $f$ et $g$ deux formules. Alors $f \Leftrightarrow g$ et $(\neq f) \Leftrightarrow (\neg g)$ sont équivalentes. 
        (Elles sont toutes deux vraies si $f$ et $g$ ont la même valeur de vérité et fausses sinon.)
    \item Soit $f$, $g$, $h$ et $i$ quatre formules. 
        Si $f \Rightarrow g$ et $h \Rightarrow i$ sont vraies, alors $(f \wedge h) \Rightarrow (g \wedge i)$ et $(f \vee h) \Rightarrow (g \vee i)$ sont vraies.
    \item Une conséquence de ces deux derniers points est que, avec les notations du second, si $f \Leftrightarrow g$ et $h \Leftrightarrow i$ sont vraies, alors $(f \wedge \neg h) \Leftrightarrow (g \wedge \neg i)$ est vraie.
\end{itemize}

\noindent \textbf{Attention :} Si $f$, $g$ et $h$ sont trois formules, savoir que $f \vee g$ et $g \vee h$ sont vraies n'implique pas que $f \vee h$ l'est également. 
(En effet, si $f$ et $h$ sont fausse alors que $g$ est vraie, les deux premières sont vraies mais la troisième est fausse.)

\subsubsection{Valeur de vérité Indéfinie}
\label{subsub:Indéfinie}

On peut étendre la logique du premier ordre en posant une troisième valeur de vérité, dite \textit{indéfinie}. 
La constante de vérité correspondante est notée $\mathsf{I}$. 
Toute formule est alors associée à une (et une seule) des trois valeurs de vérité vraie, fausse ou indéfinie. 

La table de vérité suivante donne les valeurs de formules obtenues à partir de deux formules $f$ et $g$ ainsi que d'un connecteur :
\begin{center}
\begin{tabular}{c c | c c c c c c}
    $f$          & $g$          & $\neg f$     & $f \wedge g$ & $f \vee g$ & $f \Rightarrow g$ & $f \Leftarrow g$ & $f \Leftrightarrow g$ 
    \\ \hline 
    $\mathsf{F}$ & $\mathsf{F}$ & $\mathsf{V}$ & $\mathsf{F}$ & $\mathsf{F}$ & $\mathsf{V}$    & $\mathsf{V}$     & $\mathsf{V}$
    \\
    $\mathsf{F}$ & $\mathsf{I}$ & $\mathsf{V}$ & $\mathsf{F}$ & $\mathsf{I}$ & $\mathsf{V}$    & $\mathsf{I}$     & $\mathsf{I}$
    \\
    $\mathsf{F}$ & $\mathsf{V}$ & $\mathsf{V}$ & $\mathsf{F}$ & $\mathsf{V}$ & $\mathsf{V}$    & $\mathsf{F}$     & $\mathsf{F}$
    \\ 
    $\mathsf{I}$ & $\mathsf{F}$ & $\mathsf{I}$ & $\mathsf{F}$ & $\mathsf{I}$ & $\mathsf{I}$    & $\mathsf{V}$     & $\mathsf{I}$ 
    \\ 
    $\mathsf{I}$ & $\mathsf{I}$ & $\mathsf{I}$ & $\mathsf{I}$ & $\mathsf{I}$ & $\mathsf{I}$    & $\mathsf{I}$     & $\mathsf{I}$
    \\
    $\mathsf{I}$ & $\mathsf{V}$ & $\mathsf{I}$ & $\mathsf{I}$ & $\mathsf{V}$ & $\mathsf{V}$    & $\mathsf{I}$     & $\mathsf{I}$ 
    \\
    $\mathsf{V}$ & $\mathsf{F}$ & $\mathsf{F}$ & $\mathsf{F}$ & $\mathsf{V}$ & $\mathsf{F}$    & $\mathsf{V}$     & $\mathsf{F}$ 
    \\ 
    $\mathsf{V}$ & $\mathsf{I}$ & $\mathsf{F}$ & $\mathsf{I}$ & $\mathsf{I}$ & $\mathsf{I}$    & $\mathsf{V}$     & $\mathsf{I}$
    \\
    $\mathsf{V}$ & $\mathsf{V}$ & $\mathsf{F}$ & $\mathsf{V}$ & $\mathsf{V}$ & $\mathsf{V}$    & $\mathsf{V}$     & $\mathsf{V}$ 
\end{tabular}
\end{center}
On a alors les équivalences : 
\begin{itemize}[nosep]
    \item $f \Rightarrow g$ est équivalente à $(\neg f) \vee g$,
    \item $f \Leftarrow g$ est équivalente à $f \vee (\neg g)$,
    \item $f \Leftrightarrow g$ est équivalente à $(f \wedge g) \vee ((\neg f) \wedge (\neg g))$.
\end{itemize}

On a aussi les règles additionelles : 
\begin{itemize}[nosep]
    \item toute formule vraie est équivalente à $\mathsf{V}$,
    \item toute formule fausse est équivalente à $\mathsf{F}$,
    \item toute formule indéfinie est équivalente à $\mathsf{I}$.
\end{itemize}

Si $f(x)$ est une formule dépendant d'un paramètre libre $x$, alors, 
\begin{itemize}[nosep]
    \item si $f(a)$ est vraie pour tout objet $a$ du domaine de a théorie, alors $\forall x \, f(x)$ est vraie,
    \item si $f(a)$ est vraie ou indéfinie pour tout objet $a$ du domaine de a théorie et qu'il existe au moins un d'entre eux pour lequel $f(a)$ est indéfinie, alors $\forall x \, f(x)$ est indéfinie,
    \item si $f(a)$ est fausse pour au moins un objet $a$ du domaine de a théorie, alors $\forall x \, f(x)$ est fausse.
\end{itemize}
Cela implique (en prenant la négation) : 
\begin{itemize}[nosep]
    \item si $f(a)$ est fausse pour tout objet $a$ du domaine de a théorie, alors $\exists x \, f(x)$ est fausse,
    \item si $f(a)$ est fausse ou indéfinie pour tout objet $a$ du domaine de a théorie et qu'il existe au moins un d'entre eux pour lequel $f(a)$ est indéfinie, alors $\exists x \, f(x)$ est indéfinie,
    \item si $f(a)$ est vraie pour au moins un objet $a$ du domaine de a théorie, alors $\exists x \, f(x)$ est vraie.
\end{itemize}

Le point de vue caonique en logique mathématique est de considérer que les deux seules valeurs de vérité possibles sont « vraie » et « fausse ». 
Un point de vue intermédiaire est de considérer que seules les formules ayant au moins une variable libre peuvent prendre la valeur indéfinie. 
Dans ce qui suit, on tâchera de ne tenir que des raisonnement valables avec ou sans la valeur de vérité indéfinie. 
Sauf mention contraire explicite, on considèrera qu'une formule peut prendre une des trois valeurs de vérité.

\subsubsection{Quelques schémas de raisonnement}

Pour démontrer qu'une formule est vraie, on remplacera souvent certains quantificateurs et connecteurs par des mots ayant la même signification afin de les rendre plus faciles à suivre, en suivant les règles énoncées ci-dessus. 
Nous présentons ici brièvement quelques idées souvent utilisées pour démontrer des formules, de manière informelle. 
On se place dans le cadre d'une théorie comprenant la logique du premier ordre et portant sur un certain domaine de discours définissant des objets.

\medskip

\paragraph{Raisonnement par l'absurde :} 
    Un type de raisonnement revenant souvent est le raisonnement par l'absurde : si $f$ et $g$ sont deux formules, si $f \Rightarrow g$ est vraie et $g$ est fausse, alors $f$ est nécessairement fausse. 
    En pratique, pour montrer qu'une formule $f$ est fausse, on peut donc trouver une formule $g$ telle que $g$ est fausse et $f \Rightarrow g$.

    Plus formellement, si $f$ et $g$ sont deux formules, on a : 
    \begin{equation*}
        \left((f \Rightarrow g) \wedge (\neg g) \right)
        \Leftrightarrow \left( ((\neg f) \vee g) \wedge (\neg g) \right)
        \Leftrightarrow \left( ((\neg f) \wedge (\neg g)) \vee (g \wedge (\neg g)) \right)
        \Leftrightarrow \left( ((\neg f) \wedge (\neg g)) \vee \mathsf{F} \right)
        \Leftrightarrow \left( (\neg f) \wedge (\neg g) \right) .
    \end{equation*}
    Donc, si $(f \Rightarrow g) \wedge (\neg g)$ est vraie, alors $\neg f$ est vraie, donc $f$ est fausse.

\medskip

\paragraph{Prouver une propriété de la forme $\forall x \, P(x) \Rightarrow Q(x)$ :}  
    Soit $P$ et $Q$ deux prédicats à un paramètre libre. 
    Pour prouver que la formule $\forall x \, P(x) \Rightarrow Q(x)$ est vraie, on pourra prendre un objet $x$ pouvant être n'importe-quel objet du domaine de discours de la théorie et montrer que, si $P(x)$ est vrai, alore $Q(x)$ l'est également.

\medskip

\paragraph{Prouver l'unicité d'un objet satisfaisant une propriété en montrant que deux objets la satisfaisant sont égaux :} 
    On se place ici dans le cadre de la logique du premier ordre avec égalité.
    Soit $P$ un prédicat à un paramètre libre. 
    Pour montrer qu'il existe au plus un unique objet $x$ tel que $P(x)$ est satisfait, on pourra montrer que si $x$ et $y$ sont deux objets tels que $P(x)$ et $P(y)$ sont vrais, alors $x = y$.
    Pour montrer qu'il en existe exactement un, on montrera en outre qu'il existe un objet $x$ tel que $P(x)$ est vrai.

\medskip

\paragraph{Équivalence :} 
    Soit $f$ et $g$ deux formules. 
    Si on peut montrer qur $f \Rightarrow g$ et $g \Rightarrow f$ sont vraies, alors $f \Leftrightarrow g$ est vraie.

\subsubsection{Un exemple : arc-en-ciel à minuit ?}

Pour rendre cela un peu plus concret, éxaminons un exemple d'application. 
On se restreint ici à la logique propositionelle, sans variables ni quantificateurs. 
Considérons les prédicats suivants : 
\begin{itemize}[nosep]
    \item $P_1$ : « Le soleil brille. »
    \item $P_2$ : « Il pleut. »
    \item $P_3$ : « Il y a un arc-en-ciel. »
    \item $P_4$ : « Il fait jour. »
    \item $P_5$ : « Il est minuit. »
    \item $P_6$ : « Si le soleil brille, il fait jour. »
    \item $P_7$ : « À minuit, il ne fait pas jour. »
    \item $P_8$ : « Il y a un arc-en-ciel si et seulement si le soleil brille et il pleut. »
\end{itemize}
Alors, 
\begin{itemize}[nosep]
    \item $P_6$ est équivalent à : $P_1 \Rightarrow P_4$.
    \item $P_7$ est équivalent à : $P_5 \Rightarrow \neg P_4$.
    \item $P_8$ est équivalent à : $P_3 \Rightarrow (P_1 \wedge P_2)$.
\end{itemize}

Posons-nous la question : en admettant $P_6$, $P_7$ et $P_8$, peut-il y avoir un arc-en-ciel à minuit ? 
Évidemment, non ! 
En effet, la contraposée de $P_6$ est $\neg  P_4 \Rightarrow \neg P_1$. 
Si $P_7$ et $P_6$ (et donc sa contraposée) sont vrais, alors $(P_5 \Rightarrow \neg P_4) \wedge (\neg  P_4 \Rightarrow \neg P_1)$ est vrai. 
Puisque le connecteur $\Rightarrow$ est transitif, cela implique $P_5 \Rightarrow \neg P_1$. 
Or, la contraposée de $P_8$ est $\neg (P_1 \wedge P_2) \Rightarrow \neg P_3$. 
Si $P_8$ est vrai, sa contraposée l'est aussi. 
Si, de plus, $P_1$ est faux, alors $\neg (P_1 \wedge P_2)$ est vrai, et donc $\neg P_3$ est vrai. 
Donc, si $P_8$ est vrai, $\neg P_1 \Rightarrow \neg P_3$. 
En utilisant une dernière fois la transitivité du connecteur $\Rightarrow$, on obtient donc $P_5 \Rightarrow \neg P_1$ si $P_6$, $P_7$ et $P_8$ sont vrais. 
Cela peut se récrire formellement : 
\begin{equation*}
    P_6 \wedge P_7 \wedge P_8 \Rightarrow (P_5 \Rightarrow \neg P_1). 
\end{equation*}

\subsubsection{Premier théorème d'incomplétude de Gödel}

Les deux théorèmes d'incomplétude de Gödel énoncent, en un certain sens, des limites au pouvoir démontratif d'une théorie mathématique. 
Plus précisément, ils indiquent que, sous certaines hypothèses, tous les prédicats vrais d'une théorie ne peuvent être démontrés comme tels. 
Le premier d'entre eux exprime qu'il n'existe aucun système d'axiomes cohérent dont les résultats peuvent être obtenus par un algorithme permettant de démontrer tous les résultats vrais sur les nombres naturels. 
On peut l'énoncer de manière informelle comme suit : 

\medskip

\noindent \textit{Tout système formel $F$ d'axiomes cohérent permetant de définir une arithmétique élémentaire est incomplet, au sens où il existe des prédicats exprimés dans le language de $F$ dont la valeur de vérité ne peut être démontrée vraie ni fausse à partir de $F$.}

\medskip

\noindent Cet énoncé est imprécis, entre autres puisqu'il ne définit pas ce qu'est une arithmétique élémentaire. 
Pour le préciser, considérons une théorie dont l'alphabet contient les symboles suivants : 
\begin{itemize}[nosep]
    \item Un symbole $0$ représentant une constante. 
    \item Un symbole $x$ représentant une variable, ainsi qu'un symbole $\ast$ permettant de construire d'autres variables $x^\ast$, $x^{\ast\ast}$, $x^{\ast\ast\ast}$, ... ces variables sont dites \emph{primaires}.
    \item Une fonction $S$ d'une variable, appelée \textit{successeur}. 
    \item Deux opérations binaires $+$ (addition) et $\times$ (multiplication).
    \item Les opérateurs logiques de conjonction $\wedge$, disjonction $\vee$ et négation $\neg$.
    \item Deux relations binaires $=$ (égalité) et $<$.
    \item Les parenthèses $($ et $)$.
\end{itemize}

Les formules de la théorie sont des chaines (finies) de symboles, avec les règles suivantes : 
\begin{itemize}[nosep]
    \item Si $y$ désigne une constante, $S y$ est une constante. On notera $1$ la constante $S 0$ et on supposera $1 \neq 0$.
    \item Si $y$ désigne une variable, $S y$ est une variable, dite \emph{secondaire}.
    \item Si $y$ et $z$ sont chacune une constante ou une variable, alors $y = z$ et $y < z$ sont des formules.
    \item Si $f$ est une formule, alors $(f)$ en est une. 
    \item Si $f$ est une formule, alors $\neg f$ en est une. 
    \item Si $f$ et $g$ sont deux formule n'ayant aucune variable primaire quantifiée en commun, alors $f \wedge g$ et $f \vee g$ en sont également. 
    \item Soit $f$ une formule et $v$ une variable primaire telle que ni $\forall v$ ni $\exists v$ n'apparaît dans $f$. 
        Alors $\forall v \, f$ et $\exists v \, f$ sont des formules.
\end{itemize}
Notons que l'arithmétique usuelle satisfait ces propriétés (voir section~\ref{sub:constN}).

Alternativement, on peut se limiter aux formules sans variable libre en remplaçant le règles ci-dessus par les suivantes :
\begin{itemize}[nosep]
    \item Si $y$ désigne une constante, $S y$ est une constante.
    \item Si $y$ et $z$ sont deux constantes, alors $y = z$ et $y < z$ sont des formules.
    \item Si $f$ est une formule, alors $(f)$ en est une. 
    \item Si $f$ est une formule, alors $\neg f$ en est une. 
    \item Si $f$ et $g$ sont deux formule, alors $f \wedge g$ et $f \vee g$ en sont également. 
    \item Soit $f$ une formule, $c$ une constante et $v$ une variable. 
        Soit $g$ la chaine obtenue en remplaçant $c$ par $v$ dans $f$.
        Alors, $\forall v \, g$ et $\exists v \, g$ sont des formules.
\end{itemize}

La théorie est dite \emph{cohérente} si aucune formule ne peut être montrée à la fois vraie et fausse. 
Elle est dite \emph{ω-cohérente} si, pour toute formule $f$ à une variable libre, il est impossible de montrer $\exists n \, f(n)$ si $\neg f(n)$ est démontrable pour toute constante $n$.

Enfin, la théorie est supposée \textit{effective}, c'est-à-dire qu'il est théoriquement possible d'écrire un algorithme ayant un nombre fini d'instructions donnant un par un tous ses axiomes et uniquement ses axiomes. 
On peut donner à cette définition un sens plus précis dans le cadre de l'arithmétique usuelle, et en définissant un ensemble d'instructions qu'un tel algorithme peut avoir. 
Dans la suite, on supposera que la théorie permet de définir un ensemble de nombre $\mathbb{N}$ ayant les mêmes propriétés que celui défini dan sles sections suivantes.

\medskip

\noindent \textbf{Premier théorème d'incomplétude de Gödel :} Sous ces conditions, il existe une formule $f$ de la théorie sans variable libre dont on ne peut montrer (par un algorithme fini) ni qu'elle est vraie ni qu'elle est fausse.

\medskip

L'essence de la preuve est de construire, dans le cadre de cette théorie, un prédicat $G$ équivalent à l'impossibilité de le démontrer lui-même. 
Ainsi, si $G$ est vrai, il n'est pas démontrable, et si $G$ est faux il est démontrable (ce qui est impossible si la théorie est cohérente). 
Dans le langage usuel, de tels énoncés paradoxaux sont aisés à formuler car un énoncé peut référer directement à lui-même. 
Par exemple, la phrase « Cette phrase n'est pas démontrable. » ne peut être démontrée que si elle n'est pas vraie~\footnote{Dans le même ordre d'idée, la phrase « Cette phrase est fausse. » ne peut être ni vraie ni fausse.} 
Pour démontrer le premier théorème d'incomplétude de Gödel, il suffit en quelque sorte de montrer qu'un tel énoncé existe et forme un prédicat dans le cadre de toute théorie satisfaisant les propriétés éoncées ci-dessus.

La démonstration de Gödel repose sur les \textit{nombres de Gödel} associés à chaque formule. 
De manière générale (et une fois une théorie de l'arithmétique construite, voir section~\ref{sec:arithmetique} ; on se limite ici aux entiers naturels), une \textit{numérotation de Gödel} est une fonction injective (voir une définition rigoureuse des fonctions dans le cadre de la théorie des ensembles sera donnée section~\ref{sub:fonctions}) associant un nombre à chaque symbole ou formule. 

La numérotation originelle de Gödel, que nous nommerons dans la suite simplement \textit{encodage de Gödel} par la suite, noté $\mathbf{G}$, est obtenue de la manière suivante : 
\begin{itemize}[nosep]
    \item On choisit une suite de nombres premiers distincts $p$. 
    \item À chaque symbole de la théorie et chaque variable primaire est associé un nombre distinct. 
    \item Si $n$ est un entier naturel et $x_1$, $x_2$, ..., $x_n$ sont des symboles, le nombre associé à la formule $x_1 x_2 \dots x_n$ est $p_1^{x_1} \times p_2^{x_2} \times \cdots p_n^{x_n}$.
        Plus formellement, $\mathrm{G}(x_1 x_2 \dots x_n) = \prod_{i=1}^n p_i^{x_i}$. 
\end{itemize}
Cette fonction est bien injective d'après l'unicité de la décomposition en produits de facteurs premiers (voir section~\ref{subsub:dec_fact_prem}).

\medskip

\noindent \textbf{Exemple :} Si trois variables primaires $x$, $y$ et $z$ sont représentées respectivement par les nombres $1$, $2$ et $3$, si $+$ et $=$ sont respectivement représentée par les nombres $4$ et $5$, et si la suite $p$ commence par $(2, 3, 5, 7, 11)$, alors $\mathrm{G}(x + y = z) = 2^1 \times 3^4 \times 5^2 \times 7^5 \times 11^3 = 90598973850$.

\medskip

Soit $F$ une formule. 
Si $F$ est démontrable, alors il existe une formule $P$ qui prouve $F$. 
On peut ainsi définir la fonction $\mathrm{Dem}$ de $\mathbb{N} \times \mathbb{N}$ vers ${0, 1}$ par, illustrant que « $n$ démontre $m$ » : pour tous entiers naturel $n$ et $m$, 
\begin{itemize}[nosep]
    \item Si $n$ est un nombre de Gödel associé à une formule $P$, $m$ est un nombre de Gödel associé à une formule $F$ et si $P$ démontre $F$, alors $\mathrm{Dem}(n, m) = 1$.
    \item Sinon, $\mathrm{Dem}(n, m) = 0$. 
\end{itemize}
(Cette fonction ne sera pas utilisée dans la suite, mais sert d'illustration.)

On définit la fonction $q$ de $\mathbb{N} \times \mathbb{N}$ vers ${0, 1}$ par : pour tous entiers naturel $n$ et $m$, 
\begin{itemize}[nosep]
    \item Si $n$ est un nombre de Gödel associé à une formule $P$, $m$ est un nombre de Gödel associé à une formule $F$ à un paramètre libre et si $P$ ne démontre pas $F(\mathbf{G}(F))$, alors $q(n, m) = 1$.
    \item Si $n$ est un nombre de Gödel associé à une formule $P$, $m$ est un nombre de Gödel associé à une formule $F$ à un paramètre libre et si $P$ démontre $F(\mathbf{G}(F))$, alors $q(n, m) = 0$.
    \item Sinon, $q(n, m) = 1$. 
\end{itemize}
Alors, pour toute formule $F$ à un paramètre libre, la formule $\forall y \, q(y, \mathbf{G}(F)) = 1$ est équivalente à : « il n'existe pas de preuve de $F(\mathbf{G}(F))$ ».

Définissons le prédicat à un paramètre libre $P$ par : $P(x): \forall y \, q(y, x) = 1$. 
Considérons maintenant le prédicat $Z$ défini par : $Z: P(\mathbf{G}(P))$. 
De manière informelle, $Z$ est équivalent à $\forall y \, q(y, \mathbf{G}(P))$, et donc à « il n'existe pas de preuve de $Z$ ». 
Nous avons donc construit un prédicat vrai si et seulement si il n'est pas démontrable. 

Montrons un peu plus formellement que ni $Z$ est démontrable si et seulement si elle est fausse.
\begin{itemize}[nosep]
    \item Supposons que $Z$ est démontrable. 
        Alors, il existe une formule $F$ démontrant $Z$. 
        Donc, $F$ démontre $P(\mathbf{G}(P))$.
        Donc, $q(\mathbf{G}(F), \mathbf{G}(P)) = 0$. 
        Donc, $q(\mathbf{G}(F), \mathbf{G}(P)) = 1$ est faux. 
        Donc, $\forall y \, q(y,\mathbf{G}(P)) = 1$ est faux.
        Donc, $P(\mathbf{G}(P))$ est faux.
        Donc, $Z$ est faux.
    \item Supposons que $Z$ est faux.
        ALors, $P(\mathbf{G}(P))$ est faux.
        Donc, $\forall y \, q(y,\mathbf{G}(P)) = 1$ est faux.
        Donc, $\exists y \, \neq (q(y,\mathbf{G}(P)) = 1)$ est vrai.
        Puisque, dans cette expression, $q(y,\mathbf{G}(P))$ ne peut prendre que les valeurs $0$ et $1$, on en déduit qu'il existe un entier $y$ tel que $q(y,\mathbf{G}(P)) = 0$, et donc qu'il existe une formule $F$ telle que $y = \mathbf{G}(F)$ et $F$ prouve $P(\mathbf{G}(P))$, et donc $Z$. 
\end{itemize}

\subsubsection{Second théorème d'incomplétude de Gödel}


******
