\subsection{Théorie ZF(C)}

\subsubsection{La théorie de Zermelo}
\label{sub:Zermelo}

La théorie de Zermelo, aussi dite « théorie Z » est une axiomatisation, dans le cadre de la logique du premier ordre avec égalité, de la théorie des ensembles. 
Elle fait intervenir des objets, appelés \textit{ensembles}, et leurs relations, notamment des relations binaires. 
Une de ces relations est l'\textit{appartenance}, désignée par le symbole $\in$. 
Si $x$ et $y$ sont deux ensembles, alors $x \in y$ est une proposition bien formée (il s'agit d'un terme). 
Si elle est vraie, on dira que \textit{$x$ est un élément de $y$}, que \textit{$x$ appartient à $y$}, que \textit{$x$ est dans $y$}, que \textit{$y$ contient $x$}, ou que \textit{$y$ possède $x$}.
On définit aussi la relation $\ni$ par : $x \ni y$ est équivalente à $y \in x$. 
On a donc : $\forall x \, \forall y \, (x \ni y) \Leftrightarrow (y \in x)$ et la relation $\notin$ par $\forall x \, \forall y \, (x \notin y) \Leftrightarrow \neg (x \in y)$. 
Pour l'évaluation d'une formule, les relations $\in$ et $\ni$ sont (comme toute autre relation binaire) prioritaires par rapport à l'égalité, mais pas par rapport à $\neg$.

On définit la relation d'inclusion $\subset$ par : $a \subset b$ est équivalent à $\forall x \, (x \in a) \Rightarrow (x \in b)$, autrement dit,
\begin{equation*}
    \forall a \, \forall b \, (
        (a \subset b) \Leftrightarrow (\forall x \, (x \in a) \Rightarrow (x \in b))
    ). 
\end{equation*}
Si $a \subset b$, on dira que \textit{$a$ est un sous-ensemble de $b$}, ou que \textit{$a$ est inclus dans $b$}.
Notons que, pour tout ensemble $a$, $a \subset a$ est vrai.%
\footnote{En effet, soit $x$ un ensemble, $x \in a$ a toujours la même valeur de vérité que lui-même, donc $(x \in a) \Rightarrow (x \in a)$ est vrai.}
On définit aussi la relation $\supset$ par :
\begin{equation*}
    \forall a \, \forall b \, (
        (a \supset b) \Leftrightarrow (\forall x \, (x \in a) \Leftarrow (x \in b))
    ). 
\end{equation*}

\medskip

\noindent\textbf{Lemme :} Soit $\forall a \, \forall b \, (a = b) \Leftrightarrow ((a \subset b) \wedge (b \subset a))$.

\medskip

\noindent\textbf{Démonstration :} 
    La formule $(a \subset b) \wedge (b \subset a)$ est équivalente à : $(\forall x (x \in a) \Rightarrow (x \in b)) \wedge (\forall y (y \in a) \Rightarrow (y \in b))$, et donc à $\forall x \, ((x \in a) \Rightarrow (x \in b)) \wedge ((x \in b) \Rightarrow (x \in a))$.
    Si $f$ et $g$ sont deux formules, $(f \Rightarrow g) \wedge (g \Rightarrow a)$ est équivalente à $f \Leftrightarrow g$.
    Donc, $(a \subset b) \wedge (b \subset a)$ est équivalente à $\forall x \, (x \in a) \Leftrightarrow (x \in b)$, et donc à $a = b$.
    Donc, $((a \subset b) \wedge (b \subset a)) \Leftrightarrow (a = b)$ est équivalente à $\mathsf{V}$.
    Donc, $\forall a \, \forall b \, ((a \subset b) \wedge (b \subset a)) \Leftrightarrow (a = b)$ est vraie.

    \done

\medskip

La théorie Z comporte six axiomes (l'axiome d'extensionnalité et les cinq axiomes de construction) ainsi qu'un schéma d'axiomes, correspondant à un axiome par formule à un paramètre libre. 

\medskip

\noindent\textit{\textbf{Axiome d'extensionnalité :} Si deux ensembles possèdent les mêmes éléments, alors ils sont égaux.}
\begin{equation*}
    \forall a \, \forall b \, (
        \forall x \, ((x \in a) \Leftrightarrow (x \in b)) \Rightarrow (a = b)
    ). 
\end{equation*}
La réciproque est une conséquence directe des propriétés de l'égalité en logique du premier ordre.%
\footnote{ En effet, soit deux ensembles $a$ et $b$ tels que $a = b$, et soit $x$ un ensemble, et $P$ le prédicat à un paramètre libre définit par $P y: x \in y$, puisque $a = b$, on doit avoir $P(a) \Leftrightarrow P(b)$, et donc $(x \in a) \Leftrightarrow (x \in b)$.}
On définit la relation $\neq$ par : $\forall a \, \forall b \, (a \neq b) \Leftrightarrow \neg (a = b)$.

\medskip

\noindent\textbf{Lemme :} 
    On définit la relation $R$ sur les ensembles par : soit $a$ et $b$ deux ensembles $(a \mathrel{R} b)$ a la même valeur de vérité que $(\forall x \, (x \in a) \Leftrightarrow (x \in b))$. 
    Alors, les trois prédicats suivants sont vrais :
    \begin{itemize}[nosep]
        \item $\forall x \, (x \mathrel{R} x)$ (réciprocité)
        \item $\forall x \, \forall y \, (x \mathrel{R} y) \Rightarrow (y \mathrel{R} x)$ (réflexivité)
        \item $\forall x \, \forall y \, \forall z \, ((x \mathrel{R} y) \wedge (y \mathrel{R} z)) \Rightarrow (x \mathrel{R} z)$.
    \end{itemize}
    Cela suggère que l'axiome d'extensionalité est compatible avec la définition de l'égalité en logique du premier ordre (même s'il manque le schéma d'axiomes de Leibniz pour assurer la cohérence).

\medskip

\noindent\textbf{Démonstration :} 
\begin{itemize}[nosep]
    \item Soit $x$ un ensemble. Pour tout $y$, $y \in x$ a la même valeur de vérité que $y \in x$ (trivialement, puisqu'il s'agit de la même formule).
        Donc, $\forall y \, (y \in x) \Leftrightarrow (y \in x)$. 
        Donc, $x \mathrel{R} x$.
    \item Soit $x$ et $y$ deux ensembles tels que $x \mathrel{R} y$. 
        Puisque $x = y$, on a : $\forall z \, z \in x \Leftrightarrow z \in y$. 
        Puisque le connecteur $\Leftrightarrow$ est symmétrique, on a donc : $\forall z \, z \in y \Leftrightarrow z \in y$.
        Donc, $y \mathrel{R} x$.
    \item Soit $x$, $y$ et $z$ trois ensembles tels que $x \mathrel{R} y$ et $y \mathrel{R} z$. 
        Pour tout ensemble $a$, on a $a \in x \Leftrightarrow a \in y$ et $a \in y \Leftrightarrow a \in z$. 
        Donc, par transitivité du connecteur $\Leftrightarrow$, $a \in x \Leftrightarrow a \in z$. 
        Cela étant valable pour tout ensemble $a$, on en déduit que $x \mathrel{R} z$.
\end{itemize}
\done

\medskip

\noindent\textbf{Démonstration bis :} À titre d'exercice, re-faisons ces courtes démonstrations de manière plus formelle.
\begin{itemize}[nosep]
    \item Soit $f$ la formule à deux paramètres libres $x$ et $y$ donnée par : $f: y \in x$. 
        Puisque $f \Leftrightarrow f$ est équivalente à $\mathsf{V}$, la formule $\forall x \, \forall y \, (f \Leftrightarrow f)$ est vraie. 
        Donc, $\forall x \, \forall y \, (y \in x) \Leftrightarrow (y \in x)$ est vraie.
        Donc, $\forall x \, x \mathrel{R} x$ est vraie.
    \item Soit $f$ la formule à deux paramètres libres $a$ et $x$ donnée par : $f: a \in x$, et $g$ la formule à deux paramètres libres $a$ et $y$ donnée par : $g: a \in y$. 
        Les deux formules $f \Leftrightarrow g$ et $g \Leftrightarrow f$ sont équivalentes (elles sont toutes deux vraies si $f$ et $g$ ont la même valeur de vérité et fausses sinon). 
        Donc, les formules $\forall a \, (f \Leftrightarrow g)$ et $\forall a \, (g \Leftrightarrow f)$ sont équivalentes. 
        Puisque $\forall a \, (f \Leftrightarrow g)$ est équivalente à $x \mathrel{R} y$ et $\forall a \, (g \Leftrightarrow f)$ à $y \mathrel{R} x$, on en déduit que $x \mathrel{R} y$ et $y \mathrel{R} x$ sont équivalentes. 
        Donc, $(x \mathrel{R} y) \Rightarrow (y \mathrel{R} x)$ est équivalente à $h \Rightarrow h$, où $h$ est la formule donnée par $h: x \mathrel{R} y$.
        Puisque $h \Rightarrow h$ est vraie que $h$ soit vraie ou fausse, elle est équivalente à $\mathsf{V}$. 
        Donc, $\forall x \forall y (h \Rightarrow h)$ est vraie.
        Donc, $\forall x \forall y (x \mathrel{R} y) \Rightarrow (y \mathrel{R} x)$ est vraie.
    \item Soit $f$ la formule à deux paramètres libres $a$ et $x$ donnée par : $f: a \in x$, $g$ la formule à deux paramètres libres $a$ et $y$ donnée par : $g: a \in y$, et $h$ la formule à deux paramètres libres $a$ et $z$ donnée par : $h: a \in z$. 
        Alors, $((f \Leftrightarrow g) \wedge (g \Leftrightarrow h)) \Rightarrow (f \Leftrightarrow h)$ est vraie quelles que soient les valeurs de vérité de $f$, $g$ et $h$. 
        Donc, si $\forall a \, ((f \Leftrightarrow g) \wedge (g \Leftrightarrow h))$ est vraie, alors $\forall a \, (f \Leftrightarrow h)$ est vraie.
        Donc, si $\forall a \, (f \Leftrightarrow g)$ et $\forall a \, (g \Leftrightarrow h)$ sont vraies, alors $\forall a \, (f \Leftrightarrow h)$ est vraie.
        Puisque $\forall a \, (f \Leftrightarrow g)$ est équivalente à $x \mathrel{R} y$, $\forall a \, (g \Leftrightarrow h)$ est équivalente à $y \mathrel{R} z$, et $\forall a \, (f \Leftrightarrow h)$ est équivalente à $x \mathrel{R} z$, on en déduit que $((x \mathrel{R} y) \wedge (y \mathrel{R} z)) \Rightarrow (x \mathrel{R} z)$ est toujours vraie.
        Donc, $\forall x \, \forall y \, \forall z \, ((x \mathrel{R} y) \wedge (y \mathrel{R} z)) \Rightarrow (x \mathrel{R} z)$ est vraie .
\end{itemize}

\done

\medskip

\noindent \textbf{Lemme :} La relation $\subset$ satisfait les trois propriétés suivantes : 
\begin{itemize}[nosep]
    \item \textit{Réflexivité :} $\forall x \, x \subset x$.
    \item \textit{Antisymétrie :} $\forall x \, \forall y \, (x \subset y) \wedge (y \subset x) \Rightarrow (x = y)$.
    \item \textit{Transitivité :} $\forall x \forall y \forall z \, (x \subset y) \wedge (y \subset z) \Rightarrow (x \subset z)$.
\end{itemize}

\medskip

\noindent \textbf{Démonstration :} 
\begin{itemize}[nosep]
    \item Soit $x$ un ensemble. Pour tout élément $e$ de $x$, on a (par définition), $e \in x$. 
        Donc, le prédicat $\forall e \, (e \in x) \Rightarrow (e \in x)$ est vrai.
        Donc, $x \subset x$.
    \item Soit $x$ et $y$ deux ensembles tels que $x \subset y$ et $y \subset x$.
        Soit $e$ un ensemble. 
        Si $e \in x$ est vrai, alors $e \in y$ est vrai aussi puisque $x \subset y$.
        Si $e \in x$ est faux, alors $e \in y$ est faux aussi, sans quoi on aurait $e \in y$ et donc $e \in x$ puisque $y \subset x$.
        Cela montre que $\forall e \, (e \in x) \Leftrightarrow (e \in y)$ est vrai. 
        Donc, d'après l'axiome d'extensionnalité, $x = y$ est vrai.
    \item Soit $x$, $y$ et $z$ trois ensembles tels que $x \subset y$ et $y \subset z$. 
        Soit $e$ un ensemble. 
        Si $e \in x$, alors $e \in y$ puisque $x \subset y$, et donc $e \in z$ puisque $y \subset z$. 
        Cela motre que le prédicat $\forall e \, (e \in x) \Rightarrow (e \in z)$ est vrai.
        Donc, $x \subset z$.
\end{itemize}

\done

\medskip

\noindent \textbf{Démonstration bis :} 
\begin{itemize}[nosep]
    \item Soit $f$ la formule $f: e \in x$. 
        La formule $f \Rightarrow f$ est vraie que $f$ soit vraie ou fausse, donc elle est équivalente à $\mathsf{V}$. 
        Donc, $\forall e \, (f \Rightarrow f)$ est équivalente à $\mathsf{V}$.
        Donc, $\forall e \, ((e \in x) \Rightarrow (e \in x))$ est équivalente à $\mathsf{V}$. 
        Donc, $x \subset x$ est équivalente à $\mathsf{V}$. 
        Donc, $\forall x \, x \subset x$ est vraie. 
    \item La formule $(x \subset y) \wedge (y \subset x)$ est équivalente à : $(\forall e \, (e \in x \Rightarrow e \in y)) \wedge (\forall f \, (f \in y \Rightarrow f \in x))$, et donc à $\forall e \, ((e \in x \Rightarrow e \in y) \wedge (e \in y \Rightarrow e \in x))$. 
        Puisque $(e \in x \Rightarrow e \in y) \wedge (e \in y \Rightarrow e \in x)$ est équivalente à $(e \in x \Leftrightarrow e \in y)$, la formule $(x \subset y) \wedge (y \subset x)$ est équivalente à $x = y$.
        Donc, $((x \subset y) \wedge (y \subset x)) \Rightarrow (x = y)$ est équivalente à $\mathsf{V}$. 
        Donc, $\forall x \, \forall y \, ((x \subset y) \wedge (y \subset x)) \Rightarrow (x = y)$ est vraie. 
    \item La formule $(x \subset y) \wedge (y \subset z)$ est équivalente à $(\forall e \, e \in x \Rightarrow e \in y) \wedge (\forall f \, f \in y \Rightarrow f \in z)$, donc à $\forall e \, ((e \in x \Rightarrow e \in y) \wedge (e \in y \Rightarrow e \in z))$. 
        Soit $f$, $g$ et $h$ trois formules, $((f \Rightarrow g) \wedge (g \Rightarrow h))$ est équivalente à $(f \Rightarrow h) \wedge (f \Rightarrow g)$.
        Donc, la formule $(x \subset y) \wedge (y \subset z)$ est équivalente à $\forall e \, ((e \in x \Rightarrow e \in z) \wedge (e \in x \Rightarrow e \in y))$, et donc à $(\forall e \, (e \in x \Rightarrow e \in z)) \wedge (\forall f \, (f \in x \Rightarrow f \in y))$, et donc à $(x \subset z) \wedge (x \subset y)$. 
        Puisque, si $g$ et $h$ sont deux formules, $g \wedge h \Rightarrow g$ est toujours vraie, $((x \subset z) \wedge (x \subset y)) \Rightarrow (x \subset z)$ est équivalente à $\mathsf{V}$, donc on en déduit que $((x \subset y) \wedge (y \subset z)) \Rightarrow (x \subset z)$ est équivalente à $\mathsf{V}$, donc $\forall x \, \forall y \, ((x \subset y) \wedge (y \subset z)) \Rightarrow (x \subset z)$ est vraie.
\end{itemize}

\done

\medskip

\noindent\textbf{Lemme :} La proposition $\forall a \, \forall b \, (a = b) \Leftrightarrow [(a \subset b) \wedge (b \subset a)]$ est vraie.
    Autrement dit, pour tous ensembles $a$ et $b$, la formule $a = b$ est équivalente à $(a \subset b) \wedge (b \subset a)$.

\medskip

\noindent\textbf{Démonstration :} 
    Soit $a$ et $b$ deux ensembles. 
    \begin{itemize}[nosep]
        \item Supposons d'abord que $a = b$.
            Soit $x$ tel que $x \in a$. 
            Puisque $a = b$, on a $x \in b$. 
            Donc, $\forall x \, (x \in a) \Rightarrow (x \in b)$.
            Donc, $a \subset b$. 
            Puisque l'égalité est symmétrique, on montre de même en échangeant les rôles de $a$ et $b$ que $b \subset a$.
            Donc, $(a \subset b) \wedge (b \subset a)$. 
        \item Supposons maintenant que $(a \subset b) \wedge (b \subset a)$. 
            Soit $x$ un ensemble. 
            Si $x \in a$, et puisque $a \subset b$, alors $x \in b$.
            De même, si $x \in b$, et puisque $b \subset a$, alors $x \in a$.
            Donc, $\forall x \, (x \in a) \Leftrightarrow (x \in b)$. 
            Donc, $a = b$. 
    \end{itemize}
    On a donc montré que les formules $a = b$ et $(a \subset b) \wedge (b \subset a)$ son équivalentes, au sens où chacune est vraie qi l'autre l'est (et donc, également, fausse si l'autre l'est).

   \done 

\medskip

\noindent\textbf{Démonstration bis :} 
    La formule $(a \subset b) \wedge (b \subset a)$ est équivalente à : $(\forall x \, (x \in a \Rightarrow x \in b)) \wedge (\forall y \, (y \in b \Rightarrow y \in a))$, et donc à $\forall x \, ((x \in a \Rightarrow x \in b) \wedge (x \in b \Rightarrow x \in a))$. 
    Si $f$ et $g$ sont deux formules, $(f \Rightarrow g) \wedge (g \Rightarrow f)$ est équivalente à $f \Leftrightarrow g$ (toutes deux sont vraies si $f$ et $g$ sont toutes deux vraies ou toutes deux fausses, fausses si l'une est vraie et l'autre est fausse, et (en présence de la valeur de vérité $\mathsf{I}$) indéfinies si $f$ ou $g$ l'est).
    Donc, $(x \in a \Rightarrow x \in b) \wedge (x \in b \Rightarrow x \in a)$ est équivalente à $x \in a \Leftrightarrow x \in b$.
    Donc, la formule $(a \subset b) \wedge (b \subset a)$ est équivalente à $\forall x \, (x \in a \Leftrightarrow x \in b)$, et donc à $a = b$.

    Donc, la formule $((a \subset b) \wedge (b \subset a)) \Leftrightarrow (a = b)$ est équivalente à $(a = b) \Leftrightarrow (a = b)$, et donc toujours vraie, et donc équivalente à $\mathsf{V}$. 
    Donc, la formule $\forall a \, \forall b \, ((a \subset b) \wedge (b \subset a)) \Leftrightarrow (a = b)$ est vraie. 

    \done

\medskip

\noindent\textit{\textbf{Axiome de la paire :} La paire formée par deux ensembles est un ensemble :}
\begin{equation*}
    \forall a \, \forall b \, \exists c \, \forall x \, (
        (x \in c) \Leftrightarrow ((x = a) \vee (x = b))
    ).
\end{equation*}
Si $a$ et $b$ sont deux ensembles, on note $\lbrace a, b \rbrace$ leur paire. 
Il s'agit de l'ensemble contenant $a$ et $b$ mais aucun autre (au sens de « non égal à $a$ ni à $b$ ») ensemble. 
Cet ensemble est unique d'après l'axiome d'extensionnalité.
Si de plus $b = a$, alors $\lbrace a, b \rbrace$ ne contient qu'un seul élément. 
Il peut alors être abrégé en $\lbrace a \rbrace$. 
Puisque, pour tout $x$, la formule $(x = a) \vee (x=a)$ est équivalente à $x = a$, on a : 
\begin{equation*}
    \forall x \, (x \in \lbrace a \rbrace) \Leftrightarrow (x = a). 
\end{equation*}

\medskip

\noindent\textit{\textbf{Axiome de la réunion :} Pour tout ensemble $a$, il existe un ensemble qui est l'union des éléments de $a$ :}
\begin{equation*}
    \forall a \, \exists b \, \forall x \, (
        (x \in b) \Leftrightarrow (\exists y \, ((y \in a) \wedge (x \in y)))
    ).
\end{equation*}
La réunion d'un ensemble $a$ (noté $b$ dans la formule ci-dessus) est notée $\cup a$.
Cet ensemble est unique d'après l'axiome d'extensionnalité.
Si $a$ et $b$ sont deux ensembles, $\lbrace a, b \rbrace$ est aussi un ensemble d'après l'axiome de paire. 
La réunion de cet ensemble est notée $a \cup b$. 
Soit $a$, $b$ et $c$ trois ensembles. 
On note $\lb a, b, c \rb$ l'ensemble $\lbrace a, b \rbrace \cup \lbrace c \rbrace$. 

\medskip

\noindent\textbf{Lemme :} Soit $a$, $b$ et $x$ trois ensembles. 
    Le prédicat $x \in a \cup b$ est équivalent à $(x \in a) \vee (x \in b)$.

\medskip

\noindent\textbf{Démonstration :} 
    Le prédicat $x \in a \cup b$ est équivalent à $\exists y \, y \in \lbrace a, b \rbrace \wedge x \in y$, donc à $\exists y \, (y = a \vee y = b) \wedge x \in y$, donc à $\exists y \, ((y = a \wedge x \in y) \vee (y = b \vee x \in y))$, donc à $(\exists y \, y = a \wedge x \in y) \vee (\exists z \, z = b \wedge x \in z))$.
    Si $f$ est ue formule dépendant de deux paramètres libres $x$ et $y$ et si $a$ est un ensemble, alors $\exists (y = a) \wedge f(x,y)$ est équivalente à $f(x,a)$.
    En effet, si $f(x,a)$ est fausse, alors $(y = a) \wedge f(x,y)$ est fausse pour toute valeur de $y$ et, si elle est vraie, alors elle est vraie pour une valeur de $y$ (et cette valeur est $a$). 
    Donc, $x \in a \cup b$ est équivalente à $(x \in a) \vee (x \in b)$.

    \done

\medskip

\noindent\textit{\textbf{Axiome de l'ensemble des parties :} La collection des parties d'un ensemble est un ensemble :} 
\begin{equation*}
    \forall a \, \exists b \, \forall x \, (
        (x \in b) \Leftrightarrow (x \subset a)
    ).
\end{equation*}
Cet ensemble est unique d'après l'axiome d'extensionnalité.

\medskip

\noindent\textit{\textbf{Schéma d'axiomes de compréhension :} 
Pour tout prédicat $P$ à une variable libre $x$ et chaque ensemble $a$, il existe un ensemble qui a pour éléments l'ensemble des éléments de $a$ vérifiant la propriété $P$, c'est-à-dire :}
\begin{equation*}
\forall a \, \exists b \, \forall x \, [ (x \in b) \Leftrightarrow ((x \in a) \wedge P x)]. 
\end{equation*}

Avec les mêmes notations, cet ensemble est noté $\lbrace x \in a \vert P x \rbrace$. 
Il est unique d'après l'axiome d'extensionnalité. 
(En effet, si deux ensembles satisfont l'énoncé de l'axiome obtenu pour un même ensemble et une même propriété, alors tout élémen de l'un appartient à l'autre.)
Ce schéma d'axiomes implique qu'il existe un ensemble vide, noté $\emptyset$, pourvu qu'au moins un ensemble $a$ existe—ce qui est nécessairement le cas puisque, en logique du premier ordre, les domaines d'interprétation des variables d'objets de base, ici les ensembles, sont non vides. 
On peut en effet le définir par : $\emptyset = \lbrace x \in a \vert x \neq x \rbrace$. 
Puisque tout ensemble $x$ satisfait $x = x$, il n'existe aucun $x$ tel que $x \in \emptyset$ ; autrement dit, la formule suivante est vraie : $\forall x \, x \notin \emptyset$. 
Cet ensemble est unique d'après l'axiome d'extensionnalité.

Notons que, puisque $\forall x \, x \notin \emptyset$ est vraie, $\exists x \, x \in \emptyset$ est fausse et $x \notin \emptyset$ est équivalente à $\mathsf{V}$ et $x \in \emptyset$ à $\mathsf{F}$.

\medskip

\noindent\textbf{Lemme :} Le prédicat suivant est vrai : $\forall x \, \emptyset \subset x$.

\medskip

\noindent\textbf{Démonstration :} Soit $x$ un ensemble. 
    La formule $\emptyset \subset x$ est équivalente à : $\forall e \, (e \in \emptyset) \Rightarrow (e \in x)$. 
    Or, pour tout ensemble $e$, $e \in \emptyset$ est faux, donc $(e \in \emptyset) \Rightarrow (e \in x)$ est vrai.
    Donc, $\forall e \, (e \in \emptyset) \Rightarrow (e \in x)$ est vrai. 
    Donc, $\emptyset \subset x$ est vrai.

\done

\medskip

\noindent\textbf{Démonstration bis :} 
    On veut montrer que le prédicat $P: \forall x \, \forall e \, (e \in \emptyset \Rightarrow e \in x)$ est vrai. 
    $P$ est équivalent à : $\forall x \, \forall e \, ((e \in x) \vee \neg (e \in \emptyset))$, c'est-à-dire, à : $\forall x \, \forall e \, ((e \in x) \vee (e \notin \emptyset))$.
    Puisque $\forall e \, e \notin \emptyset$ est vrai, $e \notin \emptyset$ est équivalent à $\mathsf{V}$, donc $\forall e \, ((e \in x) \vee (e \notin \emptyset))$ est équivalent à $\forall e \, ((e \in x) \vee \mathsf{V})$, donc à $\forall e \, \mathsf{V}$, et donc à $\mathsf{V}$.
    Donc, $P$ est vrai. 

    \done

\medskip

\noindent\textbf{Lemme :} Le prédicat suivant est vrai : $\forall x \, x \subset \emptyset \Rightarrow x = \emptyset$.

\medskip

\noindent\textbf{Démonstration :} Soit $x$ un ensemble satisfaisant $x \subset \emptyset$.
    Pour tout ensemble $y$, on a $y \notin \emptyset$, donc $y \notin x$.
    
\done

\medskip

\noindent\textbf{Démonstration bis :} 
    On veux montrer le prédicat $P: \forall x \, (x \subset \emptyset) \Rightarrow (x = \emptyset)$.
    Il est équivalent à : $\forall x \, (x \subset \emptyset) \Rightarrow ((x \subset \emptyset) \wedge (\emptyset \subset x))$, donc à $\forall x \, \neg (x \subset \emptyset) \vee ((x \subset \emptyset) \wedge (\emptyset \subset x))$, donc à $\forall x \, (\neg (x \subset \emptyset) \vee (x \subset \emptyset)) \wedge (\neg (x \subset \emptyset) \vee (\emptyset \subset x))$.
    Puisque $\neg (x \subset \emptyset) \vee (x \subset \emptyset)$ est toujours vrai (soit $f$ la formule $x \subset \emptyset$, il s'agit de $\neg f \vee f$, qui est vrai que $f$ soit vraie ou fausse), $P$ est équivalent à $\forall x \, (\neg (x \subset \emptyset) \vee (\emptyset \subset x))$.
    On a vu que $\forall x \, \emptyset \subset x$ est vrai.
    Donc, $\emptyset \subset x$ est équivalente à $\mathsf{V}$.
    Donc, $P$ est équivalente à $\forall x \, (\neg (x \subset \emptyset) \vee \mathsf{V})$, donc à $\forall x \, \mathsf{V}$, et donc à $\mathsf{V}$.
    Donc, $P$ est vrai.

    \done

\medskip

L'axiome de compréhension peut aussi être utilisé pour définir la différence de deux ensembles. 
Soit $A$ et $B$ deux ensembles. 
On note $A \setminus B$ l'ensemble $\lbrace x \in A \vert x \notin B \rbrace$.

\medskip

Notons qu'il s'agit bien d'un schéma d'axiomes, c'est-à-dire une méthode permettant de construire des axiomes, et non d'un seul axiome : puisqu'on ne peut pas quantifier les prédicats en logique du premier ordre, ce shéma définit un axiome pour chaque prédicat à un paramètre libre. 
En théorie Z, on considère le prédicat obtenu à partir de tout prédicat $P$ à une variable libre comme vrai.

Ce schéma peut être reformulé en notant que, si $P$ est un prédicat à une variable libre $x$ et d'autres variables libres éventuelles $a_1 \cdots a_p$, et si $\alpha 1 \dots \alpha_p$ est une collection d'ensembles pouvant remplacer $a_1 \dots a_p$, alors le prédicat $Q$ défini par $Q: P x \alpha1 \cdots \alpha_p$ a une unique variable libre $x$. 
Le schéma d'axiomes de compréhension peut ainsi être reformulé de la manière suivante : 
\textit{Pour tout prédicat $P$ à une variable libre $x$ et d'éventuels autres variables libres collectivement notées $a_1 \dots a_p$, pour chaque valeur des variables $a_1 \cdots a_p$ et chaque ensemble $b$, il existe un ensemble qui a pour éléments l'ensemble des éléments de $b$ vérifiant la propriété $P x a_1 \dots a_p$, c'est-à-dire :}
\begin{equation*}
\forall a_1 \dots a_p \, \forall b \, \exists c \, \forall x \, [ (x \in c) \Leftrightarrow ((x \in b) \wedge P x a_1 \dots a_p)]. 
\end{equation*}
(Dans cette formule, il est entendu que le premier quantificateur est absent si $P$ n'a qu'une seule variable libre.) 

\medskip

\noindent\textbf{Lemme :} Soit $A$ et $B$ deux ensembles.
    Alors, $(A \setminus B) \cup B = A \cup B$.

\medskip

\noindent\textbf{Démonstration :}
    Soit $x$ un élément de $A \cup B$. 
    Si $x \in B$, alors $x \in (A \setminus B) \cup B$.
    Sinon, $x \in A$, donc $x \in A \setminus B$, donc $x \in (A \setminus B) \cup B$.
    Donc, dans tous les cas, $x \in (A \setminus B) \cup B$.

    Soit $x$ un élément de $(A \setminus B) \cup B$.
    Alors, $x \in A \setminus B$ ou $x \in B$. 
    Si $x \in A \setminus B$, alors $x \in A$ puisque $A \setminus B \subset A$, donc $x \in A \cup B$.
    Si $x \in B$, alors $x \in A \cup B$.
    Donc, dans tous les cas, $x \in A \cup B$.

    On a donc montré que : $\forall x \, (x \in A \cup B) \Leftrightarrow (x \in (A \setminus B) \cup B)$, et donc que $(A \setminus B) \cup B = A \cup B$.

    \done

\medskip

\noindent\textbf{Démonstration bis :} 
    Le prédicat $x \in A \cup B$ est équivalent à $(x \in A) \vee (x \in B)$.
    Le prédicat $x \in (A \setminus B) \cup B$ est équivalent à $(x \in A \setminus B) \vee (x \in B)$, et donc à $((x \in A) \wedge (x \notin B)) \vee (x \in B)$.
    Ce dernier est équivalent à $((x \in A) \vee (x \in B)) \wedge ((x \notin B) \vee (x \in B))$.
    Pour toute formule $f$, $(\neg f) \vee f$ est vrai que $f$ soit vraie ou fausse, donc équivalent à $\mathsf{V}$.
    Donc, $x \in (A \setminus B) \cup B$ est équivalent à $((x \in A) \vee (x \in B)) \wedge \mathsf{V}$, donc à $(x \in A) \vee (x \in B)$, et donc à $x \in A \cup B$.
    Donc, $(x \in A \cup B) \Leftrightarrow (x \in (A \setminus B) \cup B)$ est équivalent à $(x \in A \cup B) \Leftrightarrow (x \in A \cup B)$.
    Pour toute formule $f$, $f \Leftrightarrow f$ est vrai que $f$ soit vraie ou fausse, et donc équivalent à $\mathsf{V}$.
    Donc, $\forall x \, (x \in A \cup B) \Leftrightarrow (x \in (A \setminus B) \cup B)$ est vrai.

\done

\medskip

\noindent\textbf{Lemme :} Soit $A$ et $B$ deux ensembles tels que $B \subset A$.
    Alors, $A \cup B = A$.

\medskip

\noindent\textbf{Démonstration :}
    Soit $x$ un élément de $A \cup B$. 
    Alors, $x \in A$ ou $x \in B$.
    Si $x \in B$, et puisque $B \subset A$, $x \in A$.
    Donc, $x \in A$.

    Soit $x$ un élément de $A$, on a $x \in A \cup B$.

    Ainsi, $A \cup B = A$.

    \done

\medskip

\noindent\textbf{Démonstration bis :} 
    Puisque $B \subset A$, le prédicat $\forall x \, (x \in B) \Rightarrow (x \in A)$ est vrai.
    Donc, le prédicat $(x \in B) \Rightarrow (x \in A)$ est équivaent à $\mathsf{V}$.
    Donc, le prédicat $(x \in A) \vee (x \notin B)$ est équivalent à $\mathsf{V}$.
    
    Le prédicat $x \in A \cup B$ est équivalent à $(x \in A) \vee (x \in B)$.
    Puisque, pour tout prédicat $P$, $P \wedge \mathsf{V}$ est équivalent à $P$, $x \in A \cup B$ est équivalent à $((x \in A) \vee (x \in B)) \wedge ((x \in A) \vee (x \notin B))$, et donc à $(x \in A) \vee ((x \in B) \wedge (x \notin B))$.
    Puisque, pour tout prédicat $P$, $P \wedge \neg P$ est équivalent à $\mathsf{F}$, cela est équivalent à $(x \in A) \vee \mathsf{F}$, et donc à $x \in A$.
    Donc, $x \in A \cup B$ est équivalent à $x \in A$.
    Donc, $\forall x \, x \in A \cup B \Leftrightarrow x \in A$ est vrai.
    Donc, $A \cup B = A$.

\done

\medskip

\noindent\textit{\textbf{Axiome de l'infini :} Il existe un ensemble contenant l'ensemble vide et clos par application du successeur $x \mapsto x \cup \lbrace x \rbrace$.} Formellement, cet axiome s'écrit : 
\begin{equation*}
    \exists Y \, (\emptyset \in Y) \wedge (\forall y \, ((y \in Y) \Rightarrow (y \cup \lbrace y \rbrace \in Y))). 
\end{equation*}

\medskip

L'ensemble ainsi défini contient $\emptyset$, $\lbrace \emptyset \rbrace$, $\lbrace \emptyset, \lbrace \emptyset \rbrace \rbrace$, $\lbrace \emptyset, \lbrace \emptyset \rbrace, \lbrace \emptyset, \lbrace \emptyset \rbrace \rbrace \rbrace$, ...

\subsubsection{Intersection} 

Soit $a$ et $b$ deux ensembles. 
On appelle \textit{intersection} de $a$ et $b$, notée $a \cap b$, l'ensemble 
\begin{equation*}
    a \cap b = \lbrace x \in a \vert x \in b \rbrace. 
\end{equation*}
Cet ensemble existe d'après le schéma d'axiomes de compréhension, en considérant la formule à un paramètre $P x: x \in b$. 
Il est unique d'après l'axiome d'extensionnalité.
On a : $\forall x \, x \in a \cap b \Leftrightarrow (x \in a \wedge x \in b)$. 
Notons que cette définition est symmétrique : $\forall a \, \forall b \, (a \cap b) = (b \cap a)$.  
Elle est aussi transitive : si $a$, $b$ et $c$ sont trois ensembles, on a $(a \cap b) \cap c = a \cap (b \cap c)$. 
(Ces deux propriétés sont des conséquence de la symmétrie et de la transitivité du connecteur $\wedge$.)
On pourra noter ce l'ensemble $a \cap (b \cap c)$ par $a \cap b \cap c$. 

De même, on a : $\forall x \, x \in a \cup b \Leftrightarrow (x \in a \vee x \in b)$. 
On en déduit aisément que $a \cup b = b \cup a$ et, si $c$ est un ensemble, $(a \cup b) \cup c = a \cup (b \cup c)$. 
On pourra noter ce l'ensemble $a \cup (b \cup c)$ par $a \cup b \cup c$. 

\medskip

\noindent\textbf{Lemme :} Soit $E$ un ensemble. 
    Alors $E \cup \emptyset = E$ et $E \cap \emptyset = \emptyset$.

\medskip

\noindent\textbf{Démonstration :} 
    Soit $e$ un ensemble. 
    Si $e \in E$, alors $(e \in E) \vee (e \in \emptyset)$  est vrai, donc $e \in (E \cup \emptyset)$.
    Sinon, et puisque $e \in \emptyset$ est faux, alors $(e \in E) \vee (e \in \emptyset)$  est faux, donc $e \in (E \cup \emptyset)$ est faux.
    On a donc : $\forall e \, (e \in E) \Leftrightarrow (e \in (E \cup \emptyset))$.
    Donc, $E = E \cup \emptyset$.

    Soit $e$ un ensemble. 
    Puisque $e \in \emptyset$ est faux, $(e \in \emptyset) \wedge (e \in E)$ est faux. 
    Donc, $e \in (E \cap \emptyset)$ est faux.
    Cela montre que $E \cap \emptyset = \emptyset$.

    \done

\medskip

\noindent\textbf{Démonstration bis :} 
    \begin{itemize}[nosep]
        \item Notons $P_1$ et $P_2$ les prédicats à un paramètre libre suivants : $P_1(x): x \in E \cup \emptyset$, $P_2(x): x \in E$. 
            $P_1$ est équivalent à $(x \in E) \vee (x \in \emptyset)$.
            Puisque $x \in \emptyset$ est équivalent à $\mathsf{F}$, $P_1$ est équivalent à $x \in E$.
            Donc, $P_1$ est équivalent à $P_2$.
            Donc, $P_1 \Leftrightarrow P_2$ est équivalent à $\mathsf{V}$.
            Donc, $\forall x \, P_1 \Leftrightarrow P_2$ est vrai.
            Donc, $E \cup \emptyset = E$.
        \item Notons $P_1$ le prédicat à un paramètre libre : $P_1(x): x \in E \cap \emptyset$. 
            $P_1$ est équivalent à $(x \in E) \wedge (x \in \emptyset)$.
            Puisque $x \in \emptyset$ est équivalent à $\mathsf{F}$, $P_1$ est équivalent à $\mathsf{F}$, et donc à $x \in \emptyset$.
            Donc, $P_1 \Leftrightarrow (x \in \emptyset)$ est équivalent à $\mathsf{V}$.
            Donc, $\forall x \, P_1 \Leftrightarrow (x \in \emptyset)$ est vrai.
            Donc, $E \cap \emptyset = \emptyset$.
    \end{itemize}

    \done

\subsubsection{Schéma d'axiomes de remplacement}

La théorie de Zermelo plus cet axiome donne la théorie ZF. 

\medskip

\noindent\textit{\textbf{Énoncé :} Soit $F$ une formule à deux variables libres (notées en première et second position) et d'éventuels paramètres notés $a_1 \dots a_p$. Alors, }
\begin{equation*}
    \forall a_1 \dots a_p \, 
        \left( 
            \forall x \, \forall y \, \forall z \, \left[
                (F x y a_1 \dots a_p \wedge F x z a_1 \dots a_p) \Rightarrow (z = y)
            \right]
        \right)
        \Rightarrow
        \left(
            \forall b \, \exists c \, \forall z \, \left[
                (z \in c) \Leftrightarrow (\exists x \, [(x \in b) \wedge (F x z a_1 \dots a_p)])
            \right]
        \right)
    .
\end{equation*}

\medskip

\noindent\textbf{Lemme :} Pour un choix donné des paramètres tel que le membre de gauche de l'implication est satisfait et pour tout $b$, l'ensemble $c$ définit par $\forall z \, \left[ (z \in c) \Leftrightarrow (\exists x \, [(x \in b) \wedge (F x z a_1 \dots a_p)]) \right]$ est unique d'après l'axiome d'extensionnalité.

\medskip

La démonstration de ce lemme est relativement triviale. 
Écrivons-mà cependant explicitement par soucis de clarté. 

\medskip

\noindent\textbf{Démonstration :} 
    Soit $F$ une formule à deux variables libres notées en première et seconde position et d'éventuels paramètres, collectivement notés $a$. 
    Fixons les paramètres $a$ tels que la formule
    \begin{equation*}
        \forall x \, \forall y \, \forall z \, \left[
            (F x y a \wedge F x z a) \Rightarrow (z = y)
        \right]
    \end{equation*}
    est vraie. 
    
    Soit $b$ un ensemble. 
    Soit $c_1$ et $c_2$ deux ensembles satisfaisant : 
    \begin{equation*}
        (z \in c_1) \Leftrightarrow (\exists x \, [(x \in b) \wedge (F x z a)])
    \end{equation*}
    et 
    \begin{equation*}
        (z \in c_2) \Leftrightarrow (\exists x \, [(x \in b) \wedge (F x z a)]) .
    \end{equation*}
    Alors, ,
    \begin{itemize}[nosep]
        \item Soit $z$ un ensemble. 
            Si $z \in c_1$, il existe un élément $x$ de $b$ tel que $F x z a$ est vrai. 
            Donc, $z \in c_2$.
        \item Soit $z$ un ensemble. 
            Si $z \in c_2$, il existe un élément $x$ de $b$ tel que $F x z a$ est vrai. 
            Donc, $z \in c_1$.
    \end{itemize}
    Les deux ensembles $c_1$ et $c_2$ sont donc égaux d'après l'axiome d'extensionnalité.

   \done 

\medskip

Si $F$ est une formule à deux variables libres sans autres paramètres, le schéma d'axiomes de remplacement donne :
\begin{equation*}
    \left( 
        \forall x \, \forall y \, \forall z \, \left[
            (F x y \wedge F x z) \Rightarrow (z = y)
        \right]
    \right)
    \Rightarrow
    \left(
        \forall b \, \exists c \, \forall z \, \left[
            (z \in c) \Leftrightarrow (\exists x \, [(x \in b) \wedge (F x z)])
        \right]
    \right)
.
\end{equation*}

\medskip

\noindent\textbf{Lemme :} Le schéma d'axiomes de compréhension est une conséquence du schéma d'axiomes de remplacement, obtenue en prenant $F x y : (x = y) \wedge P(x)$. 

\medskip

\noindent\textbf{Démonstration :} 
    (On peut aisément étendre cette démonstration au cas où le prédicat $P$ a d'autres paramètres que $x$ en ajoutant les mêmes paramètres à $F$.)
    On admet le schéma d'axiomes de remplacement. 
    Soit $P$ un prédicat à un paramètre libre. 
    Soit $F$ la formule à deux paramètres libres définie par $F x y : (x = y) \wedge P(x)$. 
    Pour tous $y$ et $z$, si $F x y$ et $F x z$, alors $x = y$ et $x = z$, donc $y = z$ par réflexivité et transitivité de l'égalité.
    Soit $b$ un ensemble. 
    D'après l'axiome obtenu par le schéma d'axiomes de copréhension pour la formule $F$, on peut choisir un ensemble $c$ tel que : 
    \begin{equation*}
        \forall z \, (z \in c) \Leftrightarrow \left( \exists x \, ((x \in b) \wedge (F x z)) \right). 
    \end{equation*}
    Cette formule est équivalente à : 
    \begin{equation*}
        \forall z \, (z \in c) \Leftrightarrow \left( \exists x \, ((x \in b) \wedge (x = z) \wedge P(x)) \right). 
    \end{equation*}
    Puisque la relation $\wedge$ est symmétrique et transitive, la formule $\exists x \, ((x \in b) \wedge (x = z) \wedge P(x))$ est équivalente à $\exists x \, ((x = z) \wedge  ((x \in b) \wedge P(x)))$.
    Or, pour tout $z$, la formule $\exists x \, ((x = z) \wedge ((x \in b) \wedge P(x)))$ est équivalente à $(z \in b) \wedge P(z)$.
    En effet, 
    \begin{itemize}[nosep]
        \item Si cette dernière est vraie, alors, puisque $z = z$ est toujours vrai par réciprocité de l'égalité, $(z = z) \wedge ((z \in b) \wedge P(z))$ est vraie, et donc il existe une valeur de $x$ ($z$) telle que $(x \in b) \wedge (x = z) \wedge P(x)$ est vraie. 
        \item Si elle est fausse, alors il n'existe aucune valeur de $x$ telle que $(x = z) \wedge ((x \in b) \wedge P(x))$ est vraie puisque, si $x = z$ est vrai, $(x \in b) \wedge P(x)$ a la même valeur de vérité que $(z \in b) \wedge P(z)$ et est donc fausse.
    \end{itemize}
    Ainsi, l'ensemble $c$ satisfait : 
    \begin{equation*}
        \forall z \, (z \in c) \Leftrightarrow \left( (z \in b) \wedge P(z) \right). 
    \end{equation*}
    On a donc montré que :
    \begin{equation*}
        \forall b \, \exists c \, \forall z \, (z \in c) \Leftrightarrow \left( (z \in b) \wedge P(z) \right). 
    \end{equation*}

   \done 


\medskip

\noindent\textbf{Lemme:} En présence du schéma d'axiomes de remplacement, l'axiome de la paire est une conséquence des autres. 

\medskip

\noindent\textbf{Démonstration:} Tout d'abord, d'après le schéma d'axiomes de compréhension, l'ensemble vide $\emptyset$ existe. 
Son seul sous-ensemble est lui-même. 
En effet, on a $\emptyset \subset \emptyset$ (puisque chaque ensemble est un sous-ensemble de lui-même ; une autre façon de voir cela est que $(x \in \emptyset) \Rightarrow (x \in \emptyset)$ est vraie pour tout $x$ puisque le membre de gauche est toujours faux) et, si $a \subset \emptyset$, alors $\forall x \, x \notin a$ (sans quoi on aurait $x \in a$ et donc $x \in \emptyset$, ce qui est impossible par définition de l'ensemble vide), et donc $a = \emptyset$. 
Donc, l'ensemble des parties de $\emptyset$ est l'ensemble ne contenant que $\emptyset$. 
Cet ensemble est noté $\lbrace \emptyset \rbrace$. 
Ce nouvel ensemble contient deux sous-ensembles : $\emptyset$ et $\lbrace \emptyset \rbrace$. 
(Ce sont bien des sous-ensembles car tout élément d'un de ces ensembles doit être $\emptyset$, qui est un élément de $\lbrace \emptyset \rbrace$ et, si $a \subset \lbrace \emptyset \rbrace$, $a$ ne peut contenir d'autre élément que $\emptyset$ ; il doit donc être égal soit à $\emptyset$ (s'il ne contient pas $\emptyset$) soit à $\lbrace \emptyset \rbrace$ (s'il le contient).) 
D'après l'axiome de l'ensemble des parties, l'ensemble $\lbrace \emptyset, \lbrace \emptyset \rbrace \rbrace$ contenant uniquement $\emptyset$ et $\lbrace \emptyset \rbrace$ existe donc. 

Soit $A$ et $B$ deux ensembles. 
Considérons la formule à deux variables libres $F$ définie par : 
\begin{equation*}
    F x y: [(x = \emptyset) \wedge (y = A)] \vee [(x = \lbrace \emptyset \rbrace) \wedge (y = B)].
\end{equation*}
Notons que $\lbrace \emptyset \rbrace \neq \emptyset$ puisque $\emptyset \in \lbrace \emptyset \rbrace$ et $\emptyset \notin \emptyset$. 
$F$ satisfait : 
\begin{equation*}
    \forall x \, \forall y \, \forall z \, ([ (F x y) \wedge (F x z)] \Rightarrow [y = z] ).
\end{equation*}
(Car, si le membre de gauche est vrai, soit $x = \emptyset$, $y = A$, $z = A$, soit $x = \lbrace \emptyset \rbrace$, $y = B$, $z = B$.)
Soit $C$ l'ensemble défini par l'axiome de remplacement pour $F$, en prenant pour l'ensemble noté $b$ dans la définition l'ensemble $\lbrace \emptyset, \lbrace \emptyset \rbrace \rbrace$. 
Alors, pour tout $d$, $d \in C$ si et seulement si il existe $x$ tel que $x \in \lbrace \emptyset, \lbrace \emptyset \rbrace \rbrace$ et $F x d$. 
On a donc deux (et seulement deux) possibilités : $x = \emptyset$ et $d = A$, ou $x = \lbrace \emptyset \rbrace$ et $d = B$. 
Donc, $[d \in C] \Leftrightarrow [(d = A) \vee (d = B)]$.
L'ensemble $C$ est donc la paire $\lbrace A, B \rbrace$.%
\footnote{
    Montrons cela plus rigoureusement. 
    Tout d'abords, $A$ et $B$ appartiennent à $C$. 
    En effet, on a $\emptyset \in \lbrace \emptyset, \lbrace \emptyset \rbrace \rbrace$ et $F \emptyset A$, donc $A \in C$, et $\lbrace \emptyset \rbrace \in \lbrace \emptyset, \lbrace \emptyset \rbrace \rbrace$ et $F \lbrace \emptyset \rbrace B$, donc $B \in C$.
    
    Soit $X$ un élément de $C$. 
    On peut choisir un élément $x$ de $\lbrace \emptyset, \lbrace \emptyset \rbrace \rbrace$ tel que $F x X$. 
    Cela laisse deux possibilités : $x = \emptyset$ ou $x = \lbrace \emptyset \rbrace$. 
    Si $x = \emptyset$, $F x X$ implique $X = A$.
    Si $x = \lbrace \emptyset \rbrace $, $F x X$ implique $X = B$. 
    Dans les deux cas, on a bien $(X = A) \wedge (X = B)$. 

    Ainsi, $(X \in C) \Leftrightarrow ((X = A) \vee (X = B))$ est vrai.
}

\hfill\square

\medskip

En admettant le schéma d'axiomes de remplacement, on peut donc s'affranchir du schéma d'axiomes de compréhension et de l'axiome de la paire. 
La théorie ZF est ainsi définie par quatre axiomes et un schéma d'axiomes. 

\subsubsection{Axiome de fondation} 

Cet axiome peut être inclus ou non dans la théorie ZFC, selon les auteurs. 
Dans la suite, on ne l'inclura pas sauf mention contraire explicite.

\medskip

\noindent\textit{\textbf{Énoncé :} Tout ensemble $x$ non vide possède un élément $y$ n'ayant aucun élément commun avec $x$: \\
$\forall \, x, \; \left[ x \neq \emptyset \Rightarrow (\exists y \, y \in x \, \wedge \, y \cap x = \emptyset) \right]$.} 

\medskip

\noindent\textbf{Corrolaire 1 :} Aucun ensemble ne peut être un élément de lui-même. 

\medskip

\noindent\textbf{Démonstration :} Soit $y$ un ensemble quelconque, et considérons l'ensemble $x = \lbrace y \rbrace$. 
(Cet ensemble existe d'après l'axiome de la paire : il s'agit de la paire formée par $y$ et lui-même.)
Alors, $x$ est non vide et ne contient qu'un élément ($y$). 
D'après l'axiome de fondation, on a donc $y \cap x = \emptyset$. 
Puisque $y \in x$, cela implique $y \notin y$ (sans quoi on aurait $y \in y \cap x$). 

\done

\medskip

\noindent\textbf{Corrolaire 2 :} Soit deux ensembles $x$ et $y$. Si $x \in y$, alors $y \notin x$. 

\medskip

\noindent\textbf{Démonstration :} Soit $x$ et $y$ deux ensembles tels que $x \in y$. 
Considérons l'ensemble $z = \lbrace x,y  \rbrace$ (qui existe d'après l'axiome de la paire). 
L'ensemble $z$ est non vide et ne contient que les éléments $x$ et $y$. 
Donc, d'après l'axiome de fondation, $x \cap z = \emptyset$ ou $y \cap z = \emptyset$. 
Mais $x \in y$, donc $x \in (y \cap z)$, donc la formule $y \cap z = \emptyset$ est fausse. 
On a donc $x \cap z = \emptyset$, et donc, puisque $y \in z$, $y \notin x$. 

\done

\subsubsection{Couples} 

\noindent\textbf{Définition :} Soit deux ensembles $x$ et $y$. D'après l'axiome de la paire, $\lbrace x \rbrace$ et $\lbrace x, y \rbrace$ existent. En utilisant à nouveau l'axiome de la paire, l'ensemble $\lbrace \lbrace x \rbrace, \lbrace x, y \rbrace \rbrace$ existe. On l'appelle le \textit{couple} de $x$ et $y$, noté $(x,y)$. 

\medskip

\noindent\textbf{Lemme :} Soit $a$, $b$, $c$ et $d$ quatre ensembles tels que $(a,b) = (c,d)$.
    Alors $a = c$ et $b = d$.

\medskip

\noindent\textbf{Démonstration :} On distingue deux cas selon que $a$ et $b$ sont égaux ou non. 
    Supposons d'abord que $a=b$. Alors, $(a,b) = \lbrace \lbrace a \rbrace \rbrace$. 
    Puisque $\lbrace c \rbrace \in (c,d)$ et $(c,d) = (a,b)$, on en déduit que $\lbrace c \rbrace \in \lbrace `lbrace a \rbrace \rbrace$ et donc $\lbrace c \rbrace = \lbrace a \rbrace$.
    Donc, $c \in \lbrace a \rbrace$, et donc $c = a$.
    Par ailleurs, $\lbrace c, d \rbrace \in (c,d)$, donc $\lbrace c, d \rbrace = \lbrace a \rbrace$, et donc $d = a$.
    Puisque $b = a$, on a donc bien $c = a$ et $d = b$.
    
    Supposons maintenant $a \neq b$.
    Puisque $\lbrace c \rbrace \in (c,d)$, et $(c,d) = (a,b)$, on a $\lbrace c \rbrace = \lbrace a \rbrace$ ou $\lbrace c \rbrace = \lbrace a, b \rbrace$.
    Montrons que la seconde égalité est impossible. 
    Si elle était vraie, puisque $a \in \lbrace a, b \rbrace$, on aurait $a \in \lbrace c \rbrace$, donc $a = c$, et, puisque $b \in \lbrace a,b \rbrace$, on aurait $b \in \lbrace c \rbrace$, donc $b = c$, et donc (par symmétrie et transitivité de l'égalité) $b = a$, ce qui est impossible $a \neq b$.
    Ainsi, $\lbrace c \rbrace = \lbrace a, b \rbrace$ est nécessairement fausse, et donc $\lbrace c \rbrace = \lbrace a \rbrace$. 
    Donc, $c \in \lbrace a \rbrace$, et donc $c = a$. 
    
    Puisque $\lbrace a,b \rbrace \in (a,b)$ et $(a,b)=(c,d)$, on a $\lbrace a,b \rbrace \in (c,d)$.
    Donc, $\lbrace a,b \rbrace = \lbrace c \rbrace$ ou $\lbrace a,b \rbrace = \lbrace c, d \rbrace$.
    On vient de voir que la première égalité est fausse, donc $\lbrace a,b \rbrace = \lbrace c, d \rbrace$.
    Donc, $b \in \lbrace c, d \rbrace$. 
    Donc, $b = c$ ou $b = d$. 
    Puisque $a = c$ et $b \neq a$, la première égalité est fausse. 
    Donc, $b = d$. 

   \done 

\medskip

Soit $x$ et $y$ deux ensembles et $z = (x,y)$. 
On dit parfois que $x$ est la \textit{première composante} de $z$ et $y$ sa \textit{deuxième composante}, ou \textit{seconde composante}.

\subsubsection{Produit Cartésien}

Soit $a$ et $b$ deux ensembles, $c$ l'ensemble des parties de $a$ et $d$ l'ensemble des parties de $a \cup b$. 
Soit $e$ l'ensemble des parties de $c \cup d$. 
Soit $P$ le prédicat à une variables $x$ définit par : 
\begin{equation*}
    P x: \exists \alpha \, \exists \beta \, (\alpha \in a) \wedge (\beta \in b) \wedge (x = (\alpha, \beta)).
\end{equation*}
On note $a \times b$ et on appelle \textit{produit Cartésien de $a$ et $b$} l'ensemble des éléments de $e$ satisfaisant la propriété $P$. 
Cet ensemble existe d'après le schéma d'axiomes de compréhension. 

Soit $a$ et $b$ deux ensembles et $c$ un sous-ensemble de $a \times b$. 
On appelle \textit{domaine} de $c$ l'ensemble $\lbrace x \in a \vert \exists y \, y \in b \wedge (x,y) \in c \rbrace$. 

\medskip

\noindent\textbf{Lemme :} Soit $a$, $b$, $a'$ et $b'$ quatre ensembles tels que $a' \subset a$ et $b' \subset b$. 
    Alors $a' \times b' \subset a \times b$.

\medskip

\noindent\textbf{Démonstration :}
    Soit $z$ un élément de $a' \times b'$. 
    On peut choisit un élément $x$ de $a'$ et un élément $y$ de $b'$ tels que $z = (x,y)$.
    Puisque $a'$ est un sous-ensemble de $a$, on a $x \in a$.
    Puisque $b'$ est un sous-ensemble de $b$, on a $y \in b$. 
    Donc, $(x,y) \in a \times b$.
    Donc, $z \in a \times b$.

   \done 

\medskip

\noindent\textbf{Lemme :} Soit $E$ un ensemble. 
    On a : $E \times \emptyset = \emptyset$ et $\emptyset \times E = \emptyset$.

\medskip

\noindent\textbf{Démonstration :} 
\begin{itemize}[nosep]
    \item Supposons par l'absurde qu'il existe un ensemble $z$ tel que $z \in E \times \emptyset$.
        Alors, il existe un élément $x$ de $E$ et un élément $y$ de $\emptyset$ tels que $z = (x,y)$. 
        Puisque $y \in \emptyset$ est faux pour tout ensemble $y$, cela est impossible.
        Ainsi, $z \in E \times \emptyset$ est faux pour tout ensemble $z$, et donc $E \times \emptyset = \emptyset$.
    \item Supposons par l'absurde qu'il existe un ensemble $z$ tel que $z \in \emptyset \times E$.
        Alors, il existe un élément $x$ de $\emptyset$ et un élément $y$ de $E$ tels que $z = (x,y)$. 
        Puisque $x \in \emptyset$ est faux pour tout ensemble $x$, cela est impossible.
        Ainsi, $z \in \emptyset \times E$ est faux pour tout ensemble $z$, et donc $\emptyset \times E = \emptyset$.
\end{itemize}

\done

\subsubsection{Graphe de relation binaire}

Soit $a$ et $b$ deux ensembles. 
Un \textit{graphe de relation binaire} sur $a$ et $b$ est un sous-ensemble de $a \times b$. 
À un graphe de relation binaire $G$ est associé une relation binaire $R$ définie par : $\forall a \, \forall b \, (a R b) \Leftrightarrow ((a,b) \in G)$. 
On dira alors que la relation $R$ est \textit{définie sur $a$ et $b$}.

\subsubsection{Relation d'ordre}

Soit $E$ un ensemble. 
Une relation binaire $\leq$ définie sur $E \times E$ est dite \textit{relation d'ordre} sur $E$ si elle satisfait les trois propriétés suivantes : 
\begin{itemize}[nosep]
    \item \textit{Réflexivité :} $\forall x \, x \in E \Rightarrow x \leq x$.
    \item \textit{Antisymétrie :} $\forall x \, \forall y \, x \in E \wedge y \in E \wedge (x \leq y) \wedge (y \leq x) \Rightarrow x = y$.
    \item \textit{Transitivité :} $\forall x \, \forall y \, \forall z \, x \in E \wedge y \in E \wedge z \in E \wedge (x \leq y) \mathrel{\wedge} (y \leq z) \Rightarrow x \leq z$.
\end{itemize}
Une relation d'ordre $\leq$ sur $E$ est dite \textit{relation d'ordre total} si la formule suivante est vraie : $\forall x \in E \, \forall y \in E \, (x \leq y) \vee (y \leq x)$.
Un élément $e$ de $E$ tel que : $\forall f \, f \in E \wedge f \leq e \Rightarrow f = e$ est dit \textit{minimal} (pour l'ensemble $E$ et pour la relation $\leq$) ; on dit aussi que $E$ admet $e$ pour élément minimal pour la relation $\leq$. 
Un élément $e$ de $E$ tel que $\forall x \in E \, e \leq x$ est dit \textit{plus petit élément}, ou \textit{minimum}, de $E$ (pour la relation $\leq$).
Un élément $e$ de $E$ tel que : $\forall f \, f \in E \wedge e \leq f \Rightarrow f = e$ est dit \textit{maximal} (pour l'ensemble $E$ et pour la relation $\leq$) ; on dit aussi que $E$ admet $e$ pour élément maximal pour la relation $\leq$. 
Un élément $e$ de $E$ tel que $\forall x \in E \, x \leq e$ est dit \textit{plus grand élément}, ou \textit{maximum}, de $E$ (pour la relation $\leq$).
Un ensemble muni d'une relation d'ordre est dit \textit{ordonné}.
Un ensemble muni d'une relation d'ordre total est dit \textit{totalement ordonné}.

\medskip

\noindent\textbf{Remarque :} Un ensemble a au plus un minimum et au plus un maximum. 

\medskip

\noindent\textbf{Démonstration :} Soit $E$ un ensemble et $\leq$ une relation d'ordre sur $E$. 
\begin{itemize}[nosep]
    \item Soit $a$ et $b$ deux minima de $E$ pour la relation $\leq$. 
        Alors $a \leq b$ (puisque $a$ est un minimum) et $b \leq a$ (puisque $b$ est un minimum).
        Donc, $a = b$.
    \item Soit $a$ et $b$ deux maxima de $E$ pour la relation $\leq$. 
        Alors $b \leq a$ (puisque $a$ est un maximum) et $a \leq x$ (puisque $b$ est un maximum).
        Donc, $a = b$.
\end{itemize}

\done

\medskip

\noindent\textbf{Lemme :} Soit $E$ un ensemble et $\leq$ une relation d'ordre sur $E$. 
    Alors, $E$ admet au plus un plus petit élément et au plus un plus grand élément. 

\medskip

\noindent\textbf{Démonstration :} 
    Soit $x$ et $y$ deux plus petits élément de $E$. 
    Alors, $x \leq y$ et $y \leq x$, donc $x = y$.
    
    Soit $x$ et $y$ deux plus grands élément de $E$. 
    Alors, $y \leq x$ et $x \leq y$, donc $x = y$.

    \done

\medskip

\noindent\textbf{Lemme :} Soit $E$ un ensemble et $\leq$ une relation d'ordre total sur $E$. 
    Alors, $E$ admet au plus un élément maximal et au plus un élément minimal pour la relation $\leq$. 

\medskip

\noindent\textbf{Démonstration :} 
    Soit $a$ et $b$ deux éléments maximaux de $E$ pour la relation $\leq$.
    Puisque $\leq$ est une relation d'ordre total sur $E$, $a \leq b$ ou $b \leq a$. 
    Puisque $a$ est un élément maximal, $a \leq b$ implique $b = a$.
    Puisque $b$ est un élément maximal, $b \leq a$ implique $a = b$.
    Donc, et puisque l'égalité est symmétrique, on a dans tous les cas $a = b$.
    Cela montre que $E$ admet au plus un seul élément maximal pour la relation $\leq$.
    
    Soit $a$ et $b$ deux éléments minimaux de $E$ pour la relation $\leq$.
    Puisque $\leq$ est une relation d'ordre total sur $E$, $a \leq b$ ou $b \leq a$. 
    Puisque $b$ est un élément minimal, $a \leq b$ implique $a = b$.
    Puisque $a$ est un élément minimal, $b \leq a$ implique $b = a$.
    Donc, et puisque l'égalité est symmétrique, on a dans tous les cas $a = b$.
    Cela montre que $E$ admet au plus un seul élément minimal pour la relation $\leq$.

   \done 

\medskip

\noindent \textbf{Remarques :} 
\begin{itemize}[nosep]
    \item Un élément minimal d'un ensemble totalement ordonné est aussi le minimum de cet ensemble.
    \item Un élément maximal d'un ensemble totalement ordonné est aussi le maximum de cet ensemble.
\end{itemize}

\medskip

\noindent\textbf{Lemme :} Soit $E$ un ensemble et $\leq$ une relation d'ordre sur $E$. 
    Soit $e$ un élément de $E$ tel que : $\forall x \in E \, e \leq x$. 
    Alors $e$ est un élément minimal de $E$ pour $\leq$.

\medskip

\noindent\textbf{Démonstration :} 
    Soit $x$ un élément de $E$ tel que $x \leq e$. 
    On a $(x \leq e) \wedge (e \leq x)$. 
    Par antisymétrie de la relation $\leq$, on en déduit $x = e$.

   \done 

\medskip

\noindent\textbf{Lemme :} Soit $E$ un ensemble et $\leq$ une relation d'ordre total sur $E$. 
    Soit $e$ un élément de $E$.
    Alors, le prédicat $\forall f \in E \, e \leq f$ est équivalent à dire que $e$ est l'élément minimal de $E$.

\medskip

\noindent\textbf{Démonstration :}
\begin{itemize}[nosep]
    \item Supposons le prédicat $\forall f \in E \, e \leq f$ vrai. 
        Soit $f$ un élément de $E$ tel que $f \leq e$. 
        On a alors $e \leq f$ et $f \leq e$, donc $f = e$ par antisymétrie de la relation $\leq$. 
        Ainsi, $e$ est un élément minimal de $E$ pour $\leq$.
        Puisque $\leq$ est une relation d'ordre total, cet élément minimal est unique.
    \item \textit{(Nous adoptions ici une approche un brin pédestre.)} 
        Supposons que $e$ est l'élément minimal de $E$ pour $\leq$.
        Soit $f$ un élément de $E$. 
        Puisque $\leq$ est une relation d'ordre total, $e \leq f \vee f \leq e$ est vrai.
        Puisque $e$ est l'élément minimal de $E$ pour $\leq$, $f \leq e \Rightarrow f = e$. est vrai. 
        \textit{(Ici, on pourrait directement conclure que, puisque $f \leq e$ implique $f = e$ et donc $e \leq f$, la première formule est équivalente à $e \leq f$. Dans la suite, nous montrons cela plus formellement \emph{via} le calcul des prédicats.)}
        Cette dernière formule peut se récrire en : $f = e \vee \neg (f \leq e)$.
        La conjonction de ces deux prédicats donne : $(e \leq f \vee f \leq e) \wedge (f = e \vee \neg (f \leq e))$.
        En développant cette formule, il vient : $(e \leq f \wedge f = e) \vee (e \leq f \wedge \neg (f \leq e)) \vee (f \leq e \wedge f = e) \vee (f \leq e \wedge \neg (f \leq e))$.
        Cette formule peut être simplifiée en : $(f = e) \vee (e \leq f \wedge \neg (f \leq e)) \vee (f = e) \vee \mathsf{F}$, 
        ou en $(f = e) \vee (e \leq f \wedge \neg (f \neq e))$.
        Cette formule ne peut être vraie que si $e \leq f$ (sans quoi $f = e$ et $e \leq f$ deraient fausses).
        Donc, $e \leq f$.
        Nous avons donc montré que $\forall f \in E \, e \leq f$ est vrai.
\end{itemize}
    
   \done 

\medskip

\noindent\textbf{Lemme :} Soit $E$ un ensemble et $\leq$ une relation d'ordre sur $E$. 
    La relation $\geq$ sur $E$ définie par : $\forall x \, \forall y \, x \in E \wedge y \in E \Rightarrow (x \geq y \Leftrightarrow y \leq x)$ est une relation d'ordre sur $E$. 
    En outre, si $\leq$ est une relation d'ordre total, alors $\geq$ l'est aussi.

\medskip

\noindent\textbf{Démonstration :}
\begin{itemize}[nosep]
    \item \textit{Réflexivité :} Soit $x$ un élément de $E$. On a $x \leq x$ par réflexivité de la relation $\leq$, donc $x \geq x$. 
    \item \textit{Antisymétrie :} Soit $x$ et $y$ deux éléments de $E$ tels que $x \geq y$ et $y \geq x$.
        Alors, $y \leq x$ et $x \leq y$.
        Par antisymétrie de la relation $\leq$, on en déduit que $x = y$.
    \item \textit{Transitivité :} Soit $x$, $y$ et $z$ trois éléments de $E$ tels que $x \geq y$ et $y \geq z$.
        Alors, $y \leq x$ et $z \leq y$.
        Par transitivité de la relation $\leq$, on en déduit que $z \leq x$, et donc $x \geq z$.
    \item Supposons que $\leq$ est une relation d'ordre total.
        Soit $x$ et $y$ deux éléments de $E$. 
        Alors, $x \leq y$ ou $y \leq x$.
        Donc, $y \geq x$ ou $x \geq y$.
\end{itemize}

\done

\medskip

Soit $E$ un ensemble, $\leq$ une relation d'ordre total sur $E$ et $F$ un sous-ensemble de E. 
On dit que $F$ est \textit{borné supérieurement} (dans $E$ et pour la relation $\leq$) s'il existe un élément $m$ de $E$ tel que : $\forall e \, (e \in F) \Rightarrow (e \leq m)$. 
On dit alors que cet élément est une \textit{borne supérieure} de $F$ (dans $E$ et pour la relation $\leq$).
On dit que $F$ est \textit{borné inférieurement} (dans $E$ et pour la relation $\leq$) s'il existe un élément $m$ de $E$ tel que : $\forall e \, (e \in F) \Rightarrow (m \leq e)$.
On dit alors que cet élément est une \textit{borne inférieure} de $F$ (dans $E$ et pour la relation $\leq$).
 
Une relation binaire $<$ antisymétrique, transitive et telle que $\forall x \, x \in E \Rightarrow \neg (x < x)$ (antireflexivité) est dite \textit{relation d'ordre strict}. 
(cette dernière propriété et l'antisymmétrie impliquent qu'il n'existe pas d'éléments $x$ et $y$ de $E$ tels que $(x < y) \wedge (y < x)$.)
Si $\leq$ est une relation d'ordre sur un ensemble $E$, alors la relation $<$ définie par : pour tout éléments $a$ et $b$ de $E$, $a < b \Leftrightarrow (a \leq b) \wedge (a \neq b)$ est une rélation d'ordre strict.
En effet, 
\begin{itemize}[nosep]
    \item Soit $x$ un élément de $E$, $x \neq x$ est fausse, donc $x < x$ est fausse. 
    \item Soit $x$ et $y$ deux éléments de $E$ tels que $x < y$ et $y < x$, alors $x \leq y$ et $y \leq x$, donc $x = y$. 
        La relation $<$ est bien antisymétrique. 
    \item Soit $x$, $y$ et $z$ trois éléments de $E$ lets que $x < y$ et $y < z$. 
        Alors $x \leq y$ et $y \leq z$, donc $x \leq z$. 
        Par ailleurs, si on avait $x=z$, alors $y \leq x$, et donc $y = x$, ce qui est impossible puisque $x < y$. 
        Donc, $x \neq z$. 
        On en déduit que $x < z$. 
        Ainsi, la relation $<$ est bien transitive.
\end{itemize}

\medskip

\noindent\textbf{Lemme :} Soit $E$ un ensemble et $\leq$ une relation d'ordre sur $E$. 
    La relation $<$ sur $E$ définie par : $\forall x \, \forall y \, x \in E \wedge y \in E \Rightarrow (x < y \Leftrightarrow (y \leq x \wedge x \neq y))$ est une relation d'ordre strict sur $E$. 

\medskip

\noindent\textbf{Démonstration :} 
\begin{itemize}[nosep]
    \item \textit{Antiréflexivité :} Soit $x$ un élément de $E$. 
        Puisque $x = x$, la formule $x \neq x$ est fausse, donc $x < x$ est fausse.
    \item \textit{Antisymétrie :} Soit $x$ et $y$ deux éléments de $E$ tels que $x < y$ et $y < x$. 
        Alors, $x \leq y$ et $y \leq x$. 
        Puisque $\leq$ est une relation d'ordre, cela implique $x = y$.
    \item \textit{Transitivité :} Soit $x$, $y$ et $z$ trois éléments de $E$ tels que $x < y$ et $y < z$. 
        On a $x \leq y$ et $y \leq z$. 
        Puisque $\leq$ est une relation d'ordre, cela implique $x \leq z$. 
        Par ailleurs, $z$ ne peut pas être égal à $x$ car on aurait alors $x \leq y$ et $y \leq x$, d'où $y = x$, ce qui est incompatible avec $x < y$. 
        Donc, $x \leq z$ est fausse, et donc $x < z$ est vraie.
\end{itemize}

\hfill\square

\medskip

\noindent\textbf{Lemme :} Soit $E$ un ensemble et $<$ une relation d'ordre strict sur $E$. 
    La relation $\leq$ sur $E$ définie par : $\forall x \, \forall y \, x \in E \wedge y \in E \Rightarrow (x \leq y \Leftrightarrow (y \leq x \vee x = y))$ est une relation d'ordre sur $E$. 

\medskip

\noindent\textbf{Démonstration :} 
\begin{itemize}[nosep]
    \item \textit{Réflexivité :} Soit $x$ un élément de $E$. 
        Puisque $x = x$ est vrai par réflexivité de l'égalité, $x \leq x$ est vrai.
    \item \textit{Antisymétrie :} Soit $x$ et $y$ deux éléments de $E$ tels que $x \leq y$ et $y \leq x$. 
        Alors, $x < y$ ou $x = y$. 
        De même, $y < x$ ou $x = y$. 
        Puisque $x < y$ et $y < x$ ne peuvent être simultanément vrais, on en déduit que $x = y$.
    \item \textit{Transitivité :} Soit $x$, $y$ et $z$ trois éléments de $E$ tels que $x \leq y$ et $y \leq z$. 
        On a $x < y$ ou $x = y$.
        Dans le second cas, le second prédicat de l'hypothèe donne $x \leq z$.
        Supposons maintenant $x < y$. 
        On a de même $y < z$ ou $y = z$. 
        Dans le second cas, le premier prédicat de l'hypothèe donne $x \leq z$.
        Supposons maintenant $y < z$. 
        Puisque $x < y$, $y < z$, et car $<$ est une relation d'ordre strict, donc transitive, on en déduit $x < z$, et donc $x \leq z$. 
        Le prédicat $x \leq z$ est donc vrai dans tous les cas.
\end{itemize}

\hfill\square

\medskip

\noindent\textbf{Lemme :} Soit $E$ un ensemble, $\leq$ une relation d'ordre sur $E$, et $<$ la relation d'ordre strict sur $E$ définie par : $\forall x \forall y \, (x \in E \wedge y \in E) \Rightarrow (x < y \Leftrightarrow (x \leq y \wedge x \neq y))$.
Alors, soit $x$, $y$ et $z$ trois éléments de $E$, 
\begin{itemize}[nosep]
    \item Si $x < y$ et $y \leq z$, alors $x < z$.
    \item Si $x \leq y$ et $y < z$, alors $x < z$.
\end{itemize}

\medskip

\noindent\textbf{Démonstration :} Notons d'abord que, dans les deux cas, on a $x \leq y$ et $y \leq z$, donc $x \leq z$ par transitivité de la relation $\leq$. 
    In suffit donc de montrer que $x \neq z$. 
    Supposons par l'absurde que $x = z$. 
    Alors,  
    \begin{itemize}[nosep]
        \item Dans le premier cas, on a $x < y$, donc $x \leq y$, et $y \leq x$. 
            On a donc $y = x$.
            Mais cela est incompatible avec $x < y$.
        \item Dans le second cas, on a $x \leq y$ et $y < x$, donc $y \leq x$.
            On a donc $y = x$.
            Mais cela est incompatible avec $y < x$.
    \end{itemize}
    Dans les deux cas, la formule $x = z$ est donc nécessairement fausse, donc $x \neq z$ est vraie.

   \done 

\medskip

\noindent\textbf{Lemme :} Soit $E$ un ensemble, $\leq$ une relation d'ordre sur $E$, et $<$ la relation d'ordre strict sur $E$ définie par : $\forall x \forall y \, (x \in E \wedge y \in E) \Rightarrow (x < y \Leftrightarrow (x \leq y \wedge x \neq y))$.
Alors, soit $x$ et $y$ deux éléments de $E$, les formules $x \leq y$ et $(x < y) \vee (x = y)$ sont équivalentes.

\medskip

\noindent\textbf{Démonstration :} 
    Puisque la formule $(x = y) \vee (x \neq y)$ est toujours vraie, on a : $(x \leq y) \Leftrightarrow ((x \leq y) \wedge ((x = y) \vee (x \neq y)))$. 
    Utilisant la distributivité de $\wedge$ sur $\vee$, cela donne : $(x \leq y) \Leftrightarrow (((x \leq y) \wedge (x = y)) \vee ((x \leq y) \wedge (x \neq y)))$.
    Puisque la relation $\leq$ est réflexive, $(x = y) \Rightarrow (x \leq y)$, donc $(x \leq y) \wedge (x = y)$ est équivalente à $x = y$.
    En outre, par définition de la relation $<$, $(x \leq y) \wedge (x \neq y)$ est équivalente à $x < y$. 
    Donc, $(x \leq y) \Leftrightarrow ((x = y) \vee (x < y))$.

   \done 

\medskip

\noindent\textbf{Lemme :} Soit $E$ un ensemble et $<$ une relation d'ordre strict sur $E$. 
    La relation $>$ sur $E$ définie par : $\forall x \, \forall y \, x \in E \wedge y \in E \Rightarrow (x > y \Leftrightarrow y < x)$ est une relation d'ordre strict sur $E$. 

\medskip

\noindent\textbf{Démonstration :}
\begin{itemize}[nosep]
    \item Soit $x$ un élément de $E$. Le prédicat $x < x$ est faux puisque $<$ est une relation d'ordre struct, donc $x > x$ l'est aussi. 
    \item \textit{Antisymétrie :} Soit $x$ et $y$ deux éléments de $E$ tels que $x > y$ et $y > x$.
        Alors, $y < x$ et $x < y$.
        Par antisymétrie de la relation $<$, on en déduit que $x = y$.
    \item \textit{Transitivité :} Soit $x$, $y$ et $z$ trois éléments de $E$ tels que $x > y$ et $y > z$.
        Alors, $y < x$ et $z < y$.
        Par transitivité de la relation $<$, on en déduit que $z < x$, et donc $x > z$.
\end{itemize}

\done

\medskip

Soit $E$ un ensemble, $\leq$ une relation d'ordre sur $E$ et $<$ la relation d'ordre strict définie par : pour tout éléments $a$ et $b$ de $E$, $a < b \Leftrightarrow (a \leq b) \wedge (a \neq b)$. 
Alors, soit $a$, $b$ et $c$ trois éléments de $E$ tels que $a \leq b$ et $b < c$, on a $a < c$. 
En effet, on a $a \leq c$ par transitivité de la relation $\leq$ et $a \neq c$ (sans quoi on aurait $b < a$, et donc $b \leq a$, donc $b = a$, ce qui est contradictoire avec $b < a$). 

\medskip

\noindent\textbf{Lemme :} Soit $E$ un ensemble et $\leq$ une relation d'ordre total définie sur $E$. 
    Alors la relation $>$ définie sur $E$ par : pour tous éléments $x$ et $y$ de $E$, $a > b \Leftrightarrow \neg (a \leq b)$ est une relation d'ordre strict. 

\medskip

\noindent\textbf{Démonstration :} 
\begin{itemize}[nosep]
    \item \textit{Antiréflexivité :} Soit $x$ un élément de $E$. 
        La formule $x \leq x$ est vraie, donc $x > x$ est fausse. 
    \item \textit{Antisymétrie :} Soit $x$ et $y$ deux éléments de $E$ tels que $x > y$ et $y > x$. 
        Alors, $\neg (x \leq y)$ et $\neg (y \leq x)$. 
        Puisque $\leq$ est une relation d'ordre total, cela implique $y \leq x$ et $x \leq y$, et donc $x = y$.
    \item \textit{Transitivité :} Soit $x$, $y$ et $z$ trois éléments de $E$ tels que $x > y$ et $y > z$. 
        On a $\neg (x \leq y)$ et $\neg (y \leq z)$. 
        Puisque $\leq$ est une relation d'ordre total, cela implique $y \leq x$ et $z \leq y$, et donc $z \leq x$. 
        Par ailleurs, $z$ ne peut pas être égal à $x$ car on aurait alors $y \leq x$ et $x \leq y$, d'où $y = x$, ce qui est incompatible avec $x > y$. 
        Donc, $x \leq z$ est fausse, et donc $x > z$.
\end{itemize}

\done

\medskip

\noindent\textbf{Lemme :} Soit $E$ un ensemble et $<$ une relation d'ordre strict définie sur $E$, telle que : $\forall x \in E \, \forall y \in E (x < y) \vee (y < x) \vee (x = y)$. 
    Alors la relation $\geq$ définie sur $E$ par : pour tous éléments $x$ et $y$ de $E$, $a \geq b \Leftrightarrow \neg (a < b)$ est une relation d'ordre total. 

\medskip

\noindent\textbf{Démonstration :} 
\begin{itemize}[nosep]
    \item \textit{Réflexivité :} Soit $x$ un élément de $E$. 
        La formule $x < x$ est fausse, donc $x \geq x$ est vraie. 
    \item \textit{Antisymétrie :} Soit $x$ et $y$ deux éléments de $E$ tels que $x \geq y$ et $y \geq x$. 
        Alors, $\neg (x < y)$ et $\neg (y < x)$. 
        Donc, $x = y$.
    \item \textit{Transitivité :} Soit $x$, $y$ et $z$ trois éléments de $E$ tels que $x \geq y$ et $y \geq z$. 
        On a $\neg (x < y)$ et $\neg (y < z)$. 
        Donc, $(y < x) \vee (x = y)$ et $(z < y) \vee (y = z)$. 
        Si $x = y$, alors $y \leq z$ implique $x \leq z$.
        Si $y = z$, alors $x \leq y$ implique $x \leq y$.
        Si $x \neq y$ et $y \neq z$, on a $y < x$ et $z < y$. 
        Par trannsitivité de la relation $<$, on a donc $z < x$. 
        Par antisymétrie, on a donc $\neq (x < z)$, et donc $x \geq z$. 
        La formule $x \geq z$ est ainsi vraie dans tous les cas. 
    \item Soit $x$ et $y$ deux éléments de $E$. 
        On a $(x < y) \vee (y < x) \vee (x = y)$. 
        Si $x < y$ est vraie, alors $y < x$ est fausse, donc $y \geq x$ est vraie.
        Si $y < x$ est vraie, alors $x < y$ est fausse, donc $x \geq y$ est vraie.
        Enfin, si $x = y$ est vraie, alors $x \leq y$ est vraie. 
        Dans tous les cas, on a bien $(x \leq y) \vee (y \leq x)$.
\end{itemize}

\medskip

\noindent\textbf{Vocabulaire :} Soit $E$ un ensemble et $\leq$ une relation d'ordre sur $E$. 
    Soit $\geq$, $<$ et $>$ les relations définies par : pour tous éléments $a$ et $b$ de $E$, 
    \begin{itemize}[nosep]
        \item $a \geq b \Leftrightarrow b \geq a$,
        \item $a < b \Leftrightarrow ( a \leq b \wedge a \neq b )$,
        \item $a > b \Leftrightarrow b < a$.
    \end{itemize}
    Alors, soit $a$ et $b$ deux éléments de $E$, et s'il n'y a pas d'ambiguité, 
    \begin{itemize}[nosep]
        \item si $a \leq b$, on dira que \textit{$a$ est inférieur ou égal à $b$}, 
        \item si $a \geq b$, on dira que \textit{$a$ est supérieur ou égal à $b$}, 
        \item si $a < b$, on dira que \textit{$a$ est strictement inférieur à $b$}, 
        \item si $a > b$, on dira que \textit{$a$ est strictement supérieur à $b$}.
    \end{itemize}

\medskip

\noindent\textbf{Notation :} Soit $E$ un ensemble, $\geq$ une relation d'ordre sur $E$, et $<$ la relation d'ordre strict sur $E$ définie par : pour tous éléments $a$ et $b$ de $E$, $a < b \Leftrightarrow ( a \leq b \wedge a \neq b )$. 
    Si $a_0$, $a_1$, $a_2$, ..., $a_n$ sont des éléments de $E$ (avec $a_n$ possiblement absent) et $R_1$, $R_2$, ..., $R_n$ (où $R_n$ est absent si $a_n$ l'est) des symboles chacun identique à $\leq$ ou $<$, alors la formule
    \begin{equation*}
        a_0 \mathrel{R_1} a_1 \mathrel{R_2} a_2 \dots
    \end{equation*}
    signifie :
    \begin{equation*}
        (a_0 \mathrel{R_1} a_1) \wedge (a_1 \mathrel{R_2} a_2) \dots .
    \end{equation*}

\medskip

\noindent\textbf{Définition :} Soit $E$ un ensemble et $\leq$ une relation d'ordre sur $E$. 
    La relation $\leq$ est dit un \textit{bon ordre} sur $E$ si tout sous-ensemble non vide de $E$ admet un plus petit élément.
    L'ensemble $E$ est alors dit \textit{bien ordonné}.

\medskip

\noindent\textbf{Lemme :} Soit $E$ un ensemble et $\leq$ un bon ordre sur $E$. 
    Alors $\leq$ est une relation d'ordre total sur $E$.

\medskip

\noindent\textbf{Démonstration :}
    Soit $x$ et $y$ deux éléments de $E$. 
    Alors, $\lbrace x, y \rbrace$ est un sous-ensemble non vide de $E$ (il contient au moins $x$).
    Donc, il contient un plus petit élément. 
    Si ce plus petit élément est $x$, alors $x \leq y$.
    Sinon, ce plus petit élément est $y$, donc $y \leq x$.
    Dans tous les cas, on a $x \leq y \vee y \leq x$.

    \done

\subsubsection{Partition} 

Soit $E$ et $P$ deux ensembles. 
On dit que \textit{$P$ est une partition de $E$} si les quatre propriétés suivantes sont satisfaites : 
\begin{itemize}[nosep]
    \item $\forall p \, (p \in P) \Rightarrow (p \subset E)$, 
    \item $\emptyset \notin P$, 
    \item $\forall e \, (e \in E) \Rightarrow (\exists p \, p \in P \wedge e \in p)$
    \item $\forall p \, \forall q \, (p \in P) \wedge (q \in P) \wedge ((p \cap q) \neq \emptyset) \Rightarrow (p = q)$. 
\end{itemize}

\subsubsection{Relation d'équivalence}

Soit $E$ un ensemble. 
Une relation binaire $\sim$ définie sur $E \times E$ est dite \textit{relation d'équivalence} sur $E$ si elle satisfait les trois propriétés suivantes : 
\begin{itemize}[nosep]
    \item \textit{Réflexivité :} $\forall x \, x \in E \Rightarrow x \sim x$
    \item \textit{Symétrie :} $\forall x \, \forall y \, (x \in E) \wedge (y \in E) \wedge (x \sim y) \Rightarrow (y \sim x)$.
    \item \textit{Transitivité :} $\forall x \, \forall y \, \forall z \, (x \in E) \wedge (y \in E) \wedge (z \in E) \wedge (x \sim y) \wedge (y \sim z) \Rightarrow (x \sim z)$. 
\end{itemize}

Soit $E$ un ensemble et $\sim$ une relation d'équivalence sur $E$. 
Pour tout $x \in E$, on définit la \textit{classe d'équivalence} de $x$ pour $\sim$, notée ici $[x]$, par : $[x] = \lbrace y \in E \vert y \sim x \rbrace$. 
Notons que, pour tout élément $x$ de $E$, $[x] \subset E$. 
Donc, l'ensemble des classes d'équivalences existe d'après le schéma d'axiomes de compréhensions. 
(Pour voir cela, prendre pour ensemble l'ensemble des parties de $E$ et pour propriété $P y: \exists x \, (x \in E) \wedge (y = [x])$.)

\medskip

\noindent \textbf{Lemme :} Soit $x$ et $y$ deux éléments de $E$. Si $x \sim y$, alors $[x] = [y]$. 

\medskip

\noindent \textbf{Démonstration :} 
    Suppsosons $x \sim y$. Soit $z \in [x]$. On a $z \sim x$. Par symétrie et transitivité de la relation $\sim$, on en déduit $z \sim y$. Donc, $z \in [y]$. 
    On en déduit $[x] \subset [y]$.
    Par symétrie, on a aussi $y \sim x$, et donc, en utilisant le même argument et échangeant les rôles de $x$ et $y$, on montre que $[y] \subset [x]$. 
    Ainsi, $[y] = [x]$.

\done

\medskip

\noindent \textbf{Lemme :} L'ensemble des classes d'équivalence de $E$ pour la relation $\sim$ forme une partition de $E$.

\medskip

\noindent \textbf{Démonstration :} 
Notons $F$ cet ensemble. 
Vérifions qu'il satisfait les quatre propriétés d'une partition de $E$.
\begin{itemize}
    \item Soit $f \in F$. On peut choisir un élément $y$ de $E$ tel que $f = [y]$. Puisque $[y] \subset E$, on en déduit $f \subset E$.
    \item Pour tout élément $f$ de $F$, il existe $x$ tel que $x \in E$ et $f = [x]$, et donc $x \in f$, ce qui montre que $f \neq \emptyset$. 
        Donc, $\emptyset \notin F$. 
    \item Soit $x \in E$. On a $x \in [x]$ et $[x] \in F$. 
    \item Soit $f \in F$ et $g \in F$ tels que $f \cap g \neq \emptyset$. 
        On peut choisir un élément $x$ de $f \cap g$. 
        Soit $y \in E$ et $z \in E$ tels que $f = [y]$ et $g = [z]$. 
        On a $x \sim y$ et $x \sim z$. 
        Par symétrie et transitivité de la relation $\sim$, on en déduit $y \sim z$. 
        Donc, $[y] = [z]$, et donc $f = g$.
\end{itemize}

\done

\subsubsection{Fonctions}
\label{sub:fonctions}

Soit $a$ un ensemble. 
La séquence de symboles « $\forall x \, (x \in a) \Rightarrow$ » incluse dans une formule est parfois simplifiée en « $\forall x \in a$ » ou en « $\forall x \in a,$ ».
La séquence de symboles « $\exists x \, (x \in a) \, \wedge$ » incluse dans une autre formule est parfois simplifiée en « $\exists x \in a$ » ou en « $\exists x \in a,$ ». 
Ainsi, si $f$ est une formule, la formule $\forall x \in a, f$ (éventuellement sans la virgule) est considérée comme identique à $\forall x \, (x \in a) \Rightarrow f$ (au sens où ces suites de symboles représentent la même formule) et $\exists x \in a, f$ (éventuellement sans la virgule) est considérée comme identique à $\exists x \, (x \in a) \wedge f$.

\medskip

\noindent\textbf{Définition :} Soit deux ensembles $X$ et $Y$. Une \textit{fonction}, ou \textit{application}, $f$ de $X$ vers $Y$ (ou de $X$ dans $Y$, ou de $X$ sur $Y$) est un ensemble (parfois appelé \textit{graphe}) tel que : 
\begin{equation*}
    \forall z \, [ (z \in f) \Rightarrow (\exists x \, \exists y \, [(x \in X) \wedge (y \in Y) \wedge (z = (x,y))])],
\end{equation*}
\begin{equation*}
    \forall x \, [(x \in X) \Rightarrow [\exists y \, (x,y) \in f]]
\end{equation*}
et 
\begin{equation*}
    \forall y \, \forall y' \, ([\exists x \, ((x,y) \in f \wedge (x,y') \in f)] \Rightarrow (y = y')).
\end{equation*}
La première condition est équivalente à dire que $f$ est un sous-ensemble de $X \times Y$, \textit{i.e.}, à : $f \subset X \times Y$. 
La seconde et la troisième sont équivalentes à dire que, pour tout élément $x$ de $X$, il existe un unique élément $y$ de $Y$ tel que $(x,y) \in f$, c'est-à-dire : $\forall x \, [(x \in X) \Rightarrow [\exists ! y \, (x,y) \in f]]$. 
Avec ces mêmes notations, pour tout $x$ appartenant à $X$, on note $f(x)$ (ou, quand il n'y a pas d'ambiguité, $f x$) l'unique élément $y$ de $Y$ tel que $(x,y) \in f$. 
On dit alors que $y$ est l'\textit{image} de $x$ ou que $x$ est un \textit{antécédent} de $y$ par $f$. 
On dit aussi que $f$ \textit{associe} $y$ à $x$. 

On dit que $f$ est \textit{définie sur} $X$, ou que $X$ est le \textit{domaine de définition} de $f$. 
La notation $f: X \to Y$, signifie que $f$ est une fonction de $X$ vers $Y$. 

Soit $X$ et $Y$ deux ensembles. 
L'ensemble des fonctions de $X$ vers $Y$ existe : il s'agit du sous-ensemble de l'ensemble des parties de $X \times Y$ (qui existe d'après l'axiome de l'ensemble des parties) satisfaisant la seconde et la troisième conditions ci-dessus (qui existe donc d'après le schéma d'axiomes de compréhension)%
\footnote{Pour être tout à fait rigoureux, le prédictat à employer pour utiliser l'axiome de compréhension est la conjonction de ces deux conditions, qui peut s'écrire : $(\forall x \, [(x \in X) \Rightarrow [\exists y \, (x,y) \in f]]) \wedge [\forall w \, \forall w' \, ([\exists z \, ((z,w) \in f \wedge (z,w') \in f)] \Rightarrow (w = w'))]$.}%
. 
Cet ensemble est noté $\mathcal{F}(X,Y)$, ou parfois (quand il n'y a pas d'ambiguité) $Y^X$. 
Notons que, si deux fonctions $f$ et $g$ de $X$ vers $Y$ satisfont $\forall x \in X, \, f(x) = g(x)$, alors $f = g$. 
Une fonction $f$ de $X$ vers $Y$ peut ainsi être définie de manière unique par la donnée de $f(x)$ pour tout élément $x$ de $X$.

\medskip

\noindent\textbf{Lemme :} Soit $E$ et $F$ deux ensembles non vides et $P$ un prédicat à deux paramètres libres tel que, pour tout élément $e$ de $E$, il existe un unique élément $f$ de $F$ tel que $P e f$ est vrai, \textit{i.e.}, 
\begin{equation*}
    \forall e \in E, \, 
        (\exists f \in F, \, P e f)
        \wedge 
        (\forall f \in F, \, \forall g \in F, \, P e f \wedge P e g \Rightarrow f = g) .
\end{equation*}
Alors l'ensemble $G$ définit par $G = \lbrace g \in E \times F \vert \exists e \in E \, \exists f \in F \, g = (e,f) \wedge P e f \rbrace$ est une fonction de $E$ vers $F$. 

\medskip

\noindent\textbf{Démonstration :} Montrons que l'ensemble $G$ satisfait les trois conditions pour être une fonction de $E$ vers $F$.
\begin{itemize}[nosep]
    \item Soit $g$ un élément de $G$. 
        Par définition de cet ensemble, on peut choisir un élément $e$ de $E$ et un élément $f$ de $F$ tel que $g = (e,f)$. 
        Donc, $g \in E \times F$. 
        Cela montre que $G$ est un sous-ensemble de $E \times F$.
    \item Soit $e$ un élément de $E$.
        Par définition de $P$, on peut choisir un élément $f$ de $F$ tel que $P e f$ est vrai.
        Alors, $(e,f)$ est un élément de $G$.
    \item Soit $e$ un élément de $E$ et $y$ et $y'$ deux éléments de $F$ tels que $(e,f) \in G$ et $(e,f') \in G$.
        Alors, $P e f$ et $P e f'$ sont vrais.
        Donc, $f = f'$. 
\end{itemize}

\done

\medskip

\noindent\textbf{Lemme :} Soit $E$ et $F$ deux ensembles et $f$ et $g$ deux fonctions de $E$ vers $F$. 
    On suppose que : $\forall x \in E, \, f(x) = g(x)$ est vrai. 
    Alors, $f = g$.

\medskip

\noindent\textbf{Démonstration :} 
    Soit $z$ un élément de $f$. 
    On peut choisir un élément $x$ de $E$ et un élément $y$ de $F$ tel que $z = (x,y)$. 
    Puisque $x \in E$, on peut choisir un élément $y'$ de $F$ tel que $(x,y') \in g$. 
    On a alors $y = f(x)$ et $y' = g(x)$. 
    Puisque $f(x) = g(x)$, on en déduit $y' = y$. 
    Donc, $(x,y) \in g$, et donc $z \in g$. 
    Cela montre que $f \subset g$.
    
    Soit $z$ un élément de $g$. 
    On peut choisir un élément $x$ de $E$ et un élément $y$ de $F$ tel que $z = (x,y)$. 
    Puisque $x \in E$, on peut choisir un élément $y'$ de $F$ tel que $(x,y') \in f$. 
    On a alors $y = g(x)$ et $y' = f(x)$. 
    Puisque $g(x) = f(x)$, on en déduit $y' = y$. 
    Donc, $(x,y) \in f$, et donc $z \in f$. 
    Cela montre que $g \subset f$.

    On a donc bien $f = g$. 

   \done 

\medskip

Soit $E$ et $F$ deux ensembles et $f$ une fonction de $E$ vers $F$. 
On dit que
\begin{itemize}[nosep]
    \item $f$ est \textit{injective} (ou \textit{une injection}) si $\forall x \; \forall y \, [f(x) = f(y) \Rightarrow x = y]$.
        Puisque chaque élément de $f$ est dans $E \times F$, cela est équivalent à : $\forall x \in E \; \forall y \in F \, [f(x) = f(y) \Rightarrow x = y]$.
    \item $f$ est \textit{surjective} (ou \textit{une surjection}) si $\forall y \in F \; \exists x \, [f(x) = y]$.
        Puisque chaque élément de $f$ est dans $E \times F$, cela est équivalent à : $\forall y \in F \; \exists x \in E \, [f(x) = y]$.
    \item $f$ est \textit{bijective} (ou \textit{une bijection}) si elle est à la fois injective et surjective, ce qui est équivalent à : $\forall y \in F \; \exists ! x \, [f(x) = y]$ et à : $\forall y \in F \; \exists ! x \in E \, [f(x) = y]$.
\end{itemize}
L'\textit{image} de la fonction $f$, notée $\mathrm{Im}(f)$, est l'ensemble $\lbrace y \in F \vert \exists x \, (x \in E) \wedge f(x) = y \rbrace$. 
Pour tout sous-ensemble $G$ de $F$, on note $f^{-1}(G)$ l'ensemble $\lbrace x \in E \vert f(x) \in G \rbrace$. 
S'il n'y a pas d'ambiguité, et si $y \in F$, on notera parfois $f^{-1}(y)$ l'ensemble $f^{-1}(\lbrace y \rbrace)$.
(Les ensembles ainsi obtenus pour différentes valeurs de $y$ sont deux-à-deux disjoints. 
En effet, soit $y$ et $z$ deux éléments de $F$ et $x \in E$. 
Si $x \in f^{-1}(y) \cap f^{-1}(z)$, on a $f(x) = y$ et $f(x) = z$, et donc $y = z$. 
Ainsi, si $y \neq z$, $f^{-1}(y) \cap f^{-1}(z)$ est vide.)
Notons que, pour tout élément $y$ de $F$, on a $f^{-1}(y) \neq \emptyset \Leftrightarrow y \in \mathrm{Im}(f)$. 

\medskip

\noindent\textbf{Lemme :} Soit $E$ un ensemble. 
    Soit $I$ l'ensemble $\lbrace z \in E \times E \, \exists x \in E \, z = (x,x) \rbrace$. 
    Alors, $I$ est une bijection de $E$ vers $E$, appelée \textit{fonction identité} sur $E$. 
    En outre, pour tout élément $x$ de $E$, $I(x) = x$.

\medskip 

\noindent\textbf{Démonstration :} 
\begin{itemize}[nosep]
    \item Montrons d'abord que $I$ est une fonction de $E$ vers $E$. 
        \begin{itemize}[nosep]
            \item Soit $z$ iun élément de $I$. 
                Alors il existe un élément $x$ de $E$ tel que $z = (x,x)$. 
                Donc, il existe un élément $y$ de $E$ (il suffit de prendre $y=x$) tel que $z = (x,y)$.
                La première condition est donc satisfaite.
            \item Soit $x$ un élément de $E$. 
                On a $(x,x) \in I$. 
                Donc, il existe un élément $y$ de $E$ (il suffit de prendre $y = x$) tel que $(x,y) \in E$.
                La deuxième condition est donc satisfaite.
            \item Soit $y$ et $y'$ deux éléments de $E$ et $x$ un élément de $E$ tel que $(x,y) \in I$ et $(x,y') \in I$.
                Alors, il existe deux éléments $x'$ et $x''$ de $E$ tels que $(x,y) = (x',x')$ et $(x,y') = (x'',x'')$. 
                La première égalité donne $x = x'$ et $y = x'$, donc $x = y$.
                La seconde égalité donne $x = x''$ et $y' = x''$, donc $x = y'$. 
                Donc, $y = y'$.
                La ptroisième condition est donc satisfaite.
        \end{itemize}
    \item Soit $x$ un élément de $E$.
        On a $(x,x) \in I$, donc $I(x) = x$.
    \item Montrons qu'elle est injective.
        Soit $x$ et $y$ deux éléments de $E$ tels que $I(x) = I(y)$. 
        Alors, puisque $I(x) = x$ et $I(y) = y$, et par réflexivité et ransitivité de l'égalité, $x = y$.
    \item Montrons qu'elle est surjective. 
        Soit $y$ un élément de $E$. 
        Alors, $I(y) = y$, donc il existe un élément $x$ de $E$ (il suffit de prendre $x = y$) tel que $I(x) = y$.
\end{itemize}

\done

\medskip

\noindent\textbf{Lemme :} Soit $E$ et $F$ deux ensembles, $f$ une fonction de $E$ vers $F$, $I_E$ la fonction identité ur $E$ et $I_F$ la fonction identité sur $F$. 
    Alors, $f \circ I_E = I_F \circ f = f$.

\medskip

\noindent\textbf{Démonstration :} 
    Tout d'abord, puisque $I_E$ est une fonction de $E$ vers $E$, $I_F$ une fonction de $F$ vers $F$, et $f$ une fonction de $E$ vers $F$, $f \circ I_E$ et $I_F \circ f$ sont deux fonctions de $E$ vers $F$. 
    Soit $x$ un élément de $E$. 
    On a : $(f \circ I_E)(x) = f(I_E(x)) = f(x)$ et $(I_F \circ f)(x) = I_F(f(x)) = f(x)$.
    Cela étant vrai pour tout élément $x$ de $E$, on en déduit $f \circ I_E = f$ et $I_F \circ f = f$.

   \done 

\medskip

Soit $E$ un ensemble. 
\begin{itemize}[nosep]
    \item S'il existe une fonction de $E$ vers $\emptyset$, alors $E = \emptyset$ (en effet, soit $f$ une telle fonction, si $E$ contenait un élément $x$, $f(x)$ serait un élément de $\emptyset$, ce qui est impossible).
    \item La seule fonction de $\emptyset$ vers $E$ est $\emptyset$. 
        Elle est toujours injective. 
        Elle est surjective (et donc bijective) si et seulement si $E = \emptyset$.
\end{itemize}

\medskip

Soit $E$ et $F$ deux ensembles. 
Alors, 
\begin{itemize}
    \item Si $E$ est non vide et s'il existe une injection $f$ de $E$ vers $F$, alors il existe une surjection de $F$ vers $E$. 
        En effet, une telle surjection peut être construite de la manière suivante. 
        Soit $a$ un élément de $E$.
        Soit $P$ la propriété à deux variables libres définie par : $P y x: [(y \in \mathrm{Im}(f)) \wedge (f(x) = y)] \vee [(y \notin \mathrm{Im}(f)) \wedge (x = a)]$. 
        Alors, l'ensemble $\lbrace z \in F \times E \vert \exists x \, \exists y \, (z = (y,x)) \wedge (P y x) \rbrace$ est une fonction de $F$ vers $E$ et est surjective.
    \item S'il existe une surjection $f$ de $E$ vers $F$, et si l'on admet l'axiome du choix (voir ci-dessous), alors il existe une injection de $F$ vers $E$. 
        En effet, soit $X$ l'ensemble des $f^{-1}(y)$ pour $y \in F$ (cet ensemble existe d'après l'axiome de l'ensemble des parties et le schéma d'axiome de compréhension : il s'agit de l'ensemble des parties $p$ de $F$ telles que $\exists y \, (y \in F) \wedge (p = f^{-1}(y))$), soit $g$ une fonction qui à chaque élément de cet ensemble associe un de ses éléments\footnote{Cela est possible car, pour tout élément $y$ de $F$, $f^{-1}(y)$ est non vide puisque $f$ est surjective.}, et soit $h$ l'ensemble $\lbrace z \in F \times E \vert \exists x \in E \, \exists y \in F \, (x = g(f^{-1}(y))) \wedge (z=(y,x)) \rbrace$ ; alors $h$ est une fonction injective de $F$ vers $E$. (Elle est bien injective. En effet, si $y$ et $y'$ sont deux élément de $f$ ayant la même image $x$, alors $x \in f^{-1}(y)$ et $x \in f^{-1}(y')$, donc $f(x) = y$ et $f(x) = y'$, donc $y=y'$.)
\end{itemize}

Ces deux résultats étant importants, récrivons-les et démontrons-les plus formellement.

\medskip

\noindent\textbf{Lemme :} 
Soit $E$ et $F$ deux ensembles. 
On suppose que $E$ est non vide et qu'il existe une injection de $E$ vers $F$. 
Alors, il existe une surjection de $F$ vers $E$. 

\medskip

\noindent\textbf{Démonstration :}
Soit $f$ une injection de $E$ vers $F$. 
Soit $a$ un élément de $E$ (un tel élément existe puisque $E$ est non vide). 
Définissons la propriété $P$ à deux paramètres libres par : 
\begin{equation*}
    P x y: \left[(y \in \mathrm{Im}(f)) \wedge (f(x) = y) \right]
           \vee \left[(y \notin \mathrm{Im}(f)) \wedge (x=a) \right].
\end{equation*}
Soit $g$ l'ensemble défini par : 
\begin{equation*}
    g = \lbrace
        z \in F \times E \vert
        \exists x \exists y \, (z = (y,x)) \wedge (P x y)
    \rbrace .
\end{equation*}
Montrons que $g$ est une fonction de $F$ vers $E$ et qu'elle est surjective: 
\begin{itemize}[nosep]
    \item Soit $z$ un élément de $g$. 
        Alors, on peut choisir un élément $x$ de $F$ et un élément $y$ de $E$ tels que $z = (x,y)$.
        La première condition pour être une fonction est donc satisfaite. 
    \item Soit $y$ un élément de $F$. 
        Si $y \in \mathrm{Im}(f)$, alors on peut choisir un élément $x$ de $E$ tel que $f(x) = y$. 
        On a donc $(y,x) \in F \times E$ et $P x y$. 
        Donc, $(y,x) \in g$. 
        On a donc montré que $\exists x (y,x) \in g$. 
        La deuxième condition pour être uns fonction est donc bien satisfaite.
    \item Soit $x$ et $x'$ deux éléments de $E$ et $y$ un élément de $F$ tels que $(y,x) \in g$ et $(y,x') \in g$.
        Alors, $P x y$ et $P x' y$ sont vraies.
        Si $y \in \mathrm{Im}(f)$, cela implique $f(x) = y$ et $f(x') = y$, donc $f(x) = f(x')$, et donc (puisque $f$ est injective) $x = x'$.
        Sinon, cela implique $x = a$ et $x' = a$, donc $x = x'$.
        Dans tous les cas, on a $x = x'$. 
        La troisième condition pour ếtre une fonction est donc satisfaite. 
    \item Soit $x$ un élément de $E$. On a $f(x) \in \mathrm{Im}(f)$ et $f(x) = f(x)$, donc $P x f(x)$ est vraie. 
        Puisque $x \in E$ et $f(x) \in F$, $(f(x),x) \in F \times E$. 
        Donc, $(f(x),x) \in g$. 
        Il existe donc un élément $y$ de $F$ (égal à $f(x)$) tel que $g(y) = x$. 
        Cela montre que $g$ est surjective.
\end{itemize}

\done

\medskip

\noindent\textbf{Lemme :} 
Soit $E$ et $F$ deux ensembles. 
On suppose qu'il existe une surjection de $E$ vers $F$. 
On admet également l'axiome du choix (voir ci-dessous). 
Alors, il existe une injection de $F$ vers $E$. 

\medskip

\noindent\textbf{Démonstration :} 
    Soit $f$ une surjection de $E$ vers $F$.
    Soit $\mathcal{E}$ l'ensemble des parties de $E$. 
    Soit $X$ l'ensemble définit par : 
    \begin{equation*}
        X = \lbrace
                p \in \mathcal{E} \vert
                \exists y \, (y \in F) \wedge (p = f^{-1}(\lbrace y \rbrace))
            \rbrace .
    \end{equation*}
    Soit $p$ un élément de $X$.
    On peut choisir un élément $y$ de $F$ tel que $p = f^{-1}(\lbrace y \rbrace)$.
    Puisque $f$ est surjective, on peut choisir un élément $x$ de $E$ tel que $f(x) = y$.
    Donc, $f(x) \in \lbrace y \rbrace$.
    Donc, $x \in f^{-1}(\lbrace y \rbrace)$.
    Donc, $x \in p$.
    Cela montre que $X$ ne contient pas $\emptyset$.

    D'après l'axiome du choix, il existe donc une fonction de $X$ vers $\cup X$ qui à chaque élément $x$ de $X$ associe un élément de $x$. 
    Soit $g$ une telle fonction.
    Puisque chaque élément de $X$ est un sous-ensemble de $E$, $\cup X$ en est également un. 
    En effet, soit $e$ un élément de $\cup X$, il existe un élément $x$ de $X$ tel que $e \in x$ ; puisque $x \subset E$, on a donc $e \in E$.
    Donc, $g$ est une fonction de $X$ vers $E$.
    Notons $h$ l'ensemble défini par :
    \begin{equation*}
        h = \lbrace 
                z \in F \times E \vert
                \exists x \in E \, \exists y \in F \, 
                (x = g (f^{-1}(\lbrace y \rbrace))) \wedge (z = (y,x))
            \rbrace.
    \end{equation*}
    Montrons que $h$ est une fonction de $F$ vers $E$.
    \begin{itemize}[nosep]
        \item Par définition, $h$ est un sous-ensemble de $F \times E$, et satisfait donc la première condition.
        \item Soit $y$ un élément de $F$. 
            Alors, $f^{-1}(\lbrace y \rbrace)$ est un élément de $X$.
            Soit $x$ l'élément de $E$ défini par $x = g(f^{-1}(\lbrace y \rbrace))$.
            On a $(y,x) \in h$.
        \item Soit $y$ un élément de $F$ et $x$ et $x'$ deux éléments de $E$ tels que $(y,x) \in h$ et $(y,x') \in h$.
            Alors, $x = g (f^{-1}(\lbrace y \rbrace))$ et $x' = g (f^{-1}(\lbrace y \rbrace))$.
            Donc, $x = x'$.
    \end{itemize}
    L'ensemble $h$ est donc bien une fonction de $F$ vers $E$.

    Montrons que $h$ est injective. 
    Soit $y$ et $y'$ deux éléments de $F$ tels que $h(y) = h(y')$. 
    Puisque $h(y) = g (f^{-1}(\lbrace y \rbrace))$ et $g (f^{-1}(\lbrace y \rbrace)) \in f^{-1}(\lbrace y \rbrace)$, on a $h(y) \in f^{-1}(\lbrace y \rbrace)$, et donc $f(h(y)) = y$.
    De même, puisque $h(y') = g (f^{-1}(\lbrace y' \rbrace))$ et $g (f^{-1}(\lbrace y' \rbrace)) \in f^{-1}(\lbrace y' \rbrace)$, on a $h(y') \in f^{-1}(\lbrace y' \rbrace)$, et donc $f(h(y')) = y'$.
    Puisque $h(y) = h(y')$, on en déduit que $y = y'$. 
    Ainsi, $h$ est bien injective. 

   \done 

\medskip

Soit $E$ et $F$ deux ensembles, $f$ une fonction de $E$ vers $F$ et $E'$ un sous-ensemble de $E$. 
Pour simplifier les notations, on note parfois $\lbrace f(x) \vert x \in E' \rbrace$ ou $F(E')$ l'ensemble $\lbrace y \in F \vert \exists x \, (x \in E') \wedge (f(x) = y) \rbrace$. 

\medskip

\noindent\textbf{Composition de deux fonctions :} Soit $E$, $F$ et $G$ trois ensembles. 
    Soit $f$ une fonction de $E$ vers $F$ et $g$ une fonction de $F$ vers $G$. 
    La \textit{composée} de $g$ et $f$, notée $g \circ f$, est la fonction de $E$ vers $G$ définie par : $\forall x \in E \, (g \circ f) (x) = g(f(x))$. 
    Plus formellement, $g \circ f = \lbrace z \in E \times G \vert \exists x \in E \, z = (x,g(f(x))) \rbrace$.

\medskip

\noindent\textbf{Lemme :} L'ensemble ainsi défini est bien une fonction de $E$ vers $G$.

\medskip

\noindent\textbf{Démonstration :} 
\begin{itemize}[nosep]
    \item Soit $z$ un élément de $g \circ f$. 
        Alors, on peut choisir un élément $x$ de $E$ tel que $z = (x,g(f(x)))$. 
        Puisque $f$ est une fonction de $E$ vers $F$, $f(x) \in F$.
        Puisque $g$ est une fonction de $F$ vers $G$, $g(f(x)) \in F$.
        Donc, $z \in E \times G$.
    \item Soit $x$ un élément de $E$. Alors $(x,g(f(x))) \in g \circ f$.
    \item Soit $y$ et $y'$ deux ensembles. 
        Soit $x$ un ensemlble tel que $(x,y) \in g \circ f$ et $(x,y') \in g \circ f$. 
        Alors, on peut choisir un élément $x'$ de $E$ tel que $(x,y) \in (x',g(f(x')))$ et un élément $x''$ de $E$ tel que $(x,y') = (x'',g(f(x'')))$. 
        On a donc $x = x'$, $y = g(f(x'))$, $x = x''$ et $y' = g(f(x''))$. 
        Donc, $y = g(f(x))$ et $y' = g(f(x))$. 
        Donc, $y = y'$.
\end{itemize}

\done

\medskip

\noindent\textbf{Remarque :} Avec les mêmes notations, si $f$ et $g$ sont deux injections, alors $g \circ f$ en est aussi une. 
    En effet, soit $x$ et $y$ deux éléments de $G$ tels que $(g \circ f)(x) = (g \circ f)(y)$, on a $g(f(x)) = g(f(y))$, donc $f(x) = f(y)$, et donc $x = y$.

\medskip

\noindent\textbf{Remarque :} Avec les mêmes notations, si $f$ et $g$ sont deux surjections, alors $g \circ f$ en est aussi une. 
    En effet, soit $z$ un élément de $G$, il existe un élément $y$ de $F$ tel que $g(y) = z$ et un élément $x$ de $E$ tel que $f(x) = y$ ; on a donc $(g \circ f)(x) = z$.

\medskip

\noindent\textbf{Remarque :} Avec les mêmes notations, si $f$ et $g$ sont deux bijections, alors $g \circ f$ en est aussi une. 
    En effet, il s'agit d'une injection et d'une surjection d'après les deux points précédents.

\medskip

\noindent\textbf{Lemme (associativité de la composition de fonctions) :} Soit $E$, $F$, $G$ et $H$ quatre ensembles. 
    Soit $f$, $g$ et $h$ des fonctions respectivement de $E$ vers $F$, de $F$ vers $G$ et de $G$ vers $H$. 
    Alors $h \circ (g \circ f) = (h \circ g) \circ f$.

\medskip

\noindent\textbf{Démonstration :} Montrons d'abord que $h \circ (g \circ f)$ et $(h \circ g) \circ f$ sont deux fonctions de $E$ vers $H$. 
    Puisque $f$ est une fonction de $E$ vers $F$ et $g$ une fonction de $F$ vers $G$, $g \circ f$ et une fonction de $E$ vers $G$. 
    Donc, $h \circ (g \circ f)$ est une fonction de $E$ vers $H$.
    Puisque $g$ est une fonction de $F$ vers $G$ et $h$ une fonction de $G$ vers $H$, $h \circ g$ et une fonction de $F$ vers $H$. 
    Donc, $(h \circ g) \circ f$ est une fonction de $E$ vers $H$. 

    Montrons maintenant qu'elles sont égales. 
    Soit $x$ un élément de $E$. 
    On a : $(h \circ (g \circ f))(x) = h((g \circ f)(x)) = h(g(f(x)))$.
    Par ailleurs, $((h \circ g) \circ f)(x) = (h \circ g)(f(x)) = h(g(f(x)))$. 
    Donc, $(h \circ (g \circ f))(x) = ((h \circ g) \circ f)(x)$.
    On en déduit que $h \circ (g \circ f) = (h \circ g) \circ f$.

   \done 

\medskip

\noindent\textbf{Inverse d'une bijection :} Soit $E$ et $F$ deux ensembles et $f$ une bijection de $E$ vers $F$. 
    L'ensemble $\lbrace z \in F \times E \vert \exists x \in E \, \exists y \in F \, z = (y,x) \wedge (x,y) \in f \rbrace$ est une fonction de $F$ vers $E$ (puisque, pour chaque élément $y$ de $F$, il existe un unique élément $x$ de $E$ tel que $(x,y) \in f$). On montre facilement qu'il s'agit d'une bijection (pour chaque élément $x$ de $E$, il existe un unique élément $y$ de $F$ dont l'image est $x$ : il s'agit de $f(x)$ (son image est bien $x$ par définition et, soit $z$ un élément de $F$ tel que $z \neq y$, l'image de $z$ est l'antécédant de $z$ par $f$, distinct de $x$)), appelée \textit{inverse} de $f$ et notée $f^{-1}$.  

Ce résultat étant important, récrivons-le et démontrons-le plus formellement. 

\medskip

\noindent\textbf{Lemme :} Soit $E$ et $F$ deux ensembles. On suppose qu'il existe une bijection, notée $f$, de $E$ vers $F$. Alors il existe une unique fonction $g$ de $F$ vers $E$ telle que, pour tout élément $x$ de $E$, $g(f(x)) = x$. 
En outre, cette fonction est bijective. 

\medskip

\noindent\textbf{Démonstration :}
    Soit $g$ l'ensemble défini par : 
    \begin{equation*}
        g = \lbrace z \in F \times E \vert \exists x \in E \, \exists y \in F \, z = (y,x) \wedge (x,y) \in f \rbrace.
    \end{equation*}
    Montrons d'abord que $g$ est une fonction de $F$ vers $E$. 
    \begin{itemize}[nosep]
        \item Soit $z$ un élément de $g$. Alors, il existe un élément $y$ de $F$ et un élément $x$ de $E$ tels que $z = (y,x)$.
        \item Soit $y$ un élément de $F$. Puisque $f$ est surjective, on peut choisir un élément $x$ de $E$ tel que $f(x) = y$. 
            Alors, $(y,x) \in F \times E$ et $(x,y) \in f$, donc $(y,x) \in g$. 
        \item Soit $y$ un élément de $F$ et $x$ et $x'$ deux éléments de $E$ tels que $(y,x) \in g$ et $(y,x') \in g$.
            Alors, $(x,y) \in f$ et $(x',y) \in f$, donc $f(x) = y$ et $f(x') = y$, donc $f(x) = f(x')$.
            Puisque $f$ est injective, on en déduit que $x = x'$.
    \end{itemize}
    Ainsi, $g$ est bien une fonction de $F$ vers $E$.

    Montrons qu'elle est unique. 
    Soit $h$ une fonction de $F$ vers $E$ telle que, pour tout élément $x$ de $E$, $h(f(x)) = x$. 
    Soit $y$ un élément de $F$. 
    Puisque $f$ est surjective, on peut choisir un élément $x$ de $E$ tel que $y = f(x)$.
    On a alors $g(y) = x$ et $h(y) = x$. 
    Donc, $h(y) = g(y)$. 
    Cela étant vrai pour tout élément $y$ de $F$, on en déduit que $h = g$.

    Montrons maintenant que $g$ est bijective.
    \begin{itemize}[nosep]
        \item Soit $y$ et $y'$ deux éléments de $F$ tels que $g(y) = g(y')$. 
            Puisque $(y,g(y)) \in g$, on a $(g(y),y) \in f$, donc $f(g(y)) = y$.
            De même, puisque $(y',g(y')) \in g$, on a $(g(y'),y') \in f$, donc $f(g(y')) = y'$.
            Puisque $g(y') = g(y)$, cela implique $f(g(y)) = y'$, et donc $y = y'$.
            Cela montre que $g$ est injective.
        \item Soit $x$ un élément de $E$. 
            Notons $y$ l'élément de $F$ définit par $y = f(x)$.
            Alors, $(y,x) \in F \times E$ et $(x,y) \in F$. 
            Donc, $(y,x) \in g$.
            Donc, $g(y) = x$.
            Cela montre que $g$ est surjective.
    \end{itemize}
    Puisque $g$ est injective et surjective, il s'agit bien d'une bijection.

   \done 

\medskip

\noindent\textbf{Définition :} Soit $E$ et $F$ deux ensembles, $E'$ un sous-ensemble de $E$, et $f$ une fonction de $E$ vers $F$. 
    On appelle \textit{restriction de $f$ à $E'$} la fonction $g$ de $E'$ vers $F$ définie par : pour tout élément $e$ de $E'$, $g(e) = f(e)$.

\medskip

\noindent\textbf{Notation:} Soit $E$, $F$ et $G$ trois ensembles. 
    Soit $f$ une fonction de $E$ vers $F$ et $G$ une fonction de $E$ vers $G$.
    On pourra noter $\lbrace (f(e), g(e)) \vert e \in E \rbrace$ l'ensemble $\lbrace x \in F \times G \vert \exists e \in E \, x = (f(e), g(e)) \rbrace$. 
    Si $f(e)$ est donnée par une formule explicite, alors $f(e)$ peut être remplacée par cette formule, et de même pour $g(e)$.

\subsubsection{Axiome du choix}

\noindent\textit{\textbf{Énoncé :} Pour tout ensemble $X$ d'ensembles non vides, il existe une fonction sur $X$ qui à chaque ensemble $x$ appartenant à $X$ associe un élément de $x$ :}
\begin{equation*}
    \forall X \, \left[ (\emptyset \notin X) \Rightarrow \left( \exists f: X \to \cup X \, \forall x \, [ (x \in X) \Rightarrow ( \exists y \, [((x,y) \in f) \wedge (y \in x)] ) ] \right) \right] .
\end{equation*}

\medskip

\noindent Cette formule peut se récrire plus simplement (au prix d'avoir une partie mal définie pour $x \notin X$) :
\begin{equation*}
    \forall X \, \left[ (\emptyset \notin X) \Rightarrow \left( \exists f: X \to \cup X \, \forall x \in X \, (f(x) \in x) \right) \right] .
\end{equation*}
La théorie ZF plus l'axiome du choix est appelée théorie ZFC.

\subsubsection{Lemme de Zorn (en théorie ZFC)}

Dans cette section seulement, on définit la notion de \textit{chaine} de la manière suivante.
Soit $X$ un ensemble et $\leq$ une relation d'ordre sur $X$.
Un sous-ensemble $C$ de $X$ est une \textit{chaine} de $X$ pour $\leq$ si $\leq$ est une relation d'ordre total sur $C$, autrement dit, si
\begin{equation*}
    \forall x \in C \, \forall y \in C \, x \leq y \vee y \leq x .
\end{equation*}
Pour toute chaine $C$ de $X$ pour $\leq$ et tout élément $x$ de $C$, on note $P(C, x)$ l'ensemble définit par
\begin{equation*}
    P(C, x) = \lbrace y \in C \vert y < x \rbrace ,
\end{equation*}
où $<$ est la relation d'ordre stricte définie par : pour tous éléments $x$ et $y$ de $C$, $x < y$ est équivalent à $(x \leq y) \wedge (x \neq y)$.

Notons que tout sous-ensemble d'une chaine est une chaine.%
\footnote{Montrons cela. Avec les notations précédentes, soit $C$ une chaine et $C'$ un sous-ensemble de $C$.
    Soit $x$ et $y$ deux éléments de $C'$.
    Puisque $C'$ est un sous-ensemble de $C$, $x \in C$ et $y \in C$.
    Puisque $C$ est une chaine, on a donc $x \leq y \vee y \leq x$.
    Cela montre que $C'$ est également une chaine.}
En particulier, pour toute chaine $C$ et tout élément $x$ de $C$, $P(C, x)$ est une chaine.

\medskip

\noindent\textbf{Énoncé :} Soit $X$ un ensemble et $\leq$ une relation d'ordre sur $X$.
    On suppose que toute chaine de $X$ pour $\leq$ admet une borne supérieure dans $X$.
    Alors $X$ admet (au moins) un élément maximal pour la relation $\leq$.

\medskip

On se propose de montrer cet énoncé. 
Pour ce faire, supposons par l'absurde que l'on puisse choisir un ensemble $X$ et une relation d'ordre $\leq$ sur $X$ tels que toute chaine de $X$ pour $\leq$ admet une borne supérieure dans $X$, mais que $X$ n'admet aucun élément maximal pour la relation $\leq$, et montrons que cela mène à une contradiction. 

On définit la relation d'ordre $\geq$ et les deux relations d'ordre strict $<$ et $>$ sur $X$ comme suit : pour tous éléments $x$ et $y$ de $X$, 
\begin{itemize}[nosep]
    \item $x \geq y$ est équivalent à $y \leq x$,
    \item $x < y$ est équivalent à $x \leq y \wedge x \neq y$,
    \item $x > y$ est équivalent à $y < x$.
\end{itemize}
Notons que, pour tous éléments $x$ et $y$ de $X$, $x > y$ est équivalent à $x \geq y \wedge x \neq y$.%
\footnote{
    En effet, $x > y$ est équivalent à $y < x$, donc à $y \leq x \wedge x \neq y$. 
    Puisque $y \leq x$ est équivalent à $x \geq y$, on en déduit que $x > y$ est donc équivalent à $x \geq y \wedge x \neq y$.
}
L'absence d'élément maximal implique que, pour tout élément $x$ de $X$, il existe un élément $y$ de $X$ tel que $x \leq y$ et $x \neq y$ (sans quoi $x$ serait un élément maximal) $x < y$.

Soit $C$ une chaine de $X$ pour $\leq$. 
On peut choisir une borne supérieure $u$ de $C$ dans $X$, et un élément $x$ de $X$, dit \textit{borne suprieure stricte de $C$} tel que $x > u$. 
Alors, pour tout élément $e$ de $C$, on a $e \leq u$ (puisque $u$ est une borne supérieure de $C$) et $u < x$, donc $u \leq x$, donc $e \leq x$.
En outre, avec les mêmes notations, on a $e \neq x$, sans quoi on aurait $x \leq u$ et $u < x$, donc $x \leq u$ et $u \leq x$, donc $u = x$, ce qui est impossible puisque $u < x$. 
Donc, pour tout élément $e$ de $C$, on a $e < x$. 

On note $\mathcal{X}$ l'ensemble des sous-ensembles de $X$.
Soit $\mathcal{C}$ l'ensemble des chaines de $X$. 
Il s'agit d'un sous-ensemble de l'ensemble des sous-ensembles de $X$, définit par : $\mathcal{C} = \lbrace C \in \mathcal{X} \vert \forall x \in C \, \forall y \in C \, x \leq y \vee y \leq x \rbrace$.
Pour toute chaine $C$, on note $S_C$ l'ensemble des bornes supérieures strictes de $C$ (dans $X$). 
Alors, $\left\lbrace \left( C, S_C \right) \vert C \in \mathcal{C} \right\rbrace$ est une fonction de $\mathcal{C}$ vers $\mathcal{X}$. 
(En effet, chaque élément $C$ de $\mathcal{C}$ a une unique image $S_C$.)
En outre, pour tout élément $C$ de $\mathcal{C}$, $S_C$ est non vide. 
Soit $\mathcal{S}$ l'ensemble définit par : $\mathcal{S} = \lbrace S \in \mathcal{X} \vert \exists C \in \mathcal{C} \, S = S_C \rbrace$. 
Alors, $\mathcal{S}$ est un ensemble d'ensembles non vide (puisque toute chaine admet au moins une borne supérieure stricte). 
D'après l'axiome du choix, on peut donc choisir une fonction $g$ de $\mathcal{S}$ vers $X$ telle que, pour tout élément $S$
 de $\mathcal{S}$, $g(S) \in S$. 
Soit $f = \lbrace (C, g(S_C)) \vert C \in \mathcal{C} \rbrace$. 
Alors, $f$ est une fonction de $\mathcal{C}$ vers $X$ et, pour tout élément $C$ de $\mathcal{C}$, $f(C)$ est une borne supérieure stricte de $C$.

Pour toute chaine $C$ de $X$ et tout élément $x$ de $C$, on définit le sous-ensemble $P(C, x)$ de $C$ par : 
\begin{equation*}
    P(C, x) = \lbrace y \in C \vert y < x \rbrace.
\end{equation*}
Soit $C$ une chaine de $X$. 
Un sous-ensemble $D$ de $C$ est dit \textit{segment initial} de $C$ s'il existe un élément $x$ de $C$ tel que $D = P(C, x)$.

On dit d'un sous-ensemble $A$ de $X$ qu'il est \textit{conforme} s'il satisfait les deux conditions suivantes : 
\begin{itemize}[nosep]
    \item la relation $\leq$ est un bon ordre sur $A$ (il s'agit alors d'un ordre total sur $A$, donc $A$ est une chaine), 
    \item pour tout élément $x$ de $A$, on a $x = f(P(A,x))$. 
\end{itemize}

Montrons le résultat intermédiaire suivant : 

\medskip

\noindent\textbf{Lemme :} Soit $A$ et $B$ deux sous-ensembles conformes de $X$ tels que $A \neq B$.
    Alors $A$ est un segment initial de $B$ ou $B$ est un segment initial de $A$.

\medskip

\noindent\textbf{Démonstration :} 
    Supposons que $A \neq B$. 
    Alors, il existe un élément de $A$ qui n'est pas un élément de $B$ ou un élément de $B$ qui n'est pas un élément de $A$.
    Supposons qu'il existe un élément de $A$ qui n'est pas un élément de $B$. 
    Alors, l'ensemble $A \setminus B$ est non vide. 

    L'ensemble $A \setminus B$ est un sous-ensemble non vide de $A$. 
    Puisque $\leq$ est un bon ordre sur $A$, $A \setminus B$ admet un élément minimal, noté $x$ dans la suite de cette démonstration.
    Montrons que $P(A, x) = B$.

    Soit $y$ un élément de $P(A,x)$. 
    Alors, $y \in A$ et $y < x$.
    Puisque $x$ est un élément minimal de $A \setminus B$ pour $\leq$, on en déduit que $y \neq A \setminus B$.
    Donc, $y \in B$ (sans quoi on aurait $y \in A \wedge y \notin B$, et donc $y \in A \setminus B$).
    Ainsi, $P(A,x) \subset B$.

    Il reste à montrer que $B \subset P(A,x)$. 
    Supposons par l'absurde que ce n'est pas le cas. 
    Alors, il existe un élément de $B$ qui n'est pas un élément de $P(A,x)$. 
    Donc, $B \setminus P(A, x)$ est non vide.
    Puisque $\leq$ est un bon ordre sur $B$, et puisque $B \setminus P(A,x)$ est un sous-ensemble de $B$, on en déduit que $B \setminus P(A,x)$ admet un élément minimal, noté $y$ dans la suite de cette démonstration.

    Notons que $x$ n'appartient pas à $P(B, y)$ (qui est un sous-ensemble de $B$). 
    Donc, $x \in A \setminus P(B,y)$. 
    Donc, $A \setminus P(B,y)$ est non vide. 
    Puisqu'il s'agit d'un sous-ensemble de $A$, il admet donc un élément minimal, noté $z$ dans la suite. 
    
    Puisque $\leq$ est un bon ordre que $A$, il s'agit d'un ordre total que $A$. 
    Puisque $x$ et $z$ sont deux éléments de $A$, on a $x \leq z$ ou $z \leq x$. 
    Puisque $z$ est un élément minimal de $A \setminus P(B,y)$, qui contient $x$, si $x \leq z$, alors $x = z$, donc $z \leq x$.
    On a donc toujours $z \leq x$. 

    Nous allons montrer que $P(A,z) = P(B,y)$. 
    Puisque $A$ et $B$ sont conformes, on a $z = f(P(A,z))$ et $y = f(P(B,y))$, et on aura donc $z = y$.
    Puisque $y \in B$ et $x \notin B$, on aura donc $z \neq x$, donc $z < x$, donc (puisque $z \in A$) $z \in P(A,x)$, et donc $y \in P(A,x)$, ce qui est impossible puisque $y$ est un élément de $B \setminus P(A,x)$. 
    cela montrera que l'hypothèse de départ est fausse et que $B \subset P(A,x)$. 
    On pourra alors conclure que $B = P(A,x)$.

    Soit $a$ un élément de $P(A,z)$. 
    Alors, $a \in A$ et $a < z$. 
    Puisque $z$ est un élément minimal de $A \setminus P(B,y)$ et puisque $\leq$ est une relation d'ordre total sur $A$, le prédicat $a \in A \setminus P(B,y)$ est faux (sans quoi on aurait $z \leq a$).
    Donc, $a \in P(B,y)$ (sans quoi on aurait $a \in A \setminus P(B,y)$ puisque $a \in A$).
    Cela montre que $P(A,z) \subset P(B,y)$.

    Soit $b$ un élément de $P(B,y)$. 
    Alors, $b \in B$ et $b < y$.
    Puisque $y$ est un élément minimal de $B \setminus P(A,x)$ et puisque $\leq$ est une relation d'ordre total sur $B$, $b$ ne peut appartenir à $B \setminus P(A,x)$, donc $b \in P(A,x)$, donc $b \in A$ et $b < x$.
    Si $z = x$, alors $b < z$, donc $b \in P(A, z)$.
    Sinon, $z < x$, donc, puisque $x$ est un élément minimal de $A \setminus B$ et puisque $\leq$ est une relation d'ordre total que $A$, on a $z \in B$. 
    Dans ce cas, puisque $z \in A \setminus P(B, y)$, on a $z \geq y$ (sans quoi on aurait $z < y$ et donc $a \in P(B,y)$), donc, puisque $b < y$, $b < z$, et donc $b \in P(A, z)$.
    Cela montre que $P(B, y) \subset P(A, z)$.

    Nous avons donc montré que $P(A, z) = P(B, y)$, ce qui conclut la preuve.

    \done

\medskip

Soit $A$ un sous-ensemble conforme de $X$ et $x$ un élémet de $A$. 
Soit $y$ un élément de $X$ tel que $y < x$. 
Si $y \notin A$, alors $y$ ne peut appartenir à aucun sous-ensemble conforme de $X$. 
En effet, supposons par l'absurde qu'il existe un sous-ensemble conforme $B$ de $X$ tel que $y \in B$. 
On a $A \neq B$ (puisque $y \notin A$ et $y \in B$), donc l'un des deux ensembles $A$ et $B$ est un segent initial de l'autre.
Puisque $y \in B$ et $y \notin A$, $B$ ne peut être un sous-ensemble de $A$, dons $B$ n'est pas un segment initial de $A$. 
Donc, $A$ est un segment initial de $B$. 
On peut donc choisir un élément $z$ de $B$ tel que $A = \lbrace w \in B \vert w < z$. 
Puisque $y \notin A$, $y < z$ doit être faux, donc $z = y$ ou $y \leq z$ est faux ; dans les deux cas (puisque $\leq$ est une relation d'ordre total sur $B$) $z \leq y$ est vrai. 
Puisque $y < x$, on en déduit que $y \leq x$, donc $z \leq x$. 
Mais, puisque $x \in A$, on a $x < z$, donc $x \leq z$ et $x \neq x$ sont vrais, donc $z \leq x$ est faux. 
On en déduit que l'hypothèse de départ est fausse. 

Noton $U$ l'ensemble de tous les sous-ensembles conformes de $X$.%
\footnote{
    Cet ensemble existe bien. 
    En effet, 
    \begin{itemize}[nosep]
        \item l'ensemble des sous-ensembles de $X$ existe d'après l'axiome de l'ensemble des parties, 
        \item l'ensemble des sous-ensembles conformes de $X$ existe donc d'après le schéma d'axiomes de compréhension avec le prédicat $P(x)$ équivalent à « $x$ est conforme », 
        \item l'union des sous-ensembles conformes de $X$ existe donc d'après l'axiome de la réunion.
    \end{itemize}}

\medskip

\noindent\textbf{Lemme :} $U$ est un sous-ensemble conforme de $X$.

\medskip

\noindent\textbf{Démonstration :} 
    \begin{itemize}
        \item Montrons d'abord que $\leq$ est un bon ordre sur $U$, et donc que $U$ est une chaine.
            Soit $V$ un sous-ensemble non vide de $U$. 
            Soit $v$ un élément de $V$. 
            Puisque $v \in V$, $v \in U$, donc on peut choisir un sous-ensemble conforme $A$ de $X$ tel que $v \in A$.
            L'ensemble $A$ est donc non vide et est un sou-ensemble de lui-même.
            Puisque $\leq$ est un bon ordre sur $A$, on en déduit que $A$ admet un plus petit élément $a$.
            Alors, $a$ est un plus petit élément de $U$.
            En effet, soit $u$ un élément de $u$, 
            \begin{itemize}[nosep]
                \item Si $u \in A$, alors $u < a$ est faux puisque $a$ est un plus petit élément de $A$.
                \item Sinon, $u < a$ est faux, sans quoi $u$ n'appartiendrait à aucun sous-enemble coforme de $X$, et donc n'appartiendrait pas à $U$.
            \end{itemize}
        \item Soit $x$ un élément de $U$.
            On peut choisir un sous-ensemble conforme $A$ de $X$ tel que $x \in A$. 
            Puisque $A$ est conforme, on a $x = f(P(A,x))$.
            Montrons que $P(U, x) = P(A, x)$; on aura alors $x = f(P(U,x))$, d'où le résultat attendu.
            \begin{itemize}[nosep]
                \item Soit $y$ un élément de $P(A, x)$.
                    Alors, $y \in A$ et $y < x$.
                    Puisque $A$ est un sous-ensemble de $U$, on a donc $y \in U$ et $y \in x$.
                    Donc, $u \in P(U, x)$.
                \item Soit $y$ un élément de $P(U, x)$.
                    Alors, $y < x$.
                    En outre, $y \in A$, sans quoi $y$ n'appartiendrait à aucun sous-ensemble conforme de $X$, et donc n'appartiendrait pas à $U$. 
                    Donc, $x \in P(A, x)$.
            \end{itemize}
            On a donc bien $P(U,x) = P(A,x)$.
    \end{itemize}

    \done

\medskip

Notons $x$ l'élément $f(U)$. 

\medskip

\noindent\textbf{Lemme :} $U \cup \lbrace x \rbrace$ est un sous-ensemble conforme de $X$.

\medskip

\noindent\textbf{Démonstration :} 
\begin{itemize}
    \item Montrons que $\leq$ est un bon ordre sur $U \cup \lbrace x \rbrace$.
        Soit $V$ un sous-ensemble non vide de $U \cup \lbrace x \rbrace$.
        \begin{itemize}[nosep]
            \item Si $x \notin V$, $V$ est un sous-ensemble non vide de $U$, donc, puisque $\leq$ est un bon ordre sur $U$, $V$ admet un plus petit élément.
            \item Si $x \in V$, alors $V \setminus \lbrace x \rbrace$ est un sous-ensemble de $U$ (en effet, soit $y$ un élément de cet ensemble, $y \in V$, donc $y \in U \cup \lbrace x \rbrace$, et $y \neq x$, donc $y \in U$).
                Si $V \setminus \lbrace x \rbrace$ est non vide, il admet un plus petit élément $v$. 
                Puisque $x$ est une borne supérieure stricte de $U$, $x < v$ est faux. 
                Donc, pour tout élément $y$ de $V$, $y < v$ est faux (par définition d'un plus petit élément si $y \neq x$ et par l'argument précédent sinon). 
                Donc, $v$ est un plus petit élément de $V$.
                Si $V \setminus \lbrace x \rbrace$ est vide, alors $V = \lbrace x \rbrace$, donc $x$ est un plus petit élément de $V$.
                (En effet, pour tout élément de $V$, $x = x$.)
        \end{itemize}
    \item Soit $y$ un élément de $U \cup \lbrace x \rbrace$.
        \begin{itemize}[nosep]
            \item Si $y \neq x$, on a $y \in U$. 
                Puisque $U$ est conforme, on a donc $y = f(P(U, y))$. 
                Montrons que $P(U \cup \lbrace x \rbrace, y) = P(U, y)$.
                On aura alors $y = f(P(U \cup \lbrace x \rbrace, y))$.
                \begin{itemize}[nosep]
                    \item Soit $z$ un élément de $P(U, y)$.
                        Alors, $z \in U$, donc $z \in U \cup \lbrace x \rbrace$, et $z < y$.
                        Donc, $z \in P(U \cup \lbrace x \rbrace, y)$.
                    \item Soit $z$ un élément de $P(U \cup \lbrace x \rbrace, y)$.
                        Alors $z \in U \cup \lbrace x \rbrace$ et $z < y$.
                        Puisque $x$ est une borne supérieure de $U$, $x < y$ est faux, donc $z \neq x$.
                        Donc, $z \in U$, donc $z \in P(U, y)$.
                \end{itemize}
                Ainsi, on a bien $P(U \cup \lbrace x \rbrace, y) = P(U, y)$.
            \item Sinon, $y = x$, donc $y = f(U)$. 
                Montrons que $P(U \cup \lbrace x \rbrace, x) = U$.
                On aura alors $y = f(P(U \cup \lbrace x \rbrace, y))$. 
                \begin{itemize}[nosep]
                    \item Soit $z$ un élément de $U$.
                        Alors, $z \in U \cup \lbrace x \rbrace$.
                        Soit $u$ une borne supérieure de $U$ dans $X$ telle que $x > u$.
                        Alors, $z \leq u$ et $u < x$, donc $u \leq x$, donc $z \leq x$ et $z \neq x$ (sans quoi on aurait $u < z$), donc $z < x$.
                        Donc, $z \in P(U \cup \lbrace x \rbrace, x)$.
                    \item Soit $z$ un élément de $P(U \cup \lbrace x \rbrace, x)$.
                        Alors, $z \in U \cup \lbrace x \rbrace$ et $z < x$, donc $z \neq x$, donc $z \in U$.
                \end{itemize}
        \end{itemize}
\end{itemize}

\done

\medskip

Par définition de $U$, on a donc $U \cup \lbrace x \rbrace \in U$, donc $x \in U$. 
Par définition de $x$, on peut choisir une borne supérieure $u$ de $U$ dans $X$ tel que $x > u$, donc $u \leq x$. 
Mais, puisque $x \in U$ et $u$ est un élément maximal de $U$, on a aussi $u \leq x \Rightarrow u = x$. 
Donc, $u = x$ et $x > u$, ce qui est impossible. 
On en déduit que l'hypothèse de départ est fausse, ce qui prouve le lemme de Zorn.
