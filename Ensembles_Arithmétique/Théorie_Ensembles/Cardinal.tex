\subsection{Cardinal} 

\subsubsection{Cardinal fini}

Un ensemble $E$ est dit \textit{de cardinal fini}, ou simplement \textit{fini}, s'il existe une bijection de $E$ vers un entier naturel.
S'il n'existe aucun netier naturel $n$ tel qu'il existe une bijection de $E$ vers $n$, on dit que $E$ est \textit{de cardinal infini}, ou simplement \textit{infini}. 

Soit $E$ un ensemble et $n$ un entier naturel. 
On dit que $E$ est de \textit{cardinal} $n$ s'il existe une bijection de $E$ vers $n$.%
\footnote{
    Notons que $E$ est alors de cardinal fini.
}
(Ainsi, par exemple, l'ensemble vide est le seul ensemble de cardinal $0$.)
Puisqu'il n'existe aucune bijection entre deux entiers naturels non égaux, pour tout ensemble $E$, il existe au plus un entier naturel $n$ tel que $\lvert E \rvert = n$.%
\footnote{
    Soit $E$ un ensemble et $n$ et $m$ deux entiers naturels tels que $E$ est de cardinal $n$ et de cardinal $m$. 
    Alors, il existe une bijection de $E$ vers $n$ et une bijection de $E$ vers $m$.
    Notons $f$ la première et $g$ la seconde. 
    Alors, $g^{-1}$ est une bijection de $m$ vers $E$. 
    Donc, $f \circ g^{-1}$ est une bijection de $n$ vers $m$.
    Donc, $m = n$.
}
Un ensemble de cardinal fini admet donc au plus un unique cardinal.
Soit $E$ un ensemble de cardinal fini, on note (s'il n'y a pas d'ambiguité avec d'autre notations) $\lvert E \rvert$ son cardinal. 
Si $E$ est un ensemble et $n$ un entier naturel, la notation $\abs{E} = n$ est comprise comme : « $E$ est fini et son cardinal est $n$ ». 
On notera parfois $\abs{E} = \infty$ le prédicat « $E$ est infini » et $\abs{E} \in \mathbb{N}$ le prédicat « $E$ est fini ».

Puisque l'inverse d'une bijection est une bijection, deux conséquences immédiates de ces définitions sont : 
\begin{itemize}[nosep]
    \item Un ensemble $E$ est fini si et seulement si il existe un entier naturel $n$ tel qu'il existe une bijection de $n$ vers $E$. 
    \item Soit $E$ un ensemble et $n$ un entier naturel, $E$ est de cardinal $n$ si et seulement si il existe une bijection de $n$ vers $E$. 
\end{itemize}

\medskip

\noindent\textbf{Lemme :} 
    \begin{itemize}[nosep]
        \item L'ensemble vide $\emptyset$ est le seul ensemble de cardinal $0$.
        \item Soit $a$ un ensemble. 
            Alors $\lbrace a \rbrace$ et de cardinal $1$.
        \item Soit $a$ et $b$ deux ensembles tels que $a \neq b$. 
            Alors $\lbrace a, b \rbrace$ et de cardinal $2$.
    \end{itemize}

\medskip

\noindent\textbf{Démonstration :}

\begin{itemize}[nosep]
    \item L'ensemble vide est le seu ensemble en bijection avec $0$, égal à $\emptyset$. 
        Il est donc le seul ensemble de cardinal $0$.
    \item Soit $a$ un ensemble. 
        Alors, $\lbrace (a, 0) \rbrace$ est une fonction de $\lbrace a \rbrace$ vers $1$ (son unique élément est dans $\lbrace a \rbrace \times 1$ puisque $1 = \lbrace 0 \rbrace$ et le seul élément de $\lbrace a \rbrace$, $a$, a une unique image $0$) et elle est bijective (le seul élément de $1$, $0$, a un unique antécédent, $a$).
    \item Soit $a$ et $b$ deux ensembles tels que $a \neq b$.
        Alors, $\lbrace (a,0), (b,1) \rbrace$ est une fonction de $\lbrace a, b \rbrace$ vers $\lbrace 0, 1 \rbrace$, égal à $2$ (en effet, chacun de ses deux éléments est bien dans $\lbrace a, b \rbrace \times \lbrace 0, 1 \rbrace$ et chacun des deux éléments de $\lbrace a, b \rbrace$ a une unique image ($0$ pour $a$ et $1$ pour $b$)) et elle est bijective (chaqun des deux éléments de $2$ a un unique antécédent : $a$ pour $0$ et $b$ pour $1$).
\end{itemize}

\medskip

Soit $n$ un entier naturel et $E$ un ensemble de cardinal $n$. 
Soit $f$ une bijection de $n$ vers $E$. 
On pourra noter l'ensemble $E$ par la liste de ses éléments, \textit{i.e.}, des $f(x)$ pour $x$ décrivant $[\![0, n-1]\!]$, séparés par des virgules, entre les crochets $\lbrace$ et $\rbrace$ : $E = \lbrace f(0), f(1), \dots, f(n-1) \rbrace$. 
(Dans cette expression, il est implicitement compris que $f(n-1)$ n'est pas présent si $n \leq 2$, $f(1)$ n'est pas présent si $n \leq 1$, et $f(0)$ n'est pas présent si $n = 0$.) 
Ainsi, en accord avec les notations précédemments définies, 
\begin{itemize}[nosep]
    \item $\lbrace \rbrace$ désigne l'ensemble vide,
    \item si $a$ est un ensemble, $\lbrace a \rbrace$ désigne l'ensemble admettant $a$ pour seul élément,
    \item si $a$ et $b$ sont deux ensembles, $\lbrace a, b \rbrace$ désigne l'ensemble $C$ tel que $\forall c, \, c \in C \Leftrightarrow ((c = a) \vee (c = b))$.)
\end{itemize}
Plus généralement, soit $F$ un ensemble, $n$ un entier naturel et $f$ une fonction de $n$ vers $E$. 
On peu noter $\lbrace f(0), f(1), \dots, f(n-1) \rbrace$ l'ensemble $G$ défini par : $\forall x \, x \in G \Leftrightarrow (\exists i \in [\![0, n-1]\!] \, f(i) = x)$.

\medskip

\noindent\textbf{Lemme :} Avec les mêmes notations, si $f$ est injective, alors $G$ est de cardinal $n$.

\medskip

\noindent\textbf{Démonstration :}
    Par définition, $f$ est une surjection de $n$ vers $G$. 
    En effet, soit $x$ un élément de $G$, il existe un élément $i$ de $[\![0, n-1]\!]$, et donc de $n$, tel que $g(i) = x$. 
    Si $f$ est aussi injective, alors $f$ est une bijection de $n$ vers $G$, et $f^{-1}$ est donc une bijection de $G$ vers $n$.

\done

\medskip

\noindent\textbf{Lemme :} Soit $E$ et $F$ deux ensembles finis de même cardinal. 
    On suppose que $E \subset F$. 
    Alors $E = F$. 

\medskip

\noindent\textbf{Remarque :} La réciproque est évidente : Soit $E$ et $F$ deux ensembles finis, si $E = F$, alors $E \subset F$. 
    On déduit donc le résultat suivant : si $E$ et $F$ sont finis et ont le même cardinal, alors $E \subset F \Leftrightarrow E = F$.

\medskip

\noindent\textbf{Démonstration du lemme :} 
    On procède par récurrence sur le cardinal $n$ de $E$ et $F$. 
    Si $n = 0$, on a $E = \emptyset$ et $F = \emptyset$, donc $E = F$. 

    Soit $n$ un entier naturel et supposons la propriété attendue vraie pour deux ensembles de cardinal $n$. 
    Soit $E$ et $F$ deux ensembles de cardinal $n+1$ tels que $E \subset F$. 
    Puisque le cardinal de $E$ n'est pas $0$, $E$ admet au moins un élément (sans quoi $E$ serait égal à $\emptyset$, et donc de cardinal $0$). 
    Soit $e$ un élément de $E$. 
    Puisque $E \subset F$, on a $e \in F$. 
    Soit $E'$ et $F'$ les ensembles définis par $E' = E \setminus \lbrace e \rbrace$ et $F' = E \setminus \lbrace e \rbrace$. 
    Alors, d'après le lemme précédent, $E'$ et $F'$ sont de même cardinal $n$. 
    En outre, pour tout élément $x$ de $E'$, on a $x \in E$, donc $x \in F$, et $x \neq e$, donc $x \in F'$.
    Donc, $E' \subset F'$. 
    On en déduit que $E' = F'$, et donc, puisque $E = E' \cup \lbrace e \rbrace$ et $F = F' \cup \lbrace e \rbrace$, que $E = F$. 

    Par récurrence, ce résultat est vrai pour tout entier naturel $n$.

    \done 

\medskip

\noindent\textbf{Lemme :} Soit $E$ un ensemble de cardinal fini $n$. 
    Soit $x$ tel que $x \notin E$.
    Alors $E \cup \lbrace x \rbrace$ a pour cardinal $n+1$.

\medskip

\noindent\textbf{Démonstration :} 
    Notons que, si $E$ est l'ensemble vide, alors $E \cup \lbrace x \rbrace = \lbrace x \rbrace$. 
    La fonction $f: \lbrace x \rbrace \to 1$ définie par $f(x) = 0$ est bijective, donc $E \cup \lbrace x \rbrace$ a pour cardinal $1$.

    Soit $E$ un ensemble quelconque. 
    Puisque $E$ a pour cardinal $n$, il existe une bijection $f$ de $E$ vers $n$. 
    Soit $g$ la fonction de $E \cup \lbrace x \rbrace$ vers $n+1$ définie par $g(x) = n$ et $g(y) = f(y)$ pour tout élément $y$ de $E$.
    (Cela définit bien une fonction car $x$ n'est pas dans le domaine de $f$.)%
    \footnote{
        Plus formellement, on définit l'ensemble $g$ par $g = f \cup \lbrace (x,n) \rbrace$. 
        Alors, 
        \begin{itemize}[nosep]
            \item Soit $z$ un élément de $g$, alors $z \in f$ ou $z = (x,n)$. 
                Dans le premier cas, $z \in E \times n$, donc puisque $E \subset E \cup \lbrace x \rbrace$ et $n \subset n+1$, $z \in \left( E \cup \lbrace x \rbrace \right) \times (n+1)$.
                Dans le second cas, $z \in \left( E \cup \lbrace x \rbrace \right) \times (n+1)$ puisque $x \in E \cup \lbrace x \rbrace$ et $n \in n+1$.
            \item Soit $a$ un élément de $E \cup \lbrace x \rbrace$. 
                Si $a \in E$, il existe un unique élément, noté $b$ dans la suite, tel que $(a,b) \in f$. 
                Alors, puisque $f \subset g$, $(a,b) \in g$.
                Sinon, $a = x$ et $(a, n) \in g$.
            \item Soit $a$ un élément de $E \cup \lbrace x \rbrace$ et $b$ et $c$ deux éléments de $n+1$ tels que $(a,b) \in g$ et $(a,c) \in g$.
                Si $a \in E$, alors $a \neq x$, donc $(a,b) \neq (x,n)$ et $(a,c) \neq (x,n)$.
                Donc, $(a,b) \in f$ et $(a,c) \in f$.
                Puisque $f$ est une fonction, on en déduit $b = c$.
                Sinon, on a $a = x$. 
                Puisque $x \notin E$, il n'existe aucun élément de $f$ dont la première composante est $x$. 
                Donc, $(a,b) = (x,n)$ et $(a,c) = (x,n)$, et donc $b = c$.
        \end{itemize}
        Agnsi, $g$ est bien une fonction de $E \cup \lbrace x \rbrace$ vers $n+1$.
    }
    Alors, $g$ est injective (car $f$ est injective et $n+1$ n'est pas dans l'image de $f$)%
    \footnote{
        Soit $a$ et $b$ deux éléments de $E \cup \lbrace x \rbrace$ tels que $g(a) = g(b)$.
        Si $g(a) = n$, puisque $n \notin n$, on a $a = x$ et, puisqu'alors $g(b) = n$, $b = x$, donc $a = b$. 
        Sinon, $(a,g(a))$ et $(b,g(a))$ sont deux éléments de $f$ et, puisque $f$ est injective, on a $a = b$. 
    }
    et surjective%
    \footnote{
        Soit $z$ un élément de $n+1$. 
        Si $z = n$, on a $g(x) = n$. 
        Sinon, $z < n$, donc $z \in n$ et, puisque $f$ est surjective, on peut choisir un élément $e$ de $E$ tel que $f(e) = z$, et donc $g(e) = z$.
    }
    Donc, $g$ et une bijection de $E$ vers $n+1$.
    Donc, $E$ est de cardinal $n+1$.

   \done 

\medskip 

\noindent\textbf{Lemme :} Soit $E$ un ensemble de cardinal fini $n$. 
    Soit $x$ tel que $x \in E$.
    Alors $E \setminus \lbrace x \rbrace$ a pour cardinal $n-1$.

\medskip

\noindent\textbf{Démonstration :} 
    Notons d'abord que, puisque $E$ contient au moins un élément ($x$), $E$ ne peut être vide, donc son cardinal ne peut pas être égal à $0$.
    Donc, $n-1$ est bien un entier naturel.
    Puisque $E$ a pour cardinal $n$, il existe une bijection de $E$ vers $n$. 
    Appelons-là $f$. 
    Puisque $f$ est une bijection et puisque $n-1 \in n$, on peut choisir un élément $y$ de $E$ tel que $f(y) = n-1$. 
    Soit $g$ la fonction de $E$ vers $E$ définie par : $g(x) = y$, $g(y) = x$ et $g(z) = z$ pour tout élément $z$ de $E$ tel que $z \notin \lbrace x, y \rbrace$.%
    \footnote{
        L'ensemble $g$ est bien une fonction de $E$ vers $E$.
        En effet, tout élément de $g$ est un élément de $E \times E$ et, soit $z$ un élément de $E$, 
        \begin{itemize}[nosep]
            \item Si $z \notin \lbrace x, y \rbrace$, il existe un unique élément $w$ de $E$ ($z$ lui-même) tel que $(z,w) \in g$.
            \item Si $z = x$, $y$ est le seul élément $w$ de $E$ tel que $(z,w) \in g$ (y compris si $x=y$, car alors les deux premières propriétés définissant $g$ sont équivalentes).
            \item Si $z = y$, $x$ est le seul élément $w$ de $E$ tel que $(z,w) \in g$ (y compris si $x=y$, car alors les deux premières propriétés définissant $g$ sont équivalentes).
        \end{itemize}
    }
    Soit $h$ la fonction de $E \setminus \lbrace x \rbrace$ vers $n-1$ définie par : pour tout élément $z$ de $E \setminus \lbrace x \rbrace$, $h(z) = f(g(z))$. 
    Montrons que 
    \begin{itemize}[nosep]
        \item Cela constitue une bonne définition (il s'agit de montrer que $f(g(z)) \in n-1$ pour tout élément $z$ de $E \setminus \lbrace x \rbrace$).
        \item $h$ est injective. 
        \item $h$ est surjective.
    \end{itemize}
    Ainsi, $h$ sera une bijection de $E \setminus \lbrace x \rbrace$ vers $n-1$, d'où le résultat attendu.

    Montrons le premier point. 
    Soit $z$ un élément de $E \setminus \lbrace x \rbrace$. 
    Puisque $z \in E$, $g(z)$ est bien défini et un élément de $E$, et donc $f(g(z))$ est bien défini et est un élément de $n$. 
    En outre, $g(z) \neq y$. 
    En effet, si $z \neq y$, on a $g(z) = z$, et donc $g(z) \neq y$ et, si $z = y$, on a $x \neq y$ (puisqu'alors $y \in E \setminus \lbrace x \rbrace$) et $g(z) = x$, donc $g(z) \neq y$. 
    Puisque $f$ est injective et $f(y) = n-1$, on en déduit $f(g(z)) \neq n-1$. 
    Puisque $n = (n-1) \cup \lbrace n-1 \rbrace$ et $f(g(z)) \in n$, on en déduit $f(g(z)) \in n-1$. 

    Montrons maintenant que $h$ est injective. 
    Soit $u$ et $v$ deux éléments de $E \setminus \lbrace x \rbrace$ tels que $h(u) = h(v)$. 
    Alors, $f(g(u)) = f(g(v))$. 
    Puisque $f$ est injective, on en déduit $g(u) = g(v)$. 
    Montrons que cela implique $u = v$. 
    On distingue deux cas : 
    \begin{itemize}[nosep]
        \item Si $u = y$, on a $g(u) = x$, donc $g(v) = x$. 
            Or, pour tout élément $z$ de $E \setminus \lbrace x \rbrace$ tel que $z \neq y$, on a $g(z) = z$, donc $g(z) \neq x$.
            Donc, on ne peut pas avoir $v \neq y$. 
            Donc, $v = y$, et donc $v = u$.
        \item Si $u \neq y$, on a $g(u) = u$ et donc $g(v) = u$.
            Si $v$ était égal à $y$, on aurait $g(v) = x$, ce qui est faux puisque $u \neq x$. 
            Donc, $v \neq y$, et donc $g(v) = v$. 
            Puisque $g(v) = u$, on en déduit $u = v$.
    \end{itemize}
    Dans les deux cas, on a bien $u = v$. 
    Cela montre que $h$ est injective. 

    Enfin, montrons que $h$ est surjective. 
    Soit $a$ un élément de $n-1$. 
    Puisque $f$ est surjective, on peut choisir un élément $z$ de $E$ tel que $f(z) = a$. 
    De plus, $z$ ne peut être égal à $y$ puisque $f(y) = n-1$ alors que $a < n-1$ (puisque $a \in n-1$). 
    Distinguons deux cas : 
    \begin{itemize}[nosep]
        \item Si $z = x$, alors $g(y) = z$, donc $f(g(y)) = a$. 
            En outre, puisque $z \neq y$, on a alors $y \neq x$, donc $y \in E \setminus \lbrace x \rbrace$.
        \item Sinon, $g(z) = z$, donc $f(g(z)) = a$.
            En outre, dans ce cas, $z \in E \setminus \lbrace x \rbrace$.
    \end{itemize}
    Dans les deux cas, il existe donc un élément $w$ de $E \setminus \lbrace x \rbrace$ tel que $f(g(w)) = a$, et donc $h(w) = a$.
    Cela montre que $h$ est surjective.

    \done 

\medskip

\noindent\textbf{Lemme :} Soit $n$ un entier naturel non nul. 
    Soit $E$ un ensemble de cardinal $n$. 
    Alors il existe un sous-ensemble $F$ de $E$, de cardinal $n-1$, et un élément $x$ de $E$, tels que $E = F \cup \lbrace x \rbrace$.

\medskip

\noindent\textbf{Démonstration :} Puisque $n \neq 0$, $E \neq \emptyset$. 
    On peut donc choisir un élément $x$ de $E$. 
    Soit $F = E \setminus \lbrace x \rbrace$. 
    D'après le lemme précédent, le cardinal de $F$ est $n-1$. 
    En outre, on a $E = F \cup \lbrace x \rbrace$, ce qui montre le lemme. 

    Par soucis de complétude, montrons qu'on a bien $E = F \cup \lbrace x \rbrace$. 
    \begin{itemize}[nosep]
        \item Soit $y$ un élément de $E$. 
            Si $y = x$, $y \in \lbrace x \rbrace$.
            Sinon, $y \in F$. 
            Dans les deux cas, $y \in F \cup \lbrace x \rbrace$. 
            Donc, $E \subset F \cup \lbrace x \rbrace$.
        \item Soit $y$ un élément de $F \cup \lbrace x \rbrace$. 
            Alors, $y \in F$ ou $y \in \lbrace x \rbrace$.
            Si $y \in F$, alors $y \in E$ puisque $F$ est un sous-ensemble de $E$.
            Si $x \in \lbrace x \rbrace$, alors $y = x$, et donc $y \in E$. 
            Ainsi, $F \cup \lbrace x \rbrace \subset E$.
    \end{itemize}
    On a donc bien $E = F \cup \lbrace x \rbrace$.

    \done

    \medskip

\medskip

\noindent\textbf{Lemme :} Soit $E$ et $F$ deux ensembles. 
    On suppose que $E$ est fini et $F \subset E$.
    Alors, $F$ est fini et $\abs{F} \leq \abs{E}$.
    En outre, d'après le lemme précédent, $\abs{F} = \abs{E}$ si et seulement si $F = E$.

\medskip

\noindent\textbf{Démonstration :} 
    On procéde par récurrence sur le cardinal de $E$. 
    On veut montrer que le prédicat suivant est vrai : $\forall E \, \forall F \, (\mathcal{F}(E) \wedge \abs{E} = n \wedge F \subset E) \Rightarrow (\mathcal{F}(F) \wedge \abs{F} \leq \abs{E})$, où, pour tout ensemble $X$, $\mathcal{F}(X)$ est vrai si $X$ est fini et faux sinon.

    Si $\abs{E} = 0$, alors $E = \emptyset$.
    Soit $F$ tel que $F \subset E$. 
    Alors, $F = \emptyset$. 
    Donc, $F$ est fini et de cardinal égal à celui de $E$ ($0$).
    Ainsi, $P(0)$ est vrai.

    Soit $n$ un entier naturel tel que $P(n)$ est vrai. 
    Soit $E$ un ensemble de cardinal $n+1$ et $F$ un sous-ensemble de $E$. 
    Si $F = \emptyset$, $F$ est bien fini et de cardinal $0$, donc $\abs{F} < n+1$, donc $\abs{F} \leq n+1$, donc $\abs{F} < \abs{E}$.

    Sinon, on peut choisir un élément $x$ de $F$. 
    Puisque $F$ et un sous-ensemble de $F$, on a $x \in E$. 
    Donc, $E \setminus \lbrace x \rbrace$ est de cardinal $n$.
    En outre, $F \setminus \lbrace x \rbrace \subset E \setminus \lbrace x \rbrace$. 
    En effet, soit $y$ un élément de $F \setminus \lbrace x \rbrace$, on a $y \in F$, donc $y \in E$, et $y \neq x$.
    Puisque $P(n)$ est vrai, on en déduit que $F \setminus \lbrace x \rbrace$ est fini et de cadinal inférieur ou égal à $n$. 
    Soit $m$ le cardinal de $F \setminus \lbrace x \rbrace$. 
    Puisque $x \notin F \setminus \lbrace x \rbrace$ et $F = \left( F \setminus \lbrace x \rbrace \right) \cup \lbrace x \rbrace$, on en déduit que $F$ est de cardinal $m+1$.
    En outre, puisque $m \leq n$, on a $m+1 \leq n+1$, et donc $\abs{F} \leq \abs{E}$.

    \done

\noindent\textbf{Lemme :} Soit $E$ et $F$ deux ensembles finis tels que $E \cap F = \emptyset$.
    Alors $E \cup F$ est fini et $\abs{E \cup F} = \abs{E} + \abs{F}$.

\medskip

\noindent\textbf{Démonstration :} On procède par récurrence sur le cardinal de $F$.
    Si $\abs{F} = 0$, alors $F = \emptyset$, donc $E \cup F = E$ et $\abs{E \cup F} = \abs{E} = \abs{E} + \abs{F}$. 

    Soit $n$ un entier naturel et supposons la propriété énoncée dans le lemme vraie pour tout ensemble $F$ de cardinal $n$.
    Soit $F$ un ensemble fini de cardinal $n+1$ tel que $E \cap F = \emptyset$.
    Puisque $n+1 \neq 0$, $F$ est non vide.
    Soit $x$ un élément de $F$. 
    Puisque $E \cap F = \emptyset$, $x$ ne peut être un élément de $E$. 
    Donc, $x$ n'est pas un élément de $E \cup \left( F \setminus \lbrace x \rbrace \right)$.
    Puisque $E \cup F = \left( E \cup \left( F \setminus \lbrace x \rbrace \right) \right) \cup \lbrace x \rbrace$, car $\abs{F \setminus \lbrace x \rbrace} = \abs{F} - 1 = n$ et car $E \cup \left( F \setminus \lbrace x \rbrace \right) = \emptyset$, on a donc : 
    \begin{equation*}
        \abs{E \cup F} = \abs{E \cup \left( F \setminus \lbrace x \rbrace \right)} + 1
                       = \abs{E} + \abs{F \setminus \lbrace x \rbrace} + 1
                       = \abs{E} + \abs{F} - 1 + 1
                       = \abs{E} + \abs{F}.
    \end{equation*}
    La propriété attendue est donc vraie pour tout ensemble $F$ de cardial $n+1$. 
    Par récurence, elle l'est pour tout ensemble $F$ fini.

    \done

\medskip

\noindent\textbf{Lemme :} Soit $E$ un ensemble de cardinal fini $n$. 
    Soit $F$ un sous-ensemble de $E$
    Alors $E \setminus F$ est fini et $\abs{E \setminus F} = \abs{E} - \abs{F}$.

\medskip

\noindent\textbf{Démonstration :} Notons d'abord que $F$ et $E \setminus F$ sont deux sous-ensembles de $E$, donc finis, et $\abs{F} \leq \abs{E}$.

    Montrons le reste du lemme par récurrence sur $\abs{F}$.
    Pour $\abs{F} = \emptyset$, $E \setminus F = E$, donc $\abs{E \setminus F} = \abs{E} = \abs{E} - \abs{F}$. 

    Soit $n$ un entier naturel et supposons la propriété vraie pour tout sous-ensemble $F$ de $E$ de cardinal $n$. 
    Soit $F$ un sous-ensemble de $E$ de cardinal $n+1$. 
    Puisque $n+1 \neq 0$, on peut choisir un élément $x$ de $F$. 
    Notons que $E \setminus F = \left( E \setminus \left( F \setminus \lbrace x \rbrace \right) \right) \setminus \lbrace x \rbrace$.
    Donc, et puisque $x \in E \setminus \left( F \setminus \lbrace x \rbrace \right)$, 
    \begin{equation*}
        \abs{E \setminus F} = \abs{E \setminus \left( F \setminus \lbrace x \rbrace \right)} - 1
                            = \abs{E} - n - 1
                            = \abs{E} - (n+1)
                            = \abs{E} - \abs{F}.
    \end{equation*}
    La propriété attendue est donc vraie pour tout sous-ensemble $F$ de cardial $n+1$. 
    Par récurence, elle l'est pour tout sous-ensemble fini de $E$, et donc pour tout sous-ensemble de $E$.

    \done

\medskip

\noindent\textbf{Lemme :} Soit $E$ et $F$ deux ensembles. 
    On suppose que $E$ est fini et qu'il existe une surjection de $E$ vers $F$.
    Alors $F$ est fini et $\abs{F} \leq \abs{E}$.

\medskip

\noindent\textbf{Démonstration :} On procède par récurrence forte sur le cardinal de $E$. 
    Si $\abs{E} = 0$, alors $E = \emptyset$. 
    Puisqu'il existe une surjection de $E$ vers $F$, on en déduit que $F = \emptyset$. 
    Donc, $F$ est fini et $\abs{F} = 0 = \abs{E}$, donc $\abs{F} \leq \abs{E}$.

    Soit $n$ un entier naturel et supposons le lemme vrai pour tout ensemble fini $E$ de cardinal infèrieur ou égal à $n$. 
    Soit $E$ un ensemble de cardinal $n+1$. 
    Soit $F$ un ensemble tel qu'il existe une surjection de $E$ vers $F$. 
    Soit $f$ une telle surjection. 
    Soit $x$ un élément de $E$ (un tel élément existe puisque $\abs{E} > 0$) et soit $y$ l'image de $x$ ar $f$.
    Soit $E_y$ l'ensemble des antécédents de $y$ âr $f$. 
    Alors, $E_y$ est un sous-esemble de $E$ et $E_y$ est non vide puisque $x \in E_y$.
    Soit $g$ l'ensemble défini par $g = f \setminus \lbrace z \in f \vert \exists x \in E_y \, z = (x,y) \rbrace$.
    Montrons que $g$ est une surjection de $E \setminus E_y$ vers $F \setminus \lbrace y \rbrace$.
    Cela montrera que $F \setminus \lbrace y \rbrace$ est fini et $\abs{F \setminus \lbrace y \rbrace} \leq \abs{E \setminus E_y}$, et donc, puisque $F = F \setminus \lbrace y \rbrace \cup \lbrace y \rbrace$ et $y \notin F \setminus \lbrace y \rbrace$, que $F$ est fini et $\abs{F} = \abs{F \setminus \lbrace y \rbrace} + 1$, d'où $\abs{F} \leq \abs{E \setminus E_y} + 1$. 
    Enfin, puisque $\abs{E \setminus E_y} = \abs{E} - \abs{E_y}$ et $\abs{E_y} \geq 1$, on en déduira $\abs{F} \leq \abs{E}$.

    Montrons que $g$ est une fonction de $E \setminus E_y$ vers $F \setminus \lbrace y \rbrace$.
    \begin{itemize}[nosep]
        \item Soit $z'$ un élément de $g$.
            Puisque $g$ est un sous-ensemble de $f$, $z \in f$. 
            Donc, on peut choisir un élément $x'$ de $E$ et un élément $y'$ de $F$ tels que $z' = (x',y')$.
            En outre, $x'$ ne peut être un élément de $E_y$ (si c'était le cas, $y'$ serait égal à $y$, et donc $z$ serait égal à $(x,y)$ pour un élément $x$ de $E_y$).
            Donc, $z \in f \setminus \lbrace z \in f \vert \exists x \in E_y \, z = (x,y) \rbrace$.
        \item Soit $x'$ un élément de $E \setminus E_y$. 
            Puisque $f$ est une fonction de $E$ vers $F$ et $x' \in E$, on peut choisir un élément $y'$ de $F$ tel que $(x',y') \in f$.
            En outre, $x' \notin E_y$, donc $y' \in F \setminus \lbrace y \rbrace$ et $(x',y') \in g$.
        \item Soit $y'$ et $y''$ deux éléments de $F \setminus \lbrace y \rbrace$ et $x'$ un élément de $E \setminus E_y$ tels que $(x',y') \in g$ et $(x', y'') \in g$. 
            Puisque $g$ est un sous-ensemble de $f$, on a $(x',y') \in f$ et $(x', y'') \in f$.
            Puisque $f$ est une fonction, on en déduit que $y' = y''$.
    \end{itemize}

    Montrons qu'elle est surjective. 
    Soit $y'$ un élément de $F \setminus \lbrace y \rbrace$. 
    On a $y' \in F$, donc, puisque $f$ est surjective, on peut choisir un élément $x'$ de $E$ tel que $(x',y') \in f$. 
    En outre, $f(x') \neq y$, donc $x' \neq E_y$.
    Donc, $x' \in E \setminus E_y$ et $(x',y') \in g$. 

    \done

\medskip

\noindent\textbf{Lemme :} Soit $E$ et $F$ deux ensembles. 
    On suppose que $F$ est fini et qu'il existe une injection de $E$ vers $F$.
    Alors $E$ est fini et $\abs{E} \leq \abs{F}$.

\medskip

\noindent\textbf{Démonstration :} Si $E = \emptyset$, alors $E$ est fini et $\abs{E} = 0$, donc $\abs{E} \leq \abs{F}$.
    Sinon, puisqu'il existe une injection de $E$ vers $F$, alors il existe une surjection de $F$ vers $E$.
    D'après le lemme précédent, $E$ est donc fini et $\abs{E} \leq \abs{F}$.

    \done

\medskip

\noindent\textbf{Lemme :} Soit $E$ et $F$ deux ensembles de même cardinal fini. 
    Soit $f$ une injection de $E$ vers $F$.
    Alors $f$ est une bijection. 

\medskip

\noindent\textbf{Démonstration :} On procède par récurrence sur le cardinal de $E$, noté $n$.
    Si $n = 0$, alors $E = F = \emptyset$. 
    La seule fonction de $\emptyset$ vers lui-même est $\emptyset$, qui est une bijection. 
    
    Soit $n$ un entier naturel et supposons le résultat vrai pour des ensembles de cardinal $n$. 
    Soit $E$ et $F$ deux ensembles de cardinal $n+1$ et $f$ une injection de $E$ vers $F$. 
    Montrons que $f$ est une bijection.
    Soit $e$ un élément de $E$ (un tel élément existe puisque $E$ n'est pas de cardinal $0$, et donc n'est pas l'ensemble vide). 
    Soit $g$ l'ensemble défini par : $g = f \setminus \lbrace (e, f(e)) \rbrace$. 
    Montrons que $g$ est une injection de $E \setminus \lbrace e \rbrace$ vers $F \setminus \lbrace f(e) \rbrace$. 
    \begin{itemize}[nosep]
        \item Montrons d'abord qu'il s'agit bien d'une fonction du premier ensemble vers le second. 
            \begin{itemize}[nosep]
                \item Soit $z$ un élément de $g$. 
                    Alors $z$ est un élément de $f$, donc on peut choisir un élément $x$ de $E$ et un élément $y$ de $F$ tels que $z = (x,y)$.
                    En outre, $z$ ne peut être égal à $(e,f(e))$. 
                    Puisqu'un élément de $E$ ne peut avoir qu'une image par une fonction, $x$ doit être distinct de $e$ (sans quoi on aurait $y = f(e)$ par unicité de l'image et donc $z = (e,f(e))$).
                    Puisque $f$ est injective, on doit donc avoir $y \neq f(e)$ (car le seul élément $w$ de $E$ satisfaisant $f(w) = f(e)$ est $e$).
                    Donc, $z \in (E \setminus \lbrace e \rbrace) \times (F \setminus \lbrace f(e) \rbrace)$.
                \item Soit $x$ un élément de $E \setminus \lbrace e \rbrace$. 
                    Puisque $f$ est une fonction de $E$ vers $F$, on peut choisir un élément $y$ de $F$ tel que $(x,y) \in f$. 
                    Puisque $x \neq e$, $(x,y) \neq (e,f(e))$, donc $(x,y) \in g$.
                \item Soit $x$ un élément de $E$ et $y$ et $y'$ deux éléments de $F$ tels que $(x,y) \in g$ et $(x,y') \in g$.
                    Alors, $(x,y) \in f$ et $(x,y') \in f$.
                    Puisque $f$ est une fonction, on en déduit $y = y'$.
            \end{itemize}
        \item Montrons qu'elle est injective. 
            Soit $x$ et $x'$ deux éléments de $E \setminus \lbrace e \rbrace$ tels que $g(x) = g(x')$. 
            Puisque $(x,g(x)) \in g$ et $(x',g(x')) \in g$, on a $(x,g(x)) \in f$ et $(x',g(x')) \in f$, donc $f(x) = g(x)$ et $f(x') = g(x')$, donc $f(x) = f(x')$. 
            Puisque $f$ est injective, on en déduit $x = x'$.
    \end{itemize}

    Ainsi, $g$ est une injection de $E \setminus \lbrace e \rbrace$ vers $F \setminus \lbrace f(e) \rbrace$. 
    Puisque $e \in E$ et $f(e) \in F$, ces deux ensembles sont de cardinal $n$, on en déduit que $g$ est une bijection. 
    Montrons qu'alors $f$ est surjective. 
    Soit $y$ un élément de $F$. 
    Si $y = f(e)$, alors $y$ a un antécédent par $f$ (et cet antécédent est $e$). 
    Sinon, $y \in F \setminus \lbrace f(e) \rbrace$. 
    Puisque $g$ est une bijection, donc surjective, on peut choisir un élément $x$ de $E \setminus \lbrace e \rbrace$ tel que $g(x) = y$.
    Donc, $(x,y) \in g$.
    Puisque $g$ est un sous-ensemble de $f$, on a donc $(x,y) \in f$, et donc $y$ a un antécédent par $f$ (et cet antécédent est $x$). 
    Ainsi, $f$ est surjective. 

    La fonction $f$ est donc injective et surjective. 
    C'est donc une bijection.

   \done 

\medskip

\noindent\textbf{Lemme :} Soit $E$ et $F$ deux ensembles de même cardinal fini. 
    Soit $f$ une surjection de $E$ vers $F$.
    Alors $f$ est une bijection. 

\medskip

\noindent\textbf{Démonstration :} 
    Si $\abs{E} = 0$, alors $\abs{F} = 0$ et $E = F = \emptyset$. 
    La seule fonction de $\emptyset$ vers lui-même est $\emptyset$, qui est une bijection. 

    Supposons maintenant $\abs{E} > 0$. 
    On supose par l'absurde que $f$ n'est pas bijective. 
    Puisque $f$ est surjective, $f$ n'est donc pas injective. 
    On peut donc choisir un élément $y$ de $F$ ayant au moins deux antécédents distincts par $f$. 
    Notons $E_y$ l'ensemble de ses antécédents, \textit{i.e.}, $E_y = \lbrace x \in E \vert f(x) = y \rbrace$. 
    Puisque $E_y$ est un sous-ensemble de $E$, lui-même fini, $E_y$ est fini. 
    En outre, on sait qu'il existe deux éléments $x$ et $x'$ de $E_y$ tels que $x \neq x'$, donc $\lbrace x, x' \rbrace \subset E_y$.
    Puisque $\abs{\lbrace x, x' \rbrace} = 2$ (en effet, $\lbrace (x, 0), (x', 1) \rbrace$ est une bijection de $\lbrace x, x' \rbrace$ vers $2$), on en déduit $\abs{E_y} \geq 2$.
    En outre, puisque $E_y \subset E$, $\abs{E \setminus E_y} = \abs{E} - \abs{E_y}$. 
    Donc, $\abs{E \setminus E_y} < \abs{E} - 1$. 
    Donc, $\abs{E \setminus E_y} < \abs{F \setminus \lbrace y \rbrace}$ (puisque $\abs{F \setminus \lbrace y \rbrace} = \abs{F} - 1 = \abs{E} - 1$). 

    Définissons l'ensemble $g$ par : $g = f \setminus \lbrace z \in f \vert \exists x \in E \, z = (x, y) \rbrace$. 
    Montrons que $g$ est une surjection de $E \setminus E_y$ vers $F \setminus \lbrace y \rbrace$. 
    Cela impliquera $\abs{F \setminus \lbrace y \rbrace} \leq \abs{E \setminus E_y}$, en contradiction avec le résultat précédent. 
    On pourra alors conclure que l'hypothèse de départ est fausse, et que $f$ doit donc être une bijection.
    
    Montrons d'abord que $g$ et une fonction de $E \setminus E_y$ vers $F \setminus \lbrace y \rbrace$.
    \begin{itemize}[nosep]
        \item Soit $z$ un élément de $g$. 
            Puisque $z \in f$, on peut choisir un élément $x'$ de $E$ et un élément $y'$ de $F$ tels que $z = (x',y')$.
            Par ailleurs, on doit avoir $y' \neq y$, et donc $x' \notin E_y$ (sans quoi on aurait $f(x') = y$ et donc $y' = y$). 
            Donc, $z \in \left( E \setminus E_y \right) \times \left( F \setminus \lbrace y \rbrace \right)$.
        \item Soit $x'$ un élément de $E \setminus E_y$. 
            On a $x' \in E$. 
            Donc, on peut choisir un élément $y'$ de $F$ tel que $(x',y') \in f$.
            On a alors $y' = f(x')$, donc (puisque $x' \notin E_y$) $y' \neq y$.
            Donc, $(x',y') \in g$.
        \item Soit $x'$, $y'$ et $y''$ trois ensembles tels que $(x', y') \in g$ et $(x', y'') \in g$. 
            Alors, $(x', y') \in f$ et $(x', y'') \in f$.
            Puisque $f$ est une fonction, on en déduit $y' = y''$.
    \end{itemize}
    Ainsi, $g$ est bien une fonction de $E \setminus E_y$ vers $F \setminus \lbrace y \rbrace$.

    Montrons qu'elle est surjective. 
    Soit $y'$ un élément de $F \setminus \lbrace y \rbrace$. 
    Puisque $y' \in F$ et puisque $f$ est surjective, on peut choisir un élément $x'$ de $E$ tel que $f(x') = y'$. 
    Puisque $y' \neq y$, $(x',y')$ est donc un élément de $g$, donc $g(x') = y'$. 
    La fonction $g$ est donc bien surjective.

    \done

\medskip

\noindent\textbf{Lemme :} Soit $E$ et $F$ deux ensembles finis. 
    Alors les ensembles $E \cap F$ et $E \cup F$ sont finis, et $\abs{E \cup F} = \abs{E} + \abs{F} - \abs{E \cap F}$.

\medskip

\noindent\textbf{Démonstration :} On procède par récurrence sur le cardinal de $E$. 
    On veux montrer le prédicat $P$ à un paramètre libre défini pour tout entier naturel $n$ par : $P(n): \forall E \, \forall F \, \abs{E} = n \wedge \abs{F} \in \mathbb{N} \Rightarrow \abs{E \cap F} \in \mathbb{N} \wedge \abs{E \cup F} = \abs{E} + \abs{F} - \abs{E \cap F}$. 

    Montrons d'abord $P(0)$. 
    Soit $E$ un ensemble de cardinal nul et $F$ un ensemble fini. 
    Alors, $E = \emptyset$. 
    Donc, $E \cap F = \emptyset$ et $E \cup F = F$. 
    Donc, $E \cap F$ et $E \cup F$ sont finis, $\abs{E \cap F} = 0$ et $\abs{E \cup F} = \abs{F}$. 
    Donc, $\abs{E \cup F} = \abs{E} + \abs{F} - \abs{E \cap F}$.
    Cela montre que $P(0)$ est vrai.

    Soit $n$ un entier naturel et supposons que $P(n)$ est vrai. 
    Montrons qu'alors $P(n+1)$ est vrai.
    Soit $E$ un ensemble fini de cardinal $n+1$ et $F$ un ensemble fini. 
    Puisque $n+1 > 0$, $E$ n'est pas l'ensemble vide, donc on peut choisir un élément $e$ de $E$. 
    Soit $E'$ l'ensemble $E \setminus \lbrace e \rbrace$. 
    Puisque $e \in E$, $E'$ a pour cardinal $(n+1)-1$, c'est-à-dire $n$. 
    Donc, $E' \cap F$ et $E' \cup F$ sont finis, et $\abs{E' \cup F} = n + \abs{F} - \abs{E' \cap F}$. 
    Notons $m$ le cardinal de $F$ et $k'$ celui de $E' \cap F$.
    Distinguons deux cas, selon que $e$ appartienne ou non à $F$. 
    
    Supposons d'abord $e \in F$. 
    Alors, $E \cap F = (E' \cap F) \cup \lbrace e \rbrace$%
    ~\footnote{En effet, 
        \begin{itemize}[nosep]
            \item Soit $x$ un élément de $E \cap F$. 
                Puisque $x \in E$, $x \in E'$ ou $x = e$. 
                Dans le premier cas, et puisque $x \in F$, $x \in E' \cap F$. 
                Dans le second cas, $x \in \lbrace e \rbrace$.
            \item Soit $x$ un élément de $(E' \cap F) \cup \lbrace e \rbrace$.
                Alors, $x \in E' \cap F$ ou $x = e$. 
                Dans le premier cas, $x \in E \cap F$ puisque $E'$ est un sous-ensemble de $E$.
                Dans le second cas, $x \in E \cap F$ puisque $e$ est un élément de $E$ et de $F$.
        \end{itemize}
    }.
    Puisque $E'$ ne contient pas $e$, $e \notin E' \cap F$, donc $E \cap F$ est fini et a pour cardinal $k' + 1$.
    Par ailleurs, $E \cup F = E' \cup F$%
    ~\footnote{En effet, 
        \begin{itemize}[nosep]
            \item $E' \cup F \subset E \cup F$ puisque $E' \subset E$.
            \item Soit $x$ un élément de $E \cup F$. 
                Alors, $x \in F$ ou $x \in E$.
                Dans le premier cas, $x \in E' \cup F$.
                Sinon, $x \in E$ et $x \neq e$, donc $x \in E'$, et donc $x \in E' \cup F$.
        \end{itemize}
    }.
    Donc, $E \cup F$ est fini et a pour cardinal $n + m - k'$.
    Puisque $n + m - k' = (n+1) + m - (k'+1)$, on en déduit que $\abs{E \cup F} = \abs{E} + \abs{F} - \abs{E \cap F}$.

    Supposons maintenant que $e \notin F$. 
    Alors, $E \cap F = E' \cap F$%
    ~\footnote{En effet, 
        \begin{itemize}[nosep]
            \item Soit $x$ un élément de $E \cap F$. 
                Puisque $x \in E$, $x \in E'$ ou $x = e$. 
                Puisque $x \in F$, $x \neq E$. 
                Donc, $x \in E'$ et $x \in F$, donc $x \in E' \cap F$.
            \item Soit $x$ un élément de $E' \cap F$.
                Puisque $E'$ est un sous-ensemble de $E$, $x \in E$.
                Donc, $x \in E$ et $x \in F$, donc $x \in E \cap F$.
        \end{itemize}
    }.
    Donc, $E \cap F$ est fini et a pour cardinal $k'$.
    Par ailleurs, $E \cup F = (E' \cup F) \cup \lbrace e \rbrace$%
    ~\footnote{En effet, 
        \begin{itemize}[nosep]
            \item Soit $x$ un élément de $E \cup F$. 
                Alors, $x \in F$ ou $x \in E$.
                Dans le premier cas, $x \in E' \cup F$.
                Sinon, $x \in E$, donc $x \in E'$ ou $x = e$. 
                Dans le premier cas, $x \in E' \cup F$. 
                Dans le second, $x \in \lbrace e \rbrace$
            \item Soit $x$ un élément de $(E' \cup F) \cup \lbrace e \rbrace$. 
                Alors, $x$ appartient à $E'$, à $F$ ou à $\lbrace e \rbrace$. 
                Dans le premier ou le troisième cas, $x$ appartient à $E$.
                Dans le second cas, $x$ appartient à $F$. 
                Donc, dans tous les cas, $x \in E \cup F$.
        \end{itemize}
    }.
    Puisque $e$ n'est pas un élément de $E'$ ni de $F$, on en déduit $\abs{E \cup F} = \abs{E' \cup F} + 1 = (n + m - k') + 1$. 
    Puisque $(n+1) + m - k' = (n + m - k') + 1$, on en déduit $\abs{E \cup F} = \abs{E} + \abs{F} - \abs{E \cap F}$.

    Ainsi, pour tout entier naturel $n$, $P(n) \Rightarrow P(n+1)$.
    On en conclut que $P(n)$ est vrai pour tout entier naturel $n$. 

    Soit $E$ et $F$ deux ensemble finis. 
    Puisque $\abs{E}$ est un entier naturel, $P(\abs{E})$ est vrai.
    On en déduit que les ensembles $E \cap F$ et $E \cup F$ sont finis et que $\abs{E \cup F} = \abs{E} + \abs{F} - \abs{E \cap F}$.

    \done


\subsubsection{Cas de l'ensemble des entiers naturels}

\noindent\textbf{Lemme :} L'ensemble $\mathbb{N}$ est infini. 

\medskip

\noindent\textbf{Démonstration :} Montrons par récurrence qu'il n'existe aucune bijection entre un entier naturel $n$ et $\mathbb{N}$. 
    Pour $n=0$, cela est évident car $\mathbb{N} \neq \emptyset$, donc il n'existe pas de bijection entre $\mathbb{N}$ et $0$.
    
    Traitons explicitement le cas $n=1$ (bien que cela ne soit pas strictement nécessaire pour la récurrence). 
    Ce cas est évident car, s'il existe une bijection d'un ensemble $E$ vers $1$, alors il existe une bijection de $1$ vers $E$ ; puisque $1$ ne contient qu'un seul élément ($0$) $E$ ne peux alors contenir qu'un seul élément (l'image de $0$ par cette fonction), ce qui est impossible pour $\mathbb{N}$ puisque $0 \in \mathbb{N}$, $1 \in \mathbb{N}$ et $1 \neq 0$.%
    \footnote{
        En effet, il devrait exister deux éléments de $1$ dont les images sont $0$ et $1$. 
        Si on note $f$ cette bijection, on aurait $f(0) = 0$ et $f(0) = 1$, ce qui est impossible par définition d'une fonction.
    }

    Soit $n$ un entier naturel tel qu'il n'existe pas de bijection entre $n$ et $\mathbb{N}$. 
    Montrons qu'il n'existe pas de bijection de $n+1$ vers $\mathbb{N}$. 
    Par récurrence, le résultat sera montré pour tout entier naturel.

    On procède par l'absurde : on suppose qu'une telle bijection existe, notée $f$ dans la suite, et on montre que cela mène à une contradiction. 
    Définissons la fonction $g$ de $n$ vers $\mathbb{N}$ par : pour tout élément $x$ de $n$, $g(x) = f(x)$ si $f(x) < f(n)$ et $g(x) = f(x)-1$ sinon. 
    (Cela est possible car, puisque $f$ est injective, on a $f(x) > f(n)$ dans le second cas\footnote{En effet, puisque $n \notin n$, on a $x \neq n$, donc $f(x) \neq f(n)$. Puisque, dans ce second cas, $f(x) < f(n)$ est faux, $f(x) \geq f(n)$ est vrai, et donc $f(x) > f(n)$.}, et donc $f(x) > 0$, donc $f(x) - 1$ est bien un entier naturel.)
    Montrons que $g$ est une bijection, ce qui contredira l'hypothèse faite sur $n$. 

    Montrons d'abord qu'elle est injective. 
    Soit $x$ et $y$ deux éléments de $n$ tels que $g(x) = g(y)$. 
    Si $f(x) < f(n)$ et $f(y) < f(n)$, alors $g(x) = f(x)$ et $g(y) = f(y)$. 
    Donc, $f(x) = f(y)$.
    Puisque $f$ est injective, on a donc $x=y$. 
    Si $f(x) > f(n)$ et $f(y) > f(n)$, alors $g(x) = f(x)-1$ et $g(y) = f(y)-1$. 
    Donc, $f(x)-1 = f(y)-1$, et donc $f(x) = f(y)$.
    Puisque $f$ est injective, on a donc $x=y$. 
    Si $f(x) < f(n)$ et $f(y) > f(n)$, alors $g(x) = f(x)$, donc, $g(x) < f(n)$, alors que $g(y) = f(y)-1$, donc $g(y) \geq f(n)$, ce qui est impossible puisque $g(x) = g(y)$.
    De même, $f(x) > f(n)$ et $f(y) < f(n)$ est impossible (même argument en échangeant les rôles de $x$ et $y$). 
    On a donc nécessairement $x=y$. 
    Cela montre que $g$ est injective. 

    Montrons maintenant qu'elle est surjective. 
    Soit $m$ un élément de $\mathbb{N}$. 
    Puisque $f$ est surjective, on peut choisir deux éléments $x$ et $y$ de $n+1$ tels que $f(x) = m$ et $f(y) = m+1$.
    Si $m < f(n)$, alors $x \neq n$, donc $x \in n$ et $g(x) = m$. 
    Si $m \geq f(n)$, alors $m+1 > f(n)$, donc $y \neq n$ et $g(y) = m$.
    Dans les deux cas, $m$ a donc un antécédent par $g$.
    Ainsi, $g$ est surjective. 
    C'est donc bien une bijection. 

    Cela est contradictoire avec l'hypothèse qu'il n'existe aucune bijection de $n$ sur $\mathbb{N}$. 
    On en déduit qu'il n'existe aucune bijection de $n+1$ vers $\mathbb{N}$.
    Par récurrence, on conclut que, pour tout entier naturel $n$, il n'existe aucune bijection de $n$ vers $\mathbb{N}$, et donc aucune bijection de $\mathbb{N}$ vers $n$. 
    L'ensemble $\mathbb{N}$ est donc infini.

    \done 

\medskip

\noindent\textbf{Lemme :} Soit $E$ un ensemble de cardinal infini. Alors il existe une injection de $\mathbb{N}$ vers $E$. 

\medskip

\noindent\textbf{Démonstration :} Il s'agit de montrer qu'il existe une suite $u$ d'éléments de $E$ deux-à-deux distincts. 
    Pour ce faire, on définit par récurrence une suite $u$ d'éléments de l'ensemble des parties de $\mathbb{N} \times E$ telle que : 
    \begin{itemize}[nosep]
        \item Pour tout entier naturel $n$, $v_n$ est une injection de $n$ dans $E$.
        \item Si $n$, $m$ et $k$ sont trois entiers naturels tels que $k < n$, $m \leq n$ et $k < m$, alors $v_n(k) = v_m(k)$. 
    \end{itemize}
    La suite $u$ définie par $u = \left( v_{n+1}(n) \right)_{n \in \mathbb{N}}$ sera alors une injection de $\mathbb{N}$ dans $E$. 
    En effet, si $n$ et $m$ sont deux entiers naturels tels que $u_n = u_m$, on a $v_{n+1}(n) = v_{m+1}(m)$. 
    Si $n < m$, alors $n < m+1$ et $n < n+1$, donc $v_{m+1}(n) = v_{n+1}(n)$, donc $v_{m+1}(n) = v_{m+1}(m)$, ce qui est impossible puisque $v_{m+1}$ est une injection.
    Si $m < n$, cela donne (même argument en échangeant les rôles de $n$ et $m$) $v_{n+1}(m) = v_{n+1}(n)$, ce qui est impossible puisque $v_{n+1}$ est une injection. 
    On en déduit que $n=m$. 
    La suite $u$ sera donc bien une injection de $\mathbb{N}$ dans $\mathbb{E}$.

    Posons d'abord $v_0 = \emptyset$. 
    Il s'agit bien d'une injection de $0$ dans $E$.

    Soit $n$ un entier naturel et supposons $v_n$ définit. 
    Puisque le cardinal de $E$ n'est pas $n$, $v_n$ ne peut être surjective (sans quoi elle serait une bijection de $n$ vers $E$, et donc $E$ serait de cardinal $n$, et donc fini). 
    Donc, on peut choisir un élément $x$ de $E$ tel que $x$ n'est pas dans l'image de $v_n$. 
    Posons $v_{n+1} = v_n \cup \lbrace (n, x) \rbrace$. 
    Alors, 
    \begin{itemize}[nosep]
        \item Puisque $n+1 = n \cup \lbrace n \rbrace$, $v_{n+1}$ est bien une fonction de $n+1$ vers $E$.%
            \footnote{
                Montrons cela explicitement : 
                \begin{itemize}[nosep]
                    \item Soit $z$ un éllément de $v_{n+1}$.
                        Alors, soit $z \in v_n$, et donc $z \in n \times E$, soit $z = (n,x)$ et donc $z \in \lbrace n \rbrace \times E$.
                        Puisque $n + 1 = n \cup \lbrace n \rbrace$, on a $z \in (n+1) \times E$ dans les deux cas.
                    \item Soit $m$ un élémen de $n+1$.
                        Alors $m \in n$ ou $m = n$.
                        Dans le premier cas, puisque $v_n$ est une fonction de $n$ vers $E$, on peut choisir un élément $y$ de $E$ tel que $(m,y) \in v_n$, et donc $(m,y) \in v_{n+1}$.
                        Dans le second cas, $(m,x) \in v_{n+1}$.
                    \item Soit $m$ un élément de $n+1$ et $y$ et $z$ deux éléments de $E$ tels que $(m,y) \in v_{n+1}$ et $(m,y') \in v_{n+1}$.
                        Si $m = n$, alors $m \notin n$, donc $(m,y) \notin v_n$ et $(m,y') \notin v_n$.
                        Donc, $(m,y) = (n,x)$ et $(m,y') = (n,x)$.
                        Donc, $y = x$ et $y' = x$, donc $y = y'$.
                        Sinon, $(m,y) \neq (n,x)$ et $(m,y') \neq (n,x)$, donc $(m,y) \in v_n$ et $(m,y) \in v_n$.
                        Puisque $v_n$ et une fonction, on en déduit que là aussi $y = y'$.
                \end{itemize}
            }
        \item Puisque $v_n$ est injective et que $x$ n'est pas dans son image, $v_{n+1}$ est injective. (Soit $a$ et $b$ deux éléments de $n+1$ tels que $v_{n+1}(a) = v_{n+1}(b)$, alors $a = b = n$ si $v_{n+1}(a) = x$ et $a$ et $b$ appartiennent à $n$ sinon, et donc $a = b$ car $v_n$ est injective.%
            \footnote{Détaillons un peu cet argument.
            Si $v_{n+1}(a) = x$, alors $v_{n+1}(a)$ n'est pas dans l'image de $v_n$. 
            Pour tout élément $m$ de $n$, on a donc $a \neq m$ (sans quoi on aurait $v_{n+1}(a) = v_{n+1}(m) = v_n(m)$).
            Donc, $a \in (n+1) \setminus n$. 
            Donc, $a = n$.
            Puisque $v_{n+1}(b) = x$, on a aussi (même argument en remplaçant $a$ par $b$) $b = n$, donc $a = b$. 

            Sinon, $(a,v_{n+1}(a)) \in v_n$ et $v_{n+1}(b) \neq x$, donc $(b,v_{n+1}(b)) \in v_n$, donc $(b,v_{n+1}(a)) \in v_n$. 
            Puisque $v_n$ est injective, on en déduit $a = b$. 

            Ainsi, $a = b$ dans tous les cas.
            })
        \item Soit $m$ et $k$ deux entiers naturels tels que $m \leq n + 1$ et $k < m$. 
            Si $m \leq n$, alors $v_m(k) = v_n(k)$, et donc $v_m(k) = v_{n+1}(k)$. 
            Sinon, $m = n+1$, donc $v_m = v_{n+1}$, et donc $v_m(k) = v_{n+1}(k)$.
    \end{itemize}

   \done 

\medskip

Dans la suite de cette section, le symbole $\mathbb{Y}$ désigne $\mathbb{N}$ ou $\mathbb{Z}$.

\medskip

\noindent\textbf{Définition :} Soit $\mathcal{E}$ l'ensemble des fonctions d'une partie de $\mathbb{Y}$ vers $\mathbb{Y}$. 
    On définit par récurrence deux suites d'éléments de $\mathcal{E}$, notées $\Sigma$ et $\Pi$ de la manière suivante : 
    \begin{itemize}[nosep]
        \item $\Sigma_0$ et $\Pi_0$ sont les fonctions de $\lbrace \emptyset \rbrace$ vers $\mathbb{Y}$ telles que $\Sigma_0(\emptyset) = 0$ et $\Pi_0(\emptyset) = 1$.
        \item Pour tout entier naturel $n$, $\Sigma_{n+1}$ et $\Pi_{n+1}$ sont les fonctions de l'ensemble des parties de $\mathbb{Y}$ de cardinal $n+1$ vers $\mathbb{Y}$ telles que, pour tout sous-ensemble $y$ de $\mathbb{Y}$ de cardinal $n$ et tout élément $x$ de $\mathbb{Y}$ tel que $x \notin y$, $\Sigma_{n+1} (y \cup \lbrace x \rbrace) = \Sigma_n (y) + x$ et $\Pi_{n+1} (y \cup \lbrace x \rbrace) = \Pi_n (y) \times x$. 
        (Cela est une bonne définition car tout sous-ensemble de $\mathbb{Y}$ de cardinal $n+1$ peut s'écrire sous cette forme%
        \footnote{
            Soit $E$ un ensemble de cardinal $n+1$. 
            Puisque $n$ est un entier naturel, $n+1 \neq 0$. 
            Donc, $E$ n'est pas l'ensemble vide.
            On peut donc choisir un élément $x$ de $E$. 
            Soit $F$ l'ensemble défini par $F = E \setminus \lbrace x \rbrace$.
            On a alors $\abs{F} = n$, $E = F \cup \lbrace x \rbrace$ et $x \notin F$.
        }
        et car, s'il peut s'écrire sous cette forme de plusieurs manières différentes, le résultat n'en est pas affecté. 
        Ce second point est démontré ci-dessous.)
    \end{itemize} 
    Soit $E$ un sous-ensemble de $\mathbb{Y}$ de cardinal fini $n$. 
    On note $\sum E$ l'entier $\Sigma_n (E)$ et $\prod E$ l'entier $\Pi_n(E)$.

\medskip

\noindent\textbf{Lemme :} Soit $n$ un entier naturel et $\Sigma_n$ et $\Pi_n$ définis comme ci-dessus. 
    Soit $E$ un sous-ensemble de $\mathbb{Y}$ de cardinal $n+1$. 
    Soit $F_1$ et $F_2$ deux sous-ensembles de $E$ de cardinal $n$ et $x_1$ et $x_2$ deux éléments de $E$ tels que $E = F_1 \cup \lbrace x_1 \rbrace = F_2 \cup \lbrace x_2 \rbrace$. 
    Alors $\Sigma_n (F_1) + x_1 = \Sigma_n (F_2) + x_2$ et $\Pi_n (F_1) \times x_1 = \Pi_n (F_2) \times x_2$.

\medskip

\noindent\textbf{Démonstration :}
    On procède de la manière suivante : 
    \begin{itemize}[nosep]
        \item On montre d'abord que, si $x_2 = x_1$, alors $F_2 = F_1$. 
            Dans ce cas, le résultat est alors évident. 
        \item On suppose ensuite que $x_2 \neq x_1$. 
            Alors, $n \geq 1$ (car $\lbrace x_1, x_2 \rbrace \subset E$, donc $\abs{E} \geq \abs{\lbrace x_1, x_2 \rbrace}$, donc $\abs{E} \geq 2$, donc, puisque $n = \abs{E} - 1$, $n \geq 1$), $x_2 \in F_1$ (car $x_2 \in F_1 \cup \lbrace x_1 \rbrace$ et $x_2 \notin \lbrace x_1 \rbrace$) et $x_1 \in F_2$ (même argument en échangeant les indices $1$ et $2$).
            On définit l'ensemble $G$ par $G = F_1 \setminus \lbrace x_2 \rbrace$ et montre que $F_2 \setminus \lbrace x_1 \rbrace = G$. 
        \item On a alors $\Sigma_n (F_1) + x_1 = \Sigma_{n-1} (G) + x_2 + x_1$ et $\Sigma_n (F_2) + x_2 = \Sigma_{n-1} (G) + x_1 + x_2$.
            Puisque l'addition est commutative, on en déduit $\Sigma_n (F_1) + x_1 = \Sigma_n (F_2) + x_2$.
        \item De même, $\Pi_n (F_1) \times x_1 = \Pi_{n-1} (G) \times x_2 \times x_1$ et $\Pi_n (F_2) \times x_2 = \Pi_{n-1} (G) \times x_1 \times x_2$.
            Puisque la multiplication est commutative, on en déduit $\Pi_n (F_1) \times x_1 = \Pi_n (F_2) \times x_2$.
    \end{itemize}

    Notons d'abords que $x_1 \notin F_1$ et $x_2 \notin F_2$. 
    En effet, si $x_1 \in F_1$, on aurait $F_1 = E$, ce qui est impossible puisqu'ils sont de cardinaux différents. 
    Donc, $x_1 \notin F_1$.
    De même, en remplaçant l'indice $1$ par $2$, on montre que $x_2 \notin F_2$.

    Montrons le premier point. 
    Supposons que $x_2 = x_1$. 
    Soit $e_1$ un élément de $F_1$. 
    Alors, $e_1 \in E$ et $e_1 \neq x_1$. 
    Puisque $x_2 = x_1$, on en déduit $e_1 \neq x_2$. 
    Or, puisque $E = F_2 \cup \lbrace x_2 \rbrace$ et $e_1 \in E$, on a $(e_1 \in F_2) \vee (e_1 \in \lbrace x_2 \rbrace)$. 
    Puisque $e_1 \neq x_2$, $e_1 \in \lbrace x_2 \rbrace$ est faux. 
    On en déduit donc que $e_1 \in F_2$.
    Cela montre que $F_1 \subset F_2$. 
    De même, en échangeant les rôles des indices $1$ et $2$, on montre que $F_2 \subset F_1$. 
    Ainsi, $F_1 = F_2$ et le résultat attendu est évident (car $x_2 = x_1$ et $F_2 = F_1$). 
    Dans la suite, on suppose que $x_2 \neq x_1$.

    Montrons maintenant que l'ensemble $G$ défini par $G = F_1 \setminus \lbrace x_2 \rbrace$ satisfait : $G = F_2 \setminus \lbrace x_1 \rbrace$. 
    Soit $x$ un élément de $G$. 
    Puisque $F_1$ est un sous-ensemble de $E$, on a $x \in E$. 
    Puisque $x \neq x_2$ et puisque $E = F_2 \cup \lbrace x_2 \rbrace$, on en déduit $x \in F_2$. 
    En outre, puisque $x_1 \notin F_1$, $x \neq x_1$. 
    Donc, $x \in F_2 \setminus \lbrace x_1 \rbrace$.
    Cela montre que $G \subset F_2 \setminus \lbrace x_1 \rbrace$.
    
    Soit $x$ un élément de $F_2 \setminus \lbrace x_1 \rbrace$.
    Puisque $F_2$ est un sous-ensemble de $E$, on a $x \in E$. 
    Puisque $x \neq x_1$ et puisque $E = F_1 \cup \lbrace x_1 \rbrace$, on en déduit $x \in F_1$. 
    En outre, puisque $x_2 \notin F_2$, $x \neq x_2$. 
    Donc, $x \in G$. 
    Cela montre que $F_2 \setminus \lbrace x_1 \rbrace \subset G$. 
    Ainsi, on a bien $G = F_2 \setminus \lbrace x_1 \rbrace$. 

   \done 

\medskip

\noindent\textbf{Lemme :} Soit $n$ et $E$ un sous-ensemble de $\mathbb{Y}$ de cardinal $n+1$. 
    Soit $y$ un élément de $\mathbb{Y}$.
    Soit $F$ l'ensemble définit par : $F = \lbrace x \in \mathbb{Y} \vert \exists e \in E \, x = y e \rbrace$.
    Alors, $\sum F = y \sum E$.

\medskip

\noindent\textbf{Démonstration :}
    Supposons d'abord que $y \neq 0$.

    On procède par récurrence sur $n$. 
    Si $n = 0$, alors $E = \emptyset$, donc $F = \emptyset$ et $\sum E = \sum F = 0$.
    Puisque $y \times 0 = 0$, on a bien $\sum F = y \sum E$.

    Soit $n$ un entier naturel non nul et supposons le résutat vrai. 
    Montrons qu'il reste vrai en remplaçant $n$ par $n+1$. 
    Soit $E$ un sous-ensemble de $\mathbb{Y}$ de cardinal $n+1$. 
    Soit $e$ un élément de $E$ (un tel élément existe puisque $n+1 > 0$, donc $E$ est non vide).
    Soit $E'$ l'ensemble définit par $E' = E \setminus \lbrace e \rbrace$. 
    Soit $F$ l'ensemble définit par : $F = \lbrace x \in \mathbb{Y} \vert \exists e \in E \, x = y e \rbrace$ et $F'$ l'ensemble définit par : $F' = \lbrace x \in \mathbb{Y} \vert \exists e \in E' \, x = y e \rbrace$.
    Puisque $E'$ est de cardinal $n$, $\sum F' = y \sum E'$.
    En outre, on a $F = F' \cup \lbrace y e \rbrace$ et $y e \notin F'$.
    En effet, 
    \begin{itemize}[nosep]
        \item Soit $x$ un élément de $F$.
            Alors, il existe un élémetnt $w$ de $E$ tel que $x = y w$.
            Puisque $E = E' \cup \lbrace e \rbrace$, $w \in E'$ ou $w = e$.
            Dans le premier cas, $y w \in F'$, donc $x \in F'$.
            Dans le second cas, $y w = y e$, donc $x \in \lbrace y e \rbrace$.
        \item Soit $x$ un élément de $F' \cup \lbrace y e \rbrace$. 
            Si $x \in F'$, on peut choisir un élément $w$ de $E'$ tel que $x = y w$.
            Puisque $E'$ est un sous-ensemble de $E$, $y w \in F$, donc $x \in F$.
            Sinon, $x = y e$.
            Puisque $e \in E$, on a alors $x \in F$.
        \item Si $y e \in F'$, on pourrait choisir un élément $e'$ de $E'$ tel que $y e' = y e$.
            Puisque $y > 0$, cela impliquerait $e' = e$, ce qui est impossible puisque $e \notin E'$.
            DOnc, $ye \notin F$.
    \end{itemize}
    Donc, $\sum F = \sum F' + y e = y \sum E' + y e = y (\sum E' + e) = y \sum E$.

    Par récurrece, on en déduit que le résultat est vrai pour tout entier naturel $n$.

    Supposons maintenant $y = 0$.
    Alors, $y \sum E = 0$.
    En outre, $F$ est vide si $E$ est vide ou contient $0$ pour seul élément sinon.
    Dand les deux cas, $\sum F = 0$, donc $\sum F = y \sum E$.

    \done

\subsubsection{Ensemble défini par une liste d'éléments} 

Soit $p$ un entier naturel non nul et $a_1$, $a_2$, ..., $a_p$ des ensembles. 
On note $\lbrace a_1, a_2, \dots, a_p \rbrace$ l'ensemble contenant exactement $a_1$, $a_2$, ..., $a_p$, \textit{i.e.}, 
\begin{equation*}
    \lbrace a_1, a_2, \dots, a_p \rbrace = \lbrace x \vert \exists i \, (i \in [\![1,p]\!]) \wedge (x = a_i) \rbrace.
\end{equation*}

Supposons les ensembles $a_1$, $a_2$, ..., $a_p$ sont deux-à-deux distincts. 
Alors, l'ensemble $\lbrace a_1, a_2, \dots, a_p \rbrace$, noté $E$ dans la suite de ce paragraphe, a pour cardinal $p$. 
En effet, la fonction $f$ de $p$ vers $E$ définie par : pour tout élément $i$ de $p$, $f(i) = a_{i+1}$ est
\begin{itemize}[nosep]
    \item surjective : Soit $y$ un élément de $E$, on peut choisir un entier naturel $j$ dans $[\![1,p]\!]$ tel que $y = a_j$.
        Puisque $j \geq 1$, $j-1$ est un entier naturel.
        Puisque $j \leq p$, $j-1 < p$, donc $j-1 \in p$. 
        On a $f(j-1) = a_{(j-1)+1} = a_j = y$, donc $y$ a un antécédant par $f$.
    \item injective : Soit $i$ et $j$ deux éléments de $p$ tels que $f(i) = f(j)$.
        Alors, $a_{i+1} = a_{j+1}$.
        Puisque les ensembles $a_1$, $a_2$, ..., $a_p$ sont deux-à-deux distincts (ce qui se traduit par : $\forall i \in [\![1,p]\!] \, i \neq j \Rightarrow a_i \neq a_j$), on a donc $i+1 = j+1$ (sans quoi $a_{i+1}$ serait différent de $a_{j+1}$) et donc $i = j$.
\end{itemize}
La fonction $f$ est donc bijective, ce qui montre que $\abs{E} = p$.

\subsubsection{Intervalle de \texorpdfstring{$\mathbb{N}$}{N} ou \texorpdfstring{$\mathbb{Z}$}{Z}}

\noindent\textbf{Lemme :} Soit $a$ et $b$ deux éléments de $\mathbb{N}$. 
    Alors, $[\![a, b]\!]$ est fini et
    \begin{itemize}[nosep]
        \item si $a > b$, $\abs{[\![a, b]\!]} = 0$, 
        \item sinon, $\abs{[\![a,b]\!]} = b - a + 1$.
    \end{itemize}

\medskip

\noindent\textbf{Démonstration :}
    \begin{itemize}[nosep]
        \item Supposons $a > b$. 
            Soit $x$ un élément de $[\![a,b]\!]$, on a $x \geq a$, donc $x > b$, et $x \leq b$, ce qui est impossible.
            On en déduit que $[\![a, b]\!] = \emptyset$, donc $[\![a, b]\!]$ est fini et de cardinal $0$.
        \item Supposons $a \leq b$ et considérons la fonction $f$ de $[\![a, b]\!]$ vers $b-a+1$ définie par : pour tout élément $x$ de $[\![a, b]\!]$, $f(x) = x - a$. 
            Cette fonction est bien définie car, soit $x$ un élément de $[\![a, b]\!]$, on a $x \geq a$, donc $x - a$ est bien un entier naturel, et $x \leq b$, donc $x - a \leq b - a$, donc $x-a \in b-a+1$. 
            Montrons qu'elle est bijective.
            \begin{itemize}[nosep]
                \item Soit $x$ et $y$ deux éléments de $[\![a, b]\!]$ tels que $f(x) = f(y)$.
                    Alors, $x-a = y-a$. 
                    En ajoutant $a$ des deux côtés, on obtent $x = y$.
                    Ainsi, $f$ est injective.
                \item Soit $y$ un élément de $b-a+1$. 
                    Alors, $y \leq b-a$. 
                    Donc, $y + a \leq b$.
                    En outre, puisque $y \geq 0$, $y + a \geq a$.
                    Donc, $y+a \in [\![a,b]\!]$.
                    On a  : $f(y+a) = y$. 
                    Donc, $y+a$ est un antécédent de $y$ par $f$. 
                    On en déduit que $f$ est surjective.
            \end{itemize}
    \end{itemize}

    \hfill \square

\medskip

\noindent\textbf{Lemme :} Soit $a$ et $b$ deux éléments de $\mathbb{Z}$. 
    Alors, $[\![a, b]\!]$ est fini et
    \begin{itemize}[nosep]
        \item si $a > b$, $\abs{[\![a, b]\!]} = 0$, 
        \item sinon, $\abs{[\![a,b]\!]} = \abs{b - a} + 1$.
    \end{itemize}

\medskip

\noindent\textbf{Démonstration :}
    \begin{itemize}[nosep]
        \item Supposons $a > b$. 
            Soit $x$ un élément de $[\![a,b]\!]$, on a $x \geq a$, donc $x > b$, et $x \leq b$, ce qui est impossible.
            On en déduit que $[\![a, b]\!] = \emptyset$, donc $[\![a, b]\!]$ est fini et de cardinal $0$.
        \item Supposons $a \leq b$ et considérons la fonction $f$ de $[\![a, b]\!]$ vers $\abs{b-a}+1$ définie par : pour tout élément $x$ de $[\![a, b]\!]$, $f(x) = \abs{x - a}$. 
            Cette fonction est bien définie car, soit $x$ un élément de $[\![a, b]\!]$, on a $x \geq a$, donc $x - a$ est positif, et $x \leq b$, donc $x - a \leq b - a$, donc $\abs{x-a} \leq \abs{b-a}$, donc $\abs{x-a} \in \abs{b-a} + 1$. 
            Montrons qu'elle est bijective.
            \begin{itemize}[nosep]
                \item Soit $x$ et $y$ deux éléments de $[\![a, b]\!]$ tels que $f(x) = f(y)$.
                    Alors, $\abs{x-a} = \abs{y-a}$. 
                    Puisque $x-a$ et $y-a$ sont tous deux positifs, cela implique $x - a = y - a$, et donc $x = y$.
                    Ainsi, $f$ est injective.
                \item Soit $y$ un élément de $\abs{b-a}+1$. 
                    Alors, $y \leq \abs{b-a}$. 
                    Donc, $(0,y) \leq (0, \abs{b-a})$. 
                    Puisque $b-a$ est positif (puisque $b \geq a$), $b - a = (0, \abs{b-a})$, et donc $(0, y) \leq b-a$.
                    Donc, $(0, y) + a \leq b$.
                    En outre, puisque $(0,y) \geq 0$, $(0,y) + a \geq a$.
                    Donc, $(0,y)+a \in [\![a,b]\!]$.
                    On a  : $f((0,y)+a) = \abs{(0,y)} = y$. 
                    Donc, $(0,y)+a$ est un antécédent de $y$ par $f$. 
                    On en déduit que $f$ est surjective.
            \end{itemize}
    \end{itemize}

    \hfill \square

\subsubsection{Ensemble dénombrable}

\noindent\textbf{Définition :} Un ensemble $E$ est dit \textit{dénombrable} s'il existe une bijection de $E$ vers $\mathbb{N}$. 
    De manière équivaleente, un ensemle $E$ est dénombrable si et seulement si il existe une bijection de $\mathbb{N}$ vers $E$.

\medskip

\noindent\textbf{Remarque :} Puisque $\mathbb{N}$ est infini, un ensemble dénombrable est nécéssairement infini.

\medskip

Nous montrerons ci-dessous que, $\mathbb{N}^2$ est \hyperlink{N_to_N2}{dénombrable}, et que, en fait, $\mathbb{N}^n$ est dénombrable pour tout élément $n$ de $\mathbb{N}^*$. 
Par contre, l'ensemble $\mathcal{F}\left( \mathbb{N}, \lbrace 0, 1 \rbrace \right)$ \hyperlink{F_N_0_1_non_denombrable}{n'est pas dénombrable}.

\subsubsection{Théorème de Cantor-Bernstein}

\noindent\textbf{Théorème :} Soit $E$ et $F$ deux ensembles. On suppose qu'il existe une injection de $E$ vers $F$ et une injection de $F$ vers $E$. 
    Alors, il existe une bijection de $E$ vers $F$.

\medskip

\noindent\textbf{Démonstration :} 
    Nous nous proposons de démontrer ce théorème en deux étapes : 
    \begin{itemize}[nosep]
        \item Nous montrerons d'abord le lemme suivant : Soit $A$ et $B$ deux ensembles tels que $B \subset A$. 
            On suppose qu'il existe une injection de $A$ vers $B$. 
            Alors il existe une bijection de $A$ vers $B$. 
        \item Nous en déduirons le théorème.
    \end{itemize}
    
    Commençons par le second point, qui est le plus facile. 
    On suppose le lemme vrai. 
    Soit $E$ et $F$ deux ensembles. 
    On suppose qu'il existe une injection $f$ de $E$ vers $F$ et une injection $g$ de $F$ vers $E$.
    Notons $B$ l'image de $g$ ; il s'agit d'un sous-ensemble de $E$. 
    Considérons l'application $u$ de $E$ vers $B$ définie par $u = g \circ f$. 
    Montrons que cette fonction est une injection. 
    Soit $x$ et $y$ deux éléments de $B$ tels que $u(x) = u(y)$. 
    On a : $g(f(x)) = g(f(y))$. 
    Puisque $g$ est injective, cela implique $f(x) = f(y)$. 
    Puisque $f$ est aussi injective, cela implique à son tour $x = y$. 
    Ainsi, $u$ est bien injective. 
    Donc, il existe une injection de $E vers B$.

    D'après le lemme, il existe donc une bijection de $E$ vers $B$. 
    Notons-la $l$.
    Soit $h$ la fonction de $F$ vers $B$ définie par $h(x) = g(x)$ pour tout élément $x$ de $F$. 
    Puisque $g$ est une injection, $h$ en est aussi une. 
    (Si $x$ et $y$ sont deux éléments de $F$ tels que $h(x)=h(y)$, on a $g(x) = g(y)$ et donc $x=y$.)
    En outre, elle est surjective par définition de $B$. 
    (Soit $y$ un élément de $B$, il existe un élément $x$ de $F$ tel que $g(x)=y$ et donc $h(x)=y$.)
    Donc, $h$ est une bijection.
    \footnote{Autrement dit, la fonction $g$ est une bijection de $F$ vers $B$.}
    Considérons la fonction $h^{-1} \circ l$. 
    Puisque $l$ est une bijection de $E$ vers $B$ et $h^{-1}$ une bijection de $B$ vers $F$, $h^{-1} \circ l$ est une bijection de $E$ vers $F$, ce qui montre le théorème.

    Montrons maintenant le lemme. 
    Soit $A$ et $B$ deux ensembles tels que $B \subset A$. 
    On suppose qu'il existe une injection $u$ de $A$ vers $B$. 
    On définit alors par récurrence la suite $\left( C_n \right)_{n \in \mathbb{N}}$ de sous-ensembles de $A$% 
    \footnote{Il s'agit d'une suite d'éléments de l'ensemble des parties de $A$.}
    de la manière suivante : 
    \begin{equation*}
        \left\lbrace \begin{aligned}
            & C_0 = A \setminus B \\
            & \forall n \in \mathbb{N}^* \; C_n = u \left( C_{n-1} \right)
        \end{aligned} \right. .
    \end{equation*}
    Notons $C$ la réunion de ces ensembles : $C = \lbrace y \in A \vert \, \exists n \in \mathbb{N} \, y \in C_n \rbrace$.%
    \footnote{Cet ensemble peut aussi être défini plus directement par : $C = \lbrace y \in A \vert \exists x \in A \setminus B \, \exists n \in \mathbb{N} \, u^n(x) = y \rbrace$.} 
    Notons que $u(C) \subset C$. En effet, soit $x$ un élément de $C$, il existe un élément $n$ de $\mathbb{N}$ tel que $x \in C_n$, donc $u(x) \in C_{n+1}$, et donc $u(x) \in C$.
    Définissons la fonction $v$ de $A$ vers $B$ par : pour tout élément $x$ de $A$, $v(x) = u(x)$ si $x \in C$ et $v(x) = x$ sinon. 
    (Cette définition est correcte car, pour tout $x \in A$, l'image $v(x)$ de $x$ ainsi définie est dans $B$. En effet, si $x \in C$, alors $v(x) = u(x)$ et, si $x \notin C$, $x \notin C_0$ et donc $x \in B$, d'où $v(x) \in B$.)

    Montrons que $v$ est injective. 
    Soit $x$ et $y$ deux éléments de $A$ tels que $v(x) = v(y)$. 
    Alors, 
    \begin{itemize}[nosep]
        \item Si $x \in C$, $v(x) = u(x)$. Donc, $v(y) = u(x)$. Puisque $u(C) \subset C$, cela implique $v(y) \in C$. 
            Si $y$ n'était pas dans $C$, on aurait $v(y) = y$, d'où $v(y) \notin C$, ce qui n'est pas le cas. 
            Donc, $y \in C$. 
            Donc, $v(y) = u(y)$. 
            On a donc $u(x) = u(y)$. 
            Puisque $u$ est injective, on en déduit $x = y$.
        \item Si $x \notin C$, on a $v(x) = x$. Donc, $v(y) = x$. 
            En outre, $y$ ne peut être un élément de $C$ (en effet, si $y \in C$, alors $u(y) = y$ et donc $u(y) \in C$, ce qui n'est pas le cas). 
            Donc, $v(y) = y$. 
            On en déduit que $y = x$. 
    \end{itemize}

    Motrons maintenant que $v$ est surjective. 
    Soit $y$ un élément de $B$. 
    \begin{itemize}[nosep]
        \item Si $y \notin C$, on a $v(y) = y$, donc $y$ est bien dans l'image de $v$. 
        \item Si $y \in C$, on peut choisir un élément $n$ de $\mathbb{N}$ tel que $y \in C_n$. 
            Cet entier ne peut être égal à $0$ puisque $y \in B$ (et donc $y \notin A \setminus B$). 
            Donc, $n-1$ est un entier naturel et $C_n = u(C_{n-1})$ par définition de $C_n$. 
            Il existe donc un élément $x$ de $C_{n-1}$ tel que $y = u(x)$. 
            Puisque $x \in C_{n-1}$, $x \in C$, donc $v(x) = u(x)$, et donc $v(x) = y$. 
    \end{itemize}

    Ainsi, la fonction $v$ est injective et surjective. 
    C'est donc une bijection.

   \done 

\medskip

\noindent\textbf{Corrolaire :} Soit $E$ un ensemble. On suppose qu'il existe une injection de $\mathbb{N}$ vers $E$. 
    Alors, $E$ a un cardinal infini.

\medskip

\noindent\textbf{Démonstration :} Supposons par l'absurde que ce n'est pas le cas, et montrons qu'on aboutit à une contradiction. 
    Soit $n$ le cardinal de $E$. 
    Il existe une bijection $f$ de $E$ vers $n$. 
    Puisque $n \subset \mathbb{N}$, $f$ est aussi une injection de $E$ vers $\mathbb{N}$.%
    \footnote{En effet, 
    \begin{itemize}[nosep]
        \item Soit $z$ un élément de $f$, $z$ appartient à $E \times n$ et donc à $E \times \mathbb{N}$. 
        \item Soit $x$ un élément de $E$, il existe un unique ensemble $y$ tel que $(x,y) \in f$ puisque $f$ est une fonction.
        \item La fonction $f$ est injective par définition.
    \end{itemize}
    }
    Puisqu'il existe aussi une injection de $\mathbb{N}$ vers $E$, on en déduit d'après le théorème de Cantor-Bernstein qu'il existe une bijection, notée $g$ dans la suite, de $\mathbb{N}$ vers $E$. 
    Alors, $f \circ g$ est une bijection de $\mathbb{N}$ vers $n$, ce qui est impossible puisque $\mathbb{N}$ a un cardinal infini. 
    On en déduit que l'hypothèse de départ ne peut qu'être fausse. 

   \done 

\bigskip

\hypertarget{N_to_N2}
\noindent\textbf{Exemple d'application :} Bijection entre $\mathbb{N}$ et $\mathbb{N}^2$. 
\begin{itemize}[nosep]
    \item Soit $f: \mathbb{N} \to \mathbb{N}^2$ la fonction définie par : $\forall x \in \mathbb{N} \; f(x) = (x,x)$. 
        Cette fonction est une injection de $\mathbb{N}$ dans $\mathbb{N}^2$. 
    \item Soit $g: \mathbb{N}^2 \to \mathbb{N}$ la fonction définie par : $\forall x \in \mathbb{N} \, \forall y \in \mathbb{N} \, g((x,y)) = 2^x 3^y$. 
        D'après l'unicité de la décomposition d'un entier naturel en produit de facteurs premiers (voir section~\ref{subsub:dec_fact_prem}), la fonction $g$ est injective. 
        En effet, soit deux éléments $x$ et $y$ de $\mathbb{N}^2$ tels que $g(x) = g(y)$, la première composante de $x$ doit être égale à celle de $y$ par unicité de la décomposition en facteurs premiers, et de même pour leurs secondes composantes ; donc, $x = y$. 
\end{itemize}
Il existe donc une injection de $\mathbb{N}$ dans $\mathbb{N}^2$ et une injection de $\mathbb{N}^2$ dans $\mathbb{N}$. 
D'après le théorème de Cantor-Bernstein, on en déduit qu'il existe une bijection entre $\mathbb{N}$ et $\mathbb{N}^2$.

\medskip

\noindent\textbf{Exemple d'application 2 :} Bijection entre $\mathbb{N}$ et $\mathbb{N}^n$ pour $n \in \mathbb{N}^*$. 
\begin{itemize}[nosep]
    \item Soit $f: \mathbb{N} \to \mathbb{N}^n$ la fonction définie par : $\forall x \in \mathbb{N} \; f(x) = (x, x, \dots, x)$. 
        Cette fonction est une injection de $\mathbb{N}$ dans $\mathbb{N}^n$. 
    \item Puisqu'il existe une infinité de nombres premiers distincts (voir section~\ref{subsub:defNombresPremiers}), on peut  en choisir $n$, par exemple les $n$ plus petits, notés $p_1$, $p_2$, ..., $p_n$.
        Soit $g: \mathbb{N}^n \to \mathbb{N}$ la fonction définie par : $\forall x \in \mathbb{N}^n \, g((x_1, x_2, \dots, x_n)) = p_1^{x_1} p_2^{x_2} \cdots p_n^{x_n}$. 
        D'après l'unicité de la décomposition d'un entier naturel en produits de facteurs premiers (voir section~\ref{subsub:dec_fact_prem}), la fonction $g$ est injective.%
        \footnote{En effet, si $x$ et $y$ sont éléments de $\mathbb{N}^n$ tels que $g(x) = g(y)$, alors on doit avoir $x_i = y_i$ pour tout entier $i$ dans $[\![1,n]\!]$, et donc $x = y$.}
\end{itemize}
Il existe donc une injection de $\mathbb{N}$ dans $\mathbb{N}^n$ et une injection de $\mathbb{N}^n$ dans $\mathbb{N}$. 
D'après le théorème de Cantor-Bernstein, on en déduit qu'il existe une bijection entre $\mathbb{N}$ et $\mathbb{N}^n$.

\medskip

\hypertarget{F_N_0_1_non_denombrable}
\noindent\textbf{Contre-exemple :} En guise d'exemple d'ensembles qui ne sont pas en bijection, considérons les ensembles $\mathbb{N}$ et $\mathcal{F}(\mathbb{N}, \lbrace 0, 1 \rbrace)$. 
Nous allons montrer qu'il n'existe pas de surjection du premier vers le second.
Suposons par l'absurde qu'au moins une telle surjection existe et appelons l'une d'entre elles $f$. 
Considérons l'élément $g$ de $\mathcal{F}(\mathbb{N}, \lbrace 0, 1 \rbrace)$ défini par :%
~\footnote{Plus formellement, on peut définir $g$ par :
\begin{equation*}
    g = \lbrace
        z \in \mathbb{N} \times \lbrace 0, 1 \rbrace
        \vert
        \exists x \in \mathbb{N} \, 
            z = (x, 1-f(x)(x))
    \rbrace .
\end{equation*}
On montre facilement qu'il s'agit bien d'une fonction de $\mathbb{N}$ vers $\lbrace 0,1 \rbrace$.
}
\begin{equation*}
    \forall x \in \mathbb{N} \; g(x) = 1 - f(x)(x).
\end{equation*}
Puisque $f$ est une surjection, on peut choisir un élément $x$ de $\mathbb{N}$ tel que $g = f(x)$. 
On a alors $g(x) = f(x)(x)$, d'où $1 - f(x)(x) = f(x)(x)$. 
Mais cela est impossible puisque $f(x)(x) \in \lbrace 0, 1 \rbrace$, $1 - 0 = 1$ et $1 - 1 = 0$, et donc $1 - f(x)(x) \neq f(x)(x)$. 
On en conclut que l'hypothèse de départ est fausse : il n'existe aucune bijection de $\mathbb{N}$ vers $\mathcal{F}(\mathbb{N}, \lbrace 0, 1 \rbrace)$. 

\medskip

\noindent\textbf{Définition :} Soit $n$ un entier naturel et $u$ une fonction de $n$ vers $\mathbb{Y}$. 
    Alors, 
    \begin{itemize}[nosep]
        \item Si $n = 0$, on définit $\sum u$ par $0$ et $\prod u$ par $1$.
        \item Si $n = 1$, on définit $\sum u$ par $u(0)$ et $\prod u$ par $u(0)$.
        \item Si $n > 1$, on définit $\sum u$ par $u(0) + u(1) + \cdots + u(n-1)$ et $\prod u$ par $u(0) \times u(1) \times \cdots \times u(n-1)$.
    \end{itemize}
    On noteras aussi $\sum u$ par $\sum_{i=0}^{n-1} u(i)$ et $\prod u$ par $\prod_{i=0}^{n-1} u(i)$.
