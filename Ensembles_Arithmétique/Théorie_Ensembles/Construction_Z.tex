\subsection{Construction de \texorpdfstring{$\mathbb{Z}$}{Z}}

\subsubsection{Définition}

On définit l'ensemble $\mathbb{Z}$ par : 
\begin{equation*}
    \mathbb{Z} = \left\lbrace z \in \mathbb{N} \times \mathbb{N} \middle\vert (\exists n \in \mathbb{N}, z = (0,n)) \vee (\exists n \in \mathbb{N}^*, z = (1,n)) \right\rbrace.
\end{equation*}
Pour tout élément $n$ de $\mathbb{N}$, on note parfois et s'il n'y a pas de confusion possible simplement $n$ l'élément $(0,n)$ et, si $n \neq 0$, $-n$ l'élément $(1,n)$.
On note $\mathbb{Z}^*$ l'ensemble $\mathbb{Z} \setminus \lbrace (0,0) \rbrace$.
On qualifie les éléments de $\mathbb{Z}$ d'\textit{entiers} ou \textit{entiers relatifs}, et ceux de $\mathbb{N}$ d'\textit{entiers naturels}. 

On définit deux fonctions $\mathrm{sgn}: \mathbb{Z} \to \lbrace 0, 1 \rbrace$ et $\mathrm{abs}: \mathbb{Z} \to \mathbb{N}$ de la manière suivante. 
Soit $a$ un élément de $\mathbb{Z}$. 
On peut choisir un élément $\ep$ de $\lbrace 0, 1 \rbrace$ et un élément $n$ de $\mathbb{N}$ tels que $a = (\ep,n)$. 
On pose alors $\mathrm{sgn}(a) = \ep$ et $\mathrm{abs}(a) = n$. 
Le permier est appelé \textit{signe} de l'entier $a$ et le second, aussi noté $\abs{a}$, sa \textit{valeur absolue}.

Un entier est dit \textit{nul} s'il est égal à $(0,0)$.

\subsubsection{Relation d'ordre}

\noindent\textbf{Définition :} 
    On définit la relation binaire $\leq$ sur $\mathbb{Z}$ de la manière suivante. 
    Soit $n$ et $m$ deux éléments de $\mathbb{N}$. 
    Alors, 
    \begin{itemize}[nosep]
        \item $(0,n) \leq (0,m)$ si et seulement si $n \leq m$,
        \item $(1,n) \leq (1,m)$ si et seulement si $m \leq n$,
        \item si $n \neq 0$, $(1,n) \leq (0,m)$ est vrai et $(0,m) \leq (1,n)$ est faux.
    \end{itemize}
    On définit aussi la relation $<$ par : $\forall a \in \mathbb{Z}, \forall b \in \mathbb{Z}, a < b \Leftrightarrow ((a \leq b) \wedge (a \neq b))$, la relation $\geq$ par : $\forall a \in \mathbb{Z}, \forall b \in \mathbb{Z}, a \geq b \Leftrightarrow b \leq a$, et la relation $>$ par : $\forall a \in \mathbb{Z}, \forall b \in \mathbb{Z}, a > b \Leftrightarrow ((a \geq b) \wedge (a \neq b))$.

\medskip

\noindent\textbf{Lemme :} La relation $\leq$ est une relation d'ordre sur $\mathbb{Z}$.

\medskip

\noindent\textbf{Démonstration :} Vérifions qu'elle satisfait les trois propriétés définissant une relation d'ordre : 
\begin{itemize}[nosep]
    \item \textit{Réflexivité :} Soit $x$ un entier relatif. 
        On peut choisir un élément $\ep$ de $\lbrace 0, 1 \rbrace$ et un entier naturel $n$ tel que $x = (\ep, n)$. 
        Puisque $n = n$, on a $n \leq n$, donc $x \leq x$.
    \item \textit{Antisymétrie :} Soit $x$ et $y$ deux éléments de $\mathbb{Z}$ tels que $x \leq y$ et $y \leq x$.
        On peut choisir deux éléments $\ep$ et $\eta$ de $\lbrace 0, 1 \rbrace$ et deux entiers naturels $n$ et $m$ tels que $x = (\ep, n)$ et $y = (\eta, m)$. 
        Montrons d'abord que $\ep = \eta$. 
        Si $\ep = 0$, alors $n \leq m$ implique $\eta = 0$.
        Si $\ep = 1$, alors $m \leq n$ implique $\eta = 1$. 
        Dans les deux cas, on a bien $\ep = \eta$. 
        Donc, $y = (\ep, m)$. 
        D'après les deux premières lignes de la définition de la relation $\leq$ sur $\mathbb{Z}$ (et la commutativité du connecteur $\wedge$ dans le cas $\ep = 1$), $(x \leq y) \wedge (y \leq x)$ implique donc $(n \leq m) \wedge (m \leq n)$.%
        \footnote{
            En effet, 
            \begin{itemize}[nosep]
                \item Si $\ep = 0$, $x \leq y$ implique $n \leq m$ et $y \leq x$ implique $m \leq n$.
                \item Sinon, $\ep = 1$, donc $x \leq y$ implique $m \leq n$ et $y \leq x$ implique $n \leq m$.
            \end{itemize}
        }
        Donc, $n = m$. 
        On en déduit que $x = y$.
    \item \textit{Transitivité :} Soit $x$, $y$ et $z$ trois éléments de $\mathbb{Z}$ tels que $x \leq y$ et $y \leq z$. 
        Alors, 
        \begin{itemize}[nosep]
            \item Si $\mathrm{sgn}(z) = 1$, on doit avoir $\mathrm{sgn}(y) = 1$ (puisque $y \leq z$) et $\mathrm{sgn(x)} = 1$ (puisque $x \leq y$). 
                On peut donc choisir trois entiers naturels $n$, $m$ et $k$ tels que $x = (1,n)$, $y = (1,m)$ et $z = (1,k)$. 
                En outre, on a $n \geq m$ puisque $x \geq y$ et $m \geq k$ puisque $y \geq z$. 
                Donc, $n \geq k$. 
                Donc, $(1,n) \leq (1,k)$, et donc $x \leq z$.
            \item Si $\mathrm{sgn}(z) = 0$ et $\mathrm{sgn}(y) = 1$, on a $\mathrm{sgn}(x) = 1$ puisque $x \leq y$.
                Donc, $x \leq z$.
            \item Si $\mathrm{sgn}(z) = 0$, $\mathrm{sgn}(y) = 0$, et $\mathrm{sgn}(x) = 1$, alors $x \leq z$.
            \item Si $\mathrm{sgn}(z) = 0$, $\mathrm{sgn}(y) = 0$, et $\mathrm{sgn}(x) = 0$, alors on peut choisir trois entier naturels $n$, $m$ et $k$ tels que $x = (0,n)$, $y = (0,m)$ et $z = (0,k)$.
                Puisque $x \leq y$ et $y \leq k$, on a $n \leq m$ et $m \leq k$. 
                Donc, $n \leq k$, donc $(0,n) \leq (0,k)$ et $x \leq z$.
        \end{itemize}
\end{itemize}

\done

\medskip

\noindent\textbf{Corrolaire :} La relation $\geq$ est une relation d'ordre et les relations $<$ et $>$ sont des relations d'ordre strict sur $\mathbb{Z}$.

\medskip

\noindent\textbf{Lemme :} La relation $\leq$ est une relation d'ordre total sur $\mathbb{Z}$.

\medskip

\noindent\textbf{Démonstration :} Soit $a$ et $b$ deux éléments de $\mathbb{Z}$. 
    On peut choisir deux éléments $\ep$ et $\eta$ de $\lbrace 0, 1 \rbrace$ et deux éléments $n$ et $m$ de $\mathbb{N}$ tels que $a = (\ep, n)$ et $b = (\eta, m)$.
    \begin{itemize}[nosep]
        \item Si $\ep = 0$ et $\eta = 1$, on a $b \leq a$.
        \item Si $\ep = 1$ et $\eta = 0$, on a $a \leq b$.
        \item Si $\ep = 0$, $\eta = 0$ et $n \leq m$, on a $a \leq b$.
        \item Si $\ep = 0$, $\eta = 0$ et $\neg (n \leq m)$, on a $n > m$, donc $m \leq n$, et donc $b \leq a$.
        \item Si $\ep = 1$, $\eta = 1$ et $n \leq m$, on a $b \leq a$.
        \item Si $\ep = 1$, $\eta = 1$ et $\neg (n \leq m)$, on a $n > m$, donc $m \leq n$, et donc $a \leq b$.
    \end{itemize}
    Dans tous les cas, on a donc $(a \leq b) \vee (b \leq a)$.

   \done 

\medskip

\noindent\textbf{Corrolaire :} La relation $\geq$ est une relation d'ordre total sur $\mathbb{Z}$.

\medskip

\noindent\textbf{Définitions :} Un entier $x$ est dit : 
\begin{itemize}[nosep]
    \item \textit{positif} si $x \geq 0$,
    \item \textit{négatif} si $x \leq 0$,
    \item \textit{strictement positif} si $x > 0$,
    \item \textit{strictement négatif} si $x < 0$.
\end{itemize}

\medskip

\noindent\textbf{Lemme :} Soit $a$, $b$ et $c$ trois éléments de $\mathbb{Z}$.
    Alors, $(a + c \leq b + c) \Leftrightarrow (a \leq b)$ et $(a + c < b + c) \Leftrightarrow (a < b)$.

\medskip

\noindent\textbf{Démonstration :} 
    On peut choisir trois éléments $\alpha$, $\beta$ et $\gamma$ de $\lbrace 0, 1 \rbrace$ et trois éléments $n$, $m$ et $k$ de $\mathbb{N}$ tels que $a = (\alpha, n)$, $b = (\beta, m)$ et $c = (\gamma, k)$.
    Alors, 
    \begin{itemize}[nosep]
        \item Si $\alpha = \beta = \gamma = 0$, $a + c \leq b + c$ est équivalent à $n + k \leq m + k$ et $a \leq b$ à $n \leq m$.
            Puisque $(n+k \leq m+k) \Leftrightarrow (n \leq m)$, on en déduit $(a + c \leq b + c) \Leftrightarrow (a \leq b)$.
            De même, $a + c < b + c$ est équivalent à $n + k < m + k$ et $a < b$ à $n < m$.
            Puisque $(n+k < m+k) \Leftrightarrow (n < m)$, on en déduit $(a + c < b + c) \Leftrightarrow (a < b)$.
        \item Si $\alpha = \beta = \gamma = 1$, $a + c \leq b + c$ est équivalent à $n + k \geq m + k$ et $a \leq b$ à $n \geq m$.
            Puisque $(n+k \geq m+k) \Leftrightarrow (n \geq m)$, on en déduit $(a + c \geq b + c) \Leftrightarrow (a \geq b)$.
            De même, $a + c > b + c$ est équivalent à $n + k > m + k$ et $a > b$ à $n > m$.
            Puisque $(n+k > m+k) \Leftrightarrow (n > m)$, on en déduit $(a + c > b + c) \Leftrightarrow (a > b)$.
        \item Si $\alpha = 0$ et $\beta = 1$, $a \leq b$ et $a < b$ sont faux. 
            Si, $\gamma = 0$, on a deux possibilités : 
            \begin{itemize}[nosep]
                \item Si $k < m$, $a + c = (0, n+k)$ et $b + c = (1, m-k)$, donc $a+c \leq b+c$ et $a+c < b+c$ sont faux.
                \item Si $k \geq m$, $a + c = (0, n+k)$ et $b + c = (0, k-m)$.
                    Puisque $k-m \leq k$ (puisque $k = (k-m) + m$) et $k \leq k+n$, on a $k-m \leq k+n$, donc $b+c \leq a+c$, donc $a+c \leq b+c$ et $a+c < b+c$ sont faux.
            \end{itemize}
        \item Si $\alpha = 0$ et $\beta = 1$, $a \leq b$ et $a < b$ sont faux. 
            Si, $\gamma = 1$, on a deux possibilités : 
            \begin{itemize}[nosep]
                \item Si $k \leq n$, $a + c = (0, n-k)$ et $b + c = (1, m+k)$, donc $a+c \leq b+c$ et $a+c < b+c$ sont faux.
                \item Si $k > n$, $a + c = (1, k-n)$ et $b + c = (1, m+k)$.
                    Puisque $k-n \leq k$ (puisque $k = (k-n) + n$) et $k < k+m$ (puisque $m > 0$), on a $k-n < k+m$, donc $b+c \leq a+c$, donc $a+c \leq b+c$ et $a+c < b+c$ sont faux.
            \end{itemize}
        \item Si $\alpha = 1$ et $\beta = 0$, $a \leq b$ et $a < b$ sont vrais. 
            Si, $\gamma = 0$, on a deux possibilités : 
            \begin{itemize}[nosep]
                \item Si $k < n$, $a + c = (1, n-k)$ et $b + c = (0, m+k)$, donc $a+c \leq b+c$ et $a+c \neq b+c$, et donc $a+c < b+c$, sont vrais.
                \item Si $k \geq n$, $a + c = (0, k-n)$ et $b + c = (0, k+m)$.
                    Puisque $k-n < k+m$ ($k-n < k$ puisque $n > 0$ et $k \leq k+m$), on a donc $a+c \leq b+c$ et $a+c \neq b+c$, et donc $a+c < b+c$. 
            \end{itemize}
        \item Si $\alpha = 1$ et $\beta = 0$, $a \leq b$ et $a < b$ sont vrais. 
            Si, $\gamma = 1$, on a deux possibilités : 
            \begin{itemize}[nosep]
                \item Si $k \leq m$, $a + c = (1, n+k)$ et $b + c = (0, m-k)$, donc $a+c \leq b+c$ et $a+c \neq b+c$, et donc $a+c < b+c$, sont vrais.
                \item Si $k > m$, $a + c = (1, n+k)$ et $b + c = (1, k-m)$.
                    Puisque $k-m < k+n$ ($k+n > k$ puisque $n > 0$ et $k-m \leq k$), on a donc $a+c \leq b+c$ et $a+c \neq b+c$, et donc $a+c < b+c$. 
            \end{itemize}
        \item Supposons $\alpha = \beta = 0$ et $\gamma = 1$. 
            Alors, 
            \begin{itemize}[nosep]
                \item Si $k \leq n$ et $k \leq m$, on a $a+c = (0,n-k)$ et $b+c =(0,m-k)$.
                    Donc, $(a+c \leq b+c) \Leftrightarrow (n-k \leq m-k)$ et $(a+c = b+c) \Leftrightarrow (n-k = m-k)$.
                    Puisque $(n-k)+k = n$ et $(m-k)+k = m$, $(n-k \leq m-k) \Leftrightarrow (n \leq m)$ et $(n-k = m-k) \Leftrightarrow (n = m)$. 
                    Donc, $(n-k \leq m-k) \Leftrightarrow (a \leq b)$ et $(n-k = m-k) \Leftrightarrow (a = b)$. 
                    Donc, $(a+c \leq b+c) \Leftrightarrow (a \leq b)$ et $(a+c = b+c) \Leftrightarrow (a = b)$. 
                    Donc, $(a+c \leq b+c) \Leftrightarrow (a \leq b)$ et $(a+c < b+c) \Leftrightarrow (a < b)$. 
                \item Si $k > n$ et $k > m$, on a $a+c = (1,k-n)$ et $b+c =(1,k-m)$.
                    Donc, $(a+c \leq b+c) \Leftrightarrow (k-n \geq k-m)$ et $(a+c = b+c) \Leftrightarrow (k-n = k-m)$.
                    Or, $k-n \geq k-m$ est équivalent à $n \leq m$ et $k-n = k-m$ à $n = m$.
                    Donc, $(k-n \geq k-m) \Leftrightarrow (a \leq b)$ et $(k-n = k-m) \Leftrightarrow (a = b)$. 
                    Donc, $(a+c \leq b+c) \Leftrightarrow (a \leq b)$ et $(a+c < b+c) \Leftrightarrow (a < b)$. 
                \item Si $k \leq n$ et $k > m$, alors $m < n$, donc $b < a$, donc $a \leq b$ et $a < b$ sont faux.
                    En outre, $a+c = (0, n-k)$ et $b + c = (1, k-m)$, donc $b+c < a+c$, donc $a+c \leq b+c$ et $a+c < b+c$ sont faux.
                \item Si $k > n$ et $k \leq m$, alors $n < m$, donc $a < b$, donc $a \leq b$ et $a < b$ sont vrais.
                    En outre, $a+c = (1, k-n)$ et $b + c = (0, m-k)$, donc $a+c < b+c$, donc $a+c \leq b+c$ et $a+c < b+c$ sont vrais.
            \end{itemize}
        \item Supposons $\alpha = \beta = 1$ et $\gamma = 0$. 
            Alors, 
            \begin{itemize}[nosep]
                \item Si $k < n$ et $k < m$, on a $a+c = (1,n-k)$ et $b+c =(1,m-k)$.
                    Donc, $(a+c \leq b+c) \Leftrightarrow (n-k \geq m-k)$ et $(a+c = b+c) \Leftrightarrow (n-k = m-k)$.
                    Puisque $(n-k)+k = n$ et $(m-k)+k = m$, $(n-k \geq m-k) \Leftrightarrow (n \geq m)$ et $(n-k = m-k) \Leftrightarrow (n = m)$. 
                    Donc, $(n-k \geq m-k) \Leftrightarrow (a \leq b)$ et $(n-k = m-k) \Leftrightarrow (a = b)$. 
                    Donc, $(a+c \leq b+c) \Leftrightarrow (a \leq b)$ et $(a+c = b+c) \Leftrightarrow (a = b)$. 
                    Donc, $(a+c \leq b+c) \Leftrightarrow (a \leq b)$ et $(a+c < b+c) \Leftrightarrow (a < b)$. 
                \item Si $k \geq n$ et $k \geq m$, on a $a+c = (0,k-n)$ et $b+c =(0,k-m)$.
                    Donc, $(a+c \leq b+c) \Leftrightarrow (k-n \leq k-m)$ et $(a+c = b+c) \Leftrightarrow (k-n = k-m)$.
                    Or, $k-n \leq k-m$ est équivalent à $n \geq m$ et $k-n = k-m$ à $n = m$.
                    Donc, $(k-n \leq k-m) \Leftrightarrow (a \leq b)$ et $(k-n = k-m) \Leftrightarrow (a = b)$. 
                    Donc, $(a+c \leq b+c) \Leftrightarrow (a \leq b)$ et $(a+c < b+c) \Leftrightarrow (a < b)$. 
                \item Si $k < n$ et $k \geq m$, alors $m < n$, donc $b > a$, donc $a \leq b$ et $a < b$ sont vrais.
                    En outre, $a+c = (1, n-k)$ et $b + c = (0, k-m)$, donc $b+c > a+c$, donc $a+c \leq b+c$ et $a+c < b+c$ sont vrais.
                \item Si $k \geq n$ et $k < m$, alors $n < m$, donc $a > b$, donc $a \leq b$ et $a < b$ sont faux.
                    En outre, $a+c = (0, k-n)$ et $b + c = (1, m-k)$, donc $a+c > b+c$, donc $a+c \leq b+c$ et $a+c < b+c$ sont faux.
            \end{itemize}
    \end{itemize}
    Dans tous les cas, on a bien $(a+c \leq b+c) \Leftrightarrow a \leq b$ et $(a+c < b+c) \Leftrightarrow a < b$.
    
    \done

\subsubsection{Addition}

\noindent\textbf{Définition :} 
    On définit l'opération $+$ sur $\mathbb{Z}$ (vue comme une fonction de $\mathbb{Z} \times \mathbb{Z}$ vers $\mathbb{Z}$) de la manière suivante.
    Soit $n$ et $m$ deux élément de $\mathbb{N}$.
    Alors, 
    \begin{itemize}[nosep]
        \item $(0,n) + (0,m) = (0, n+m)$,
        \item si $n \neq 0$ et $m \neq 0$, $(1,n) + (1,m) = (1, n+m)$ ;
        \item si $n \neq 0$ et $n \leq m$, $(1,n) + (0,m) = (0, m-n)$ ;
        \item si $n \neq 0$ et $n > m$, $(1,n) + (0,m) = (1, n-m)$ ;
        \item si $m \neq 0$ et $n < m$, $(0,n) + (1,m) = (1, m-n)$ ;
        \item si $m \neq 0$ et $n \geq m$, $(0,n) + (1,m) = (0, n-m)$.
    \end{itemize}

\medskip

\noindent\textbf{Lemme :} Soit $z$ un élément de $\mathbb{Z}$. 
    Alors $z + 0 = z$ et $0 + z = z$.

\medskip

\noindent\textbf{Démonstration :} Éxaminons tour à tour les deux cas possibles : 
    \begin{itemize}[nosep]
        \item S'il existe un entier naturel $n$ tel que $z = (0,n)$, alors $z + 0 = (0,n+0) = (0,n) = z$ et $0 + z = (0,0+n) = (0,n) = z$.
        \item S'il existe un entier naturel non nul $n$ tel que $z = (1,n)$, alors $z + 0 = (1,n) + (0,0) = (1,n-0) = (1,n) = z$ et $0 + z = (0,0) + (1,n) = (1,n-0) = (1,n) = z$.
    \end{itemize}
    
   \done 

\medskip

\noindent\textbf{Lemme :} L'addition est commutative : pour tous éléments $a$ et $b$ de $\mathbb{Z}$, $a + b = b + a$.

\medskip

\noindent\textbf{Démonstration :} Éxaminons les différents cas possibles : 
    \begin{itemize}[nosep]
        \item Si $\mathrm{sgn}(a) = 0$ et $\mathrm{sgn}(b) = 0$, alors $a + b = (0, \abs{a} + \abs{b})$ et $b + a = (0, \abs{b} + \abs{a})$.
            Puisque l'addition d'entiers naturels est commutative, on a $\abs{a} + \abs{b} = \abs{b} + \abs{a}$, et donc $a + b = b + a$.
        \item Si $\mathrm{sgn}(a) = 1$ et $\mathrm{sgn}(b) = 1$, alors $a + b = (1, \abs{a} + \abs{b})$ et $b + a = (1, \abs{b} + \abs{a})$.
            Puisque l'addition d'entiers naturels est commutative, on a $\abs{a} + \abs{b} = \abs{b} + \abs{a}$, et donc $a + b = b + a$.
        \item Si $\mathrm{sgn}(a) = 0$, $\mathrm{sgn}(b) = 1$ et $\abs{a} \geq \abs{b}$, alors $a + b = (0, \abs{a} - \abs{b})$ et $b + a = (0, \abs{a} - \abs{b})$, donc $a + b = b + a$.
        \item Si $\mathrm{sgn}(a) = 0$, $\mathrm{sgn}(b) = 1$ et $\abs{a} < \abs{b}$, alors $a + b = (1, \abs{b} - \abs{a})$ et $b + a = (1, \abs{b} - \abs{a})$, donc $a + b = b + a$.
        \item Si $\mathrm{sgn}(a) = 1$, $\mathrm{sgn}(b) = 0$ et $\abs{a} > \abs{b}$, alors $a + b = (1, \abs{a} - \abs{b})$ et $b + a = (1, \abs{a} - \abs{b})$, donc $a + b = b + a$.
        \item Si $\mathrm{sgn}(a) = 1$, $\mathrm{sgn}(b) = 0$ et $\abs{a} \leq \abs{b}$, alors $a + b = (0, \abs{b} - \abs{a})$ et $b + a = (0, \abs{b} - \abs{a})$, donc $a + b = b + a$.
    \end{itemize}
    Dans tous les cas, on a bien $a + b = b + a$.

   \done 

\medskip

\noindent\textbf{Lemme :} L'addition est associative : pour tous éléments $a$, $b$ et $c$ de $\mathbb{Z}$, $a + (b + c) = (a + b) + c$.

\medskip

\noindent\textbf{Démonstration :} 
    Soit $a$, $b$ et $c$ trois éléments de $\mathbb{Z}$. 
    On peut choisir trois éléments $\kappa$, $\mu$ et $\nu$ de $\lbrace 0, 1 \rbrace$ et trois éléments $k$, $m$ et $n$ de $\mathbb{N}$ tels que $a = (\kappa, k)$, $b = (\mu, m)$ et $c = (\nu, n)$. 
    Alors, 
    \begin{itemize}[nosep]
        \item Si $\kappa = \mu = \nu = 0$, on a $(a+b)+c = (0,k+m)+c = (0,(k+m)+n)$ et $a+(b+c) = a+(0,m+n) = (0,k+(m+n))$. 
            Puisque l'addition d'entiers naturels est associative, $(k+m)+n = k+(m+n)$. 
            Donc, $(0,(k+m)+n) = (0,k+(m+n))$, et donc $(a+b)+c = a+(b+c)$.
        \item Si $\kappa = \mu = \nu = 1$, on a $(a+b)+c = (1,k+m)+c = (1,(k+m)+n)$ et $a+(b+c) = a+(1,m+n) = (1,k+(m+n))$. 
            Puisque l'addition d'entiers naturels est associative, $(k+m)+n = k+(m+n)$. 
            Donc, $(1,(k+m)+n) = (0,k+(m+n))$, et donc $(a+b)+c = a+(b+c)$.
        \item Si $\kappa = \mu = 0$ et $\nu = 1$, on a $(a+b)+c = (0,k+m)+c$.
            Donc, $(a+b)+c = (0,(k+m)-n)$ si $k+m \geq n$ et $(a+b)+c = (1,n-(k+m))$ sinon. 
            Éxaminons les différentes possibilités pour $a+(b+c)$. 
            \begin{itemize}[nosep]
                \item Si $n > m$ et $(n-m) > k$, alors $n > k+m$ et $a + (b+c) = a + (1,n-m) = (1, (n-m)-k) = (1, n-(m+k))$.
                    Donc, $a+(b+c) = (a+b)+c$.
                \item Si $n > m$ et $(n-m) \leq k$, alors $n \leq k+m$ et $a + (b+c) = a + (1,n-m) = (0, k-(n-m)) = (0, (k+m)-n)$.
                    Donc, $a+(b+c) = (a+b)+c$.
                \item Si $n \leq m$, alors $n \leq k+m$ et $a+(b+c) = a + (0, m-n) = (0, k+(m-n)) = (0, (k+m)-n)$.
                    Donc, $a+(b+c) = (a+b)+c$.
            \end{itemize}
        \item Si $\kappa = \mu = 1$ et $\nu = 0$, on a $(a+b)+c = (1,k+m)+c$.
            Donc, $(a+b)+c = (1,(k+m)-n)$ si $k+m > n$ et $(a+b)+c = (0,n-(k+m))$ sinon. 
            Éxaminons les différentes possibilités pour $a+(b+c)$. 
            \begin{itemize}[nosep]
                \item Si $n \geq m$ et $(n-m) \geq k$, alors $n \geq k+m$ et $a + (b+c) = a + (0,n-m) = (0, (n-m)-k) = (0, n-(m+k))$.
                    Donc, $a+(b+c) = (a+b)+c$.
                \item Si $n \geq m$ et $(n-m) < k$, alors $n < k+m$ et $a + (b+c) = a + (0,n-m) = (1, k-(n-m)) = (1, (k+m)-n)$.
                    Donc, $a+(b+c) = (a+b)+c$.
                \item Si $n < m$, alors $n < k+m$ et $a+(b+c) = a + (1, m-n) = (1, k+(m-n)) = (1, (k+m)-n)$.
                    Donc, $a+(b+c) = (a+b)+c$.
            \end{itemize}
        \item Si $\mu = \nu$ et $\mu \neq \kappa$, on se ramène au deux cas précédents en notant que $a$ et $c$ jouent des rôles interchangeables.
            En effet, si on définit les trois entiers $\bar{a}$, $\bar{b}$ et $\bar{c}$ par $\bar{a} = c$, $\bar{b} = b$ et $\bar{c} = a$, on a $\bar{a} + (\bar{b} + \bar{c}) = (\bar{a} + \bar{b}) + \bar{c}$ d'après les deux cas précédents.
            En utilisant quatre fois la commutativité de l'addition, cela donne $(\bar{c} + \bar{b}) + \bar{a} = \bar{c} + (\bar{b} + \bar{a})$, et donc $(a + b) + c = a + (b + c)$.
        \item Si $\mu = \nu$ et $\mu \neq \kappa$, on se ramène au cas précédent de la manière suivante.
            Puisque $\mathrm{sgn}(a) = \mathrm{sgn}(c)$ et $\mathrm{sgn}(a) \neq \mathrm{sgn}(b)$, on a : $(a+b) + c = (b+a) + c = b + (a+c)$. 
            En utilisant la commutativité de l'addition, il vient : $(a + b) + c = (a + c) + b$. 
            Puisque $\mathrm{sgn}(a) = \mathrm{sgn}(c)$, on a d'après les cas précédents : $(a + c) + b = a + (c + b) = a + (b + c)$.
            Donc, $(a + b) + c = a + (b + c)$.
    \end{itemize}

    \done

\subsubsection{Opposé}

\noindent\textbf{Définition :} On définit l'opération $-$ sur $\mathbb{Z}$, vue comme une fonction de $\mathbb{Z}$ vers $\mathbb{Z}$, de la manière suivante : 
\begin{itemize}[nosep]
    \item $-(0,0) = (0,0)$ ;
    \item soit $n$ un entier naturel non nul, $-(0,n) = (1,n)$.
    \item soit $n$ un entier naturel non nul, $-(1,n) = (0,n)$.
\end{itemize}

\medskip

\noindent\textbf{Lemme :} Soit $z$ un élément de $\mathbb{Z}$. 
    Alors $-(-z) = z$.

\medskip

\noindent\textbf{Démonstration :} Éxaminons tout à tour les trois cas possibles : 
\begin{itemize}[nosep]
    \item Si $z = (0,0)$, alors $-z = z$, donc $-(-z) = -z = z$.
    \item S'il existe un entier naturel non nul $n$ tel que $z = (0,n)$, alors $-z = (1,n)$, donc $-(-z)) = (0,n) = z$.
    \item S'il existe un entier naturel non nul $n$ tel que $z = (1,n)$, alors $-z = (0,n)$, donc $-(-z)) = (1,n) = z$.
\end{itemize}
Dans tous les cas, on a donc bien $-(-z) = z$.

\done

\medskip

\noindent\textbf{Lemme :} Soit $z$ un élément de $\mathbb{Z}$. 
    Alors $z + (-z) = (0,0)$.

\medskip

\noindent\textbf{Démonstration :} Éxaminons tout à tour les trois cas possibles : 
\begin{itemize}[nosep]
    \item Si $z = (0,0)$, alors $-z = z$, donc $z + (-z) = (0,0) + (0,0) = (0,0)$.
    \item S'il existe un entier naturel non nul $n$ tel que $z = (0,n)$, alors $-z = (1,n)$, donc $z + (-z)) = (0,n) + (1,n) = (0,n-n) = (0,0)$.
    \item S'il existe un entier naturel non nul $n$ tel que $z = (1,n)$, alors $-z = (0,n)$, donc $z + (-z)) = (1,n) + (0,n) = (0,n-n) = (0,0)$.
\end{itemize}
Dans tous les cas, on a donc bien $z + (-z) = (0,0)$.

\done

\subsubsection{Soustraction}

\noindent\textbf{Définition :} 
    On définit l'opération $-$ sur $\mathbb{Z}$ (vue comme une fonction de $\mathbb{Z} \times \mathbb{Z}$ vers $\mathbb{Z}$) de la manière suivante.
    Soit $n$ et $m$ deux élément de $\mathbb{Z}$.
    Alors, 
    \begin{itemize}[nosep]
        \item si $n = (0,0)$, alors $m - n = m$ ;
        \item sinon, $m - n = m + (-n)$.
    \end{itemize}

\medskip

\noindent\textbf{Lemme :} Pour tout élément $z$ de $\mathbb{Z}$, on a :
    \begin{itemize}[nosep]
        \item $z - z = 0$,
        \item $z - 0 = z$,
        \item $0 - z = -z$.
    \end{itemize}

\medskip

\noindent\textbf{Démonstration :} Soit $z$ un élément de $\mathbb{Z}$. On a :
    \begin{itemize}[nosep]
        \item $z - z = z + (-z) = 0$,
        \item $z - 0 = z$ par définition,
        \item $0 - z = 0 + (-z) = -z$.
    \end{itemize}

   \done 

\subsubsection{Multiplication}

\noindent\textbf{Définition :} 
    On définit l'opération $\times$ sur $\mathbb{Z}$ (vue comme une fonction de $\mathbb{Z} \times \mathbb{Z}$ vers $\mathbb{Z}$) de la manière suivante.
    Soit $n$ et $m$ deux élément de $\mathbb{N}$.
    Alors, 
    \begin{itemize}[nosep]
        \item si $n = 0$, $(0,n) \times (0,m) = (0, 0)$ et $(0,n) \times (1,m) = (0,0)$ ;
        \item si $m = 0$, $(0,n) \times (0,m) = (0, 0)$ et $(1,n) \times (0,m) = (0,0)$ ;
        \item si $n \neq 0$ et $m \neq 0$, $(0,n) \times (0,m) = (0, n \times m)$ ;
        \item si $n \neq 0$ et $m \neq 0$, $(1,n) \times (0,m) = (1, n \times m)$ ;
        \item si $n \neq 0$ et $m \neq 0$, $(0,n) \times (1,m) = (1, n \times m)$ ;
        \item si $n \neq 0$ et $m \neq 0$, $(1,n) \times (1,m) = (0, n \times m)$. 
    \end{itemize}
    Ces règles sont équivalentes à : soit $a$ et $b$ deux entiers,
    \begin{itemize}[nosep]
        \item si $a = 0$, alors $a \times b = b \times a = 0$,
        \item si $a \neq 0$ et $b \neq 0$, $a \times b = (\ep, \abs{a} \times \abs{b})$, où $\ep$ est égal à $0$ si $\mathrm{sgn}(a) = \mathrm{sgn}(b)$ et $1$ sinon.
    \end{itemize}
    Le symbole $\times$ est parfois omis quand il n'y a pas de confusion possible.

\medskip

\noindent\textbf{Lemme :} Soit $a$ et $b$ deux entiers relatifs.
    Si $a \times b = 0$, alors $a = 0$ ou $b = 0$.

\medskip

\noindent\textbf{Démonstration :}  
    On peut choisir deux entiers naturels $n$ et $m$ et deux éléments $\ep$ et $\eta$ de $\lbrace 0, 1 \rbrace$ tel que $a = (\ep, n)$ et $b = (\eta, m)$. 
    Donc, on peut choisir un élément $\mu$ de $\lbrace 0, 1 \rbrace$ tel que $a \times b = (\mu, n \times m)$.
    (Avec $\mu = 0$ si $\ep = \eta$ ou $n = 0$ ou $m = 0$, et $\mu = 1$ si $\ep \neq \eta$, $n \neq 0$ et $m \neq 0$.)
    Si $a \times b = (0,0)$, on a donc $n \times m = 0$, donc $n = 0$ ou $m = 0$.
    Si $n = 0$, $\ep$ doit être égal à $0$ (puisque $(\ep,n) \in \mathbb{Z}$), donc $a = (0,0)$.
    Sinon, $m = 0$, donc $\eta$ doit être égal à $0$ (puisque $(\eta,m) \in \mathbb{Z}$), donc $b = (0,0)$.

   \done 

\medskip

\noindent\textbf{Lemme :} La multiplication est commutative : pour tous éléments $a$ et $b$ de $\mathbb{Z}$, $a \times b = b \times a$.

\medskip

\noindent\textbf{Démonstration :} Soit $a$ et $b$ deux entiers relatifs. 
    Soit $n$ et $m$ deux entiers naturels et $\ep$ et $\eta$ deux éléments de $\lbrace 0, 1 \rbrace$ tels que $a = (\ep, n)$ et $b = (\eta, m)$. 
    Alors, 
    \begin{itemize}[nosep]
        \item Si $n = 0$ ou $m = 0$, $a \times b = (0,0)$ et $b \times a = (0,0)$, donc $a \times b = b \times a$. 
        \item Sinon, on a $a \times b = (\mu, n \times m)$ et $b \times a = (\mu, m \times n)$, où $\mu$ est égal à $0$ si $\ep = \eta$ et $1$ sinon.
            Puisque la multiplication d'entiers naturels est commutative, $n \times m = m \times n$, donc $a \times b = b \times a$.
    \end{itemize}

   \done 

\medskip

\noindent\textbf{Lemme :} La multiplication est associative : pour tous éléments $a$, $b$ et $c$ de $\mathbb{Z}$, $a \times (b \times c) = (a \times b) \times c$.

\medskip

\noindent\textbf{Démonstration :} On distingue différents cas : 
    \begin{itemize}[nosep]
        \item Si $a = 0$, $(a \times b) \times c = 0 \times c = 0$ et $a \times (b \times c) = 0$.
        \item Si $b = 0$, $(a \times b) \times c = 0 \times c = 0$ et $a \times (b \times c) = a \times 0 = 0$.
        \item Si $c = 0$, $(a \times b) \times c = 0$ et $a \times (b \times c) = a \times 0 = 0$.
        \item Supposons que $a$, $b$ et $c$ sont non nuls. 
            Distinguons alors selon les valaurs possibles de $(\mathrm{sgn}(a), \mathrm{sgn}(b), \mathrm{sgn}(c))$, qui est un élément de $\lbrace 0, 1 \rbrace^3$ :
            \begin{itemize}[nosep]
                \item S'il est égal à $(0,0,0)$, on a 
                    $(a \times b) \times c = (0,\abs{a} \times \abs{b}) \times c = (0, (\abs{a} \times \abs{b}) \times \abs{c})$ et 
                    $a \times (b \times c) = a \times (0, \abs{b} \times \abs{c}) = (0, \abs{a} \times (\abs{b} \times \abs{c}))$.
                \item S'il est égal à $(0,0,1)$, on a 
                    $(a \times b) \times c = (0,\abs{a} \times \abs{b}) \times c = (1, (\abs{a} \times \abs{b}) \times \abs{c})$ et 
                    $a \times (b \times c) = a \times (1, \abs{b} \times \abs{c}) = (1, \abs{a} \times (\abs{b} \times \abs{c}))$.
                \item S'il est égal à $(0,1,0)$, on a 
                    $(a \times b) \times c = (1,\abs{a} \times \abs{b}) \times c = (0, (\abs{a} \times \abs{b}) \times \abs{c})$ et 
                    $a \times (b \times c) = a \times (1, \abs{b} \times \abs{c}) = (0, \abs{a} \times (\abs{b} \times \abs{c}))$.
                \item S'il est égal à $(0,1,1)$, on a 
                    $(a \times b) \times c = (1,\abs{a} \times \abs{b}) \times c = (0, (\abs{a} \times \abs{b}) \times \abs{c})$ et 
                    $a \times (b \times c) = a \times (0, \abs{b} \times \abs{c}) = (0, \abs{a} \times (\abs{b} \times \abs{c}))$.
                \item S'il est égal à $(1,0,0)$, on a 
                    $(a \times b) \times c = (1,\abs{a} \times \abs{b}) \times c = (1, (\abs{a} \times \abs{b}) \times \abs{c})$ et 
                    $a \times (b \times c) = a \times (0, \abs{b} \times \abs{c}) = (1, \abs{a} \times (\abs{b} \times \abs{c}))$.
                \item S'il est égal à $(1,0,1)$, on a 
                    $(a \times b) \times c = (1,\abs{a} \times \abs{b}) \times c = (0, (\abs{a} \times \abs{b}) \times \abs{c})$ et 
                    $a \times (b \times c) = a \times (1, \abs{b} \times \abs{c}) = (0, \abs{a} \times (\abs{b} \times \abs{c}))$.
                \item S'il est égal à $(1,1,0)$, on a 
                    $(a \times b) \times c = (0,\abs{a} \times \abs{b}) \times c = (0, (\abs{a} \times \abs{b}) \times \abs{c})$ et 
                    $a \times (b \times c) = a \times (1, \abs{b} \times \abs{c}) = (0, \abs{a} \times (\abs{b} \times \abs{c}))$.
                \item S'il est égal à $(1,1,1)$, on a 
                    $(a \times b) \times c = (0,\abs{a} \times \abs{b}) \times c = (1, (\abs{a} \times \abs{b}) \times \abs{c})$ et 
                    $a \times (b \times c) = a \times (0, \abs{b} \times \abs{c}) = (1, \abs{a} \times (\abs{b} \times \abs{c}))$.
            \end{itemize}
            Notons que $(\abs{a} \times \abs{b}) \times \abs{c} = \abs{a} \times (\abs{b} \times \abs{c})$ puisque la multiplication d'entiers est associative.
    \end{itemize}
    Dans tous les cas, on a bien $(a \times b) \times c = a \times (b \times c)$.

   \done 

\medskip

\noindent\textbf{Lemme :} La multiplication est distributive sur l'addition : pour tous éléments $a$, $b$ et $c$ de $\mathbb{Z}$, $a \times (b + c) = (a \times b) + (a \times c)$.

\medskip

\noindent\textbf{Démonstration :} 
\begin{itemize}[nosep]
    \item Si $\mathrm{sgn}(a) = \mathrm{sgn}(b) = \mathrm{sgn}(c) = 0$, alors $a \times (b + c) = a \times (0, \abs{b} + \abs{c}) = (0, \abs{a} \times (\abs{b} + \abs{c}))$ et $(a \times b) + (a \times c) = (0, \abs{a} \times \abs{b}) + (0, \abs{a} \times \abs{c}) = (0, (\abs{a} \times \abs{b}) + (\abs{a} \times \abs{c}))$.
        Puisque, sur $\mathbb{N}$, la multiplication est distributive sur l'addition, on a $\abs{a} \times (\abs{b} + \abs{c}) = (\abs{a} \times \abs{b}) + (\abs{a} \times \abs{c})$, et donc $a \times (b + c) = (a \times b) + (a \times c)$.
    \item Si $\mathrm{sgn}(a) = \mathrm{sgn}(b) = \mathrm{sgn}(c) = 1$, alors $a \times (b + c) = a \times (1, \abs{b} + \abs{c}) = (0, \abs{a} \times (\abs{b} + \abs{c}))$ et $(a \times b) + (a \times c) = (0, \abs{a} \times \abs{b}) + (0, \abs{a} \times \abs{c}) = (0, (\abs{a} \times \abs{b}) + (\abs{a} \times \abs{c}))$.
        Puisque, sur $\mathbb{N}$, la multiplication est distributive sur l'addition, on a $\abs{a} \times (\abs{b} + \abs{c}) = (\abs{a} \times \abs{b}) + (\abs{a} \times \abs{c})$, et donc $a \times (b + c) = (a \times b) + (a \times c)$.
    \item Si $a = (0,0)$, alors $a \times (b+c) = (0,0)$ et $(a \times b) + (a \times c) = 0 + 0 = 0$, donc $a \times (b + c) = (a \times b) + (a \times c)$.
    \item Si $b = (0,0)$, alors $a \times (b+c) = a \times c$ et $(a \times b) + (a \times c) = 0 + (a \times c) = a \times c$, donc $a \times (b + c) = (a \times b) + (a \times c)$.
    \item Si $c = (0,0)$, alors $a \times (b+c) = a \times b$ et $(a \times b) + (a \times c) = (a \times b) + 0 = a \times b$, donc $a \times (b + c) = (a \times b) + (a \times c)$.
    \item Si $\mathrm{sgn}(a) = 1$, $\mathrm{sgn}(b) = \mathrm{sgn}(c) = 0$, $b \neq 0$ et $c \neq 0$, alors $a \times (b + c) = a \times (0, \abs{b} + \abs{c}) = (1, \abs{a} \times (\abs{b} + \abs{c}))$ et $(a \times b) + (a \times c) = (1, \abs{a} \times \abs{b}) + (1, \abs{a} \times \abs{c}) = (1, (\abs{a} \times \abs{b}) + (\abs{a} \times \abs{c}))$.
        Puisque, sur $\mathbb{N}$, la multiplication est distributive sur l'addition, on a $\abs{a} \times (\abs{b} + \abs{c}) = (\abs{a} \times \abs{b}) + (\abs{a} \times \abs{c})$, et donc $a \times (b + c) = (a \times b) + (a \times c)$.
    \item Si $\mathrm{sgn}(a) = 0$, $\mathrm{sgn}(b) = \mathrm{sgn}(c) = 1$ et $a \neq 0$, alors $a \times (b + c) = a \times (1, \abs{b} + \abs{c}) = (1, \abs{a} \times (\abs{b} + \abs{c}))$ et $(a \times b) + (a \times c) = (1, \abs{a} \times \abs{b}) + (1, \abs{a} \times \abs{c}) = (1, (\abs{a} \times \abs{b}) + (\abs{a} \times \abs{c}))$.
        Puisque, sur $\mathbb{N}$, la multiplication est distributive sur l'addition, on a $\abs{a} \times (\abs{b} + \abs{c}) = (\abs{a} \times \abs{b}) + (\abs{a} \times \abs{c})$, et donc $a \times (b + c) = (a \times b) + (a \times c)$.
    \item Supposons $\mathrm{sgn}(a) = \mathrm{sgn}(c) = 0$, $\mathrm{sgn}(b) = 1$ et $a \neq 0$. 
        Alors, $(a \times b) + (a \times c) = (1, \abs{a} \times \abs{b}) + (0, \abs{a} \times \abs{c})$. 
        Cette quantité est égale à $(1, \abs{a} \times \abs{b} - \abs{a} \times \abs{c})$ si $\abs{a} \times \abs{b} > \abs{a} \times \abs{c}$, \textit{i.e.}, si $\abs{b} > \abs{c}$, et à $(0, \abs{a} \times \abs{c} - \abs{a} \times \abs{b})$ sinon.
        Par ailleurs, $b + c$ est égal à $(1, \abs{b} - \abs{c})$ si $\abs{b} > \abs{c}$ et $(0, \abs{c} - \abs{b})$ sinon.
        Donc, $a \times (b + c)$ est égal à $(1, \abs{a} (\abs{b} - \abs{c}))$ dans le premier cas et à $(0, \abs{a} (\abs{c} - \abs{b}))$ dans le second.
        Puisque $\abs{a} (\abs{b} - \abs{c}) = \abs{a} \abs{b} - \abs{a} \abs{c}$ dans le premier cas et $\abs{a} (\abs{b} - \abs{c}) = \abs{a} \abs{c} - \abs{a} \abs{b}$ dans le second, on en déduit $a \times (b + c) = (a \times b) + (a \times c)$.
    \item Supposons $\mathrm{sgn}(a) = \mathrm{sgn}(c) = 1$, $\mathrm{sgn}(b) = 0$ et $b \neq 0$. 
        Alors, $(a \times b) + (a \times c) = (1, \abs{a} \times \abs{b}) + (0, \abs{a} \times \abs{c})$. 
        Cette quantité est égale à $(1, \abs{a} \times \abs{b} - \abs{a} \times \abs{c})$ si $\abs{a} \times \abs{b} > \abs{a} \times \abs{c}$, \textit{i.e.}, si $\abs{b} > \abs{c}$, et à $(0, \abs{a} \times \abs{c} - \abs{a} \times \abs{b})$ sinon.
        Par ailleurs, $b + c$ est égal à $(0, \abs{b} - \abs{c})$ si $\abs{b} \geq \abs{c}$ et $(1, \abs{c} - \abs{b})$ sinon.
        Donc, $a \times (b + c)$ est égal à $(1, \abs{a} (\abs{b} - \abs{c}))$ si $\abs{b} > \abs{c}$ et à $(0, \abs{a} (\abs{c} - \abs{b}))$ sinon (y compris si $\abs{b} = \abs{c}$, puisqu'alors $c = -b$).
        Puisque $\abs{a} (\abs{b} - \abs{c}) = \abs{a} \abs{b} - \abs{a} \abs{c}$ dans le premier cas et $\abs{a} (\abs{b} - \abs{c}) = \abs{a} \abs{c} - \abs{a} \abs{b}$ dans le second, on en déduit $a \times (b + c) = (a \times b) + (a \times c)$.
    \item Les deux derniers cas, où $a$ et $b$ sont de même signe et $c$ de signe différent avec $a$, $b$ et $c$ non nuls, se ramènent aux cas précédents par commutativité de l'addition.
        En effet, les cas précédents montrent que $a \times (c + b) = (a \times c) + (a \times b)$, et donc, par commutativité de l'addition, $a \times (b + c) = (a \times b) + (a \times c)$.
\end{itemize}

\done

\subsubsection{Puissance}

\noindent\textbf{Puissance d'entiers relatifs :} Soit $E$ l'ensemble des fonctions de $\mathbb{Z}$ dans $\mathbb{Z}$. 
    On définit la suite $\mathrm{Exp}$ d'éléments de $E$ par récurrence de la manière suivante : 
    \begin{itemize}[nosep]
        \item Pour tout élément $m$ de $\mathbb{Z}$, $\mathrm{Exp}(0)(m) = (0,1)$.
        \item Pour tout élément $n$ de $\mathbb{N}$, pour tout élément $m$ de $\mathbb{Z}$, $\mathrm{Exp}(n+1)(m) = \mathrm{Exp}(n)(m) \times m$.
    \end{itemize}
    Notons que, pour tout élément $m$ de $\mathbb{Z}$, on a $\mathrm{Exp}(1)(m) = m$. 
    Dans la suite, pour tous éléments $n$ et $m$ de $\mathbb{Z}$, on notera l'entier $\mathrm{Exp}(n)(m)$ par $m^n$. 
    Pour touts éléments $n$ et $m$ de $\mathbb{Z}$, on a donc $m^0=1$, $m^1 = m$ et $m^{n+1} = m^n \times m$. 
    L'exponentiation est prioritaire sur la multiplication et l'addition. 
    Par exemple, si $a$, $b$ et $c$ sont trois éléments de $\mathbb{Z}$, $a^b \times c$ est équivalent à $(a^b) \times c$ et $a^b + c$ est équivalent à $(a^b) + c$.

\medskip

\noindent\textbf{Lemme :} Soit $n$ et $m$ deux entiers naturels. 
    Alors, $(0,m)^n = (0,m^n)$.

\medskip

\noindent\textbf{Démonstration :} On procède par récurrence sur $n$.
    Pour $n = 0$, on a $(0,m)^n = (0,1)$ et $(0,m^n) = (0,1)$, donc $(0,m)^n = (0,m^n)$.
    Soit $n$ un entier naturel tel que $(0,m)^n = (0,m^n)$. 
    Alors, $(0,m)^{n+1} = (0,m)^n \times (0,m) = (0,m^n) \times (0,m) = (0, m^n \times m) = (0, m^{n+1})$. 
    Donc, $(0,m)^{n+1} = (0,m^{n+1})$.
    Par récurence, on en déduit que le résultat est vrai pour tout entier narurel $n$.

   \done 

\medskip

\noindent\textbf{Lemme :} Soit $n$ un entier naturel et $m$ un entier naturel non nul. 
    Alors, $(1,m)^{2n} = (0,m^{2n})$ et $(1,m)^{2n+1} = (0,m^{2n+1})$.

\medskip

\noindent\textbf{Démonstration :} On procède par récurrence sur $n$.
    Pour $n = 0$, on a $(1,m)^{2n} = (1,m)^0 = (0,1) = (0,m^0) = (0,m^{2 n})$ et $(1,m)^{2n+1} = (1,m)^{2 n} \times (1,m) = (0,1) \times (1,m) = (1,m) = (1,m^1) = (1,m^{2 n + 1})$. 
    Donc, $(1,m)^{2n} = (0,m^{2 n})$ et $(1,m)^{2n+1} = (1,m^{2 n + 1})$. 
    
    Soit $n$ un entier naurel tel que $(1,m)^{2n} = (0,m^{2 n})$ et $(1,m)^{2n+1} = (1,m^{2 n + 1})$. 
    Alors, $(1,m)^{2(n+1)} = (1,m)^{2n + 2} = (1,m)^{(2n+1)+1} = (1,m)^{2n+1} \times (1,m) = (1, m^{2n+1}) \times (1,m) = (0,m^{2n+1} \times m) = (0, m^{2n+1+1}) = (0, m^{2(n+1)})$ et $(1,m)^{2(n+1)+1} = (1,m)^{2(n+1)} \times (1,m) = (0, m^{2(n+1)}) \times (1,m) = (1,m^{2(n+1)} \times m) = (1, m^{2(n+1)+1})$. 
    Donc, $(1,m)^{2(n+1)} = (0, m^{2(n+1)})$ et $(1,m)^{2(n+1)+1} = (1, m^{2(n+1)+1})$. 
    Par récurrence, on en le résultat attendu est vrai pour tout entier naturel $n$.

   \done 

\subsubsection{Factoriel}

\noindent\textbf{Définition :} Soit $n$ un entier relatif. 
    Si $n \geq 0$, alors on peut choisir un entier naturel $m$ tel que $n = (0,m)$. 
    On pose alors $n! = m!$. 
    Sinon, on pose $n! = 0$.
