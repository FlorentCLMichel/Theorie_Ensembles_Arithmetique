\subsection{Polynômes}

\subsubsection{Définition} 

\noindent\textbf{Définition :} Soit $(A, +, \times)$ un anneau commutatif. 
    On note $0_A$ l'élément neutre de $A$ pour $+$.
    On définit l'anneau (commutatif)\footnote{Voir ci-dessous pour la démonstration qu'il s'agit bien d'un anneau commutatif.} des polynômes sur $(A, +, \times)$, $(\mathbf{A}, +, \times)$ de la manière suivante : 
    \begin{itemize}[nosep]
        \item Définition de $\mathbf{A}[X]$ : Pour tout $x$, $x \in A$ si et seulement si on peut choisir un entier naturel $n$, appelé \textit{degré} de $x$ et noté $\mathrm{deg}(x)$, et $n+1$ éléments $a_0, a_1, \dots, a_n$ de $A$ tels que $x = (a_0, a_1, \cdots, a_n)$ et $a_n \neq 0_A \vee n = 0$.
            Le polynôme $(0_A)$ est dit \textit{nul}.
        \item Addition : Soit $\mathbf{a}$ et $\mathbf{b}$ deux éléments de $\mathbf{A}$, $n$ et $m$ deux entiers naturels, et $a_0$, $a_1$, ..., $a_n$, $b_0$, $b_1$, ..., $b_m$ des éléments de $A$ tels que $\mathbf{a} = (a_0, a_1, \dots, a_n)$ et $\mathbf{b} = (b_0, b_1, \dots, b_m)$.
            Soit $k$ le maximum de $\lbrace n, m \rbrace$ et $l$ son minimum.
            On définit les éléments $c_0$, $c_1$, ... $c_k$ de $A$ par : 
            \begin{itemize}[nosep]
                \item Pour tout élément $i$ de $[\![0, l]\!]$, $c_i = a_i + b_i$.
                    Notons que si $n = m$, alors $l = k$.
                \item Si $n > m$ (alors, $k = n$ et $l = m$), pour tout $i$ dans $[\![l+1, k]\!]$, $c_i = a_i$.
                \item Si $n < m$ (alors, $k = m$ et $l = n$), pour tout $i$ dans $[\![l+1, k]\!]$, $c_i = b_i$.
            \end{itemize}
            Si au moins un de ces éléments est différent de $0_A$, alors l'ensemble des éléments $i$ de $[\![0, k]\!]$ tels que $c_i \neq 0_A$ est un sous-ensemble non vide de $\mathbb{N}$, donc il admet un unique élément maximal $d$ ; sinon, on pose $d = 0$.
            On définit alors $\mathbf{a} + \mathbf{b}$ par le polynôme $(c_0, c_1, \dots, c_d)$.
            Sinon, on définit $\mathbf{a} + \mathbf{b}$ par le polynôme nul $(0_A)$.
        \item Multiplication : Avec les mêmes notations, $\mathbf{a} \times \mathbf{b}$ est le polynôme $(d_0, d_1, \dots d_{n+m})$ défini par : pour tout élément $i$ de $[\![0, n+m]\!]$, on définit l'élément $d_i$ de $A$ par :
            \begin{equation*}
                d_i = \sum_{j=\mathrm{max}(0,i-m)}^{\mathrm{min}(i,n)} a_j \times b_{i-j}.
            \end{equation*}
            (Notons que cette expression est bien définie puisque $j$ est toujours compris entre $0$ et $n$ inclus et $i-j$ entre $0$ et $m$ inclus.
            Notons aussi que $d_{n+m} = a_n \times b_m$.)
            Si au moins un de ces éléments est différent de $0_A$, alors l'ensemble des éléments $i$ de $[\![0, n+m]\!]$ tels que $d_i \neq 0_A$ est un sous-ensemble non vide de $\mathbb{N}$, donc il admet un unique élément maximal $d$ ; sinon, on pose $d = 0$.
            On définit alors $\mathbf{a} \times \mathbf{b}$ par le polynôme $(d_0, d_1, \dots, d_d)$.
            Sinon, on définit $\mathbf{a} \times \mathbf{b}$ par le polynôme nul $(0_A)$.
    \end{itemize}
    L'ensemble des polynômes peut être noté $A[X]$ ou de manière équivalente avec $X$ remplacé par un autre symbole non encore défini. 
    Soit $n$ un entier naturel et $a_0$, $a_1$, ..., $a_n$ des éléments de $A$ tels que $n = 0$ ou $a_n \neq 0$.
    Le polynôme $(a_0, a_1, \dots, a_n)$ pourra être noté $a_0 + a_1 X + \cdots + a_n X^n$, en omettant éventuellement les termes de coefficient nul.
    Pour alléger les notations, on pourra utiliser le symbole $\sum$ : avec les notations précédentes, si $k$ et $l$ sont deux éléments de $[\![0, n]\!]$, alors, 
    \begin{itemize}[nosep]
        \item si $l \geq k$, $\sum_{i=k}^l a_i X^i$ désigne $a_k X^k + a_{k+1} X^{k+1} + \cdots + a_l X^l$ ;
        \item si $l < k$, $\sum_{i=k}^l a_i X^i$ désigne le polynôme nul.
    \end{itemize}

\medskip

\noindent\textbf{Notation :} 
    Soit $(A, +, \times)$ un anneau commutatif et $0_A$ son élément neutre pour $+$.
    Pour tout élément $\mathbf{a}$ de $A[X]$, on définit la suite des coefficients de $\mathbf{a}$, également notée $\mathbf{a}$ quand il n'y a pas de confusion possible, de la manière suivante : pour tout entier naturel $i$, 
\begin{itemize}[nosep]
    \item Si $i$ est inférieur ou égal au degré de $\mathbf{a}$, alors $\mathbf{a}_i$ est le $i$ème coefficient de $\mathbf{a}$ en partant de $0$.
    \item Sinon, on pose $\mathbf{a}_i = 0_A$.
\end{itemize}
Deux polynômes sont égaux si et seulement si les suites de leurs coefficients le sont.
En effet, 
\begin{itemize}[nosep]
    \item S'ils sont égaux, ils ont la même suite de coefficients par construction.
    \item Si les suites de leurs coefficients sont égales, alors ils ont alors le même degré (indice du denier élément non nul, ou $0$ si tous les éléments sont nuls) et les mêmes éléments (élements de cette suite d'indice inférieur ou égal au degré).
\end{itemize}
Notons que toute suite $u$ d'éléments de $A$ telle qu'il existe un entier naturel $n_0$ tel que $\forall n \in \mathbb{N} \, n > n_0 \Rightarrow u_n = 0$ est la suite de coefficients d'un polynôme. 
En effet, avec ces notations, soit $m$ le plus grand entier naturel tel que $u_m \neq 0_A$ si la suite $u$ a au moins un coefficient non nul ou $m = 0$ sinon, alors $u$ est la suite de coefficients du polynôme $(u_0, u_1, \dots, u_m)$.

\medskip

\noindent\textbf{Remarque :} 
    Soit $(A, +, \times)$ un anneau commutatif, $0_A$ l'élément neutre de $(A, +)$, et $e$ un élément de $A$.
    Soit $\mathbf{e}$ le polynôme $(e)$.
    Alors, $\mathbf{e}_0 = e$ et, pour tout entier $n$ strictement positif, $\mathbf{e}_n = 0_A$.
    En particluer, si $e = 0_a$, on a $\mathbf{e}_n = 0_A$ pour tout entier naturel $n$.

\medskip

\noindent\textbf{Remarque :} 
    Soit $(A, +, \times)$ un anneau commutatif.
    L'ensemble $A[X]$ pourrait être défini de manière presque équivalente par l'ensemble des suites d'éléments de $A$ nulles à partir d'un certain indice, ou, de manière équivalente, pour lesquelles il existe un nombre fini d'indices donnant une valeur non nulle.

\medskip

\noindent\textbf{Lemme :}
    Soit $(A, +, \times)$ un anneau commutatif.
    Soit $\mathbf{a}$ et $\mathbf{b}$ deux éléments de $A[X]$. 
    Soit $\mathbf{c}$ et $\mathbf{d}$ les polynômes donnés par $\mathbf{c} = \mathbf{a} + \mathbf{b}$ et $\mathbf{d} = \mathbf{a} \times \mathbf{b}$.
    Alors, les suites d'éléments de $\mathbf{c}$ et $\mathbf{d}$ sont données par : pour tout entier naturel $n$, 
    \begin{equation*}
        \mathbf{c}_n = \mathbf{a}_n + \mathbf{b}_n, 
        \qquad
        \mathbf{d}_n = \sum_{i=0}^n \mathbf{a}_i \times \mathbf{b}_{n-i}
        .
    \end{equation*}

\medskip

\noindent\textbf{Démonstration :} 
\begin{itemize}[nosep]
    \item \textit{Addition :} 
        Définissons la suite $e$ d'éléments de $A$ par $e_n = \mathbf{a}_n + \mathbf{b}_n$ pour tout entier naturel $n$.
        Montrons que, pour tout entier naturel $n$, $e_n = 0_A$ si $n > \mathrm{max}(\mathrm{deg}(\mathbf{a}), \mathrm{deg}(\mathbf{b}))$ et $e_n = c_n$ sinon, où $c_n$ est définit comme dans la définition de l'addition de polynômes. 
        Cela montrera que le polynôme dont la suite de coefficients est $e_n$ existe et que, soit $m$ le plus grand entier naturel tel que $c_m \neq 0_A$ si un tel entier existe, ou $m = 0$ sinon, ce polynôme est $(c_0, c_1, \dots, c_m)$, c'est à dire exactement $\mathbf{a} + \mathbf{b}$.
        \begin{itemize}[nosep]
            \item Si $n > \mathrm{max}(\mathrm{deg}(\mathbf{a}), \mathrm{deg}(\mathbf{b}))$, alors $n > \mathrm{deg}(\mathbf{a})$, donc $\mathbf{a}_n = 0_A$, et $n > \mathrm{deg}(\mathbf{b})$, donc $\mathbf{b}_n = 0_A$.
                Donc, $\mathbf{a}_n + \mathbf{b}_n = 0_A$.
                Donc, $e_n = 0_A$.
            \item Sinon, on a $n \leq \mathrm{deg}(\mathbf{a}) \vee n \leq \mathrm{deg}(\mathbf{b})$. 
                On a alors trois possibilités.\footnote{
                    On utilise ici le fait que, si $f$ et $g$ sont deux formules, alors $f \vee g$ est équivalente à $(f \wedge g) \vee (f \wedge (\neg g)) \vee ((\neg f) \wedge g)$.
                    (Toutes deux sont vraies si $f$ est vraie ou $g$ est vraie et fausses si $f$ et $g$ sont fausses.)
                }
                \begin{itemize}[nosep]
                    \item Si $n \leq \mathrm{deg}(\mathbf{a}) \wedge n > \mathrm{deg}(\mathbf{b})$, alors $\mathbf{b}_n = 0_A$, donc $e_n = \mathbf{a}_n$. 
                        Avec les notations de la définition, cela donne $e_n = a_n$.
                    \item Si $n > \mathrm{deg}(\mathbf{a}) \wedge n \leq \mathrm{deg}(\mathbf{b})$, alors $\mathbf{a}_n = 0_A$, donc $e_n = \mathbf{b}_n$. 
                        Avec les notations de la définition, cela donne $e_n = b_n$.
                    \item Si $n \leq \mathrm{deg}(\mathbf{a}) \wedge n \leq \mathrm{deg}(\mathbf{b})$, alors $e_n = \mathbf{a}_n + \mathbf{b}_n$. 
                        Avec les notations de la définition, cela donne $e_n = a_n + b_n$.
                \end{itemize}
                Dans les trois cas, on a bien $e_n = c_n$.
        \end{itemize}
    \item \textit{Multiplication :} 
        Définissons la suite $f$ d'éléments de $A$ par $f_n = \sum_{i=0}^n \mathbf{a}_i \times \mathbf{b}_{n-i}$ pour tout entier naturel $n$.
        Montrons que, pour tout entier naturel $n$, $f_n = 0$ si $n > \mathrm{deg}(\mathbf{a}) + \mathrm{deg}(\mathbf{b})$ (et donc $f_n = \mathbf{d}_n$) et $f_n = \mathbf{d}_n$ sinon.
        \begin{itemize}[nosep]
            \item Soit $n$ un entier naturel strictement supérieur à $\mathrm{deg}(\mathbf{a}) + \mathrm{deg}(\mathbf{b})$.
                Soit $i$ un entier naturel inférieur ou égal à $n$. 
                Alors, soit $i > \mathrm{deg}(\mathbf{a})$, donc $\mathbf{a}_n = 0_A$, soit $i \leq \mathrm{deg}(\mathbf{a})$, donc $n - i \geq n - \mathrm{deg}(\mathbf{a})$, donc $n - i > \mathrm{deg}(\mathbf{b})$, donc $\mathbf{b}_n = 0_A$.
                Dans les deux cas, on a $\mathbf{a}_i \times \mathbf{b}_{n-i} = 0_A$.
                Donc, $\sum_{i=0}^n \mathbf{a}_i \times \mathbf{b}_{n-i} = \sum_{i=0}^n 0_A = 0_A$.
            \item Soit $n$ un entier naturel inférieur ou égal à $\mathrm{deg}(\mathbf{a}) + \mathrm{deg}(\mathbf{b})$.
                Pour tout entier naturel $i$ inférieur ou égal à $n$ et strictement supérieur à $\mathrm{deg}(\mathbf{a})$, on a $\mathbf{a}_i = 0_A$, donc $\mathbf{a}_i \times \mathbf{b}_{n-i} = 0_A$.
                En outre, pour tout entier naturel $i$ strictement inférieur à $n - \mathrm{deg}(\mathbf{b})$, $n - i > \mathrm{deg}(\mathbf{b})$, donc $\mathbf{b}_{n-i} = 0_A$, donc $\mathbf{a}_i \times \mathbf{b}_{n-i} = 0_A$.
                Pour tout entier naturel $i$ inférieur ou égal à $n$, $\mathbf{a}_i \times \mathbf{b}_{n-i}$ ne peut donc être non nul que si $n - \mathrm{deg}(\mathbf{b}) \leq i \leq \mathrm{deg}(\mathbf{a})$.
                Puisque, avec les mêmes notations, on a $0 \leq i$ puisque $i$ est un entier naturel, l'inégalité de gauche implique $\mathrm{max}(0, n - \mathrm{deg}(\mathbf{b})) \leq i$.
                Puisque, avec les mêmes notations $i \leq n$, l'inégalité de droite implique $i \leq \mathrm{min}(n, \mathrm{deg}(\mathbf{a}))$.
                On a donc $\sum_{i = 0}^n \mathbf{a}_i \times \mathbf{b}_{n-i} = \sum_{i = \mathrm{max}(0, n - \mathrm{deg}(\mathbf{b}))}^{\mathrm{min}(n, \mathrm{deg}(\mathbf{a}))} \mathbf{a}_i \times \mathbf{b}_{n-i}$, ce qui montre que $\mathbf{d}_n$ est bien le coefficient d'indice $n$ du polynôme $\mathbf{a} \times \mathbf{b}$.
        \end{itemize}
\end{itemize}

\done

\medskip

\noindent\textbf{Lemme :} Soit $(A, +, \times)$ un anneau commutatif et unifère. 
    Alors, $(A[X], +, \times)$ est unifère.

\medskip

\noindent\textbf{Définition :} Soit $A$ un anneau commutatif et $\mathbf{a}$ un élément de $A[X]$.
    Pour tout entier naturel $n$, le coefficient $\mathbf{a}_n$ est appelé \emph{coefficient d'ordre $n$} du polynôme $\mathbf{a}$.

\medskip

\noindent\textbf{Preuve qu'il s'agit bien d'un anneau commutatif (et unifère si $(A, +, \times)$ l'est) :} Montrons qu'il s'agit bien d'un anneau, avec pour éléments neutres $(0_A)$ et, si $(A, +, \times)$ est unifère, $(1_A)$, où $1_A$ est l'élément neutre de $A$ pour $\times$. 
Dans cette démonstration, $\mathbf{a}$, $\mathbf{b}$ et $\mathbf{c}$ sont trois éléments arbitraires de $A[X]$.
Nous notons $\mathbf{0}$ le polynôme $(0_A)$ et, si $(A, +, \times)$ est unifère, $\mathbf{1}$ le polynôme $(1_A)$.
\begin{itemize}[nosep]
    \item \textit{$(A[X], +)$ est un groupe abélien :}
        \begin{itemize}[nosep]
            \item $(A[X], +)$ est un magma puisque $+$ est une loi de composition interne sur $A[X]$.
            \item \textit{L'opération $+$ est commutative :} 
                Pour tout entier naturel $n$, on a (puisque $(A, +)$ est commutatif)
                $(\mathbf{a} + \mathbf{b})_n = \mathbf{a}_n + \mathbf{b}_n = \mathbf{b}_n + \mathbf{a}_n = (\mathbf{b} + \mathbf{a})_n$.
                Donc, les suites des coeffcients de $\mathbf{a} + \mathbf{b}$ et $\mathbf{b} + \mathbf{a}$ sont identiques.
                Donc, $\mathbf{a} + \mathbf{b} = \mathbf{b} + \mathbf{a}$.
            \item \textit{Le polynôme $(0_A)$ est neutre pour $+$ :} 
                Soit $n$ un entier naturel.
                On a : $(\mathbf{a} + \mathbf{0})_n = \mathbf{a}_n + \mathbf{0}_n$.
                En outre $\mathbf{0}_n = 0_A$ (en effet, cela est vrai pour $n = 0$ par définition de $\mathbf{0}$ et pour $n > 0$ car alors $n > \mathrm{deg}(\mathbf{0})$).
                Donc, $(\mathbf{a} + \mathbf{0})_n = \mathbf{a}_n + 0_A = \mathbf{a}_n$.
                Cela montre que la suite des coefficients de $\mathbf{a} + \mathbf{0}$ est identique à celle de $\mathbf{a}$, et donc que $\mathbf{a} + \mathbf{0} = \mathbf{a}$.
                Puisque l'addition de polynômes est commutatives, cela implique en outre $\mathbf{0} + \mathbf{a} = \mathbf{a}$.
            \item \textit{L'opération $+$ est associative :} 
                Notons $\mathbf{d}$ le polynôme $\mathbf{a} + (\mathbf{b} + \mathbf{c})$ et $\mathbf{e}$ le polynôme $(\mathbf{a} + \mathbf{b}) + \mathbf{c}$. 
                Pour tout entier naturel $n$, on a : $\mathbf{d}_n = \mathbf{a}_n + (\mathbf{b} + \mathbf{c})_n = \mathbf{a}_n + (\mathbf{b}_n + \mathbf{c}_n = (\mathbf{a}_n + \mathbf{b}_n) + \mathbf{c}_n$ (puisque $(A, +)$ est associatif), donc $\mathbf{d}_n = (\mathbf{a} + \mathbf{b})_n + \mathbf{c}_n = \mathbf{e}_n$.
                Donc, $\mathbf{d} = \mathbf{e}$.
            \item \textit{Tout élément de $A[X]$ admet un inverse pour l'opération $+$ :}
                Dans ce paragrahe seulement, pour tout élément $e$ de $A$, on note $\tilde{e}$ l'inverse de $e$ pour l'opération $+$ (qui existe puisque $(A, +)$ est un groupe). 
                On note $n$ le degré de $\mathbf{a}$ et $a$ la suite de ses coefficients. 
                Montrons que le polynôme $\tilde{\mathbf{a}}$ définit par $\tilde{\mathbf{a}} = (\tilde{a}_0, \tilde{a}_1, \dots, \tilde{a}_n)$ (qui est bien un polynôme puisque soit $n = 0$ soit $a_n \neq 0_A$ et donc $\tilde{a}_n \neq 0_A$) est un inverse de $\mathbf{a}$ pour l'opération $+$.
                Notons $d$ la suite des coefficients de $\mathbf{a} + \tilde{\mathbf{a}}$.
                Soit $n$ un entier naturel. 
                Si $m > n$, alors $\mathbf{a}_m = 0_A$ et $\tilde{\mathbf{a}}_m = 0_A$, donc $d_m = 0_A + 0_A = 0_A$.
                Sinon, $d_m = a_m + \tilde{a}_m = 0_A$.
                Donc, $d_m = 0_a$ pour tout entier naturel $m$.
                On en déduit que $\mathbf{a} + \tilde{\mathbf{a}} = \mathbf{0}_A$.
        \end{itemize}
    \item \textit{L'opération $\times$ est commutative :}
        Soit $n$ un entier naturel.
        On a puisque l'anneau $(A, +, \times)$ est commutatif : 
        \begin{equation*}
            (\mathbf{a} \times \mathbf{b})_n
            = \sum_{i=0}^n \mathbf{a}_i \times \mathbf{b}_{n-i}
            = \sum_{i=0}^n \mathbf{b}_{n-i} \times \mathbf{a}_i. 
        \end{equation*}
        Soit $f$ la fonction de $[\![0, n]\!]$ vers lui-même qui à tout élément $k$ associe $n-k$. 
        Il s'agit d'une bijection, puisque tout élément $l$ a pour unique antécédent $n-l$. 
        On a donc :
        \begin{equation*}
            (\mathbf{a} \times \mathbf{b})_n
            = \sum_{i=0}^n \mathbf{b}_{f(n-i)} \times \mathbf{a}_{f(i)}. 
            = \sum_{i=0}^n \mathbf{b}_{i} \times \mathbf{a}_{n-i}
            = (\mathbf{b} \times \mathbf{a})_n .
        \end{equation*}
        Cela étant vrai pour tout entier naturel $n$, on en déduit que $\mathbf{a} \times \mathbf{b} = \mathbf{b} \times \mathbf{a}$.
    \item \textit{L'opération $\times$ est associative :}
        Notons $\mathbf{d}$ le polynôme $\mathbf{a} \times (\mathbf{b} \times \mathbf{c})$ et $\mathbf{e}$ le polynôme $(\mathbf{a} \times \mathbf{b}) \times \mathbf{c}$. 
        Pour tout entier naturel $n$, on a : $\mathbf{d}_n = \mathbf{a}_n \times (\mathbf{b} \times \mathbf{c})_n = \mathbf{a}_n \times (\mathbf{b}_n \times \mathbf{c}_n = (\mathbf{a}_n \times \mathbf{b}_n) \times \mathbf{c}_n$ (puisque $(A, \times)$ est associatif), donc $\mathbf{d}_n = (\mathbf{a} \times \mathbf{b})_n \times \mathbf{c}_n = \mathbf{e}_n$.
        Donc, $\mathbf{d} = \mathbf{e}$.
    \item \textit{L'opération $\times$ est distributive sur $+$ :} 
        Montrons que $\mathbf{a} \times (\mathbf{b} + \mathbf{c}) = (\mathbf{a} \times \mathbf{b}) + (\mathbf{a} \times \mathbf{c})$.
        Puisque l'opération $\times$ est commutative, cela montrera également $(\mathbf{b} + \mathbf{c}) \times \mathbf{a} = (\mathbf{b} \times \mathbf{a}) + (\mathbf{c} \times \mathbf{a})$.
        Notons $\mathbf{d}$ le polynôme $\mathbf{a} \times (\mathbf{b} + \mathbf{c})$ et $\mathbf{e}$ le polynôme $(\mathbf{a} \times \mathbf{b}) + (\mathbf{a} \times \mathbf{c})$.
        Pour tout entier naturel $n$, on a 
        \begin{equation*}
            \mathbf{d}_n = \sum_{i=0}^n \mathbf{a}_i \times (\mathbf{b} + \mathbf{c})_{n-i}
                         = \sum_{i=0}^n \mathbf{a}_i \times (\mathbf{b}_{n-i} + \mathbf{c}_{n-i})
                         = \sum_{i=0}^n \left[ (\mathbf{a}_i \times \mathbf{b}_{n-i}) + (\mathbf{a}_i \times \mathbf{c}_{n-i}) \right]
        \end{equation*}
        et
        \begin{equation*}
            \mathbf{e}_n = (\mathbf{a} \times \mathbf{b})_n + (\mathbf{a} \times \mathbf{c})_n
                         = \sum_{i=0}^n  (\mathbf{a}_i \times \mathbf{b}_{n-i}) 
                            + \sum_{j=0}^n (\mathbf{a}_j \times \mathbf{c}_{n-j}) 
                         = \sum_{i=0}^n \left[ (\mathbf{a}_i \times \mathbf{b}_{n-i}) + (\mathbf{a}_i \times \mathbf{c}_{n-i}) \right] .
        \end{equation*}
        Donc, $\mathbf{d}_n = \mathbf{e}_n$ pour tout entier naturel $n$.
        Donc, $\mathbf{d} = \mathbf{e}$.
    \item \textit{Si l'anneau $(A, +, \times)$ est unifère, le polynôme $(1_A)$ est neutre pour $\times$ :}
        Puisque l'opération $\times$ est commutative, il suffit de montrer que $\mathbf{1} \times \mathbf{a} = \mathbf{a}$ : cela impliquera $\mathbf{a} \times \mathbf{1} = \mathbf{a}$.
        Soit $n$ un entier naturel. 
        On a : 
        \begin{equation*}
            (\mathbf{1} \times \mathbf{a})_n 
            = \sum_{i=0}^n \mathbf{1}_i \times \mathbf{a}_{n-i}
            = \sum_{i=0}^0 \mathbf{1}_i \times \mathbf{a}_{n-i} + \sum_{i=1}^n \mathbf{1}_i \times \mathbf{a}_{n-i}
            = \mathbf{1}_0 \times \mathbf{a}_n + \sum_{i=1}^n \mathbf{1}_i \times \mathbf{a}_{n-i}.
        \end{equation*}
        Puisque $\mathbf{1}_0 = 1_A$, $\mathbf{1}_0 \times \mathbf{a}_n = \mathbf{a}_n$.
        En outre, pour tout élément $i$ de $[\![1, n]\!]$, $\mathbf{1}_i = 0_A$, donc $\mathbf{1}_i \times \mathbf{a}_{n-i} = 0_A$.
        Donc, 
        \begin{equation*}
            (\mathbf{1} \times \mathbf{a})_n 
            = \mathbf{a}_n + \sum_{i=1}^n 0_A
            = \mathbf{a}_n + 0_A
            = \mathbf{a}_n .
        \end{equation*}
        On en déduit que la suite des coefficients de $\mathbf{1} \times \mathbf{a}$ est égale à celle de $\mathbf{a}$, et donc que $\mathbf{1} \times \mathbf{a}$ = $\mathbf{a}$.
\end{itemize}

\done

%\noindent\textbf{Preuve qu'il s'agit bien d'un anneau commutatif :} Montrons qu'il s'agit bien d'un anneau, avec pour éléments neutres $(0_A)$ et $(1_A)$, où $1_A$ est l'élément neutre de $A$ pour $\times$. 
%Dans cette démonstration, $\mathbf{a}$, $\mathbf{b}$ et $\mathbf{c}$ sont trois éléments arbitraires de $A[X]$, $n_a$, $n_b$ et $n_c$ sont trois entiers naturels et $a_0$, $a_1$, ..., $a_{n_a}$, $b_0$, $b_1$, ..., $b_{n_b}$ et $c_0$, $c_1$, ..., $c_{n_c}$ sont des éléments de $A$ tels que $(a_{n_a} \neq 0) \vee (n_a = 0)$, $(b_{n_b} \neq 0) \vee (n_b = 0)$ et $(c_{n_c} \neq 0) \vee (n_c = 0)$, et $\mathbf{a} = (a_0, a_1, \dots, a_n)$, $\mathbf{b} = (b_0, b_1, \dots, b_n)$ et $\mathbf{c} = (c_0, c_1, \dots, c_n)$. 
%\begin{itemize}[nosep]
%    \item \textit{$(A[X], +)$ est un groupe abélien :}
%        \begin{itemize}[nosep]
%            \item $(A[X], +)$ est un magma puisque $+$ est une loi de composition interne sur $A[X]$.
%            \item \textit{L'opération $+$ est commutative :}
%                Traitons séparément les deux cas $n_a \geq n_b$ et $_a < n_b$. 
%                Si $n_a \geq n_b$, alors $\mathbf{a} + \mathbf{b} = (c_0, c_1, \dots, c_{n_a})$ et $\mathbf{b} + \mathbf{a} = (d_0, d_1, \dots, d_{n_a})$ où, pour tout élément $i$ de $[\![0, n_a]\!]$,
%                \begin{itemize}[nosep]
%                    \item si $i \leq n_b$, $c_i = a_i + b_i$ et $d_i = b_i + a_i$, et donc $c_i = d_i$ puisque le groupe $(A, +)$ est abélien ;
%                    \item sinon, $c_i = a_i$ et $d_i = a_i$, donc $c_i = d_i$.
%                \end{itemize}
%                On a donc bien $\mathbf{a} + \mathbf{b} = \mathbf{b} + \mathbf{a}$.
%
%                Sinon, $n_a < n_b$. 
%                Alors $\mathbf{a} + \mathbf{b} = (c_0, c_1, \dots, c_{n_b})$ et $\mathbf{b} + \mathbf{a} = (d_0, d_1, \dots, d_{n_b})$ où, pour tout élément $i$ de $[\![0, n_b]\!]$,
%                \begin{itemize}[nosep]
%                    \item si $i \leq n_a$, $c_i = a_i + b_i$ et $d_i = b_i + a_i$, et donc $c_i = d_i$ puisque le groupe $(A, +)$ est abélien ;
%                    \item sinon, $c_i = b_i$ et $d_i = b_i$, donc $c_i = d_i$.
%                \end{itemize}
%                On a donc à nouveau $\mathbf{a} + \mathbf{b} = \mathbf{b} + \mathbf{a}$.
%            \item \textit{Le polynôme $(0_A)$ est neutre pour $+$ :}
%                Puisque $n_a$ est un entier naturel, $n_a \geq 0$.
%                Donc, le minimum de $\lbrace 0, n_a \rbrace$ est $0$ et son maximum est $n_a$.
%                Notons $\mathbf{d}$ le polynôme $\mathbf{a} + (0_A)$. 
%                Soit $n_d$ un entier naturel et $d_0$, $d_1$, ..., $d_{n_d}$ des éléments de $A$ tels que $\mathbf{d} = (d_0, d_1, \dots, d_{n_d})$.
%                Par définition de l'addition, on a $n_d = n_a$, $d_0 = a_0 + 0_A = a_0$ et, pour tout élément $i$ de $[\![1, n_d]\!]$, $d_i = a_i$.
%                Donc, $(d_0, d_1, \dots, d_{n_d}) = (a_0, a_1, \dots, a_{n_a})$.
%                Donc, $\mathbf{d} = \mathbf{a}$.
%                Puisque l'opération $+$ est commutative, on en déduit que $(0_A) + \mathbf{a} = \mathbf{a} + (0_A) = \mathbf{a}$.
%            \item \textit{L'opération $+$ est associative :}
%                En principe, il y a six cas différents à traiter. 
%                Mais nous pouvons réduire ce nombre en notant que, pour prouver $\mathbf{a} + (\mathbf{b} + \mathbf{c}) = (\mathbf{a} + \mathbf{b}) + \mathbf{c}$, les rôles de $\mathbf{a}$ et $\mathbf{c}$ sont interchangeables. 
%                En effet, puisque l'addition est commutative (et par symétrie de l'égalité), cette équation est équivalente à $\mathbf{c} + (\mathbf{b} + \mathbf{a}) = (\mathbf{c} + \mathbf{b}) + \mathbf{a}$. 
%                Sans perte de généralité (et quitte à échanger les noms de $\mathbf{a}$ et $\mathbf{c}$), on peut donc supposer $d_a \leq d_c$. 
%                (Puisque, si ce n'est pas le cas, alors $d_c \leq d_a$.)
%                Il n'y a donc que trois cas à traiter : $d_b \leq d_a$, $d_a < d_b \leq d_c$ et $d_c < d_b$.
%                Notons $d_{ab}$, $d_{bc}$, $d_{abc1}$ et $d_{abc2}$ les quatre entiers naturels et $\mathrm{ab}_0$, $\mathrm{ab}_1$, ..., $\mathrm{ab}_{d_{ab}}$, $\mathrm{bc}_0$, $\mathrm{bc}_1$, ..., $\mathrm{ab}_{d_{bc}}$, $\mathrm{abc1}_0$, $\mathrm{abc1}_1$, ..., $\mathrm{abc1}_{d_{abc1}}$, $\mathrm{ab}_{d_{bc}}$, $\mathrm{abc2}_0$, $\mathrm{abc2}_1$, ..., $\mathrm{abc2}_{d_{abc2}}$ les éléments de $A$ tels que $\mathbf{a} + \mathbf{b} = \left( \mathrm{ab}_0, \mathrm{ab}_1, \dots, \mathrm{ab}_{d_{ab}} \right)$, $\mathbf{b} + \mathbf{c} = \left( \mathrm{bc}_0, \mathrm{bc}_1, \dots, \mathrm{bc}_{d_{bc}} \right)$, $\mathbf{a} + (\mathbf{b} + \mathbf{c}) = \left( \mathrm{abc1}_0, \mathrm{abc1}_1, \dots, \mathrm{abc1}_{d_{abc1}} \right)$ et $(\mathbf{a} + \mathbf{b}) + \mathbf{c} = \left( \mathrm{abc2}_0, \mathrm{abc2}_1, \dots, \mathrm{abc2}_{d_{abc2}} \right)$.
%                Il s'agit de montrer que $d_{abc1} = d_{abc2}$ et que, pour tout entier naturel $i$ inférieur ou égal à $d_{abc1}$, $\mathrm{abc1}_i = \mathrm{abc2}_i$.
%                \begin{itemize}[nosep]
%                    \item Cas $d_b \leq d_a$ : 
%                        On a alors $d_b \leq d_a \leq d_c$.
%                        Donc, $d_{ab} = d_a$, donc $d_{abc2} = d_c$ et $d_{bc} = d_c$, donc $d_{abc1} = d_c$. 
%                        Donc, $d_{abc2} = d_{abc1}$.
%                        ***
%                    \item Cas $d_a < d_b \leq d_c$ : ***
%                    \item Cas $d_c < d_b$ : ***
%                \end{itemize}
%            \item \textit{Tout élément de $A[X]$ admet un inverse pour l'opération $+$ :}
%                Dans ce paragrahe seulement, pour tout élément $e$ de $A$, on note $\tilde{e}$ l'inverse de $e$ pour l'opération $+$ (qui existe puisque $(A, +)$ est un groupe). 
%                Montrons que le polynôme $(\tilde{a}_0, \tilde{a}_1, \dots, \tilde{a}_{n_a})$ (qui est bien un polynôme puisque soit $n_a = 0$ soit $a_{n_a} \neq 0_A$ et donc $\tilde{a}_{n_a} \neq 0_A$) est un inverse de $\mathbf{a}$ pour l'opération $+$.
%                ***
%        \end{itemize}
%    \item \textit{L'opération $\times$ est commutative :}
%        ***
%    \item \textit{L'opération $\times$ est associative :}
%        ***
%    \item \textit{L'opération $\times$ est distributive sur $+$ :} 
%        Montrons que $\mathbf{a} \times (\mathbf{b} + \mathbf{c}) = (\mathbf{a} \times \mathbf{b}) + (\mathbf{a} \times \mathbf{c})$.
%        Puisque l'opération $\times$ est commutative, cela montrera également $(\mathbf{b} + \mathbf{c}) \times \mathbf{a} = (\mathbf{b} \times \mathbf{a}) + (\mathbf{c} \times \mathbf{a})$.
%        ***
%    \item \textit{Le polynôme $(1_A)$ est neutre pour $\times$ :}
%        ***
%\end{itemize}

\medskip

\noindent\textbf{Évaluation d'un polynôme :} Soit $(A, +, \times)$ un anneau commutatif et $\mathbf{a}$ un polynôme sur $A$.
    On peut choisir un entier naturel $n$ et $n+1$ éléments $a_0$, $a_2$, ..., $a_n$ de $A$ tels que $\mathbf{a} = (a_0, a_1, \dots, a_n)$.
    Pour tout élément $a$ de $A$, on note $\mathbf{a}(a)$ l'élément $\sum_{i=0}^n a_i a^i$. 

\medskip

\noindent\textbf{Lemme :} Soit $(A, +, \times)$ un anneau unifère commutatif et $a$ un élément de $A$.
    La fonction de $A[X]$ vers $A$ qui à tout élément $\mathbf{a}$ de $A[X]$ associe $\mathbf{a}(a)$ est un morphisme d'anneaux.

\medskip

\noindent\textbf{Démonstration :} 
    Notons $f$ la fonction considérée et $1_A$ l'élément neutres de $A$ pour $\times$.  
    ***

\subsubsection{Degré}

\noindent\textbf{Définition (degré) (rappel) :} Soit $(A, +, \times)$ un anneau commutatif et $\mathbf{a}$ un élément de $A[X]$.
    Soit $n$ un entier naturel et $a_0$, $a_1$, ..., $a_n$ des éléments de $A$ tels que $\mathbf{a} = (a_0, a_1, \dots, a_n)$.
    L'entier naturel $n$ est appelé \textit{degré} de $\mathbf{a}$, et peut être noté $\mathrm{deg}(\mathbf{a})$. 
\medskip

\noindent\textbf{Lemme :} Soit $\mathcal{A}$ un anneau commmutatif et $\mathbf{a}$ et $\mathbf{b}$ deux polynômes sur $\mathcal{A}$, de degrés respectifs $d_a$ et $d_b$. 
    On suppose que le produit de deux éléments de l'anneau ditincts de l'élément neutre pour l'addition l'est aussi.
    Alors, $\mathbf{a} \times \mathbf{b}$ a pour degré $d_a + d_b$ sauf si $\mathbf{a}$ ou $\mathbf{b}$ est le polynôme nul, auquel cas $\mathbf{a} \times \mathbf{b}$ a pour degré $0$.

\medskip

\noindent\textbf{Démonstration :} Évident d'après la définition de la multiplication (avec ces notations, si ni $\mathbf{a}$ ni $\mathbf{b}$ n'est le polynôme nul, alors $d_{n+m} \neq 0_A$.).

\subsubsection{Racines}

\noindent\textbf{Définition (racine) :} Soit $(A, +, \times)$ un anneau commutatif et $\mathbf{a}$ un élément de $A[X]$.
    On note $0_A$ l'élément neutre de $A$ pour $+$.
    Un élément $r$ de $A$ est dit \textit{racine} de $\mathbf{a}$ si $\mathbf{a}(r) = 0_A$.

\medskip

\noindent\textbf{Remarque :} Notons qu'un polynôme peut, en général, avoir plus de racines distinctes que son degré. 
    Considérons par exemple l'anneau $(\mathbb{Z}_6, +, \times)$ (voir définition section~\ref{sub:def_Z_nZ}), d'élément neutre pour $+$ $\bar{0}$, et le polynôme $\mathbf{p} = X^2 - \bar{5} X$. 
    On a : $\mathbf{p}(\bar{0}) = \bar{0}$, $\mathbf{p}(1) = \bar{1} - \bar{5} = \bar{2}$, $\mathbf{p}(2) = \bar{4} - \overline{10} = \overline{-6} = \bar{0}$, $\mathbf{p}(3) = \bar{9} - \overline{15} = \overline{-6} = \bar{0}$, $\mathbf{p}(4) = \overline{16} - \overline{20} = \bar{2}$ et $\mathbf{p}(5) = \overline{25} - \overline{25} = \bar{0}$. 
    Donc, $\mathbf{p}$, bien que de degré $2$, a $4$ racines distinctes ($\bar{0}$, $\bar{2}$, $\bar{3}$ et $\bar{5}$).

