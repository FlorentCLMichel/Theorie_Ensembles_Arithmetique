\subsection{Polynômes}

\subsubsection{Définition} 

\noindent\textbf{Définition :} Soit $(A, +, \times)$ un anneau commutatif. 
    On note $0_A$ l'élément neutre de $A$ pour $+$, dit \textit{nul}.
    On définit l'anneau (commutatif) des polynômes sur $(A, +, \times)$, $(\mathbf{A}, +, \times)$ de la manière suivante : 
    \begin{itemize}[nosep]
        \item Définition de $\mathbf{A}$ : Pour tout $x$, $x \in A$ si et seulement si on peut choisir un entier naturel $n$, appelé \textit{degré} de $x$, et $n+1$ éléments $a_0, a_1, \dots, a_n$ de $A$ tels que $x = (a_0, a_1, \cdots, a_n)$ et $a_n \neq 0_A \vee n = 0$.
            On note $0_{\mathbf{A}}$ le polynôme nul $(0_{\mathbf{A}})$.
            Le polynôme $(0_A)$ est dit \textit{nul}.
        \item Addition : Soit $\mathbf{a}$ et $\mathbf{b}$ deux éléments de $\mathbf{A}$, $n$ et $m$ deux entiers naturels, et $a_0$, $a_1$, ..., $a_n$, $b_0$, $b_1$, ..., $b_m$ des éléments de $A$ tels que $\mathbf{a} = (a_0, a_1, \dots, a_n)$ et $\mathbf{b} = (b_0, b_1, \dots, b_m)$.
            Soit $k$ le maximum de $\lbrace n, m \rbrace$ et $l$ son minimum.
            On définit les éléments $c_0$, $c_1$, ... $c_k$ de $A$ par : 
            \begin{itemize}[nosep]
                \item Pour tout $i$ dans $[\![0, l]\!]$, $c_i = a_i + b_i$.
                    Notons que si $n = m$, alors $l = k$.
                \item Si $n > m$ (alors, $k = n$ et $l = m$), pour tout $i$ dans $[\![l+1, k]\!]$, $c_i = a_i$.
                \item Si $n < m$ (alors, $k = m$ et $l = n$), pour tout $i$ dans $[\![l+1, k]\!]$, $c_i = b_i$.
            \end{itemize}
            Si au moins un de ces éléments est différent de $0_A$, alors l'ensemble des éléments $i$ de $[\![0, k]\!]$ tels que $c_i \neq 0_A$ est un sous-ensemble non vide de $\mathbb{N}$, donc il admet un unique élément maximal $d$ ; sinon, on pose $d = 0$.
            On définit alors $\mathbf{a} + \mathbf{b}$ par le polynôme $(c_0, c_1, \dots, c_d)$.
            Sinon, on définit $\mathbf{a} + \mathbf{b}$ par le polynôme nul $(0_A)$.
        \item Multiplication : Avec les mêmes notations, $\mathbf{a} \times \mathbf{b}$ est le polynôme $(d_0, d_1, \dots d_{n+m})$ défini par : pour tout élément $i$ de $[\![0, n+m]\!]$, on définit l'élément $d_i$ de $A$ par :
            \begin{equation*}
                d_i = \sum_{j=i}^{\mathrm{min}(i,n)} a_j \times b_{i-j}.
            \end{equation*}
            (Notons que $d_{n+m} = a_n \times b_m$.)
            Si au moins un de ces éléments est différent de $0_A$, alors l'ensemble des éléments $i$ de $[\![0, n+m]\!]$ tels que $d_i \neq 0_A$ est un sous-ensemble non vide de $\mathbb{N}$, donc il admet un unique élément maximal $d$ ; sinon, on pose $d = 0$.
            On définit alors $\mathbf{a} \times \mathbf{b}$ par le polynôme $(d_0, d_1, \dots, d_d)$.
            Sinon, on définit $\mathbf{a} \times \mathbf{b}$ par le polynôme nul $(0_A)$.
    \end{itemize}
    L'ensemble des polynômes peut être noté $A[X]$ ou de manière équivalente avec $X$ remplacé par un autre symbole non encore défini. 
    Soit $n$ un entier naturel et $a_0$, $a_1$, ..., $a_n$ des éléments de $A$ tels que $n = 0$ ou $a_n \neq 0$.
    Le polynôme $(a_0, a_1, \dots, a_n)$ pourra être noté $a_0 + a_1 X + \cdots + a_n X^n$, en omettant éventuellement les termes de coefficient nul.
    Pour alléger les notations, on pourra utiliser le symbole $\sum$ : avec les notations précédentes, si $k$ et $l$ sont deux éléments de $[\![0, n]\!]$, alors, 
    \begin{itemize}[nosep]
        \item si $l \geq k$, $\sum_{i=k}^l a_i X^i$ désigne $a_k X^k + a_{k+1} X^{k+1} + \cdots + a_l X^l$ ;
        \item si $l < k$, $\sum_{i=k}^l a_i X^i$ désigne le polynôme nul.
    \end{itemize}

\medskip

\noindent\textbf{Notation :} 
    Soit $(A, +, \times)$ un anneau commutatif.
    Pour tout élément $\mathbf{a}$ de $A[X]$, on définit la suite des coefficients de $\mathbf{a}$, également notée $\mathbf{a}$ quand il n'y a pas de confusion possible, de la manière suivante : pour tout entier naturel $i$, 
\begin{itemize}[nosep]
    \item Si $i$ est inférieur ou égal au degré de $\mathbf{a}$, alors $\mathbf{a}_i$ est le $i$ème coefficient de $\mathbf{a}$ en partant de $0$.
    \item Sinon, on pose $\mathbf{a}_i = 0$.
\end{itemize}
Deux polynômes sont égaux si et seulement si les suites de leurs coefficients le sont.
En effet, 
\begin{itemize}[nosep]
    \item S'ils sont égaux, ils ont la même suite de coefficients par construction.
    \item Si les suites de leurs coefficients sont égales, alors ils ont alors le même degré (index du denier élément non nul, ou $0$ si tous les éléments sont nuls) et les mêmes éléments (élements de cette suite d'index inférieur ou égal au degré).
\end{itemize}

\medskip

\noindent\textbf{Lemme :}
    Soit $(A, +, \times)$ un anneau commutatif.
    Soit $\mathbf{a}$ et $\mathbf{b}$ deux éléments de $A[X]$. 
    Soit $\mathbf{c}$ et $\mathbf{d}$ les polynômes donnés par $\mathbf{c} = \mathbf{a} + \mathbf{b}$ et $\mathbf{d} = \mathbf{a} \times \mathbf{b}$.
    Alors, les suites d'éléments de $\mathbf{c}$ et $\mathbf{d}$ sont données par : pour tout entier naturel $n$, 
    \begin{equation*}
        \mathbf{c}_n = \mathbf{a}_n + \mathbf{b}_n, 
        \qquad
        \mathbf{d}_n = \sum_{i=0}^n \mathbf{a}_i \times \mathbf{b}_{n-i}
        .
    \end{equation*}

\medskip

\noindent\textbf{Démonstration :} ***

\medskip

\noindent\textbf{Remarque :} 
    Soit $(A, +, \times)$ un anneau commutatif.
    L'ensemble $A[X]$ pourrait être défini de manière presque équivalente par l'ensemble des suites d'éléments de $A$ nulles à partir d'un certain rang.

\medskip

\noindent\textbf{Preuve qu'il s'agit bien d'un anneau commutatif :} Montrons qu'il s'agit bien d'un anneau, avec pour éléments neutres $(0_A)$ et $(1_A)$, où $1_A$ est l'élément neutre de $A$ pour $\times$. 
Dans cette démonstration, $\mathbf{a}$, $\mathbf{b}$ et $\mathbf{c}$ sont trois éléments arbitraires de $A[X]$, $n_a$, $n_b$ et $n_c$ sont trois entiers naturels et $a_0$, $a_1$, ..., $a_{n_a}$, $b_0$, $b_1$, ..., $b_{n_b}$ et $c_0$, $c_1$, ..., $c_{n_c}$ sont des éléments de $A$ tels que $(a_{n_a} \neq 0) \vee (n_a = 0)$, $(b_{n_b} \neq 0) \vee (n_b = 0)$ et $(c_{n_c} \neq 0) \vee (n_c = 0)$, et $\mathbf{a} = (a_0, a_1, \dots, a_n)$, $\mathbf{b} = (b_0, b_1, \dots, b_n)$ et $\mathbf{c} = (c_0, c_1, \dots, c_n)$. 
\begin{itemize}[nosep]
    \item \textit{$(A[X], +)$ est un groupe abélien :}
        \begin{itemize}[nosep]
            \item $(A[X], +)$ est un magma puisque $+$ est une loi de composition interne sur $A[X]$.
            \item \textit{L'opération $+$ est commutative :}
                Traitons séparément les deux cas $n_a \geq n_b$ et $_a < n_b$. 
                Si $n_a \geq n_b$, alors $\mathbf{a} + \mathbf{b} = (c_0, c_1, \dots, c_{n_a})$ et $\mathbf{b} + \mathbf{a} = (d_0, d_1, \dots, d_{n_a})$ où, pour tout élément $i$ de $[\![0, n_a]\!]$,
                \begin{itemize}[nosep]
                    \item si $i \leq n_b$, $c_i = a_i + b_i$ et $d_i = b_i + a_i$, et donc $c_i = d_i$ puisque le groupe $(A, +)$ est abélien ;
                    \item sinon, $c_i = a_i$ et $d_i = a_i$, donc $c_i = d_i$.
                \end{itemize}
                On a donc bien $\mathbf{a} + \mathbf{b} = \mathbf{b} + \mathbf{a}$.

                Sinon, $n_a < n_b$. 
                Alors $\mathbf{a} + \mathbf{b} = (c_0, c_1, \dots, c_{n_b})$ et $\mathbf{b} + \mathbf{a} = (d_0, d_1, \dots, d_{n_b})$ où, pour tout élément $i$ de $[\![0, n_b]\!]$,
                \begin{itemize}[nosep]
                    \item si $i \leq n_a$, $c_i = a_i + b_i$ et $d_i = b_i + a_i$, et donc $c_i = d_i$ puisque le groupe $(A, +)$ est abélien ;
                    \item sinon, $c_i = b_i$ et $d_i = b_i$, donc $c_i = d_i$.
                \end{itemize}
                On a donc à nouveau $\mathbf{a} + \mathbf{b} = \mathbf{b} + \mathbf{a}$.
            \item \textit{Le polynôme $(0_A)$ est neutre pour $+$ :}
                Puisque $n_a$ est un entier naturel, $n_a \geq 0$.
                Donc, le minimum de $\lbrace 0, n_a \rbrace$ est $0$ et son maximum est $n_a$.
                Notons $\mathbf{d}$ le polynôme $\mathbf{a} + (0_A)$. 
                Soit $n_d$ un entier naturel et $d_0$, $d_1$, ..., $d_{n_d}$ des éléments de $A$ tels que $\mathbf{d} = (d_0, d_1, \dots, d_{n_d})$.
                Par définition de l'addition, on a $n_d = n_a$, $d_0 = a_0 + 0_A = a_0$ et, pour tout élément $i$ de $[\![1, n_d]\!]$, $d_i = a_i$.
                Donc, $(d_0, d_1, \dots, d_{n_d}) = (a_0, a_1, \dots, a_{n_a})$.
                Donc, $\mathbf{d} = \mathbf{a}$.
                Puisque l'opération $+$ est commutative, on en déduit que $(0_A) + \mathbf{a} = \mathbf{a} + (0_A) = \mathbf{a}$.
            \item \textit{L'opération $+$ est associative :}
                En principe, il y a six cas différents à traiter. 
                Mais nous pouvons réduire ce nombre en notant que, pour prouver $\mathbf{a} + (\mathbf{b} + \mathbf{c}) = (\mathbf{a} + \mathbf{b}) + \mathbf{c}$, les rôles de $\mathbf{a}$ et $\mathbf{c}$ sont interchangeables. 
                En effet, puisque l'addition est commutative (et par symétrie de l'égalité), cette équation est équivalente à $\mathbf{c} + (\mathbf{b} + \mathbf{a}) = (\mathbf{c} + \mathbf{b}) + \mathbf{a}$. 
                Sans perte de généralité (et quitte à échanger les noms de $\mathbf{a}$ et $\mathbf{c}$), on peut donc supposer $d_a \leq d_c$. 
                (Puisque, si ce n'est pas le cas, alors $d_c \leq d_a$.)
                Il n'y a donc que trois cas à traiter : $d_b \leq d_a$, $d_a < d_b \leq d_c$ et $d_c < d_b$.
                Notons $d_{ab}$, $d_{bc}$, $d_{abc1}$ et $d_{abc2}$ les quatre entiers naturels et $\mathrm{ab}_0$, $\mathrm{ab}_1$, ..., $\mathrm{ab}_{d_{ab}}$, $\mathrm{bc}_0$, $\mathrm{bc}_1$, ..., $\mathrm{ab}_{d_{bc}}$, $\mathrm{abc1}_0$, $\mathrm{abc1}_1$, ..., $\mathrm{abc1}_{d_{abc1}}$, $\mathrm{ab}_{d_{bc}}$, $\mathrm{abc2}_0$, $\mathrm{abc2}_1$, ..., $\mathrm{abc2}_{d_{abc2}}$ les éléments de $A$ tels que $\mathbf{a} + \mathbf{b} = \left( \mathrm{ab}_0, \mathrm{ab}_1, \dots, \mathrm{ab}_{d_{ab}} \right)$, $\mathbf{b} + \mathbf{c} = \left( \mathrm{bc}_0, \mathrm{bc}_1, \dots, \mathrm{bc}_{d_{bc}} \right)$, $\mathbf{a} + (\mathbf{b} + \mathbf{c}) = \left( \mathrm{abc1}_0, \mathrm{abc1}_1, \dots, \mathrm{abc1}_{d_{abc1}} \right)$ et $(\mathbf{a} + \mathbf{b}) + \mathbf{c} = \left( \mathrm{abc2}_0, \mathrm{abc2}_1, \dots, \mathrm{abc2}_{d_{abc2}} \right)$.
                Il s'agit de montrer que $d_{abc1} = d_{abc2}$ et que, pour tout entier naturel $i$ inférieur ou égal à $d_{abc1}$, $\mathrm{abc1}_i = \mathrm{abc2}_i$.
                \begin{itemize}[nosep]
                    \item Cas $d_b \leq d_a$ : 
                        On a alors $d_b \leq d_a \leq d_c$.
                        Donc, $d_{ab} = d_a$, donc $d_{abc2} = d_c$ et $d_{bc} = d_c$, donc $d_{abc1} = d_c$. 
                        Donc, $d_{abc2} = d_{abc1}$.
                        ***
                    \item Cas $d_a < d_b \leq d_c$ : ***
                    \item Cas $d_c < d_b$ : ***
                \end{itemize}
            \item \textit{Tout élément de $A[X]$ admet un inverse pour l'opération $+$ :}
                Dans ce paragrahe seulement, pour tout élément $e$ de $A$, on note $\tilde{e}$ l'inverse de $e$ pour l'opération $+$ (qui existe puisque $(A, +)$ est un groupe). 
                Montrons que le polynôme $(\tilde{a}_0, \tilde{a}_1, \dots, \tilde{a}_{n_a})$ (qui est bien un polynôme puisque soit $n_a = 0$ soit $a_{n_a} \neq 0_A$ et donc $\tilde{a}_{n_a} \neq 0_A$) est un inverse de $\mathbf{a}$ pour l'opération $+$.
                ***
        \end{itemize}
    \item \textit{L'opération $\times$ est commutative :}
        ***
    \item \textit{L'opération $\times$ est distributive sur $+$ :} 
        Montrons que $\mathbf{a} \times (\mathbf{b} + \mathbf{c}) = (\mathbf{a} \times \mathbf{b}) + (\mathbf{a} \times \mathbf{c})$.
        Puisque l'opération $\times$ est commutative, cela montrera également $(\mathbf{b} + \mathbf{c}) \times \mathbf{a} = (\mathbf{b} \times \mathbf{a}) + (\mathbf{c} \times \mathbf{a})$.
        ***
    \item \textit{Le polynôme $(1_A)$ est neutre pour $\times$ :}
        ***
\end{itemize}

\medskip

\noindent\textbf{Évaluation d'un polynôme :} Soit $(A, +, \times)$ un anneau commutatif et $\mathbf{a}$ un polynôme sur $A$.
    On peut choisir un entier naturel $n$ et $n+1$ éléments $a_0$, $a_2$, ..., $a_n$ de $A$ tels que $\mathbf{a} = (a_0, a_1, \dots, a_n)$.
    Pour tout élément $a$ de $A$, on note $\mathbf{a}(a)$ l'élément $\sum_{i=0}^n a_i a^i$. 

\medskip

\noindent\textbf{Lemme :} Soit $(A, +, \times)$ un anneau commutatif et $a$ un élément de $A$.
    La fonction de $A[X]$ vers $A$ qui à tout élément $\mathbf{a}$ de $A[X]$ associe $\mathbf{a}(a)$ est un morphisme d'anneaux.

\medskip

\noindent\textbf{Démonstration :} ***

\subsubsection{Degré}

\noindent\textbf{Définition (degré) (rappel) :} Soit $(A, +, \times)$ un anneau commutatif et $\mathbf{a}$ un élément de $A[X]$.
    Soit $n$ un entier naturel et $a_0$, $a_1$, ..., $a_n$ des éléments de $A$ tels que $\mathbf{a} = (a_0, a_1, \dots, a_n)$.
    L'entier naturel $n$ est appelé \textit{degré} de $\mathbf{a}$, et peut être noté $\mathrm{deg}(\mathbf{a})$. 
\medskip

\noindent\textbf{Lemme :} Soit $\mathcal{A}$ un anneau commmutatif et $\mathbf{a}$ et $\mathbf{b}$ deux polynômes sur $\mathcal{A}$, de degrés respectifs $d_a$ et $d_b$. 
    On suppose que le produit de deux éléments de l'anneau ditincts de l'élément neutre pour l'addition l'est aussi.
    Alors, $\mathbf{a} \times \mathbf{b}$ a pour degré $d_a + d_b$ sauf si $\mathbf{a}$ ou $\mathbf{b}$ est le polynôme nul, auquel cas $\mathbf{a} \times \mathbf{b}$ a pour degré $0$.

\medskip

\noindent\textbf{Démonstration :} Évident d'après la définition de la multiplication (avec ces notations, si ni $\mathbf{a}$ ni $\mathbf{b}$ n'est le polynôme nul, alors $d_{n+m} \neq 0_A$.).

\subsubsection{Racines}

\noindent\textbf{Définition (racine) :} Soit $(A, +, \times)$ un anneau commutatif et $\mathbf{a}$ un élément de $A[X]$.
    On note $0_A$ l'élément neutre de $A$ pour $+$.
    Un élément $r$ de $A$ est dit \textit{racine} de $\mathbf{a}$ si $\mathbf{a}(r) = 0_A$.

\medskip

\noindent\textbf{Remarque :} Notons qu'un polynôme peut, en général, avoir plus de racines distinctes que son degré. 
    Considérons par exemple l'anneau $(\mathbb{Z}_6, +, \times)$ (voir définition section~\ref{sub:def_Z_nZ}), d'élément neutre pour $+$ $\bar{0}$, et le polynôme $\mathbf{p} = X^2 - \bar{5} X$. 
    On a : $\mathbf{p}(\bar{0}) = \bar{0}$, $\mathbf{p}(1) = \bar{1} - \bar{5} = \bar{2}$, $\mathbf{p}(2) = \bar{4} - \overline{10} = \overline{-6} = \bar{0}$, $\mathbf{p}(3) = \bar{9} - \overline{15} = \overline{-6} = \bar{0}$, $\mathbf{p}(4) = \overline{16} - \overline{20} = \bar{2}$ et $\mathbf{p}(5) = \overline{25} - \overline{25} = \bar{0}$. 
    Donc, $\mathbf{p}$, bien que de degré $2$, a $4$ racines distinctes ($\bar{0}$, $\bar{2}$, $\bar{3}$ et $\bar{5}$).

