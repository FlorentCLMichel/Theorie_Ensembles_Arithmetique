\subsection{Éléments de théorie des groupes}

Dans cette section, nous donnons quelques concepts de base de théorie des groupes. 

\subsubsection{Définitions}

\noindent\textbf{Définition (magma) :} Un \textit{magma} $\mathcal{M}$ est un couple formé par un ensemble $M$ et une \textit{loi de composition interne} $\cdot$ sur $M$ (parfois appelée \textit{opération}), c'est-à-dire une fonction de $M \times M$ vers $M$. 
    Si $a$ et $b$ sont deux éléments de $M$, on note $a \cdot b$ l'image de $(a,b)$ par $\cdot$.

\medskip

\noindent\textbf{Définition (élément neutre) :} Soit $(M,\cdot)$ un magma.
    Un élément $e$ de $M$ est dit \textit{élément neutre} si 
    \begin{equation*}
        \forall m \in M, \, e \cdot m = m \wedge m \cdot e = m.
    \end{equation*}
    Un magma admettant un élément neutre est dit \textit{unifère} ou \textit{unitère}.

\medskip

\noindent\textbf{Lemme :} Un magma admet au plus un élément neutre.

\medskip

\noindent\textbf{Démonstration :} Soit $(M,\cdot)$ un magma et $e$ et $f$ deux éléments identités pour ce magma.
    Puisque $e$ est un élément identité, $e \cdot f = f$.
    Puisque $f$ est un élément identité, $e \cdot f = e$.
    Par commutativité et transitivité de l'égalité, on en déduit $e = f$.

    \hfill \square

\medskip

\noindent\textbf{Lemme :} Soit $(M, +)$ un magma unifère et $e$ son élément neutre.
    Soit $n$ un entier naturel non nul.
    Alors, l'élément $(\cdots ((e + e) + e) + \cdots +e)$, où $e$ apparaît $n$ fois, est égal à $e$.

\medskip

\noindent\textbf{Démonstration :} On procède par récurrence sur $n$. 
    Pour $n = 1$, le résultat est évident puisque $e = e$.
    Soit $n$ un entier naturel non nul tel que le résultat est vrai. 
    Soit $x$ l'élément $(\cdots ((e+e) + e) + \cdots + e)$, où $e$ apparaît $n+1$ fois. 
    Alors, on peut noter $x = (y + e)$, où $y$ est l'élément défini dans le lemme avec $e$ apparaissant $n$ fois. 
    Donc, $y = e$.
    Donc, $x = e$. 
    On en déduit que le résultat attendu est vrai en remplaçant $n$ par $n+1$. 
    Par récurrence, il est donc vrai pour tout entier naturel non nul $n$.

    \done

\medskip

\noindent\textbf{Définition (morphisme de magmas) :} Soit $\left(M, \cdot \right)$ et $\left(N, \ast \right)$ deux magmas.
    Une fonction $f$ de $M$ vers $N$ est dite \textit{morphisme de magmas} de $\left(M, \cdot \right)$ vers $\left(N, \ast \right)$ si elle satisfait : 
    \begin{equation*}
        \forall (a,b) \in M^2, \, f (a \cdot b) = f(a) \ast f(b) .
    \end{equation*}

\medskip

\noindent\textbf{Définition (noyau) :} Soit $(M_1, \dot)$ un magma et $\mathcal{M}_2$ un magma unifère.  
    Soit $0$ l'élément neutres de $\mathcal{M}_2$ et $f$ un morphisme de magmas de $(M_1, \dot)$ vers $\mathcal{M}_2$.
    On appelle \emph{noyau} de $f$, noté $\mathrm{Ker}(f)$, l'ensemble des antécéédants de $0$ par $f$: $\mathrm{Ker}(f) = \lbrace x \in M_1 \vert f(x) = 0 \rbrace$.

\medskip

\noindent\textbf{Définition (isomorpisme de magmas) :} Soit $\mathcal{M}$ et $\mathcal{N}$ deux magmas.
    Un morphise de magmas $f$ de $\mathcal{M}$ vers $\mathcal{N}$ est dit \textit{isomorphisme de magmas} s'il est également une bijection.

\medskip

\noindent\textbf{Lemme :} L'image d'un élément neutre par un morphisme de magmas surjectif (et donc, en particulier, par un isomorphisme) est un élément neutre.

\medskip

\noindent\textbf{Démonstration :} Soit $\left(M, \cdot \right)$ et $\left(N, \ast \right)$ deux magmas et soit $f$ un morphisme surjectif du premier vers le second. 
    Soit $e$ un élément neutre de $\left(M, \cdot \right)$. 
    On a $f(e) \in N$. 
    Soit $y$ un élément de $N$. 
    Il s'agit de montrer que $f(e) \ast y = y$ et $y \ast f(e) = y$.

    Puisque $f$ est surjectif, on peut choisir un élément $x$ de $M$ tel que $f(x) = y$. 
    Donc, $f(e) \ast y = f(e) \ast f(x)$. 
    Puisque $f$ est un morphisme de magmas, $f(e) \ast f(x) = f(e \cdot x)$. 
    Puisque $e$ est un élément neutre pour $\cdot$, $e \cdot x = x$. 
    Donc, $f(e) \ast f(x) = f(x)$.
    Donc, $f(e) \ast y = y$. 

    De même, puisque $f$ est un morphisme de magmas, $f(x) \ast f(e) = f(x \cdot e)$. 
    Puisque $e$ est un élément neutre pour $\cdot$, $x \cdot e = x$. 
    Donc, $f(x) \ast f(e) = f(x)$.
    Donc, $y \ast f(x) = y$. 

    \hfill \square

\medskip

\noindent\textbf{Corolaire :} Soit $\mathcal{M}$ et $\mathcal{N}$ deux magmas. 
    On suppose qu'il existe un morphisme de magmas surjectif de $\mathcal{M}$ vers $\mathcal{N}$. 
    Si $\mathcal{M}$ est unifère, alors $\mathcal{N}$ l'est aussi. 
    (Car l'image par le morphisme de l'élément neutre de $\mathcal{M}$ est un élément neutre pour $\mathcal{N}$.)

\medskip

\noindent\textbf{Lemme :} L'inverse d'un isomorphisme de magmas est un isomorphisme de magmas.

\medskip

\noindent\textbf{Démonstration :} Soit $\left(M, \cdot \right)$ et $\left(N, \ast \right)$ deux magmas et soit $f$ un isomorphisme du premier vers le second. 
    Soit $g$ l'inverse de $f$ (qui existe et est une bijection puisque $f$ est une bijection). 
    Montrons que $g$ est un morphisme de magmas. 
    Puisque $g$ est également une bijection, il s'agira alors d'un isomorphisme. 

    Soit $a$ et $b$ deux éléments de $N$. 
    Puisque $f$ est une bijection, elle est surjective, donc on peut choisir deux éléments $c$ et $d$ de $M$ tels que $a = f(c)$ et $b = f(d)$. 
    On a alors : $g(a) \cdot g(b) = g(f(c)) \cdot g(f(d))$.
    Puisque $g$ est l'inverse de $f$, cela donne : $g(a) \cdot g(b) = c \cdot d$. 

    Par ailleurs, $g(a \ast b) = g(f(c) \ast f(d))$. 
    Puisque $f$ est un morphisme de magmas on a $f(c) \ast f(d) = f (c \cdot d)$, donc cela donne $g(a \ast b) = g(f(c \cdot d))$.
    Puisque $g$ est l'inverse de $f$, cela donne : $g(a \ast b) = c \cdot d$. 
    Donc, $g(a \ast b) = g(a) \cdot g(b)$. 

    Cela étant vrai pour tous éléments $a$ et $b$ de $M$, on en déduit que $g$ est un morphisme de magmas, et donc un isomorphisme, de $\left(M, \cdot \right)$ vers $\left(N, \ast \right)$.

    \hfill \square

\medskip

\noindent\textbf{Définition (associativité) :} Soit $(M,\cdot)$ un magma.
    La loi de composition interne $\cdot$ est dite \textit{associative} si
    \begin{equation*}
        \forall a \in M, \, \forall b \in M, \, \forall c \in M , \,  
        (a \cdot b) \cdot c = a \cdot (b \cdot c).
    \end{equation*}
    Le magma $(M, \cdot)$ est alors dit \textit{associatif}. 
    Un magma associatif est aussi appelé \textit{demi-groupe}.

\medskip

\noindent\textbf{Lemme :} Soit $\mathcal{M}_1$ et $\mathcal{M}_2$ deux magmas.
    On suppose que $\mathcal{M}_1$ est associatif et qu'il existe un morphisme de magmas surjectif de $\mathcal{M}_1$ vers $\mathcal{M}_2$.
    Alors $\mathcal{M}_2$ est associatif.

\medskip

\noindent\textbf{Démonstration :} Soit $(M,\cdot)$ et $(N,\ast)$ deux magmas. 
    On suppose que le premier est associatif et qu'il existe un morphisme de magmas surjectif, $f$, du premier vers le second. 
    Soit $a$, $b$ et $c$ trois éléments de $N$. 
    Puisque $f$ est surjectif, on peut choisir trois éléments $d$, $e$ et $g$ de $M$ tels que $f(d) = a$, $f(e) = b$ et $f(g) = c$.
    On a alors : $(a \ast b) \ast c = (f(d) \ast f(e)) \ast f(g)$.
    Puisque $f$ est un morphisme de magmas, $(f(d) \ast f(e)) \ast f(g) = f(d \cdot e) \ast f(g) = f((d \cdot e) \cdot g)$.
    Puisque $\cdot$ est associative, $f((d \cdot e) \cdot g) = f(d \cdot (e \cdot g))$. 
    En utilisant à nouveau le fait que $f$ est un morphisme de magmas, on obtient : $f(d \cdot (e \cdot g)) = f(d) \ast f(e \cdot g)$ et $f(d) \ast f(e \cdot g)) = f(d) \ast (f(e) \ast f(g))$. 
    Enfin, puisque $d$, $e$ et $g$ sont des antécédents respectifs de $a$, $b$ et $c$ par $f$, on a $f(d) \ast (f(e) \ast f(g)) = a \ast (b \ast c)$. 
    En combinant ces formules et en utilisant la transitivité de l'égalité, il vient : $(a \ast b) \ast c = a \ast (b \ast c)$.

    \hfill \square

\medskip

\noindent\textbf{Définition (commutativité) :} Soit $(M,\cdot)$ un magma.
    La loi de composition interne $\cdot$ est dite \textit{commutative} si
    \begin{equation*}
        \forall a \in M, \, \forall b \in M,  
        a \cdot b = b \cdot a.
    \end{equation*}
    Le magma $(M, \cdot)$ est alors dit \textit{commutatif} ou \textit{abélien}.

\medskip

\noindent\textbf{Lemme :} Soit $\mathcal{M}_1$ et $\mathcal{M}_2$ deux magmas.
    On suppose que $\mathcal{M}_1$ est commutatif et qu'il existe un morphisme de magmas surjectif de $\mathcal{M}_1$ vers $\mathcal{M}_2$.
    Alors $\mathcal{M}_2$ est commutatif.

\medskip

\noindent\textbf{Démonstration :} Soit $(M,\cdot)$ et $(N,\ast)$ deux magmas. 
    On suppose que le premier est commutatif et qu'il existe un morphise de magmas surjectif, $f$, du premier vers le second. 
    Soit $a$ et $b$ deux éléments de $N$. 
    Puisque $f$ est surjectif, on peut choisir deux éléments $d$ et $e$ de $M$ tels que $f(d) = a$ et $f(e) = b$.
    On a alors : $a \ast b = f(d) \ast f(e)$.
    Puisque $f$ est un morphisme de magmas, $f(d) \ast f(e) = f(d \cdot e)$.
    Puisque $\cdot$ est commutative, $d \cdot e = e \cdot d$, donc $f(d \cdot e) = f(e \cdot d)$. 
    En utilisant à nouveau le fait que $f$ est un morphisme de magmas, on obtient : $f(e \cdot d) = f(e) \ast f(d)$. 
    Enfin, puisque $d$ et $e$ sont des antécédents respectifs de $a$ et $b$ par $f$, on a $f(e) \ast f(d) = b \ast a$. 
    En combinant ces formules et en utilisant la transitivité de l'égalité, il vient : $a \ast b = b \ast a$.

    \hfill \square

\medskip

\noindent\textbf{Définition (monoïde) :} Un magma unifère et associatif est appelé \textit{monoïde}.

\medskip

\noindent\textbf{Définition (monoïde abélien) :} Un magma unifère, associatif et abélien est dit \textit{monoïde abélien}.

\medskip

\noindent\textbf{Définition (puissance) :} Soit $(M,\cdot)$ un monoïde abélien et $1$ son élément neutre.
    On définit la suite $\mathrm{P}$ de fonctions de $M$ vers $M$ par : 
    \begin{itemize}[nosep]
        \item pour tout élément $m$ de $M$, $P_0(m) = 1$,
        \item pour tout entier naturel $n$, pour tout élément $m$ de $M$, $P_{n+1}(m) = P_n(m) \cdot n$.
    \end{itemize}
    Pour tout élément $m$ de $M$ et tout entier naturel $n$, on notera $m^n$ l'élément $P_n(m)$.

\medskip

\noindent\textbf{Lemme :} Soit $(M,\cdot)$ un monoïde abélien, $m$ un élément de $M$, et $a$ et $b$ deux entiers naturels. 
    Alors, $m^{a + b} = m^a \cdot m^b$.

\medskip

\noindent\textbf{Démonstration :} On procéde par récurrence sur $b$.
    Fixons $m$ et $a$.
    On veut montrer que le prédicat à un paramère libre $P$ définit par : $P(b): m^{a+b} = m^a \cdot m^b$ est vrai pour tout entier naturel $b$.
    \begin{itemize}[nosep]
        \item $P(0)$ est équivalent à $m^a = m^a$, qui est vrai.
        \item Soit $b$ un entier naturel tel que $P(b)$ est vrai. 
            Alors, $m^{a+(b+1)} = m^{(a+b)+1} = m^{a+b} \cdot m = (m^a \cdot m^b) \cdot m = m^a \cdot (m^b \cdot m) = m^a \cdot m^{b+1}$.
            Donc, $P(b+1)$ est vrai.
    \end{itemize}
    Par récurrence, $P(b)$ est vrai pour tout entier naturel $b$.

    \done

\medskip

\noindent\textbf{Lemme :} Soit $(M,\cdot)$ un monoïde abélien, $m$ un élément de $M$, et $a$ et $b$ deux entiers naturels. 
    Alors, $m^{a \times b} = (m^a)^b$.

\medskip

\noindent\textbf{Démonstration :} On procéde par récurrence sur $b$.
    Fixons $m$ et $a$.
    On veut montrer que le prédicat à un paramère libre $P$ définit par : $P(b): m^{a \times b} = (m^a)^b$ est vrai pour tout entier naturel $b$.
    \begin{itemize}[nosep]
        \item $P(0)$ est équivalent à $m^{a \times 0} = (m^a)^0$, donc à $m^0 = e$, qui est vrai.
        \item Soit $b$ un entier naturel tel que $P(b)$ est vrai. 
            Alors, $m^{a \times (b+1)} = m^{(a \times b) + a} = m^{a \times b} \times m^a = (m^a)^b \times m^a = (m^a)^{b+1}$.
            Donc, $P(b+1)$ est vrai.
    \end{itemize}
    Par récurrence, $P(b)$ est vrai pour tout entier naturel $b$.

    \done

\medskip

\noindent\textbf{Lemme :} Soit $(M, \cdot)$ un monoïde abélien, $a$ et $b$ deux éléments de $M$, et $n$ un entier naturel. 
    Alors, $(a \cdot b)^n = a^n \cdot b^n$.

\medskip

\noindent\textbf{Démonstration :} On procéde par récurrence sur $n$.
    Soit $P$ le prédicat à un paramètre libre définit par : $P(n): (a \cdot b)^n = a^n \cdot b^n$.
    Notons $e$ l'élément neutre de $(M, \cdot)$.

    On a : $(a \cdot b)^0 = e$ et $a^0 \cdot b^0 = e \cdot e = e$.
    Donc, $(a \cdot b)^0 = a^0 \cdot b^0$.
    Donc, $P(0)$ est vrai.

    Soit $n$ un entier naturel tel que $P(n)$ est vrai. 
    Alors, $(a \cdot b)^{n+1} = (a \cdot b)^n \cdot (a \cdot b) = (a^n \cdot b^n) \cdot (a \cdot b) = (a^n \cdot a) \cdot (b^n \cdot b) = a^{n+1} \cdot b^{n+1}$.
    Donc, $P(n+1)$ est vrai.

    Par récurrence, on en déduit que $P(n)$ est vrai pour tout entier naturel $n$.

    \done

\medskip

\noindent\textbf{Définition (morphisme de monoïdes) :} Un morphisme de magmas d'un monoïde vers un autre est dit \textit{morphisme de monoïdes}.

\medskip

\noindent\textbf{Définition (isomorpisme de monoïdes) :} Un isomorphisme de magmas d'un monoïde vers un autre est dit \textit{isomorphisme de monoïdes}.

\medskip

\noindent\textbf{Lemme :} L'inverse d'un isomorphisme de monoïdes est un isomorphisme de monoïdes.

\medskip

\noindent\textbf{Démonstration :} Conséquence directe du même résultat pour un isomorphisle de magmas.

\medskip

\noindent\textbf{Définition (inverse) :} Soit $(M,\cdot)$ un monoïde et $e$ son élément neutre. 
    Soit $m$ un élément de $M$. 
    Un élément $n$ de $M$ est dit \textit{inverse} de $m$ (pour $\cdot$) si $m \cdot n = e \wedge  n \cdot m = e$.

\medskip

\noindent\textbf{Remarque :} Soit $(M,\cdot)$ un monoïde et $e$ son élément neutre.
    Alors, $e$ est son propre inverse.
    
\medskip

\noindent\textbf{Lemme :} Soit $(M,\cdot)$ un monoïde et $m$ un élément de $M$.
    Alors, $m$ admet au plus un seul inverse pour $\cdot$.

\medskip

\noindent\textbf{Démonstration :} Soit $(M,\cdot)$ un monoïde et $e$ son élément neutre. 
    Soit $m$ un élément de $M$. 
    Soit $n$ et $o$ deux inverse de $m$ pour $\cdot$. 
    Alors, $(n \cdot m) \cdot o = e \cdot o = o$.
    Par ailleurs, $n \cdot (m \cdot o) = n \cdot e = n$.
    Puisque $\cdot$ est associative, $(n \cdot m) \cdot o = n \cdot (m \cdot o)$. 
    Donc, $o = n$.

    \hfill \square

\medskip

\noindent\textbf{Lemme :} Soit $(M,\cdot)$ un monoïde et $m$ et $n$ deux éléments de $M$.
    Si $n$ est l'inverse de $m$, alors $m$ est l'inverse de $n$.

\medskip

\noindent\textbf{Démonstration :} 
    Notons $e$ l'élément neutre de $(M, \cdot)$.
    Si $n$ est l'inverse de $m$, alors $m \cdot n = n \cdot m = e$.
    Donc, $n \cdot m = m \cdot n = e$.
    Donc, $m$ est l'inverse de $n$.

    \done

\medskip

\noindent\textbf{Lemme :} Soit $(M,\cdot)$ un monoïde et $a$ et $b$ deux éléments de $M$.
    On suppose que $a$ a un inverse $c$ et $b$ a un inverse $d$. 
    Alors, $a \cdot b$ a un inverse, et son inverse est $d \cdot c$.

\medskip

\noindent\textbf{Démonstration :} 
    Notons $e$ l'élément neutre de $(M, \cdot)$.
    On a : $(a \cdot b) \cdot (d \cdot c) = a \cdot (b \cdot (d \cdot c)) = a \cdot ((b \cdot d) \cdot c) = a \cdot (e \cdot c) = a \cdot c = e$.

    \done

\medskip

\noindent\textbf{Définition (groupe) :} Soit $(M,\cdot)$ un monoïde. 
    Si chaque élément de $M$ admet un inverse pour $\cdot$, alors $(M,\cdot)$ est appelé \textit{groupe}.

\medskip

\noindent\textbf{Lemme :} Soit $(G, \cdot)$ un groupe, $1$ son élément neutre et $a$ un élément de $G$
    Si $a \cdot b = 1$, alors $b$ est l'inverse de $a$.

\medskip

\noindent\textbf{Démonstration :}
    Soit $c$ l'inverse de $a$.
    On a : $c \cdot (a \cdot b) = c \cdot 1$, donc $(c \cdot a) \cdot b = c$, donc $1 \cdot b = c$, donc $b = c$.

    \done

\medskip

\noindent\textbf{Définition (groupe abélien) :} Soit $(M,\cdot)$ un monoïde abélien. 
    Si chaque élément de $M$ admet un inverse pour $\cdot$, alors $(M,\cdot)$ est appelé \textit{groupe abélien}.

\medskip

\noindent\textbf{Définition (morphisme de groupes) :} Un morphisme de magmas d'un groupe vers un autre est dit \textit{morphisme de groupes}.

\medskip

\noindent\textbf{Lemme :} Soit $(A, +)$ et $(B, \oplus)$ deux groupes. 
    Notosns $0_A$ l'élément neutre du premier et $0_B$ celui du second. 
    Soit $f$ un morphisme de groupes de $(A, +)$ vers $(B, \oplus)$.
    Alors, 
    \begin{itemize}[nosep]
        \item $f(0_A) = 0_B$,
        \item soit $a$ un élément de $A$ et $b$ son inverse, alors $f(b)$ est l'inverse de $f(a)$, 
        \item $f$ est injective si et seulement si son noyau est $\lbrace 0_A \rbrace$.
    \end{itemize}

\medskip

\noindent\textbf{Démonstration :}
    \begin{itemize}[nosep]
        \item On a : $f(0_A) = f(0_A + 0_A) = f(0_A) \oplus f(0_A)$. 
            Soit $c$ l'inverse de $f(0_A)$.
            On a : $c \oplus f(0_A) = c \oplus (f(0_A) \oplus f(0_A)) = (c \oplus f(0_A)) \oplus f(0_A)$.
            Donc, $0_B = 0_B \oplus f(0_A) = f(0_A)$.
        \item On a : $f(a) \oplus f(b) = f(a + b) = f(0_A) = 0_B$.
            Donc, $f(b)$ est l'inverse de $f(a)$.
        \item 
            \begin{itemize}[nosep]
                \item Si $f$ est injective, $0_B$ ne peut admettre au plus qu'un antécédant pat $f$. 
                    Puisque $0_A$ est un tel antécédant, il ne peut y en avoir d'autre, donc son noyaux est $\lbrace 0_A \rbrace$.
                \item Supposons que le noyau de $f$ est $\lbrace 0_A \rbrace$
                    Alors, $0_A$ est le seul antécédant de $0_B$ par $f$.
                    Soit $a$ et $b$ deux éléments de $A$ tels que $f(a) = f(b)$. 
                    Soit $c$ l'inverse de $a$.
                    On a : $f(c) \oplus f(a) = f(c + a) = 0_B$.
                    Donc, $f(c) \oplus f(b) = 0_B$.
                    Donc, $f(c + b) = 0_B$.
                    Donc, $c + b = 0_A$, donc $a + (c + b) = a + 0_A$, donc $(a + c) + b = a$, donc $0_A + b = a$, donc $b = a$.
                    Cela montre que $f$ est injective.
            \end{itemize}
    \end{itemize}

    \done

\medskip

\noindent\textbf{Définition (isomorphisme de groupes) :} Un isomorphisme de magmas d'un groupe vers un autre est dit \textit{isomorphisme de groupes}.

\medskip

\noindent\textbf{Lemme :} L'inverse d'un isomorphisme de groupes est un isomorphisme de groupes.

\medskip

\noindent\textbf{Démonstration :} Conséquence direct du même résultat pour un isomorphisme de magmas.

\medskip

\noindent\textbf{Définition (puissance négative) :} Soit $(G,\cdot)$ un groupe abélien. 
    Pour tout entier $n$ strictement négatif et tout élément $g$ de $G$, on note $g^n$ l'élément $h^n$, où $h$ est l'inverse de $g$.

\medskip

\noindent\textbf{Lemme :} Soit $(G,\cdot)$ un groupe abélien, $e$ l'élément neutre de $G$ et $g$ un élément de $G$.
    Pour tout entier naturel $n$, $g^n \cdot g^{-n} = e$.

\medskip

\noindent\textbf{Démonstration :} 
    On procède par récurrence sur $n$. 
    Soit $P$ le prédicat à un paramètre libre définit par : $P(n): g^n \cdot g^{-n} = e$.
    Soit $h$ l'inverse de $g$. 
    \begin{itemize}[nosep]
        \item $P(0)$ est équivalent à $g^0 \cdot g^0 = e$, donc à $e \cdot e = e$, qui est vrai.
        \item Soit $n$ un entier naturel tel que $P(n)$ est vrai.
            On a : $g^{n+1} \cdot g^{-(n+1)} = g^{n+1} \cdot h^{n+1} = g^n \cdot g \cdot h^n \cdot h = g^n \cdot g \cdot h \cdot h^n = g^n \cdot e \cdot h^n = g^n \cdot h^n = g^n \cdot g^{-n} = e$.
            Donc, $P(n+1)$ est vrai.
    \end{itemize}
    Par récurrence, $P(n)$ est vrai pour tout entier naturel $n$.

    \done

\medskip

\noindent\textbf{Lemme :} Soit $(G,\cdot)$ un groupe abélien, $g$ un élément de $G$, et $a$ et $b$ deux entiers.
    Alors, $g^{a+b} = g^a \times g^b$.

\medskip

\noindent\textbf{Démonstration :} Soit $h$ l'inverse de $g$. 
    \begin{itemize}[nosep]
        \item Si $a \geq 0$ et $b \geq 0$, on se ramène au cas déjà démontré.
        \item Si $a \geq 0$, $b < 0$ et $a+b \geq 0$ on a $g^{a+b} \cdot g^{-b} = g^{(a+b) + (-b)}= g^a$, donc $g^{a+b} = g^a \cdot g^b$.
        \item Si $a \geq 0$, $b < 0$ et $a+b < 0$ on a $g^{a+b} \cdot g^{-a} = h^{-(a+b)} h^a = h^{-b} = g^b$, donc $g^{a+b} = g^b \cdot g^a = g^a \cdot g^b$.
        \item Si $a < 0$, $b \geq 0$ et $a+b \geq 0$ on a $g^{a+b} \cdot g^{-a} = g^{(a+b) + (-a)} = g^b$, donc $g^{a+b} = g^b \cdot g^a = g^a \cdot g^b$.
        \item Si $a < 0$, $b \geq 0$ et $a+b < 0$ on a $g^{a+b} \cdot g^{-b} = h^{-(a+b)} h^b = h^{-a} = g^a$, donc $g^{a+b} = g^a \cdot g^b$.
        \item Si $a < 0$ et $b < 0$, on a $-(a+b) < 0$, $-a \geq 0$ et $-b \geq 0$, donc $g^{a+b} = h^{-(a+b)} = h^{(-a) + (-b)} = h^{-a} \cdot h^{-b} = g^a \cdot g^b$.
    \end{itemize}
    Dans tous les cas, on a bien $g^{a+b} = g^a \cdot g^b$.

    \done

\medskip

\noindent\textbf{Lemme :} Soit $(G,\cdot)$ un groupe abélien, $g$ un élément de $G$, et $a$ et $b$ deux entiers naturels. 
    Alors, $g^{a \times b} = (g^a)^b$.

\medskip

\noindent\textbf{Démonstration :} Soit $h$ l'inverse de $g$. 
    \begin{itemize}[nosep]
        \item Si $a \geq 0$ et $b \geq 0$, on se ramène au cas déjà démontré.
        \item Si $a \geq 0$ et $b < 0$, on a $g^{a \times b} = g^{-(a \times (-b))} = h^{a \times (-b)} = (h^a)^{-b}$ et $(g^a)^b = (g^{-a})^{-b} = (h^a)^{-b}$.
        \item Si $a < 0$ et $b \geq 0$, on a $g^{a \times b} = g^{-((-a) \times b)} = h^{(-a) \times b} = (h^{-a})^b$ et $(g^a)^b = (h^{-a})^b$.
        \item Si $a < 0$ et $b < 0$, on a $g^{a \times b} = g^{(-a) \times (-b)} = (g^{-a})^{-b}$ et $(g^a)^b$ est égal à $(g^{-a})^{-b}$ puisque $g^{-a}$ est l'inverse de $g^a$.
    \end{itemize}
    Dans tous les cas, on a bien $g^{a \times b} = (g^a)^b$.

\medskip

\noindent\textbf{Lemme :} Soit $(G, \cdot)$ un groupe abélien, $a$ et $b$ deux éléments de $G$, et $n$ un entier. 
    Alors, $(a \cdot b)^n = a^n \cdot b^n$.

\medskip

\noindent\textbf{Démonstration :} 
\begin{itemize}[nosep]
    \item Si $n \geq 0$, il s'agit d'un résultat déjà démontré.
    \item Supposons $n < 0$. 
        Soit $c$ l'inverse de $a$ et $d$ l'inverse de $b$.
        Alors, $(a \cdot b)^n = (d \cdot c)^{-n} = d^{-n} \cdot c^{-n} = b^n \cdot a^n = a^n \cdot b^n$.
\end{itemize}
Dans les deux cas, on a bien $(a \cdot b)^n = a^n \cdot b^n$.

\done

\medskip

\noindent\textbf{Définition (cyclicité) :} Un groupe abélien $(G, \cdot)$ est dit $\textit{cyclique}$ s'il existe un élément $g$ de $G$ tel que : 
    \begin{equation*}
        \forall x \in G, \, \exists n \in \mathbb{N}, \, g^n = x.
    \end{equation*}
    Un tel élément $g$ est dit \textit{générateur} du groupe.

\medskip

\noindent\textbf{Définition (sous-groupe) :} Soit $(G, \cdot)$ un groupe et $H$ un sous-ensemble non-vide de $G$.
    Si $(H, \cdot)$ est un groupe, alors il est dit \textit{sous-groupe} de $(G, \cdot)$.

\medskip

\noindent\textbf{Lemme :} Soit $(G, \cdot)$ un groupe et $H$ un sous-ensemble non-vide de $G$ tel que $(H, \cdot)$ est un groupe.
    Soit $e$ lélément neutre de $(G, \cdot)$. 
    Alors $e \in H$.

\medskip

\noindent\textbf{Démonstration :} Puisque $(H, \cdot)$ est un groupe, donc unifere, $H$ est non vide. 
    Soit $h$ un élément de $H$. 
    Puisque $(H, \cdot)$ est un groupe, l'inverse $l$ de $h$ appartient aussi à $H$, et donc $l \cdot h$ également. 
    Puisque $l$ est l'inverse de $H$, $l \cdot h = e$, ce aui conclut la preuve.

    \done

\subsubsection{Quelques résultats}

\noindent\textbf{Lemme :} Parmis les ensembles construits précédemment, 
    \begin{itemize}[nosep]
        \item $(\mathbb{N},+)$, $(\mathbb{N},\times)$ et $(\mathbb{Z},\times)$ sont des monoïdes abéliens,
        \item $(\mathbb{Z},+)$ est un groupe abélien.
    \end{itemize}

\medskip

\noindent\textbf{Démonstration :} Nous avons déjà démontré tous les éléments nécessaires. 
    En effet, 
    \begin{itemize}[nosep]
        \item Les operations $+$ et $\times$ sont des lois de compositions internes sur $\mathbb{N}$ et $\mathbb{Z}$, donc $(\mathbb{N},+)$, $(\mathbb{N},\times)$, $(\mathbb{Z},+)$ et $(\mathbb{Z},\times)$ sont des magmas.
        \item Les operations $+$ et $\times$ sont associatives et admettent chacune un élément neutre ($0$ pour la première et $1$ pour la seconde) dans $\mathbb{N}$ et $\mathbb{Z}$, donc $(\mathbb{N},+)$, $(\mathbb{N},\times)$, $(\mathbb{Z},+)$ et $(\mathbb{Z},\times)$ sont des monoïdes.
        \item Les operations $+$ et $\times$ sont commutatives, donc $(\mathbb{N},+)$, $(\mathbb{N},\times)$, $(\mathbb{Z},+)$ et $(\mathbb{Z},\times)$ sont des monoïdes abéliens.
        \item Pour tout élément $z$ de $\mathbb{Z}$, on a $z + (-z) = (-z) + z = 0$, donc $-z$ est un inverse de $z$ pour $+$. 
            $(\mathbb{Z},+)$ est donc un groupe abélien.
    \end{itemize}

\subsubsection{Groupe fini}

\noindent\textbf{Définition :} Soit $(G, \cdot)$ un groupe. Si $G$ est fini, $(G, \cdot)$ est dit \textit{groupe fini}.
    Le cardinal de $G$ est parfois appelé cardinal du groupe $(G, \cdot)$.

\medskip

\noindent\textbf{Lemme :} Soit $(G, \cdot)$ un groupe commutatif fini, $1$ son élément neutre et $n$ le cardinal de $G$. 
    Soit $g$ un élément de $G$. 
    Alors, on peut choisir un entier $m$ tel que $m \neq 0$, $m \leq n$ et $g^m = 1$. 
    Soit $E$ l'ensemble des entiers naturels $m$ tels que $0 < m \leq n$ et $g^m = 1$.
    Il s'agit d'un sous-ensemble non vide de $\mathbb{N}$, donc il admet un plus petit élément. 
    Ce dernier (qui est inférieur ou égal à $n$ et strictement supérieur à $0$) est appelé \textit{ordre} de $g$.

\medskip

\noindent\textbf{Démonstration :} Soit $f$ la fonction de $[\![0, n]\!]$ (égal à $n+1$) vers $G$ qui à tout élément $m$ de $[\![0, n]\!]$ associe $g^m$. 
    Puisque le cardinal de $G$ est strictement inférieur à $n+1$, $f$ ne peut être injective.
    Donc, on peut choisir deux éléments $a$ et $b$ de $[\![0,n]\!]$ tels que $a \neq b$ et $f(a) = f(b)$, et donc $g^a = g^b$. 
    \begin{itemize}[nosep]
        \item Si $a > b$, on a $g^{a-b} = e$.
            Puisque $a \leq n$ et $a > b$, on a $a-b > 0$ et $a-b \leq n$. 
        \item Sinon, $a < b$ et on a $g^{b-a} = e$.
            Puisque $b \leq n$ et $b > a$, on a $b-a > 0$ et $b-a \leq n$. 
    \end{itemize}

    \done

\medskip

\noindent\textbf{Lemme :} Soit $(G, \cdot)$ un groupe commutatif fini et $g$ un élément de $G$. 
    Alors, $g$ est un générateur de $(G, \cdot)$ si et seulement si il est d'ordre $\abs{G}$. 

\medskip

\noindent\textbf{Démonstration :} 
Notons $n$ le cardinal de $G$.
Notons que, puisque $(G,\cdot)$ est un groupe, il admet au moins un élément neutre, donc $n > 0$.
Soit $f$ la fonction de $[\![0, n-1]\!]$ (égal à $n$) vers $G$ définie par : pour tout entier naturel $m$ strictement inférieur à $n$, $f(m) = g^m$.
\begin{itemize}[nosep]
    \item Supposons que $g$ est d'ordre $n$.
        Soit $a$ et $b$ deux éléments de $[\![0, n-1]\!]$ tels que $f(a) = f(b)$.
        Alors, $g^a = g^b$, donc $g^{a-b} = g^{b-a} = e$, donc $g^{\abs{a-b}} = e$.
        Puisque $a < n$ et $b < n$, $a-b < n$ et $b-a < n$, donc $\abs{a-b} < n$.
        Puisque $g$ est d'ordre $n$, on doit donc avoir $\abs{a-b} = 0$.
        Donc, $a = b$.
        Cela montre que $f$ est inective.
        Puisque $G$ et $[\![0, n-1]\!]$ ont le même cardinal $n$, $f$ est donc bijective.
        Donc, pour tout élément $h$ de $G$, on peut choisir un élément $m$ de $[\![0, n-1]\!]$ tel que $g^m = h$.
        Donc, $g$ est un générateur de $(G,\cdot)$.
    \item Supposons que $g$ n'est pas d'ordre $n$. 
        Soit $m$ l'ordre de $g$.
        Alors, $m \neq 0$ et $g^m = e = g^0$, donc $f(m) = f(0)$.
        Donc, $f$ n'est pas injective.
        Puisque $G$ et $[\![0, n-1]\!]$ ont le même cardinal $n$, $f$ n'est donc pas surjective (sans quoi elle serait bijective, et donc injective).
        Donc, on peut choisir un élément $h$ de $g$ tel que, pour tout élément $k$ de $[\![0, n-1]\!]$, $g^k \leq h$.
        Soit $k$ un entier naturel quelconque et $q$ et $l$ le quotient et le reste de la division euclidienne de $k$ par $m$. 
        ALors, $l < m$, donc $l < n$.
        On a : $g^k = g^{q m + l} = (g^m)^q \cdot g^l = g^l$.
        Donc, $g^k \neq h$.
        Ainsi, $g$ n'est pas un générateur de $(G,\cdot)$.
\end{itemize}

\done

\subsubsection{Groupe quotient}

\noindent\textbf{Définition :} Soit $(G, \cdot)$ un groupe abélien et $(H, \cdot)$ un sous-groupe de $(G, \cdot)$.
    On définit l'ensemble $G \divslash H$ comme l'ensemble des classes d'équivalences de $G$ pour la relation $R$ définie par : $\forall g \in G \, \forall g' \in G \, g R g' \Leftrightarrow g^{-1} \cdot g' \in H$, oú un exposant $-1$ indique l'inverse. 
    Pour tout élément $g$ de $G$, on note $\bar{g}$ la classe d'équivalence de $g$ pour $R$.
    On définit la loi de composition interne $\cdot$ sur $G \divslash H$ de la manière suivante : soit $c_1$ et $c_2$ deux éléments de $G \divslash H$, on peut choisir un éléments $g_1$ de $c_1$ et un élément $g_2$ de $c_2$ ; on pose alors $c_1 \cdot c_2 = \overline{g_1 \cdot g_2}$. 
    Alors, 
    \begin{itemize}[nosep]
        \item La loi de composition interne $\cdot$ sur $G \divslash H$ est bien définie.
        \item $(G \divslash H, \cdot)$ est un groupe abélien, appelé \textit{groupe quotient} de $(G, \cdot)$ et $(H, \cdot)$.
    \end{itemize}

\medskip

\noindent\textbf{Démonstration :} Pour le premier point, il s'agit de montrer que le résultat ne dépends pas du choix de $g_1$ et $g_2$.
    Soit $c_1$ et $c_2$ deux éléments de $G \divslash H$. 
    Soit $g_1$ et $g_3$ deux éléments de $c_1$ et $g_2$ et $g_4$ deux éléments de $c_2$.
    On a alors $g_1 R g_3$ et $g_2 R g_4$.
    On peut donc choisir deux éléments $h_1$ et $h_2$ de $H$ tels que $g_1^{-1} \cdot g_3 = h_1$ et $g_2^{-1} \cdot g_4 = h_2$.
    On a alors : $(g_1 \cdot g_2)^{-1} \cdot (g_3 \cdot g_4) = (g_2^{-1} \cdot g_1^{-1}) \cdot (g_3 \cdot g_4) = (g_1^{-1} \cdot g_3) \cdot \cdot (g_2^{-1} \cdot g_4) = h_1 \cdot h_2$. 
    Puisque $(H, \cdot)$ est un groupe, on en déduit que $(g_1 \cdot g_2)^{-1} \cdot (g_3 \cdot g_4) \in H$, donc $(g_1 \cdot g_2) R (g_3 \cdot g_4)$, et donc $\overline{g_1 \cdot g_2} = \overline{g_3 \cdot g_4}$.

    Montrons maintenant qu'il s'agit d'un groupe abélien : 
    \begin{itemize}[nosep]
        \item Par définition, $\cdot$ est une loi de composition interne sur $G \divslash H$.
        \item Commutativité : Soit $c_1$ et $c_2$ deux éléments de $G \divslash H$.
            Soit $g_1$ un élément de $c_1$ et $g_2$ un élément de $c_2$. 
            On a : $c_1 \cdot c_2 = \overline{g_1 \cdot g_2}$ et $c_2 \cdot c_1 = \overline{g_2 \cdot g_1}$.
            Puisque $(G, \cdot)$ est abélien, $g_1 \cdot g_2 = g_2 \cdot g_1$, donc $c_1 \cdot c_2 = c_2 \cdot c_1$.
        \item Soit $e$ l'élément neutre de $(G, \cdot)$. 
            Soit $c$ un élément de $G \divslash H$ et $g$ un élément de $c$. 
            On a : $c \cdot \bar{e} = \overline{g \cdot e} = \bar{g} = c$.
            Par commutativité, cela implique également $\bar{e} \times c = c$.
            Donc, $\bar{e}$ est un élément neutre de $G \divslash H$ pour $\cdot$.
        \item Associativité : Soit $c_1$, $c_2$ et $c_3$ trois éléments de $G \divslash H$.
            Soit $g_1$ un élément de $c_1$, $g_2$ un élément de $c_2$ et $g_3$ un élément de $c_3$.
            On a : $c_1 \cdot (c_2 \cdot c_3) = c_1 \cdot \overline{g_2 \cdot g_3} = \overline{g_1 \cdot (g_2 \cdot g_3)} = \overline{(g_1 \cdot g_2) \cdot g_3} = \overline{g_1 \cdot g_2} \cdot c_3 = (c_1 \cdot c_2) \cdot c_3$.
        \item Soit $c$ un élément de $G \divslash H$ et $g$ un élément de $c$. 
            Notons $g^{-1}$ l'inverse de $g$ et $e$ l'élément neutre de $(G,\cdot)$. 
            Alors, $c \cdot \bar{g^{-1}} = \overline{g \cdot g^{-1}} = \bar{e}$.
            Par commutativité, cela implique également $\bar{g^{-1}} \times c = \bar{e}$.
            Puisque $\bar{e}$ est un élément neutre pour $\cdot$, on conclut que $\bar{g^{-1}}$ est un inverse de $c$.
    \end{itemize}

    \done

\subsubsection{Anneaux et corps}

\noindent\textbf{Définition (anneau) :} Soit $A$ un ensemble et $+$ et $\times$ deux lois de composition interne sur $A$. 
    Le triplet $(A, +, \times)$ est un \textit{anneau} si les propriétés suivantes sont satisfaites : 
    \begin{itemize}[nosep]
        \item $(A,+)$ est un groupe abélien,
        \item $(\times, \times)$ est un magma associatif,
        \item $\times$ est distributive sur $+$ : pour tous éléments $a$, $b$ et $c$ de $A$, on a $a \times (b + c) = (a \times b) + (a \times c)$ et $(a + b) \times c = (a \times c) + (b \times c)$.
    \end{itemize}

\medskip

\noindent\textbf{Lemme :} Soit $(A, +, \times)$ un anneau. 
On note $0$ l'élément neutre de $(A, +)$.
Pour tout élément $a$ de $A$, $0 \times a = a \times 0 = 0$.

\medskip

\noindent\textbf{Démonstration :} 
    Notons $b$ l'élément $0 \times a$. 
    On a : $b + b = (0 \times a) + (0 \times a) = (0 + 0) \times a = 0 \times a = b$.
    Soit $\bar{b}$ l'inverse de $b$ pour l'opération $+$ (qui existe car $(A, +)$ est un groupe).
    Alors, $b = 0 + b = (\bar{b} + b) + b = \bar{b} + (b + b) = \bar{b} + b = 0$. 

    Notons $d$ l'élément $a \times 0$. 
    On a : $d + d = (a \times 0) + (a \times 0) = a \times (0 + 0) = a \times 0 = d$.
    Soit $\bar{d}$ l'inverse de $d$ pour l'opération $+$ (qui existe car $(A, +)$ est un groupe).
    Alors, $d = d + 0 = d + (d + \bar{d}) = (d + d) + \bar{d} = d + \bar{d} = 0$. 

    \done

\medskip

\noindent\textbf{Lemme :} Soit $(A, +, \times)$ un anneau. 
On note $0$ l'élément neutre de $(A, +)$.
On suppose que tout élément de $A$ distinct de $0$ admet un inverse pour $\times$.
Soit $a$ et $b$ deux éléments de $A$ tels que $a \times b = 0$.
Alors $a = 0$ ou $b = 0$.

\medskip

\noindent\textbf{Démonstration :} Si $a = 0$, le résultat est vrai.
Sinon, $a$ admet un inverse, pour $\times$, noté $\bar{a}$.
Alors, $b = (\bar{a} \times a) \times b = \bar{a} \times (a \times b) = \bar{a} \times 0 = 0$.

\done

\medskip

\noindent\textbf{Remarque :} Cela est faux en général si on enlève l'hypothèse sur $A$. 
Considérons par exemple l'anneau $(\mathbb{Z}_N, +, \times)$ (voir section~\ref{sub:def_Z_nZ}) où $N$ est un entier naturel strictement supérieur à $1$ non premier. 
Soit $d$ un diviseur de $N$ distinct de $1$ et $N$, et $k$ l'entier naturel tel que $d \times k = N$. 
Alors $d > 1$, $d < N$, $k > 1$ et $k < N$, donc $\bar{d} \ neq \bar{0}$ et $\bar{k} \ neq \bar{0}$ mais $\bar{d} \times \bar{k} = \bar{N} = 0$.

\medskip

\noindent\textbf{Définition (anneau unifère) :} Soit $A$ un ensemble et $+$ et $\times$ deux lois de composition interne sur $A$ tels que $(A, +, \times)$ est un anneau. 
    L'anneau $(A, +, \times)$ est dit \textit{unifère}, ou \textit{unitaire}, si la loi de composition interne $\times$ admet un élément neutre.

\medskip

Soit $(A, +, \times)$ un anneau unifère, $0$ l'élément neutre de $+$, et $1$ l'élément neutre de $\times$. 
Si $0 = 1$, alors $A$ ne contient qu'un seul élément. 
En effet, pour tout élément $a$ de $A$, on a $a = 1 \times a$, donc $a = 0 \times a$, donc $a = 0$. 
Un anneau ne contennat qu'un seul élément est dit \textit{nul}. 

\medskip

\noindent\textbf{Définition (morphisme d'anneaux) :} Soit $(A, +, \times)$ et $(B, \oplus, \otimes)$ deux anneaux unifères.
    Notons $1_A$ l'élément neutre de $A$ pour $\times$ et $1_B$ celui de $B$ pour $\otimes$.
    Soit $f$ une fonction de $A$ vers $B$. 
    Alors, $f$ est un \emph{morphisme d'anneaux} de $(A, +, \times)$ vers $(B, \oplus, \otimes)$ si (et seulement si)
    \begin{itemize}[nosep]
        \item pour tous éléments $a$ et $b$ de $A$, $f(a + b) = f(a) \oplus f(b)$,
        \item pour tous éléments $a$ et $b$ de $A$, $f(a \times b) = f(a) \otimes f(b)$,
        \item $f(1_A) = 1_B$.
    \end{itemize}

\medskip

\noindent\textbf{Remarque :} Soit $(A, +, \times)$ et $(B, \oplus, \otimes)$ deux anneaux unifères.
    Notons $0_A$ l'élément neutre de $A$ pour $+$ et $0_B$ celui de $B$ pour $\oplus$.
    Soit $f$ un morphisme d'anneaux de $(A, +, \times)$ vers $(B, \oplus, \otimes)$.
    Alors, $f$ est un morphisme de groupes de $(A, +)$ vers $(B, \oplus)$.
    Donc, 
    \begin{itemize}[nosep]
        \item $f(0_A) = 0_B$, 
        \item soit $a$ et $b$ deux éléments de $a$ tels que $b$ est l'inverse de $a$ pour $+$, alors $f(b)$ es l'inverse de $f(a)$ pour $\oplus$.
    \end{itemize}

\medskip

\noindent\textbf{Définition (anneau commutatif) :} Soit $A$ un ensemble et $+$ et $\times$ deux lois de composition interne sur $A$ tels que $(A, +, \times)$ est un anneau. 
    L'anneau $(A, +, \times)$ est dit \textit{commutatif} si la loi de composition interne $\times$ est commutative.

\medskip

\noindent\textbf{Définition (corps) :} Soit $K$ un ensemble et $+$ et $\times$ deux lois de composition interne sur $K$. 
    Le triplet $(K, +, \times)$ est un \textit{corps} si les propriétés suivantes sont satisfaites : 
    \begin{itemize}[nosep]
        \item $(A,+,\times)$ est un anneau commutatif, unifère, et non nul,
        \item en notant $0$ l'élément neutre de $+$ et $1$ celui de $\times$, tout élément de $K$ distinct de $0$ admet un inverse pour $\times$, c'est-à-dire : $\forall k \in K \, (k \neq 0) \Rightarrow (\exists l \in K \, k \times l = l \times k = 1)$.
    \end{itemize}

\medskip

\noindent\textbf{Remarque :} Soit $K$ un ensemble et $+$ et $\times$ deux lois de composition interne sur $K$ tels que $(K, +, \times)$ est un corps. 
    Soit $0$ l'élément neutre de $+$.
    Notons $K^*$ l'ensemble $K \setminus \lbrace 0 \rbrace$.
    Si $a$ et $b$ sont deux éléments de $K^*$, alors $a \times b$ est encore un élément de $K^*$. 
    (En effet, il ne peut être égal à $0$ : puisque $a$ est non nul, il adet un inverse $c$ pout $\times$, et on a $c \times (a \times b) = b$, donc $c \times (a \times b) \neq 0$, donc $a \times b \neq 0$.)
    Donc, $(K^*, \times)$ est un magma.
    En outre, 
    \begin{itemize}[nosep]
        \item La loi de composition interne $\times$ est associative.
        \item La loi de composition interne $\times$ est commutative.
        \item La loi de composition interne $\times$ admet un élément neutre $1$ (qui est distinct de $0$ puisque l'anneau $(K, +, \times)$ est non nul, et donc apppartient à $K^*$).
        \item Soit $a$ un élément de $K^*$. 
            Puisque $a \in K$ et $a \neq 0$, $a$ admet un inverse $b$ pour $\times$ dans $K$.
            Puisque $a \times b = 1$ et $1 \neq 0$, $b \neq 0$, donc $b \in K^*$.
            Donc, $a$ admet un inverse pour $\times$ dans $K^*$.
    \end{itemize}
    Ainsi, $(K^*, \times_*)$, où $\times_*$ désigne la restriction de la loi de composition interne $\times$ à $K^*$, est un groupe abélien.

\medskip

\noindent\textbf{Lemme :} Soit $K$ un ensemble et $+$ et $\times$ deux lois de composition interne sur $K$ tels que $(K, +, \times)$ est un corps. 
    Soit $0$ l'élément neutre de $+$.
    Soit $a$ et $b$ deux éléments de $K$ tels que $a \neq 0$ et $b \neq 0$.
    Alors, $a \times b \neq 0$.

\medskip

\noindent\textbf{Démonstration :} Soit $c$ l'inverse de $a$ pour $\times$.
    On a : $c \times (a \times b) = (c \times a) \times b = b$.
    Donc, : $c \times (a \times b) \neq 0$.
    Donc, : $a \times b \neq 0$.

    \done

\medskip

\noindent\textbf{Notation :} Soit $(A, +, \times)$ un anneau, $0_A$ son élément neutre pour $+$ et, si l'anneau est unifère $1_A$ son élément neutre pour $\times$.
    Si $n$ et $m$ sont deux entiers tels que $n \geq 0$ et $a_n$, $a_{n+1}$, ..., $a_m$ des éléments de $A$ (aucun élément si $m < n$), alors 
    \begin{itemize}[nosep]
        \item On pourra noter $\sum_{i=n}^m a_i$ (où $i$ est une variable) l'élément $a_0 + a_1 + \cdots + a_m$ (ou $0_A$ si $m < n$).
        \item Si $(A, +, \times est unifère)$, on pourra noter $\prod_{i=n}^m a_i$ (où $i$ est une variable) l'élément $a_0 \times a_1 \times \cdots \times a_m$ (ou $1_A$ si $m < n$).
    \end{itemize}
    On peut définir ces notations plus rigoureusement de manière suivante. 
    Soit $a$ une suite d'éléments de $A$.
    Soit $n$ un entier naturel.
    Pour tout entier strictement négatif $m$, on définit $\sum_{i=n}^{n+m} a_i$ comme égal à $0_A$ et $\prod_{i=n}^{n+m} a_i$ comme égal à $1_A$.
    On définit $\sum_{i=n}^n a_i$ et $\prod_{i=n}^n a_i$ comme égaux à $a_n$.
    Enfin, pour tout entier naturel $m$, on définit $\sum_{i=n}^{n+(m+1)} a_i$ comme égal à $\left( \sum_{i=n}^{n+m} a_i \right) + a_{n+(m+1)}$ et $\prod_{i=n}^{n+(m+1)} a_i$ comme égal à $\left( \prod_{i=n}^{n+m} a_i \right) \times a_{n+(m+1)}$.

\medskip

\noindent\textbf{Remarque :} On peut étendre les définitions ci-dessus au cas d'un $n$-uplet $(b_0, b_1, \dots b_{n-1})$ d'éléments de $A$, où $n$ est un entier naturel strictement supérieur à $0$, en définissant la suite $a$ d'éléments de $A$ par : 
\begin{itemize}[nosep]
    \item pour tout élément $m$ de $[\![0, n-1]\!]$, $a_m = b_m$, 
    \item pour $\sum$, pour tout entier $m$ supérieur ou égal à $n$, $a_m = 0_A$, 
    \item si $(A? +, \times)$ est unifère, pour $\prod$, pour tout entier $m$ supérieur ou égal à $n$, $a_m = 1_A$.
\end{itemize}
On pose alors, pour tous entiers naturels $k$ et $l$ strictement : 
\begin{equation*}
    \sum_{i=k}^l b_i = \sum_{i=k}^l a_i
\end{equation*}
et, si $(A, +, \times)$ est unifère
\begin{equation*}
    \prod_{i=k}^l b_i = \prod_{i=k}^l a_i.
\end{equation*}

\medskip

\noindent\textbf{Lemme :} Soit $(A, +, \times)$ un anneau, $n$ et $m$ deux entiers naturels, et $a$ une suite d'éléments de $A$ et $b$ un élément de $A$. 
    Alors,
    \begin{equation*}
        \sum_{i=n}^m (a_i \times b) = \left( \sum_{i=n}^m a_i \right) \times b 
    \end{equation*}
    et
    \begin{equation*}
        \sum_{i=n}^m (b \times a_i) = b \times \left( \sum_{i=n}^m a_i \right) .
    \end{equation*}

\medskip

\noindent\textbf{Démonstration :} 
    Notons $0_A$ l'élément neutre de $(A, +)$.
    Si $m < n$, on a $\sum_{i=n}^m (a_i \times b) = 0_A$, $\left( \sum_{i=n}^m a_i \right) \times b = 0_A \times b = 0_A$, $\sum_{i=n}^m (b \times a_i) = 0_A$ et $b \times \left( \sum_{i=n}^m a_i \right) = b \times 0_A = 0_A$, donc les deux formules sont vraies.
    Montrons par récurrence qur $m$ que cela reste le cas pour $m \geq n$.

    Pour $m = n$, $\sum_{i=n}^m (a_i \times b)$ et $\left( \sum_{i=n}^m a_i \right) \times b$ sont tous deux égaux à $a_n \times b$, donc égaux, et $\sum_{i=n}^m (b \times a_i)$ et $b \times \left( \sum_{i=n}^m a_i \right)$ sont tous deux égaux à $b \times a_n$, donc égaux.

    Soit $m$ un entier tel que les deux formules du lemme sont vraies.
    Alors, 
    \begin{equation*}
        \sum_{i=n}^{m+1} (a_i \times b) 
        = \left( \sum_{i=n}^m (a_i \times b) \right) + a_{m+1} \times b
        = \left( \sum_{i=n}^m a_i \right) \times b + a_{m+1} \times b
        = \left( \left( \sum_{i=n}^m a_i \right) + a_{m+1} \right) \times b 
        = \left( \sum_{i=n}^{m+1} a_i \right) \times b 
    \end{equation*}
    et
    \begin{equation*}
        \sum_{i=n}^{m+1} (b \times a_i) 
        = \left( \sum_{i=n}^m (b \times a_i) \right) + b \times a_{m+1}
        = b \times \left( \sum_{i=n}^m a_i \right) + b \times a_{m+1}
        = b \times \left( \left( \sum_{i=n}^m a_i \right) + a_{m+1} \right)
        = b \times \left( \sum_{i=n}^{m+1} a_i \right) .
    \end{equation*}
    Ce sont les deux formulles du lemme avec $m$ remplacé par $m+1$.
    Par récurrence, ce résultat est vrai pour tout entier $m$ supérieur ou égal à $n$.

    \done

\medskip

\noindent\textbf{Lemme :} Soit $(A, +, \times)$ un anneau, $n$, $m$ et $k$ trois entiers naturels et $a$ une suite d'éléments de $A$. 
    On suppose que $n \leq k \leq m$
    Alors,
    \begin{equation*}
        \left( \sum_{i=n}^k a_i \right) + \left( \sum_{i=k+1}^m a_i \right) = \sum_{i=n}^m a_i
    \end{equation*}
    et, si $(A, +, \times)$ est unifère,
    \begin{equation*}
        \left( \prod_{i=n}^k a_i \right) \times \left( \prod_{i=k+1}^m a_i \right) = \prod_{i=n}^m a_i.
    \end{equation*}

\medskip

\noindent\textbf{Démonstration :}  
    On procède par récurrence sur $m$.
    Notons $0_A$ l'élément neutres de $(A, +)$ et, si l'anneau est unifère, $1_A$ celui de $(A, \times)$.
    Pour $m = n$, on a nécessairement $k = n$, donc $k+1 > m$ et 
    \begin{equation*}
        \left( \sum_{i=n}^k a_i \right) + \left( \sum_{i=k+1}^m a_i \right) 
        = \left( \sum_{i=n}^n a_i \right) + 0_A
        = \sum_{i=n}^n a_i 
        = \sum_{i=n}^m a_i .
    \end{equation*}
    De même, si $(A, +, \times)$ est unifère, 
    \begin{equation*}
        \left( \prod_{i=n}^k a_i \right) \times \left( \prod_{i=k+1}^m a_i \right) 
        = \left( \prod_{i=n}^n a_i \right) \times 1_A
        = \prod_{i=n}^n a_i 
        = \prod_{i=n}^m a_i .
    \end{equation*}

    Soit $m$ un entier naturel supérieur ou égal à $n$ tel que le résultat soit vrai.
    Soit $k$ un entier tel que $n \leq k \leq m+1$. 
    Alors, $k \leq m$ ou $ = m+1$.
    Si $k = m+1$, alors $k+1 > m+1$, donc 
    \begin{equation*}
        \left( \sum_{i=n}^k a_i \right) + \left( \sum_{i=k+1}^{m+1} a_i \right)
        = \left( \sum_{i=n}^{m+1} a_i \right) + 0_A
        = \sum_{i=n}^{m+1} a_i 
    \end{equation*}
    et, si $(A, +, \times)$ est unifère, 
    \begin{equation*}
        \left( \prod_{i=n}^k a_i \right) \times \left( \prod_{i=k+1}^{m+1} a_i \right)
        = \left( \prod_{i=n}^{m+1} a_i \right) \times 1_A
        = \prod_{i=n}^{m+1} a_i .
    \end{equation*}
    Sinon, on a
    \begin{equation*}
        \left( \sum_{i=n}^k a_i \right) + \left( \sum_{i=k+1}^{m+1} a_i \right)
        = \left( \sum_{i=n}^k a_i \right) + \left( \sum_{i=k+1}^m a_i \right) + a_{m+1}
        = \left( \sum_{i=n}^k a_m \right) + a_{m+1}
        = \sum_{i=n}^{m+1} a_i
    \end{equation*}
    et, si $(A, +, \times)$ est unifère, 
    \begin{equation*}
        \left( \prod_{i=n}^k a_i \right) \times \left( \prod_{i=k+1}^{m+1} a_i \right)
        = \left( \prod_{i=n}^k a_i \right) \times \left( \prod_{i=k+1}^m a_i \right) + a_{m+1}
        = \left( \prod_{i=n}^k a_m \right) \times a_{m+1}
        = \prod_{i=n}^{m+1} a_i .
    \end{equation*}
    Le résultat attendu est donc vrai au rang $m+1$.
    Par récurrence, il l'est pour tout entier $m$ supérieur ou égal à $n$.

    \done

\noindent\textbf{Lemme :} Soit $(A, +, \times)$ un anneau, $n$ et $m$ deux entiers naturels, et $a$ et $b$ deux suites d'éléments de $A$. 
    Alors,
    \begin{equation*}
        \sum_{i=n}^m (a_i + b_i) = \left( \sum_{i=n}^m a_i \right) + \left( \sum_{i=n}^m b_i \right) .
    \end{equation*}

\medskip

\noindent\textbf{Démonstration :} 
    Notons $0_A$ l'élément neutre de $(A, +)$.
    Si $m < n$, alors $\sum_{i=n}^m (a_i + b_i) = \left( \sum_{i=n}^m a_i \right) = \left( \sum_{i=n}^m b_i \right) = 0_A$, donc le résultat est évident.
    Montrons par récurrence sur $m$ que le lemme reste vrai pour $m \geq n$.

    Si $m = n$, alors 
    \begin{equation*}
        \left( \sum_{i=n}^m a_i \right) + \left( \sum_{i=n}^m b_i \right)
        = a_n + b_n
        = \sum_{i=n}^m (a_i + b_i).
    \end{equation*}

    Soit $n$ et $m$ deux entiers naturels tels que l'énoncé du lemme est vrai et $m \geq n$.
    Alors, 
    \begin{equation*}
        \sum_{i=n}^{m+1} (a_i + b_i) 
        = \left( \sum_{i=n}^m (a_i + b_i) \right) + (a_{m+1} + b_{m+1})
        = \left( \left( \sum_{i=n}^m a_i \right) + \left( \sum_{i=n}^m b_i \right) \right) + (a_{m+1} + b_{m+1})
    \end{equation*}
    \begin{equation*}
        \sum_{i=n}^{m+1} (a_i + b_i) 
        = \left( \left( \sum_{i=n}^m a_i \right) + a_{m+1} \right) + \left( \left( \sum_{i=n}^m b_i \right) + b_{m+1} \right)
        = \left( \sum_{i=n}^m a_i \right) + \left( \sum_{i=n}^m b_i \right) .
    \end{equation*}
    L'énoncé du lemme reste vrai en remplaçant $m$ par $m+1$.
    Par récurrence, il est donc vrai pour tout entier $m$ supérieur ou égal à $n$. 

    \done

\medskip

\noindent\textbf{Lemme :} Soit $(A, +, \times)$ un anneau commutatif, $n$ et $m$ deux entiers naturels, et $a$ et $b$ deux suites d'éléments de $A$. 
    Alors,
    \begin{equation*}
        \prod_{i=n}^m (a_i \times b_i) = \left( \prod_{i=n}^m a_i \right) \times \left( \prod_{i=n}^m b_i \right) .
    \end{equation*}

\medskip

\noindent\textbf{Démonstration :} Identique à la précédente en remplaçant $\sum$ par $\prod$ et $+$ par $\times$.

\medskip

\noindent\textbf{Lemme :} Soit $(A, +, \times)$ un anneau, $n$, $m$, $k$ et $l$ quatre entiers naturels, et $a$ et $b$ deux suites d'éléments de $A$. 
    Alors,
    \begin{equation*}
        \left( \sum_{i=n}^m a_i \right) \times \left( \sum_{j=k}^l b_j \right) = \sum_{i=n}^m \sum_{j=k}^l a_i \times b_j .
    \end{equation*}

\medskip

\noindent\textbf{Démonstration :} 
    Notons $0_A$ l'élément neutre de $(A, +)$.
    Si $m < n$, on a $\sum_{i=n}^m a_i = 0_A$ et $\sum_{i=n}^m \sum_{j=k}^l a_i \times b_j = 0_A$, donc les deux membres de l'égalité sont égaux à $0_A$, et donc égaux.
    Montrons par récurrence que le résultat reste vrai pour $m \geq n$. 

    Si $m = n$, on a
    \begin{equation*}
        \left( \sum_{i=n}^m a_i \right) \times \left( \sum_{j=k}^l b_j \right) = a_n \times \sum_{j=k}^l b_j
    \end{equation*}
    et
    \begin{equation*}
        \sum_{i=n}^m \sum_{j=k}^l a_i \times b_j = \sum_{j=k}^l a_n \times b_j = a_n \times \sum_{j=k}^l b_j,
    \end{equation*}
    donc le résultat est vrai.

    Supposons maontenant que, pour un certain netier $m$,
    \begin{equation*}
        \left( \sum_{i=n}^m a_i \right) \times \left( \sum_{j=k}^l b_j \right) = \sum_{i=n}^m \sum_{j=k}^l a_i \times b_j .
    \end{equation*}
    Alors, 
    \begin{equation*}
        \left( \sum_{i=n}^{m+1} a_i \right) \times \left( \sum_{j=k}^l b_j \right)
        = \left( \left( \sum_{i=n}^m a_i \right) + a_{m+1} \right) \times \left( \sum_{j=k}^l b_j \right)
        = \left( \sum_{i=n}^m a_i \right) \times \left( \sum_{j=k}^l b_j \right) + a_{m+1} \times \left( \sum_{j=k}^l b_j \right)
    \end{equation*}
    \begin{equation*}
        \left( \sum_{i=n}^{m+1} a_i \right) \times \left( \sum_{j=k}^l b_j \right)
        = \left( \sum_{i=n}^m \sum_{j=k}^l a_i \times b_j \right) + \left( \sum_{j=k}^l a_{m+1} \times b_j \right)
        = \left( \sum_{i=n}^{m+1} \sum_{j=k}^l a_i \times b_j \right) .
    \end{equation*}
    Le résultat reste vrai au rang $m+1$.
    Par récurrrence, il est vrai pour tout entier $m$ tel que $m \geq n$.

    \done

\medskip

\noindent\textbf{Lemme :} Soit $(A, +, \times)$ un anneau, $n$ et $m$ deux entiers naturels tels que $m \geq n$, $a$ une suite d'éléments de $A$ et $f$ une bijection de $[\![n, m]\!]$ vers lui-même. 
    Alors,
    \begin{equation*}
        \sum_{i=n}^m a_{f(i)} = \sum_{i=n}^m a_i.
    \end{equation*}

\medskip

\noindent\textbf{Démonstration :} 
    On procède par récurrence sur $m$. 
    Pour $m = n$, on a $[\![n, m]\!] = \lbrace n \rbrace$, donc la seule bijection (en fait, la seule fonction) de cet ensemble vers lui-même est la fonction identité $\lbrace (n, n) \rbrace$. 
    En notant $f$ cette fonction, on a 
    \begin{equation*}
        \sum_{i=n}^m a_{f(i)} = a_n = \sum_{i=n}^m a_i.
    \end{equation*}

    Soit $m$ un entier naturel supérieur ou égal à $n$ pour lequel le résultat est vrai.
    Soit $f$ une bijection de $[\![n, m+1]\!]$ vers lui-même.
    Soit $k$ l'unique antécédent de $m+1$ par $f$.
    Supposons d'abord $k \leq m$.
    On a alors : 
    \begin{equation*}
        \sum_{i=n}^{m+1} a_{f(i)}
        = \sum_{i=n}^k a_{f(i)} + \sum_{i=k+1}^{m+1} a_{f(i)}
        = \sum_{i=n}^k a_{f(i)} + \sum_{i=k+1}^m a_{f(i)} + a_{f(m+1)}.
    \end{equation*}
    Puisque le groupe $(A, +)$ est abélien, cela donne : 
    \begin{equation*}
        \sum_{i=n}^{m+1} a_{f(i)}
        = \sum_{i=n}^{k-1} a_{f(i)} + a_{f(m+1)} + \sum_{i=k+1}^m a_{f(i)} + a_{m+1} .
    \end{equation*}
    Définissons la fonction $g$ de $[\![n, m]\!]$ vers lui-même de la manière suivante : pour tout élément $x$ de $[\![n, m]\!]$, $g(x) = f(x)$ si $x \neq k$ et $g(k) = f(m+1)$. 
    On a alors : 
    \begin{equation*}
        \sum_{i=n}^{m+1} a_{f(i)}
        = \sum_{i=n}^{k-1} a_{g(i)} + a_{g(k)} + \sum_{i=k+1}^m a_{g(i)} + a_{m+1} 
        = \sum_{i=n}^k a_{g(i)} + \sum_{i=k+1}^m a_{g(i)} + a_{m+1} 
        = \sum_{i=n}^m a_{g(i)} + a_{m+1} .
    \end{equation*}
    En outre, $g$ est bijective. 
    En effet, soit $y$ un élément de $[\![n, m]\!]$, 
    \begin{itemize}[nosep]
        \item Si $y \neq f(m+1)$, son unique antécédent par $g$ est $f^{-1}(y)$ (qui est bien dans $[\![n, m]\!]$ puisqu'il ne peut être égal à $m+1$).
        \item Si $y = f(m+1)$, $k$ est un antécédent de $y$, et tout autre élément $x$ de $[\![n, m]\!]$ ne peut être un antécédent de $y$ (sans quoi on aurait $f(x) = f(m+1)$), donc $y$ admet à nouveau un unique antécédent.
    \end{itemize}
    Donc, 
    \begin{equation*}
        \sum_{i=n}^{m+1} a_{f(i)}
        = \sum_{i=n}^m a_{i} + a_{m+1} 
        = \sum_{i=n}^{m+1} a_{i} .
    \end{equation*}

    Sinon, $k = m+1$. 
    Définissons la fonction $g$ de $[\![n, m]\!]$ vers lui-même par $g(x) = f(x)$ pour tout élément $x$ de cet ensemble. 
    Cette fonction est bien définie puisque, pour tout tel élément $x$, $x \neq m+1$ donc $f(x) \leq m$, et est bojective puisque, pour tout élément $y$ de $[\![n, m]\!]$, il existe un unique antécédent $x$ de $y$ par $g$—il s'agit de son unique antécédent par $f$ (qui est bien dans cet ensemble puisqu'il ne peut être égal à $m+1$, puisque $f(m+1) = m+1$). 
    On a alors : 
    \begin{equation*}
        \sum_{i=n}^{m+1} a_{f(i)}
        = \sum_{i=n}^m a_{f(i)} + a_{f(m+1)}
        = \sum_{i=n}^m a_{f(i)} + a_{m+1}
        = \sum_{i=n}^m a_{i} + a_{m+1}
        = \sum_{i=n}^{m+1} a_{i}.
    \end{equation*}

    Dans les deux cas, le résultat atttendu est vrai au rang $m+1$.
    Par récurrence, il est vrai pour tout entier $m$ supérieur ou égal à $n$.

    \done

\medskip

\noindent\textbf{Lemme :} Soit $(A, +, \times)$ un anneau unifère et commutatif, $n$ et $m$ deux entiers naturels tels que $m \geq n$, $a$ une suite d'éléments de $A$ et $f$ une bijection de $[\![n, m]\!]$ vers lui-même. 
    Alors,
    \begin{equation*}
        \prod_{i=n}^m a_{f(i)} = \prod_{i=n}^m a_i.
    \end{equation*}

\medskip

\noindent\textbf{Démonstration :} Même que ci-dessus en remplaçant $\sum$ par $\prod$ et $+$ par $\times$.
