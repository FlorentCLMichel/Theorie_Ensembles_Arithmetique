\subsection{Nombres premiers}

\subsubsection{Définition}
\label{subsub:defNombresPremiers}

Un entier naturel $p$ est dit \textit{premier} s'il admet exactement deux entiers naturels distincts pour diviseurs: $1$ et lui-même.
On note $\mathbb{P}$ l'ensemble des nombres premiers.

Les premiers nombres premiers sont $2$, $3$, $5$, $7$, $11$, $13$, $17$, $19$ et $23$.

Le seul nombre premier pair est $2$. 
Les deux seuls nombres premiers séparés de $1$ sont $2$ et $3$. 
(En effet, les nombres premiers strictement supérieurs à $2$ sont tous impairs, et donc séparés d'au moins $2$.)\footnote{Soit $p$ et $q$ deux nombres premiers strictement supérieurs à $2$. Si $p-q=1$, $p=q+1$, alors $p$ est pair ou $q$ est pair, ce qui est impossible puisque aucun d'eux n'est divisible par $2$.}
Deux nombres premiers $p$ et $q$ sont dits \textit{jumeaux} si $p-q=2$ ou $q-p=2$. 

\bigskip

\noindent\textbf{Lemme :} Tout entier naturel strictement supérieur à $1$ est divisible par au moins un nombre premier.

\medskip

\noindent\textbf{Démonstration :} On procède par récurrence forte. 
Montrons que, pour tout entier naturel $n$, $n \leq 1$ ou $n$ est divisible par au moins un nombre premier.
Le lemme est évidemment vrai pour les nombres $0$ et $1$ (qui sont tous deux inférieurs ou égaux à $1$) et pour $2$, qui est lui-même premier (et, comme tout entier naturel, divisible par lui-même). 

Soit $n$ un entier naturel strictement supérieur à $2$ et supposons que tout entier compris entre $2$ et $n-1$ soit divisible par au moins un nombre premier. 

Si $n$ est divisible par un entier $l$ compris entre $2$ et $n-1$ (inclus), alors il existe un nombre permier $p$ divisant $l$. 
Il existe alors deux entiers naturels $k$ et $q$ tels que $n = k l$ et $l = q p$, d'où $n = (k q) p$. 
Donc, $p$ divise $n$. 

Si $n$ n'est divisible par aucun entier compris entre $2$ et $n-1$, alors il n'est divisible par aucun entier strictement supérieur à $1$ autre que lui-même. 
En effet, soit $m$ un tel entier, soit $m < n$, et donc $m$ ne divise pas $n$, soit $m > n$, auquel cas il ne peut être un diviseur de $n$.
Donc, $n$ n'est divisible que par $1$ et par lui-même (qui sont bien distincts puisque $n > 1$). 
Par définition, $n$ est donc un nombre premier. 
Puisque $n$ est divisible par lui-même, il est divisible par un nombre premier. 

Dans les deux cas, $n$ est divisible par un nombre premier. 
Par récurrence forte, on en déduit que le résultat est vrai pour tout entier naturel, et donc que tout entier naturel strictement supérieur à $1$ est divisible par au moins un nombre premier.

\done

\bigskip

\noindent\textbf{Lemme :} Il existe une infinité de nombres premiers.

\medskip

\noindent\textbf{Démonstration :} On procède par l'absurde. 
Supposons qu'il n'existe qu'un nombre fini $N \in \mathbb{N}^*$ de nombres premiers, notés $p_1$, $p_2$, ..., $p_N$. 
Soit $q \equiv 1 + \prod_{i=1}^N p_i$. 
On a $q > 1$.
Pour tout $i \in [\![1, N]\!]$, on a $q = 1 \, [p_i]$, donc le reste de la division euclidienne de $q$ par $p_i$ est $1$, et $q$ n'est pas divisible par $p_i$.  
Donc, $q$ n'est divisible par aucun nombre premier, ce qui contredit le lemme précédent. 
On en conclut que l'hypothèse est fausse. 

\done

\bigskip

\noindent\textbf{Lemme :} Soit $p$, $q$ et $r$ trois nombres premiers tels que $r > q > p$ et $r-q = q-p$. Alors $p=3$ ou $q-p$ est divisible par $3$.

\medskip

\noindent\textbf{Démonstration :} Notons $a = q-p$. 
Soit $b$ le reste de la division euclidienne de $p$ par $3$ et $c$ celui de la division euclidienne de $a$ par $3$. 
Si $c = 0$, le résultat est immédiat.
Si $b = 0$, p est divisible par $3$. 
Puisque $p$ est premier, on en déduit que $p=3$ et le résultat est établi. 
Montrons que les autres cas, \textit{i.e.} ceux où $b$ et $c$ sont chacun égaux à $1$ ou $2$, sont impossibles. 

Si $b = c = 1$, alors $r = (1 + 1 + 1) \, [3] = 3 \, [3] = 0 \, [3]$. 
Donc, $r$ est divisible par $3$. 
Puisque $r$ est premier, on en déduit que $r=3$, ce qui est impossible car $p$ et $q$ sont deux nombres premiers distincts strictement inférieurs à $r$ et qu'il n'existe qu'un seul nombre premier ($2$) strictement inférieur à $3$. 

De même, si $b = c = 2$, alors $r = (2 + 2 + 2) \, [3] = 6 \, [3] = 0 \, [3]$, ce qui est impossible comme nous venons de le voir.

Si $b = 1$ et $c = 2$ ou si $b = 2$ et $c = 1$, alors $q = (1 + 2) \, [3] = 3 \, [3] = 0 \, [3]$. 
Donc, $q$ est divisible par $3$.
Puisque $q$ est premier, on en déduit que $q=3$. 
Puisque $p$ est un nombre premier strictement inférieur à $q$, il ne peut qu'être égal à $2$. 
Donc, $a = 3-2 = 1$, d'où $r = q + 1 = 4$. 
Mais $4$ n'est pas un nombre premier (puisque $4 = 2 \times 2$ est divisible par $2$), donc ce cas est impossible.

\done

\bigskip

Des fonctions Haskell détarminant si un entier naturel est premier et calculant les premier nombres premiers sont données en appendice~\ref{app:Haskell_primalité}.

\subsubsection{Théorème de Bachet-Bézout}

\bigskip

\noindent\textbf{Théorème (Bachet-Bézout) :} 
Soit $a$ et $b$ deux entiers naturels et $c$ leur plus grand diviseur commun. 
Il existe deux entiers relatifs $p$ et $q$ tels que $p a + q b = c$. 

\medskip

\noindent\textbf{Démonstration :} 
Soit $E = \lbrace p a + q b \vert (p,q) \in \mathbb{Z}^2 \, \wedge \, p a + q b > 0 \rbrace$. 
$E$ est un sous-ensemble non vide (il contient au moins $a$, obtenu pour $p=1$ et $q=0$) de $\mathbb{N}$, donc il admet un plus petit élément $r$. 
En outre, par définition de $E$, $r > 0$. 
Montrons qu'il s'agit du plus grand diviseur commun de $a$ et $b$, ce qui prouvera le théorème. 
Pour ce faire, nous procédons en deux temps. 
Nous montrons d'abord que $r$ divise $a$ et $b$, puis qu'il n'existe ausun plus grand diviseur commun à ces deux nombres.

Notons $s$ le reste de la division euclidienne de $a$ par $r$. 
Il existe un entier naturel $k$ tel que $a = k r + s$, et $0 \leq s < r$. 
Soit $p$ et $q$ deux entiers relatifs tels que $p a + q b = r$. 
On a: $p a + q b = k a + s$, et donc $(p-k) a + q b = s$. 
Si $s > 0$, $s$ serait donc un élément de $E$. 
Or, cela est impossible car $s < r$. 
On en déduit que $s = 0$, et donc que $r$ divise $a$. 
On montre de même, par le même argument et en échangeant les rôles de $a$ et $b$, que $r$ divise $b$. 
Ainsi, $r$ est un diviseur commun de $a$ et $b$. 

Supposons maintenant par l'absurde qu'il existe un autre diviseur commun à $a$ et $b$, noté $t$, tel que $t > r$. 
On peut choisir deux entiers naturels $u$ et $v$ tels que $a = u t$ et $b = v t$. 
Soit $p$ et $q$ deux entiers relatifs tels que $p a + q b = r$. 
On a: $r = p u t + q v t$, d'où $r = (p u + q v) t$. 
Puisque $r$ et $t$ sont tous deux strictement positifs, $p u + q v$ doit l'être aussi. 
Mais un entier strictement positif est supérieur ou égal à $1$, et donc $r \geq t$, en contradiction avec l'hypothèse. 
On en déduit qu'il n'existe aucun diviseur commun à $a$ et $b$ strictement supérieur à $r$.

Ainsi, $r$, qui est un élément de $E$ et peut donc s'écrire $q a + p b$ avec $(p, q) \in \mathbb{Z}^2$, est le plus grand diviseur commun de $a$ et $b$.

\done

\bigskip
\noindent\textbf{Lemme :} 
Soit $p$ un nombre premier et $a$ et $b$ deux entiers naturels. 
Si $p$ divise $a b$, alors $p$ divise $a$ ou $p$ divise $b$.

\medskip

\noindent\textbf{Démonstration :} 
Si $p$ divise $a$, le résultat est vrai. 
Supposons que $p$ ne divise pas $a$. 
Puisque les seuls diviseurs de $p$ sont $1$ et lui-même, et car $p$ ne divise pas $a$, $1$ est le seul diviseur commun à $p$ et $a$, et donc leur plus grand diviseur commun. 
D'après le théorème précédent, on peut donc choisir deux entiers relatifs $q$ et $r$ tels que $q p + r a = 1$. 
Multiplions cette équation par $b$. 
Il vient : $q p b + r a b = b$. 
Puisque $p$ divise $a b$, on peut choisir un entier naturel $k$ tel que $a b = k p$. 
Donc, $q p b + k p = b$. 
Cette équation peut se récrire : $(q b + k) p = b$. 
Puisque $p$ et $b$ sont tous deux strictement positifs, $p b + k$ doit l'être aussi, et $p$ divise donc $b$.

\done

\bigskip
\noindent\textbf{Corrolaire :} 
Soit $N \in \mathbb{N}$ tel que $N > 1$, $p$ un nombre premier, et $a_1$, $a_2$, ..., $a_N$ des entiers naturels. 
Si $p$ divise $a_1 a_2 \cdots a_N$, alors il existe $i \in [\![1,N]\!]$ tel que $p$ divise $a_i$.

\medskip

\noindent\textbf{Démonstration :} 
On procède par récurrence sur $N$. 
Pour $N = 2$, il s'agit du lemme précédent. 
Supposons maintenant $N > 2$ et l'assertion vraie au rang $N-1$. 
Puisque $p$ divise $a_1 a_2 \cdots a_N$, égal à $(a_1 a_2 \cdots a_{N-1}) a_N$, d'après le lemme précédent, il divise $a_1 a_2 \cdots a_{N-1}$ ou $a_N$. 
Dans le second cas, l'assertion est vraie. 
Dans le premiers, par hypothèse de récurrence, il existe $i \in [\![1,N-1]\!]$, et donc $i \in [\![1,N]\!]$, tel que $p$ divise $a_i$ et l'assertion est vraie. 
Par récurrence, elle est donc vraie pour tout $N \in \mathbb{N}$ tel que $N > 1$. 

\done

\medskip

\noindent\textbf{Corrolaire :} 
Soit $N \in \mathbb{N}$ tel que $N > 1$, $p$ un nombre premier, et $a$ un entier naturel.
Si $p$ divise $a^N$, alors $p$ divise $a$.

\medskip

\noindent\textbf{Démonstration :} 
Application du corrolaire précédent avec $a_1 = a_2 = \cdots = a_N = a$.

\done

\medskip

\noindent\textbf{Corrolaire :} Soit $a$, $b$, et $c$ trois entiers deux-à-deux premiers entre eux. Alors $ab$ et $c$ sont aussi premiers entre eux.

\medskip 

\noindent\textbf{Démonstration :} Avec les mêmes notations, soit $d$ un diviseur de $ab$ et $c$. D'après le théorème de Bachet-Bézout, $d$ divise $a$ ou $b$. 
Puisque $a$ et $c$ sont permiers entre eux et que $b$ et $c$ sont permiers entre eux, $d$ ne peut donc qu'être égal à $1$. 
Réciproqiement, $1$ divise tout entier et donc $ab$ et $c$.
Donc, $1$ est le pgcd de $ab$ et $c$.

\done

\medskip 

\noindent\textbf{Corrolaire :} Soit $a$ et $b$ deux entier premier entre eux, et $c$ un entier. Si $a$ et $b$ divisent $c$, alors $ab$ divise $c$.

\medskip

\noindent\textbf{Démonstration :} 
    Soit $a$ et $b$ deux entiers naturels premiers entre eux, et $c$ un entier tel que $a$ et $b$ divisent $c$.
    D'après le théorème de Bachet-Bézout, on peux choisir deux entiers relatifs $u$ et $v$ tels qur $u a + v b = 1$. 
    Multiplions cette expression par $c$. Il vient : $u a c + v b c = c$. 
    Puisque $a$ et $b$ divisent $c$, on peut choisir deux entiers naturels $d$ et $e$ tels que $a d = c$ et $b e = c$. 
    Remplaçant $c$ dans le membre de gauche de l'équation précédente donne alors : $u a b e + v b a d = c$. 
    Factorisant $a b$, il vient : $(u e + v d) a b = c$. 
    Puisque $a$, $b$ et $c$ sont positifs, $u e + v d$ l'est aussi. 
    Cela montre que $a b$ divise $c$. 

\done

\subsubsection{Théorème du reste chinois}

\medskip

\noindent\textbf{Théorème :} Soit $p$ un entier naturel strictement supérieur à $1$. Soit $n_1$, $n_2$, ..., $n_p$ des entiers naturels strictement supérieurs à $1$ deux-à-deux premiers entre eux. 
Soit $N = n_1 n_2 \cdots n_p$. 
Pour tout $p$-uplets d'entiers naturels $(a_1, a_2, \dots, a_p)$ tel que $a_i < n_i$ pour chaque $i \in [\![1,p]\!]$, il existe un unique entier naturel $n$ tel que $n < N$ et, pour chaque $i \in [\![1,p]\!]$, $n \equiv a_i \, [n_i]$.

\medskip

\noindent\textbf{Démonstration :} 
\begin{itemize}
    \item \textit{Unicité :} Supposons avoir deux tels entiers $n$ et $m$. 
    Sans perte de généralité, on peut supposer $n \geq m$. 
    (Si ce n'ets pas le cas, on se ramène à cette situation en inversant les noms de $n$ et $m$.)
    Alors, pour tout $i \in [\![1,p]\!]$, $n-m \equiv 0 \, [n_i]$ ; autrement dit, $n_i$ divise $n-m$. 
    D'après l'un des corrolaires du théorème de Bachet-Bézout, on en déduit que $N$ divise $n-m$. 
    Puisque $n-m \geq 0$ (par définition d'un entier naturel) et $n-m < N$, on en déduit que $n-m=0$, et donc $n=m$.
    \item \textit{Existence :} Les deux ensembles $[\![0, N-1]\!]$ et $[\![0, n_1-1]\!] \times [\![0, n_2-1]\!] \times \cdots \times [\![0, n_p-1]\!]$ ont le même cardinal fini $N$. D'après le résultat précédent, la fonction du permiers vers le second qui à un entier $n \in [\![0,N-1]\!]$ associe le $p$-uplet de ses restes modulo $a_1$, $a_2$, ..., $a_p$ est injective. Donc, elle est aussi surjective.
\end{itemize}

\done

\medskip

\noindent\textbf{Corrolaire :} Soit $a$ et $b$ deux entiers naturels premiers entre eux et strictemetnt supérieurs à $2$. Alors il existe au moins quatre entiers naturels $x_1$, $x_2$, $x_3$ et $x_4$ chacun strictement inférieur à $ab$ et tels que, pour tout $i \in \lbrace 1, 2, 3, 4 \rbrace$, $x_i^2 \equiv 1 \, [a b]$.

\medskip

\noindent\textbf{Démonstration :} 
    D'après le théorème du reste chinois, on peut choisir quatre entiers $x_1$, $x_2$, $x_3$ et $x_4$ strictement inférieurs à $ab$ tels que $x_1 \equiv 1 \, [a]$, $x_1 \equiv 1 \, [b]$, $x_2 \equiv -1 \, [a]$, $x_2 \equiv 1 \, [b]$, $x_3 \equiv 1 \, [a]$, $x_3 \equiv -1 \, [b]$, $x_4 \equiv -1 \, [a]$, $x_4 \equiv -1 \, [b]$. 
    (Ces quatre entiers sont bien deux à deux distincts car aucune paire de deux d'entre eux n'a le même reste par division euclidienne par $a$ ou $b$ ; en effet, puisque $a>2$, $-1$ n'est pas égal à $1$ modulo $a$ ni $b$.) 
    Pour tout $i \in \lbrace 1, 2, 3, 4 \rbrace$, on a $x_i^2 \equiv 1 \, [a]$ et $x_i^2 \equiv 1 \, [b]$. 
    
    Soit $i \in \lbrace 1, 2, 3, 4 \rbrace$. 
    Soit $y$ le reste de la division euclidienne de $x^2$ par $ab$. 
    On a $y \equiv x^2 \, [a]$ et $y \equiv x^2 \, [b]$. 
    Donc, $y \equiv 1 \, [a]$ et $y \equiv 1 \, [b]$. 
    Puisque $1 \equiv 1 \, [a]$ et $1 \equiv 1 \, [b]$ et que $1$ et $y$ sont deux entiers naturels strictement inférieurs à $a b$, on déduit du théorème du reste chinois que $y = 1$.  

    Cela étant vrai pour chaque valeur de $i$, on en déduit : $\forall i \in \lbrace 1, 2, 3, 4 \rbrace \, x_i^1 \equiv 1 \, [a b]$.

\done


\subsubsection{Décomposition en facteurs premiers}
\label{subsub:dec_fact_prem}

\bigskip

\noindent\textbf{Théorème :} 
Tout entier naturel non nul $a$ peut s'écrire comme un produit de facteurs premiers, \textit{i.e.}, il existe $N \in \mathbb{N}^*$, $\left( p_1, p_2, \dots, p_N \right) \in \mathbb{P}^{N}$ et $\left( n_1, n_2, \dots, n_N \right) \in \mathbb{N}^{* N}$ tels que
\begin{equation*}
    a = \prod_{i=1}^N p_i^{n_i},
\end{equation*}
où le produit vide correspond à l'entier naturel $1$.
Cette décomposition est unique à l'ordre près des facteurs. 
(Où un nombre élevé à une puissance $n \in \mathbb{N}$ est vu comme le produit de $n$ fois ce nombre.)

\medskip

\noindent\textbf{Démonstration :} 

\begin{itemize}
    \item \textit{Existence :} On procède par récurrence forte sur $a$. 
    Pour $a=1$, le produit vide convient. 
    Supposons maintenant $a > 1$. 
    Si $a$ n'admet aucun autre diviseur que $1$ et lui-même, il est premier et $a$ est une décomposition en facteurs premiers. 
    Sinon, on peut choisir un entier naturel non nul $b$ tel que $b > 1$, $b < a$, et $b$ divise $a$. 
    On peut aussi choisir un entier naturel non nul $k$ tel que $k b = a$. 
    On a $k < a$. 
    Par hypothèse de récurrence, $b$ et $k$ admettent chacun une décomposition en produit de facteurs premiers. 
    Notons-les respectivement $B$ et $K$.
    Alors, $B K$ est une décomposition en produit de facteurs premiers pour $a$. 
    On conclut que $a$ admet une décomposition en produit de facteurs premiers. 
    Par récurrence forte, cela est vrai pour tout $a \in \mathbb{N}^*$.
    \item \textit{Unicité :} 
    Supposons par l'absurde que $a$ admette deux décompositions en produits de facteurs premiers différentes (autrement que par l'ordre des facteurs). 
    Il existe au moins un nombre premier $p$ apparaissant avec des puissances différentes dans ces deux décompositions (cette puissance étant possiblement zéro dans une des deux décompositions si $p$ n'y apparaît pas). 
    Notons ces deux puissances $n_1$ et $n_2$, avec $n_2 > n_1$. 
    On peut choisir deux entiers naturels $k_1$ et $k_2$ tels que $a = p^{n_1} k_1 = p^{n_2} k_2$ et $k_1$ peut s'écrire comme produit de facteurs premiers distincts de $p$. 
    On a: $p^{n_2 - n_1} k_2 = k_1$. 
    Donc, $p$ divise $k_1$. 
    En utilisant le corrolaire du lemme du théorème de Bachet-Bézout, on en déduit qu'il divise au moins l'un des facteurs premiers de l'écriture de $k_1$ sus-mentionnée. 
    Mais cela est impossible car chacun d'eux est premier et distinct de $p$, et n'admet donc pas $p$ pour diviseur.
    On en déduit que $a$ ne peut admettre deux décompositions en produits de facteurs premiers distinctes (sauf par l'ordre des facteurs).
\end{itemize}

\done

\subsubsection{Petit théorème de Fermat}

\noindent\textbf{Théorème :} Soit $p$ un nombre premier et $a$ un entier naturel non multiple de $p$. 
Alors, $a^{p-1} \equiv 1 \, [p]$.

\medskip

\noindent\textbf{Démonstration :} Notons, pour tout $i \in [\![1, p-1]\!]$, $a_i = i a$ et $r_i$ le reste de la division euclidienne de $a_i$ par $p$. 
Montrons d'abords que les $r_i$ ainsi définis sont deux à deux distincts. 
On procède par l'absurde. 
Soit $(i,j) \in [\![1, p-1]\!]^2$ tels que $j > i$ et $r_j = r_i$. 
On a alors $a_j \equiv a_i \, [p]$, et donc $p$ divise $a_j - a_i$. 
Puisque $a_j - a_i = a \, (j-i)$ et $p$ et $a$ sont premiers entre eux, d'après de lemme du théorème de Bachet-Bézout, $p$ divise $j-i$. 
Mais cela est impossible puisque $j-i$ est un entier naturel non nul strictement inférieur à $p$. 
Cela montre que les $r_i$ pour $i \in [\![1, p-1]\!]$ sont deux à deux distincts. 

Puisque chacun d'eux appartient à $[\![1, p-1]\!]$ (aucun d'eux ne peut être nul d'après le lemme du théorème de Bachet-Bézout puisque, pour chacune des valeur de $i$, $p$ ne divise ni $i$ ni $a$, et donc pas $a_i$), l'ensemble $\lbrace r_i \vert i \in [\![1, p-1]\!] \rbrace$ est inclus dans $[\![1, p-1]\!]$ et a le même cardinal fini. 
On en déduit que ces deux ensembles sont égaux. 
On a donc : $\prod_{i=1}^{p-1} r_i = (p-1)!$, et donc, puisque $a_i \equiv r_i \, [p]$ pour tout $i \in [\![1, p-1]\!]$, $\prod_{i=1}^{p-1} a_i \equiv (p-1)! \, [p]$.

Soustrayant puis factorisant $(p-1)!$, il vient : $(p-1)! \left( a^{p-1} - 1 \right) = 0 \, [p]$. 
Donc, $p$ divise $(p-1)! \left( a^{p-1} - 1 \right)$. 
Puisque $p$ ne divise ni $1$, ni $2$, ..., ni $p-1$, et d'après le second lemme du théorème de Bachet-Bézout, on en déduit que $p$ divise $a^{p-1} - 1$. 
On a donc $a^{p-1} - 1 \equiv 0 \, [p]$, et donc $a^{p-1} = 1 \, [p]$.

\done

\medskip

\noindent\textbf{Remarque :} Avec les mêmes notations, il n'est pas toujours vrai que $p-1$ est le plus petit entier naturel $n$ tel que $a^n \equiv 1 \, [p]$.
    Un contre-exemple est donné par $p = 17$ et $a = 2$: on a $2^8 = 256 = (15 \times 17) + 1$, donc, bien que $17$ est premier, $2$ n'est pas un multiple de $17$ et $8 < 17 - 1$, $2^8 \equiv 1 \, [17]$. 
    Une autre classe de contre-exemples est donnée par tout nombre premier $p$ strictement supérieur à $2$ et $a = 1$ : on a alors $a^n = 1 \, [p]$ pour tout entier naturel $p$.
