\subsection{Nombres premiers}

\subsubsection{Définition}
\label{subsub:defNombresPremiers}

Un entier naturel $p$ est dit \textit{premier} s'il admet exactement deux entiers naturels distincts pour diviseurs: $1$ et lui-même.
On note $\mathbb{P}$ l'ensemble des nombres premiers.

Les premiers nombres premiers sont $2$, $3$, $5$, $7$, $11$, $13$, $17$, $19$ et $23$.

Le seul nombre premier pair est $2$. 
Les deux seuls nombres premiers séparés de $1$ sont $2$ et $3$. 
(En effet, les nombres premiers strictement supérieurs à $2$ sont tous impairs, et donc séparés d'au moins $2$.)\footnote{Soit $p$ et $q$ deux nombres premiers strictement supérieurs à $2$. Si $p-q=1$, $p=q+1$, alors $p$ est pair ou $q$ est pair, ce qui est impossible puisque aucun d'eux n'est divisible par $2$.}
Deux nombres premiers $p$ et $q$ sont dits \textit{jumeaux} si $p-q=2$ ou $q-p=2$. 

\bigskip

\noindent\textbf{Lemme :} Tout entier naturel strictement supérieur à $1$ est divisible par au moins un nombre premier.

\medskip

\noindent\textbf{Démonstration :} On procède par récurrence forte. 
Montrons que, pour tout entier naturel $n$, $n \leq 1$ ou $n$ est divisible par au moins un nombre premier.
Le lemme est évidemment vrai pour les nombres $0$ et $1$ (qui sont tous deux inférieurs ou égaux à $1$) et pour $2$, qui est lui-même premier (et, comme tout entier naturel, divisible par lui-même). 

Soit $n$ un entier naturel strictement supérieur à $2$ et supposons que tout entier compris entre $2$ et $n-1$ soit divisible par au moins un nombre premier. 

Si $n$ est divisible par un entier $l$ compris entre $2$ et $n-1$ (inclus), alors il existe un nombre premier $p$ divisant $l$. 
Il existe alors deux entiers naturels $k$ et $q$ tels que $n = k l$ et $l = q p$, d'où $n = (k q) p$. 
Donc, $p$ divise $n$. 

Si $n$ n'est divisible par aucun entier compris entre $2$ et $n-1$, alors il n'est divisible par aucun entier strictement supérieur à $1$ autre que lui-même. 
En effet, soit $m$ un tel entier, soit $m < n$, et donc $m$ ne divise pas $n$, soit $m > n$, auquel cas il ne peut être un diviseur de $n$.
Donc, $n$ n'est divisible que par $1$ et par lui-même (qui sont bien distincts puisque $n > 1$). 
Par définition, $n$ est donc un nombre premier. 
Puisque $n$ est divisible par lui-même, il est divisible par un nombre premier. 

Dans les deux cas, $n$ est divisible par un nombre premier. 
Par récurrence forte, on en déduit que le résultat est vrai pour tout entier naturel, et donc que tout entier naturel strictement supérieur à $1$ est divisible par au moins un nombre premier.

\done

\bigskip

\noindent\textbf{Lemme :} Il existe une infinité de nombres premiers.

\medskip

\noindent\textbf{Démonstration :} Supposons par l'absurde que l'ensemble des nombres premiers et fini. 
Soit $N$ le cardinal de $\mathbb{P}$ ; $N$ est donc un entier naturel. 
Notons que $N \geq 3$ puisque $2$, $3$ et $5$ sont premiers.
On peut choisir une bijection $b$ de $N$ vers $\mathbb{P}$. 
Pour tout entier naturel $i$ strictement inférieur à $N$, on note $p_{i+1}$ l'entier naturel $b(i)$.
Ainsi, $\mathbb{P} = \lbrace p_1, p_2, \dots, p_N \rbrace$. 

Soit $q$ l'entier définit par : $q \equiv 1 + \prod_{i=1}^N p_i$. 
Puisque tout nombre premier est strictement supérieur à $0$, $\prod_{i=1}^N p_i > 0$~\footnote{Cela se démontre facilement par récurrence sur $N$.}, donc $q > 1$.
Pour tout entier naturel $i$ tel que $i \in [\![1, N]\!]$, on a $q \equiv 1 \, [p_i]$, donc le reste de la division euclidienne de $q$ par $p_i$ est $1$, et $q$ n'est pas divisible par $p_i$.  
Donc, $q$ n'est divisible par aucun nombre premier, ce qui contredit le lemme précédent. 
On en conclut que l'hypothèse de départ est fausse. 

\done

\bigskip

\noindent\textbf{Lemme :} Soit $p$, $q$ et $r$ trois nombres premiers tels que $r > q$, $q > p$ et $r-q = q-p$. Alors $p=3$ ou $q-p$ est divisible par $3$.

\medskip

\noindent\textbf{Démonstration :} Notons $a$ l'entier $q-p$. 
Soit $b$ le reste de la division euclidienne de $p$ par $3$ et $c$ celui de la division euclidienne de $a$ par $3$. 
Si $c = 0$, le résultat est immédiat.
Si $b = 0$, $p$ est divisible par $3$. 
Puisque $p$ est premier, on en déduit que $p=3$ et le résultat est établi. 
Montrons que les autres cas, \textit{i.e.} ceux où $b$ et $c$ sont chacun égaux à $1$ ou $2$, sont impossibles. 
Pour ce faire, on note que $r = p + (q - p) + (r - q) = p + 2 a$, donc $r \equiv b + 2 c \, [3]$.

Si $b = c = 1$, alors $r \equiv (1 + 2) \, [3] \equiv 3 \, [3] \equiv 0 \, [3]$. 
Donc, $r$ est divisible par $3$. 
Puisque $r$ est premier, on en déduit que $r=3$, ce qui est impossible car $p$ et $q$ sont deux nombres premiers distincts strictement inférieurs à $r$ et qu'il n'existe qu'un seul nombre premier ($2$) strictement inférieur à $3$. 

De même, si $b = c = 2$, alors $r \equiv (2 + 4) \, [3] \equiv 6 \, [3] \equiv 0 \, [3]$, ce qui est impossible comme nous venons de le voir.

Si $b = 1$ et $c = 2$ ou si $b = 2$ et $c = 1$, alors $q \equiv (1 + 2) \, [3] \equiv 3 \, [3] \equiv 0 \, [3]$. 
Donc, $q$ est divisible par $3$.
Puisque $q$ est premier, on en déduit que $q=3$. 
Puisque $p$ est un nombre premier strictement inférieur à $q$, il ne peut qu'être égal à $2$. 
Donc, $a = 3-2 = 1$, d'où $r = q + 1 = 4$. 
Mais $4$ n'est pas un nombre premier (puisque $4 = 2 \times 2$ est divisible par $2$), donc ce cas est impossible.

\done

\bigskip

Des fonctions Haskell déterminant si un entier naturel est premier et calculant les premier nombres premiers sont données en appendice~\ref{app:Haskell_primalité}. 
(Ces fonctions sont données à titre d'illustration uniquement, et ne sont pas particulièrement efficaces. 
En particulier, il est en général plus rapide pour calculer les premiers nombre premiers d'utiliser le crible d'Érastosthène, dont une implémentation en C++ et Rust est donnée en appendice~\ref{app:code_erastosthene}.)

\subsubsection{Théorème de Bachet-Bézout}

\bigskip

\noindent\textbf{Théorème (Bachet-Bézout) :} 
Soit $a$ et $b$ deux entiers naturels et $c$ leur plus grand diviseur commun. 
Il existe deux entiers relatifs $p$ et $q$ tels que $p a + q b = c$. 

\medskip

\noindent\textbf{Démonstration :} 
Soit $E = \lbrace n \in \mathbb{N}^* \vert \exists (p,q) \in \mathbb{Z}^2 \, p a + q b \rbrace$. 
$E$ est un sous-ensemble non vide (il contient au moins $a$, obtenu pour $p=1$ et $q=0$) de $\mathbb{N}$, donc il admet un unique élément minimal $r$. 
En outre, par définition de $E$, $r > 0$. 
Montrons qu'il s'agit du plus grand diviseur commun de $a$ et $b$, ce qui prouvera le théorème. 
Pour ce faire, nous procédons en deux temps. 
Nous montrons d'abord que $r$ divise $a$ et $b$, puis qu'il n'existe aucun diviseur commun à ces deux nombres qui soit strictement supérieur à $r$.

Notons $s$ le reste de la division euclidienne de $a$ par $r$. 
Il existe un entier naturel $k$ tel que $a = k r + s$, et $0 \leq s < r$. 
Soit $p$ et $q$ deux entiers relatifs tels que $p a + q b = r$. 
On a: $p a + q b = k a + s$, et donc $(p-k) a + q b = s$. 
Si $s$ était strictement positif, $s$ serait un élément de $\mathbb{N}^*$ et donc de $E$. 
Or, cela est impossible car $s < r$ et $r$ est un élément minimal de $E$. 
On en déduit que $s = 0$, et donc que $r$ divise $a$. 
On montre de même, par le même argument et en échangeant les rôles de $a$ et $b$, que $r$ divise $b$. 
Ainsi, $r$ est un diviseur commun de $a$ et $b$. 

Supposons maintenant par l'absurde qu'il existe un autre diviseur commun à $a$ et $b$, noté $t$, tel que $t > r$. 
On peut choisir deux entiers naturels $u$ et $v$ tels que $a = u t$ et $b = v t$. 
Soit $p$ et $q$ deux entiers relatifs tels que $p a + q b = r$. 
On a: $r = p u t + q v t$, d'où $r = (p u + q v) t$. 
Puisque $r$ et $t$ sont tous deux strictement positifs, $p u + q v$ doit l'être aussi. 
Mais un entier strictement positif est supérieur ou égal à $1$, et donc $r \geq t$, en contradiction avec l'hypothèse. 
On en déduit qu'il n'existe aucun diviseur commun à $a$ et $b$ strictement supérieur à $r$.

Ainsi, $r$, qui est un élément de $E$ et peut donc s'écrire $q a + p b$ avec $(p, q) \in \mathbb{Z}^2$, est le plus grand diviseur commun de $a$ et $b$.

\done

\medskip

\noindent\textbf{Lemme :} 
Soit $p$ un nombre premier et $a$ et $b$ deux entiers naturels. 
Si $p$ divise $a b$, alors $p$ divise $a$ ou $p$ divise $b$.

\medskip

\noindent\textbf{Démonstration :} 
Si $p$ divise $a$, le résultat est vrai. 
Supposons que $p$ ne divise pas $a$. 
Puisque les seuls diviseurs de $p$ sont $1$ et lui-même, et car $p$ ne divise pas $a$, $1$ est le seul diviseur commun à $p$ et $a$, et donc leur plus grand diviseur commun. 
D'après le théorème précédent, on peut donc choisir deux entiers relatifs $q$ et $r$ tels que $q p + r a = 1$. 
Multiplions cette équation par $b$. 
Il vient : $q p b + r a b = b$. 
Puisque $p$ divise $a b$, on peut choisir un entier naturel $k$ tel que $a b = k p$. 
Donc, $q p b + r k p = b$. 
Cette équation peut se récrire : $(q b + r k) p = b$. 
Puisque $p$ et $b$ sont tous deux strictement positifs, $p b + r k$ doit l'être aussi, et $p$ divise donc $b$.

\done

\bigskip
\noindent\textbf{Corolaire :} 
Soit $p$ un nombre premier, $N$ un entier naturel, et $a_1$, $a_2$, ..., $a_N$ des entiers naturels (si $N \neq 0$). 
Si $p$ divise $\prod_{i=1}^N a_i$ (pris égal à $1$ si $N = 0$), alors il existe un élément $i$ de $[\![1,N]\!]$ tel que $p$ divise $a_i$.

\medskip

\noindent\textbf{Démonstration :} 
On procède par récurrence sur $N$. 
Soit $P$ le prédicat à un paramètre libre défini par : $P(N): \forall a \in \mathbb{N}^N \, p \vert \prod_{i=1}^N a_i \Rightarrow \exists i \in [\![1,N]\!] \, \vert a_i$.

Puisque $p$ est premier, $p > 1$, donc $p$ ne divise pas $1$. 
Donc, $P(0)$ est vrai.

Si $N = 1$ et $p \vert \prod_{i = 1}^N a_i$, alors, puisque $ \prod_{i = 1}^N= a_1$, $p \vert a_1$.
Donc, $P(1)$ est vrai.

Le prédicat $P(2)$ est également vrai d'après le lemme précédent. 

Soit $N$ un entier naturel tel que $P(N)$ est vrai. 
Si $N = 0$ ou $N = 1$, alors $N+1 = 1$ ou $N+1 = 2$, donc $P(N+1)$ est vrai.
Supposons maintenant $N \geq 2$.

Soit $a_1$, $a_2$, ..., $a_{N+1}$ des entiers naturels tels que $p \vert a_1 a_2 \cdots a_{N+1}$.
Puisque $p$ divise $a_1 a_2 \cdots a_{N+1}$, égal à $(a_1 a_2 \cdots a_N) a_{N+1}$, d'après le lemme précédent, il divise $a_1 a_2 \cdots a_N$ ou $a_{N+1}$. 
Dans le second cas, il existe bien un élément $i$ de $[\![1, N+1]\!]$ ($N+1$) tel que $p \vert a_i$. 
Dans le premiers, par hypothèse de récurrence, il existe un élément $i$ de $[\![1,N]\!]$, satisfaisant donc $i \in [\![1,N+1]\!]$, tel que $p$ divise $a_i$. 
Ainsi, $P(N+1)$ est vrai.

Par récurrence, $P(N)$ est donc vrai pour tout entier naturel $N$. 

\done

\medskip

\noindent\textbf{Corolaire :} 
Soit $N$ un entier naturel, $p$ un nombre premier, et $a$ un entier naturel.
Si $p$ divise $a^N$, alors $p$ divise $a$.

\medskip

\noindent\textbf{Démonstration :} 
C'est une application directe du corolaire précédent avec $a_1 = a_2 = \cdots = a_N = a$.

\done

\medskip

\noindent\textbf{Corolaire :} Soit $a$, $b$, et $c$ trois entiers tels que $a$ et $c$ sont premiers entre eux et $b$ et $c$ sont premiers entre eux. Alors $ab$ et $c$ sont aussi premiers entre eux.

\medskip 

\noindent\textbf{Démonstration :} Soit $d$ un diviseur de $ab$ et $c$. 
On veut montrer que $d$ doit être égal à $1$. 
Supposons par l'absurde que $d > 1$. 
Alors, $d$ admet au moins un diviseur premier $e$.
Puisque $e \vert d$ et $d \vert ab$, on a $e \vert ab$.
Donc, $e \vert a$ ou $e \vert b$.
De même, puisque $e \vert d$ et $d \vert c$, on a $e \vert c$.
Donc, $(e \vert a \wedge e \vert c) \vee (e \vert b \wedge e \vert c)$.
Cela est impossible puisque $e > 0$, $a$ et $c$ sont premiers entre eux et que $b$ et $c$ sont premiers entre eux. 
On en déduit que l'hypothèse de départ est fausse et que $d$ doit être égal à $1$. 

Réciproquement, $1$ divise tout entier et donc $ab$ et $c$.
Donc, $1$ est le pgcd de $ab$ et $c$.

\done

\medskip 

\noindent\textbf{Corolaire :} Soit $a$ et $b$ deux entier naturels non nuls, et $c$ un entier naturel non nul premier avec $a$ et avec $b$. 
Alors, $c$ est premier avec $ab$.

\medskip

\noindent\textbf{Démonstration :}
    Soit $d$ un diviseur commun de $c$ et $a b$. 
    Supposons par l'absurde que $d > 1$.
    Alors $d$ est un entier naturel strictement supérieur à $1$, donc il admet un diviseur premier $p$. 
    Puisque $d \vert ab$ et $d \vert c$, $p \vert ab$, donc $p \vert a$ ou $p \vert b$, et $p \vert c$. 
    Cela est impossible puisque $a$ et $c$ sont premiers entre eux (donc $p \vert a \wedge p \vert c$ est impossible) et $b$ et $c$ sont premiers entre eux (donc $p \vert b \wedge p \vert c$ est impossible). 
    On en déduit que l'hypothèse est fausse, et donc que $d=1$.
    
    \done

\medskip 

\noindent\textbf{Corolaire :} Soit $n$ un entier naturel non nul et $a_1$, $a_2$, ..., $a_n$ des entiers naturels non nuls (où $a_n$ est absent si $n \leq 2$ et $a_2$ est absent si $n = 1$). 
    Soit $b$ un entier naturel non nul premier avec $a_1$, $a_2$, ..., $a_n$.
    Alors, $b$ est premier avec $a_1 a_2 \dots a_n$.

\medskip

\noindent\textbf{Démonstration :}
    On procède par récurrence sur $n$. 
    Pour $n$ égal à $1$, le résultat est évident puisque $a_1 a_2 \dots a_n = a_1$.

    Soit $m$ un entier naturel non nul et supposons l'énoncé vrai pour $n = m$.
    Soit $a_1$, $a_2$, ..., $a_m$, $a_{m+1}$ des entiers naturels non nul et $b$ un entier naturel non nul premier avec chacun d'entre eux.
    Alors, $b$ est premier avec $a_1 a_2 \dots a_m$. 
    Puisqu'il est également premier avec $a_{m+1}$, on en déduit d'après le corolaire précédent que $b$ est premier avec $a_1 a_2 \dots a_m a_{m+1}$.
    Donc, l'énoncé est vrai pour $n = m+1$. 
    
    Par récurrence, il l'est pour tout entier naturel non nul $n$.

    \done

\medskip 

\noindent\textbf{Corolaire :} Soit $a$ et $b$ deux entier premier entre eux, et $c$ un entier. Si $a$ et $b$ divisent $c$, alors $ab$ divise $c$.

\medskip

\noindent\textbf{Démonstration :} 
    Soit $a$ et $b$ deux entiers naturels premiers entre eux, et $c$ un entier tel que $a$ et $b$ divisent $c$.
    D'après le théorème de Bachet-Bézout, on peux choisir deux entiers relatifs $u$ et $v$ tels que $u a + v b = 1$. 
    Multiplions cette expression par $c$. Il vient : $u a c + v b c = c$. 
    Puisque $a$ et $b$ divisent $c$, on peut choisir deux entiers naturels $d$ et $e$ tels que $a d = c$ et $b e = c$. 
    Remplaçant $c$ dans le membre de gauche de l'équation précédente donne alors : $u a b e + v b a d = c$. 
    Factorisant $a b$, il vient : $(u e + v d) a b = c$. 
    Puisque $a$, $b$ et $c$ sont positifs, $u e + v d$ l'est aussi. 
    Cela montre que $a b$ divise $c$. 

\done

\medskip 

\noindent\textbf{Corolaire :} Soit $p$ un entier naturel strictement supérieur à $1$, $a_1$, $a_2$, ..., $a_p$ des entiers naturels deux-à-deux premier entre eux, et $b$ un entier. Si $a_i$ divise $b$ pour tout élément $i$ de $[\![1,p]\!]$, alors $a_1 \times a_2 \times \cdots \times a_p$ divise $b$.

\medskip

\noindent\textbf{Démonstration :} On procède par récurrence sur $p$. 
    Pour $p = 2$, il s'agit du corolaire précédent.
    Soit $p$ un entier naturel supérieur ou égal à $2$ et supposons l'énoncé vrai pour cette valeur. 
    Soit $a_1$, $a_2$, ..., $a_p$, $a_{p+1}$ des entiers naturels deux-à-deux premiers entre eux et $b$ un entier tel que : $\forall i \in [\![1, p+1]\!] \, a_i \vert b$. 
    Montrons que $a_1 \times a_2 \times \cdots \times a_p \times a_{p+1} \vert b$.

    Puisque les nombres $a_1$, $a_2$, ..., $a_p$, $a_{p+1}$ sont deux-à-deux premiers entre eux, $a_1 \times a_2 \times \cdots \times a_p$ et $a_{p+1}$ sont premiers entre eux. 
    Puisque $a_1 \times a_2 \times \cdots \times a_p \vert b$ (par hypothèse de récurrence) et $a_{p+1} \vert b$, on conclut d'après le corolaire précédent que $a_1 \times a_2 \times \cdots \times a_p \times a_{p+1} \vert b$.

    \done

\medskip

\noindent\textbf{Lemme :} Soit $a$, $b$ et $c$ trois entiers naturels tels que $a \vert b c$ et $a$ et $b$ sont premiers entre eux. 
    Alors, $a \vert c$.

\medskip

\noindent\textbf{Démonstration :} Puisque $a$ et $b$ sont premiers entre eux, on peux choisir deux entiers $u$ et $v$ tels que $u a + v b = 1$. 
    Donc, $u a c + v b c = c$.
    Puisque $a \vert b c$, on peut choisir un entier naturel $k$ tel que $b c = k a$.
    On a donc $u a c + v k a = c$, donc $(u c + v k) a = c$.
    Puisque $a$ et $c$ sont tous deux positifs, $u c + v k$ doit être positif, donc il s'agit d'un entier naturel.
    Donc, $a \vert c$.

    \done

\subsubsection{Plus petit commun multiple}

\noindent\textbf{Définition :} Soit $a$ et $b$ deux entiers naturels non nuls. L'ensemble de leurs multiples communs non nuls est un sous-ensemble de $\mathbb{N}$ non vide (il contient au moins $a \times b$). Il admet donc un unique élément minimal, appelé \textit{plus petit commun multiple}, ou \textit{ppcm}, de $a$ et $b$. 

\medskip

\noindent\textbf{Lemme :} Soit $a$ et $b$ deux entiers naturels non nuls, $c$ leur pgcd, et $d$ leur ppcm.
    Alors, $c \times d = a \times b$.

\medskip

\noindent\textbf{Lemme :}
    Soit $q$ et $r$ le quotient et le reste de la division Euclidienne de $a \times b$ par $c$.
    Puisque $c$ est un diviseur de $a$, il est aussi un diviseur de $a \times b$, donc $r = 0$.
    Donc, $c \times q = a \times b$.
    Il suffit donc de montrer que $q = d$. 

    Puisque $c$ est un diviseur de $a$ et de $b$, on peut choisir deux entiers naturels $k$ et $l$ tels que $a = k c$ et $b = l c$.
    On a donc $q c = k b c$ et $q c = l a c$.
    Puisque $c$ est non nul, cela donne $q = k b$ et $q = l a$.
    Donc, $q$ est un multiple de $a$ et de $b$. 
    En outre, puisque $a \neq 0$ et $b \neq 0$, $a \times b \neq 0$, donc $q \neq 0$.
    Donc, $q$ est un multiple non nul de $a$ et de $b$. 

    Montrons que c'est le plus petit. 
    Supposons par l'absurde que ce n'est pas le cas. 
    Soit $m$ un entier naturel non nul et deux entiers naturels non nuls $n$ et $o$ tels que $m = n a$ et $m = o b$.
    Soit $a'$ et $b'$ les quotients des divisions euclidiennes de $a$ et $b$ par $c$.
    On a : $m = n c a' = o c b'$. 
    Puisque $c \neq 0$, cela implique $n a' = o b'$.
    Donc, $a'$ divise $o b'$.
    Puisque $a'$ et $b'$ sont premiers entre eux, cela implique que $a$ divise $o$.
    On peut donc choisir un entier naturel $t$ tel que $o = t a'$. 
    En outre, $t$ est non nul puisque $o$ l'est.
    On a alors $m = t c a' b'$.
    Donc, $m c = t a b$.
    Puisque $t > 0$, on a donc $m c \geq a b$, donc $m c \geq q c$. 
    Puisque $c \neq 0$, cela implique $m \geq q$. 
    Ainsi, $q$ est bien le plus petit multiple commun non nul de $a$ et $b$.

    \done

\subsubsection{Théorème du reste chinois}

\medskip

\noindent\textbf{Théorème :} Soit $p$ un entier naturel strictement supérieur à $1$. Soit $n_1$, $n_2$, ..., $n_p$ des entiers naturels strictement supérieurs à $1$ deux-à-deux premiers entre eux. 
Soit $N$ l'entier naturel défini par : $N = n_1 n_2 \cdots n_p$. 
Pour tout $p$-uplet d'entiers naturels $(a_1, a_2, \dots, a_p)$ tel que $a_i < n_i$ pour chaque élément $i$ de $[\![1,p]\!]$, il existe un unique entier naturel $n$ tel que $n < N$ satisfaisant : $\forall i \in [\![1,p]\!] \, n \equiv a_i \, [n_i]$.

\medskip

\noindent\textbf{Démonstration :} 
\begin{itemize}
    \item \textit{Unicité :} Supposons avoir deux tels entiers $n$ et $m$. 
    Sans perte de généralité, on peut supposer $n \geq m$. 
    (Si ce n'est pas le cas, on se ramène à cette situation en inversant les noms de $n$ et $m$.)
    Alors, pour tout élément $i$ de $[\![1,p]\!]$, $n-m \equiv 0 \, [n_i]$ ; autrement dit, $n_i$ divise $n-m$. 
    D'après l'un des corolaires du théorème de Bachet-Bézout, on en déduit que $N$ divise $n-m$. 
    Puisque $n-m \geq 0$ (par définition d'un entier naturel) et $n-m < N$, on en déduit que $n-m=0$, et donc $n=m$.
    \item \textit{Existence :} Les deux ensembles $[\![0, N-1]\!]$ et $[\![0, n_1-1]\!] \times [\![0, n_2-1]\!] \times \cdots \times [\![0, n_p-1]\!]$ ont le même cardinal fini $N$. D'après le résultat précédent, la fonction du premier vers le second qui à un entier $n$ appartenant à $[\![0,N-1]\!]$ associe $(n \mathrel{\%} n_1, n \mathrel{\%} n_2, \dots, n \mathrel{\%} n_p)$ est injective. Donc, elle est bijective.
\end{itemize}

\done

\medskip

\noindent\textbf{Corolaire :} Soit $a$ et $b$ deux entiers naturels premiers entre eux et strictement supérieurs à $2$. Alors il existe au moins quatre entiers naturels $x_1$, $x_2$, $x_3$ et $x_4$ deux à deux distincts, chacun strictement inférieur à $ab$ et tels que, pour tout élément $i$ de $\lbrace 1, 2, 3, 4 \rbrace$, $x_i^2 \equiv 1 \, [a b]$.

\medskip

\noindent\textbf{Démonstration :} 
    D'après le théorème du reste chinois, on peut choisir quatre entiers $x_1$, $x_2$, $x_3$ et $x_4$ strictement inférieurs à $ab$ tels que $x_1 \equiv 1 \, [a]$, $x_1 \equiv 1 \, [b]$, $x_2 \equiv (a-1) \, [a]$, $x_2 \equiv 1 \, [b]$, $x_3 \equiv 1 \, [a]$, $x_3 \equiv (b-1) \, [b]$, $x_4 \equiv (a-1) \, [a]$, $x_4 \equiv (b-1) \, [b]$. 
    (Ces quatre entiers sont bien deux à deux distincts car aucune paire de deux d'entre eux n'a les deux mêmes restes par les divisions euclidiennes par $a$ et par $b$ ; en effet, puisque $a>2$ et $b > 2$, $a-1$ n'est pas égal à $1$ modulo $a$ (ce sont deux entiers naturels strictement inférieurs à $a$ et distincts puisque $a-2 > 0$) et $b-1$ n'est pas égal à $1$ modulo $b$.) 
    Puisque $(a-1) \times (a-1) = a^2 - 2 a + 1$ et $(b-1) \times (b-1) = b^2 - 2 b + 1$, on a $(a-1) \times (a-1) \equiv 1 \, [a]$ et $(b-1) \times (b-1) = 1 \, [b]$.
    Donc, pour tout élément $i$ de $\lbrace 1, 2, 3, 4 \rbrace$, on a $x_i^2 \equiv 1 \, [a]$ et $x_i^2 \equiv 1 \, [b]$. 
    
    Soit $i$ un élément de $\lbrace 1, 2, 3, 4 \rbrace$. 
    Soit $y$ le reste de la division euclidienne de $x_i^2$ par $ab$. 
    On a $y \equiv x_i^2 \, [a]$ et $y \equiv x_i^2 \, [b]$. 
    Donc, $y \equiv 1 \, [a]$ et $y \equiv 1 \, [b]$. 
    Puisque $1 \equiv 1 \, [a]$ et $1 \equiv 1 \, [b]$ et que $1$ et $y$ sont deux entiers naturels strictement inférieurs à $a b$, on déduit du théorème du reste chinois que $y = 1$.  

    Cela étant vrai pour chaque valeur de $i$, on en déduit : $\forall i \in \lbrace 1, 2, 3, 4 \rbrace \, x_i^2 \equiv 1 \, [a b]$.

\done


\subsubsection{Décomposition en produit de facteurs premiers}
\label{subsub:dec_fact_prem}

\bigskip

\noindent\textbf{Théorème :} 
Tout entier naturel non nul peut s'écrire comme un produit de facteurs premiers, \textit{i.e.}, pour tout entier naturel non nul $a$, il existe un entier naturel $N$, un $N$-uplet $\left( p_1, p_2, \dots, p_N \right)$ de nombres premiers deux-à-deux distincts et un $N$-uplet $\left( n_1, n_2, \dots, n_N \right)$ d'entiers naturels non nuls tels que
\begin{equation*}
    a = \prod_{i=1}^N p_i^{n_i}.
\end{equation*}
Le couple $\left( \left( p_1, p_2, \dots, p_N \right), \left( n_1, n_2, \dots, n_N \right)\right)$ est appelé \textit{décomposition de $a$ en produit de facteurs premiers}.

Cette décomposition est unique à l'ordre près des facteurs : soit $a$ un entier naturel non nul, si $N$ et $M$ sont deux entiers naturels, $\left( p_1, p_2, \dots, p_N \right)$ un $N$-uplet de nombres premiers deux-à-deux distincts, $\left( q_1, q_2, \dots, q_M \right)$ un $M$-uplet de nombres premiers deux-à-deux distincts, $\left( n_1, n_2, \dots, n_N \right)$ un $N$-uplet d'entiers naturels non nuls et $\left( m_1, m_2, \dots, m_m \right)$ un $M$-uplet d'entiers naturels non nuls tels que 
\begin{equation*}
    a = \prod_{i=1}^N p_i^{n_i} = \prod_{j=1}^M q_j^{m_j},
\end{equation*}
alors, pour tout élément $i$ de $[\![1,N]\!]$, on peut choisir un élément $j$ de $[\![1,M]\!]$ tel que $q_j = p_i$ et $m_j = n_i$.

(Notons que l'entier $j$ ainsi définit est unique, et ne peut être le même pour deux valeurs différentes de $i$. 
Cela définit donc une injection de $[\![1,N]\!]$ vers $[\![1,M]\!]$. 
Puisque le premier est de cardinal $N$ et le second de cardinal $M$, cela implique $N \geq M$. 
En échangeant les rôles des deux décompositions, on montre de même $M \leq N$, et donc $N = M$.)

\medskip

\noindent\textbf{Démonstration :} 

\begin{itemize}
    \item \textit{Existence :} On procède par récurrence forte sur $a$. 
        Pour $a=1$, $a$ est égal au produit vide, donc $(\emptyset, \emptyset)$ convient. 
        Supposons maintenant $a > 1$ et le résultat vrai pour tout entier naturel strictement inférieur à $a$. 
        Alors, on peut choisir un nombre premier $p$ tel que $p \vert a$.
        On peut donc choisir un entier naturel non nul $q$ tel que $p q = a$. 
        Puisque $p > 1$, $q < a$ (sans quoi on aurait $p q > a$).
        Donc, $q$ admet une décomposition en produit de facteurs premiers. 
        On peut donc choisir un entier naturel $N$, un $N$-uplet de nombres premiers $\left( p_1, p_2, \dots, p_N \right)$ deux-à-deux distincts, et un $N$-uplet d'entiers naturels non nuls $\left( n_1, n_2, \cdots n_N \right)$ tels que $k = \prod_{i=1}^N p_i^{n_i}$.
        On a donc $p \prod_{i=1}^N p_i^{n_i} = a$. 
        S'il existe un élément $i_0$ de $[\![1,N]\!]$ tel que $p_{i_0} = p$, alors, définissant, pour tout élément $i$ de $[\![1,N]\!]$, $n_i' = n_i + 1$ si $i = i_0$ et $n_i' = n_i$ sinon, on a : $\prod_{i_1}^N p_i^{n_i'} = p \prod_{i=1}^N p_i^{n_i} = a$, donc $\left(\left( p_1, pi2, \dots, p_N \right), \left(n_1', n_2', \dots, n_N' \right) \right)$ est une décomposition de $a$ et produit de facteurs premiers. 
        Sinon, $\left( \left( p_1, p_2, \dots, p_N, p \right), \left( n_1, n_2, \dots, n_N, 1 \right) \right)$ est une décomposition de $a$ en produit de facteurs premiers.
        Dans tous les cas, $a$ admet bien une décomposition en produit de facteurs premiers.
        Par récurrence forte, on conclut que le résultat est vrai pour tout entier naturel non nul.

    \item \textit{Unicité :} 
        Supposons par l'absurde que $a$ admette deux décompositions en produits de facteurs premiers différentes (autrement que par l'ordre des facteurs). 
        Il existe au moins un nombre premier $p$ apparaissant avec des puissances différentes dans ces deux décompositions (cette puissance étant possiblement zéro dans une des deux décompositions si $p$ n'y apparaît pas), c'est-à-dire, avec les notations de l'énoncé, un élément $i$ de $[\![1,N]\!]$ tel que $p_i$ n'apparaisse pas dans la seconde décomposition ou, si $j$ désigne l'élément de $[\![1,M]\!]$ tel que $p_i = q_j$, $n_i \neq m_j$. 
        Notons ces deux puissances $n_1$ et $n_2$, avec $n_2 > n_1$ (avec $n_1 = 0$ dans le premier cas). 
        On peut choisir deux entiers naturels $k_1$ et $k_2$ tels que $a = p^{n_1} k_1 = p^{n_2} k_2$ et $k_1$ peut s'écrire comme produit de facteurs premiers distincts de $p$. 
        On a: $p^{n_2 - n_1} k_2 = k_1$ (puisque $p^{n_1} p^{n_2 - n_1} k_2 = p^{n_2} k_2 = p^{n_1} k_1$ et $p^{n_1} \neq 0$). 
        Puisque $n_2 > n_1$, $p^{n_2 - n_1} k_2 = p p^{n_2 - n_1 - 1} k_2$ et $p^{n_2 - n_1 - 1} k_2$ est un entier naturel.
        Donc, $p$ divise $k_1$. 
        En utilisant le corolaire du lemme du théorème de Bachet-Bézout, on en déduit qu'il divise au moins l'un des facteurs premiers de l'écriture de $k_1$ susmentionnée. 
        Mais cela est impossible car chacun d'eux est premier et distinct de $p$, et n'admet donc pas $p$ pour diviseur.
    On en déduit que $a$ ne peut admettre deux décompositions en produits de facteurs premiers différent autrement que par l'ordre des facteurs.
\end{itemize}

\done

\subsubsection{Représentation schématique}

(Cette section n'est pas destinée à être rigoureuse, mais seulement à donner une certaine intuition de la décomposition en facteurs premiers.)

Un entier naturel strictement supérieur à $1$ peut être représenté schématiquement par une série de blocs représentant chacun de ses facteurs premiers (avec multiplicité). 
D'après le résultat précédent, cette représentation est unique à l'ordre près des blocs, qui peuvent (puisque la multiplication est commutative) être réarrangés de manière quelconque. 
L'entier $1$ pourra être représenté par la série vide.
Notons que la série représentant un entier a un unique bloc si et seulement si ce nombre est premier.
Puisque la multiplication est associative, accoler les séries représentant deux entiers naturels non nuls donne la série représentant leur produit.

Quelques exemples : 
\begin{multicols}{3}
\begin{itemize}[nosep]
    \item $1$: 
    \item $2$: \inbox{2}
    \item $3$: \inbox{3}
    \item $4$: \inbox{2}\inbox{2}
    \item $5$: \inbox{5}
    \item $6$: \inbox{2}\inbox{3}
    \item $7$: \inbox{7}
    \item $8$: \inbox{2}\inbox{2}\inbox{2}
    \item $9$: \inbox{3}\inbox{3}
    \item $10$: \inbox{2}\inbox{5}
    \item $11$: \inbox{11}
    \item $12$: \inbox{2}\inbox{2}\inbox{3}
    \item $13$: \inbox{13}
    \item $14$: \inbox{2}\inbox{7}
    \item $15$: \inbox{3}\inbox{5}
    \item $16$: \inbox{2}\inbox{2}\inbox{2}\inbox{2}
    \item $17$: \inbox{17}
    \item $18$: \inbox{2}\inbox{3}\inbox{3}
    \item $19$: \inbox{19}
    \item $20$: \inbox{2}\inbox{2}\inbox{5}
    \item $21$: \inbox{3}\inbox{7}
    \item $22$: \inbox{2}\inbox{11}
    \item $23$: \inbox{23}
    \item $24$: \inbox{2}\inbox{2}\inbox{2}\inbox{3}
    \item $25$: \inbox{5}\inbox{5}
    \item $26$: \inbox{2}\inbox{13}
    \item $27$: \inbox{3}\inbox{3}\inbox{3}
\end{itemize}
\end{multicols}

Exemples de multiplications : 
\begin{multicols}{2}
\begin{itemize}[nosep]
    \item $1 \times 2 = 2$: $\times \, \inbox{2} = \inbox{2}$
    \item $2 \times 2 = 4$: \inbox{2} \times \,\inbox{2} = \inbox{2}\inbox{2}
    \item $2 \times 3 = 6$: \inbox{2} \times \,\inbox{3} = \inbox{2}\inbox{3}
    \item $2 \times 4 = 8$: \inbox{2} \times \,\inbox{2}\inbox{2} = \inbox{2}\inbox{2}\inbox{2}
    \item $6 \times 4 = 24$: \inbox{2}\inbox{3} \times \,\inbox{2}\inbox{2} = \inbox{2}\inbox{3}\inbox{2}\inbox{2} = \inbox{2}\inbox{2}\inbox{2}\inbox{3}
    \item $7 \times 3 = 21$: \inbox{7} \times \inbox{3} = \inbox{7}\inbox{3} = \inbox{3}\inbox{7}
\end{itemize}
\end{multicols}

Notons que : 
\begin{itemize}[nosep]
    \item Soit $a$ et $b$ deux entiers naturels non nuls.
        Alors, $a$ est un multiple de $b$ si et seulement si la représentation schématique de $b$ est (possiblement après réarrangements) incluse dans celle de $a$. \\
        \textit{Exemple 1 :} $80$ est un multiple de $20$ :
        \begin{itemize}[nosep]
            \item $80$: \textcolor{blue}{\inbox{2}\inbox{2}\inbox{5}}\inbox{2}\inbox{2}
            \item $20$: \textcolor{blue}{\inbox{2}\inbox{2}\inbox{5}}
        \end{itemize}
        \textit{Exemple 2 :} $80$ n'est pas un multiple de $60$ :
        \begin{itemize}[nosep]
            \item $80$: \textcolor{blue}{\inbox{2}\inbox{2}\inbox{5}}\inbox{2}\inbox{2}
            \item $60$: \textcolor{blue}{\inbox{2}\inbox{2}\inbox{5}}\textcolor{red}{\inbox{3}}
        \end{itemize}
    \item Soit $a$ et $b$ deux entiers naturels non nuls et $c$ leur pgcd.
        Alors, la représentation schématique de $c$ est donnée par la partie commune de celles de $a$ et $b$.
        \textit{Exemple :} Le pgcd de $100$ et $30$ est $10$
        \begin{itemize}[nosep]
            \item $10$: \textcolor{blue}{\inbox{2}\inbox{5}}\textcolor{green!50!black}{\inbox{2}\inbox{5}}
            \item $30$: \textcolor{blue}{\inbox{2}\inbox{5}}\textcolor{red}{\inbox{3}}
            \item \textcolor{blue}{$2 \times 5 = 10$}
        \end{itemize}
\end{itemize}

\subsubsection{Petit théorème de Fermat}

\noindent\textbf{Théorème :} Soit $p$ un nombre premier et $a$ un entier naturel non multiple de $p$. 
Alors, $a^{p-1} \equiv 1 \, [p]$.

\medskip

\noindent\textbf{Démonstration :} Pour tout élément $i$ de $[\![1, p-1]\!]$, on définit l'entier naturels $a_i$ par $a_i = i a$ et on note $r_i$ le reste de la division euclidienne de $a_i$ par $p$. 
Montrons d'abords que les $r_i$ ainsi définis sont deux à deux distincts. 
Cela montrera que la fonction de $p-1$ vers $\lbrace r_1, r_2, \dots, r_{p-1} \rbrace$ associant $r_{i+1}$ à tout élément $i$ de $p-1$ est injective.
Elle est aussi surjective (pour tout élément $e$ de cet ensemble, il existe par définition un élément $i$ de $[\![1,p-1]\!]$ tel que $e = r_i$ ; en posant $j = i-1$, on a $j \in p-1$ et $e = r_{j+1}$). 
Donc, cela démontrera qu'il s'agit d'une bijection, et donc que $[\![1, p-1]\!]$ et $\lbrace r_1, r_2, \dots, r_{p-1} \rbrace$ sont de même cardinal $p-1$.

On procède par l'absurde. 
Soit $i$ et $j$ deux éléments de $[\![1, p-1]\!]$ tels que $j \neq i$ et $r_j = r_i$. 
On suppose en outre que $j > i$. 
(Si ce n'est pas le cas, alors $i > j$ et on se ramène à ce cas en échangeant les rôles de $i$ et $j$.)
On a alors $a_j \equiv a_i \, [p]$, et donc $p$ divise $a_j - a_i$. 
Puisque $a_j - a_i = a \, (j-i)$ et $p$ et $a$ sont premiers entre eux, d'après de lemme du théorème de Bachet-Bézout, $p$ divise $j-i$. 
Mais cela est impossible puisque $j-i$ est un entier naturel non nul strictement inférieur à $p$. 
Cela montre que les $r_i$ pour $i \in [\![1, p-1]\!]$ sont deux à deux distincts. 

En outre, aucun d'eux ne peut être nul d'après le lemme du théorème de Bachet-Bézout puisque, pour chacune des valeur de $i$ dans $[\![1,p-1]\!]$, $p$ ne divise ni $i$ (puisque $i$ est un entier strictement positif et strictement inférieur à $p$) ni $a$, et donc pas $a_i$.
Donc, chacun d'eux appartient à $[\![1, p-1]\!]$.

Ainsi, l'ensemble $\lbrace r_1, r_2, \dots, r_{p-1} \rbrace$ est inclus dans $[\![1, p-1]\!]$ et a le même cardinal fini. 
On en déduit que ces deux ensembles sont égaux. 
On a donc : $\prod_{i=1}^{p-1} r_i = (p-1)!$, et donc, puisque $a_i \equiv r_i \, [p]$ pour tout élément $i$ de $[\![1, p-1]\!]$, $\prod_{i=1}^{p-1} a_i \equiv (p-1)! \, [p]$.
Puisque $\prod_{i=1}^{p-1} a_i = (p-1)! \, a^{p-1}$, cela donne $(p-1)! \, a^{p-1} \equiv (p-1)! \, [p]$.

Soustrayant puis factorisant $(p-1)!$, il vient : $(p-1)! \left( a^{p-1} - 1 \right) = 0 \, [p]$. 
Donc, $p$ divise $(p-1)! \left( a^{p-1} - 1 \right)$. 
Puisque $p$ ne divise ni $1$, ni $2$, ..., ni $p-1$, et d'après le premier corolaire du théorème de Bachet-Bézout, on en déduit que $p$ divise $a^{p-1} - 1$. 
On a donc $a^{p-1} - 1 \equiv 0 \, [p]$, et donc $a^{p-1} = 1 \, [p]$.

\done

\medskip

\noindent\textbf{Remarque :} Avec les mêmes notations, il n'est pas toujours vrai que $p-1$ est le plus petit entier naturel $n$ tel que $a^n \equiv 1 \, [p]$.
    Un contre-exemple est donné par $p = 17$ et $a = 2$: on a $2^8 = 256 = (15 \times 17) + 1$, donc, bien que $17$ est premier, $2$ n'est pas un multiple de $17$ et $8 < 17 - 1$, $2^8 \equiv 1 \, [17]$. 
    Une autre classe de contre-exemples est donnée par tout nombre premier $p$ strictement supérieur à $2$ et $a = 1$ : on a alors $a^n = 1 \, [p]$ pour tout entier naturel $p$.

\medskip

\noindent\textbf{Lemme :} Soit $p$ un nombre premier. 
    Soit $a$ et $b$ deux entiers naturels tel que $a$ n'est pas un multiplde de $p$ et soit $s$ un entier naturel non nul.
    Soit $g$ le pgcd de $s$ et $p-1$.
    On suppose que $a^s \equiv b^s \, [p]$.
    Alors, $a^g \equiv b^g \, [p]$.

    En partisulier, si $s$ et $p-1$ sont premiers entre eux, $g = 1$, donc $a \equiv b \, [p]$.
    Si on impose en outre $a < p$ et $b < p$, cela implique $a = b$.

\medskip

\noindent\textbf{Démonstration :} 

D'après le théorème de Bachet-Bézout, on peut choisir deux entiers relatifs $l$ et $m$ tels que $g = l s + m (p-1)$. 
Puisque $a^s \equiv b^s \, [p]$, on a $a^{\abs{l} s} \equiv b^{\abs{l} s} \, [p]$.
En outre, d'après le petit théorème de Fermat, $a^{p-1} \equiv 1 \, [p]$ et $b^{p-1} \equiv 1 \, [p]$, donc $a^{\abs{m} (p-1)} \equiv 1 \, [p]$ et $b^{\abs{m} (p-1)} \equiv 1 \, [p]$.
Puisque $g > 0$, $s > 0$ et $p-1 > 0$, on ne peut avoir $l < 0$ et $m < 0$ (ce qui impliquerait $g < 0$).

\begin{itemize}[nosep]
    \item Si $l \geq 0$ et $m \geq 0$, on a $a^g = a^{\abs{l} s} \times a^{\abs{m} (p-1)}$, donc $a^g \equiv a^{\abs{l} s} \, [p]$.
        De même, $b^g \equiv b^{\abs{l} s} \, [p]$.
        Donc, $a^g \equiv b^g \, [p]$.
    \item Si $m < 0$, on a $l \geq 0$, $a^g \times a^{\abs{m} (p-1)} = a^{\abs{l} s}$ et $b^g \times b^{\abs{m} (p-1)} = b^{\abs{l} s}$.
        Donc, $a^g \times a^{\abs{m} (p-1)} \equiv b^g \times b^{\abs{m} (p-1)} \, [p]$.
        Donc, $a^g \equiv b^g \, [p]$.
    \item Si $l < 0$, on a $m \geq 0$, $a^g \times a^{\abs{l} s} = a^{\abs{m} (p-1)}$ et $b^g \times b^{\abs{l} s} = b^{\abs{m} (p-1)}$.
        Donc, $a^g \times a^{\abs{l} s} \equiv 1 \, [p]$ et $b^g \times b^{\abs{l} s} \equiv 1 \, [p]$. 
        Donc, $p$ divise $a^g \times a^{\abs{l} s} - b^g \times b^{\abs{l} s}$. 
        En outre, puisque $a^{\abs{l} s} \equiv b^{\abs{l} s} \, [p]$, on peut choisir un entier $k$ tel que $b^{\abs{l} s} = a^{\abs{l} s} + k p$.
        Donc, $p$ divise $a^g \times a^{\abs{l} s} - b^g \times a^{\abs{l} s}$~\footnote{
            En effet, on peut choisir un entier $q$ tel que $a^g \times a^{\abs{l} s} - b^g \times b^{\abs{l} s} = q p$, et donc $a^g \times a^{\abs{l} s} - b^g \times a^{\abs{l} s} = q p + b^g \times k \times p = (q + b^g \times k) \times p$.
        }%
        , égal à $(a^g \times - b^g \times) a^{\abs{l} s}$. 
        Puisque $p$ est premier et puisque $a$ n'est pas un multiple de $p$, $a^{\abs{l} s}$ n'en n'est pas un non plus.
        Donc, $p$ divise $a^g \times - b^g \times$.
\end{itemize}

\done

\medskip

\noindent\textbf{Remarque :} Le dernier résultat n'est pas vrai en général si $s$ et $p-1$ ne sont pas premiers entre eux.
    En effet, soit $g$ leurs pgcd et $l$ et $m$ deux entiers naturels tels que $s = l g$ et $p-1 = m g$, soit $r$ le reste de la division Euclidienne de $a^m$ par $p$ pour tout entier $a$ non multiple de $p$, on a $r^s \equiv 1 \, [p]$. 
    Puisque $g > 1$, $m < p-1$. 
    Puisque le groupe $(\mathbb{Z} \divslash (p \mathbb{Z}))^*$ est cyclique (voir section \ref{subsubsec:cyclicity_Z_pZ}), il admet au moins un générateur. 
    Choisissant ce générateur pour $a$, on a ne peut avoir $a^m \equiv 1 \, [p]$, donc $r$ est un élément de $[\![0, p-1]\!]$ distinct de $1$ tel que $r^s \equiv 1 \, [p]$.
