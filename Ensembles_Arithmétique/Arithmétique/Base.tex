\subsection{Écriture en base \texorpdfstring{$b$}{b}}
\label{sub:base}

Soit $b$ un entier naturel strictement supérieur à $1$. 

\bigskip 

\noindent\textbf{Théorème :} Soit $n \in \mathbb{N}$. Il existe un unique entier naturel $m$ et une unique séquence $(u_{m-1}, u_{m-2}, \dots, u_0)$ tels que $0 \leq u_i < b$ pour tout $i \in [\![0,n-1]\!]$, $u_{m-1} = 0$, et $\sum_{i=0}^{m-1} u_i b^{i} = n$. 

\medskip 

\noindent\textbf{Définition :} Pour $n \neq 0$, cette séquence est appelée \textit{écriture de $n$ en base $b$}. L'écriture de $0$ en base $b$ est $(0)$. On omettra parfois les parenthèses et virgules quand il n'y a pas de confusion possible. 

\medskip

\noindent\textbf{Démonstration :} 
On procède par recurrence forte. 
Soit $b$ un entier naturel strictement supérieur à $1$. 
L'écriture de $0$ en base $b$ existe et est uniquepar définition. 
Soit $n \in \mathbb{N}^*$ et supposons que tout entier naturel strictement inférieur à $n$ admet une unique écriture en base $b$. 
Distinguons deux cas selon que $n$ est ou non un multiple de $b$. 

Supposons d'abord qu'il n'en est pas un. 
Notons $(u_{l-1}, u_{l-2}, \dots, u_0)$ l'écriture de $n-1$ en base $b$, où $l \in \mathbb{N}^*$. 
L'entier naturel $u_0$ est le reste de la division euclidienne de $n-1$ par $b$, donc $u_0 < b-1$ (puisque, si $u_0 = b-1$, on aurait $n-1 \equiv b-1 \, [b]$ et donc $n \equiv 0 \, [b]$). 
Donc, $(u_{l-1}, u_{l-2}, \dots, u_1, u_0+1)$ satisfait la définition d'une écriture en base $b$. 
Puisque $\sum_{i=0}^l-1 u_i b^i = n-1$, $ (u_0 + 1) \sum_{i=1}^l-1 u_i b^i = n$, donc il s'agit d'une écriture de $n$ en base $b$. 

Montrons qu'elle est unique. 
Supposons avoir deux écritures de $n$ en base $b$, $(u_{l-1}, u_{l-2}, \dots, u_1, u_0)$ et $(v_{m-1}, v_{m-2}, \dots, v_1, v_0)$, où $l$ et $m$ sont deux entiers naturels non nuls. 
Alors, $u_0$ et $v_0$ doivent être égaux au reste de la division euclidienne de $n$ par $b$, et donc distincts de $0$. 
Donc, $(u_{l-1}, u_{l-2}, \dots, u_1, u_0-1)$ et $(v_{m-1}, v_{m-2}, \dots, v_1, v_0-1)$ sont deux écriturse en base $b$ de $n-1$. 
Par hypothèse de récurrence, elles doivent être identiques, et donc $(u_{l-1}, u_{l-2}, \dots, u_1, u_0) = (v_{m-1}, v_{m-2}, \dots, v_m, v_0)$. 

Supposons maintenant que $b$ divise $n$. 
Soit $q$ le quotient de la division enclidienne de $n$ par $b$. 
$q < n$, donc, par hypothèse de récurrence, il adment une unique écriture $(u_{l-1}, u_{l-2}, \dots, u_0)$ en base $b$, où $l \in \mathbb{N}^*$. 
Alors, $(u_{l-1}, u_{l-2}, \dots, u_0, 0)$ satisfait les conditions d'une écriture en base $b$. 
C'est celle de $\sum_{i=0}^{l-1} u_i b^{i+1} = b q = n$. 

Montrons qu'elle est unique. 
Supposons avoir deux écritures de $n$ en base $b$, $(u_{l-1}, u_{l-2}, \dots, u_1, u_0)$ et $(v_{m-1}, v_{m-2}, \dots, v_1, v_0)$, où $l$ et $m$ sont deux entiers naturels non nuls. 
Alors, $u_0$ et $v_0$ doivent être égaux au reste de la division euclidienne de $n$ par $b$, et donc égaux à $0$. 
On a donc: $n = \sum_{i=1}^{l-1} u_i b^{i} = \sum_{j=1}^{m-1} v_j b^{j}$. 
Soit $q = \sum_{i=1}^{l-1} u_i b^{i-1}$. 
On a aussi : $q = \sum_{j=1}^{m-1} v_j b^{j-1}$. 
Donc, $(u_{l-1}, u_{l-2}, \dots, u_1)$ et $(v_{m-1}, v_{m-2}, \dots, v_1)$ sont deux écritures en base $b$ de $q$. 
Par hypothèse de recurrence, elles sont identiques. 
Puisque $v_0 = u_0 = 0$, les deux écritures de $n$ sont dont identiques. 

Par récurrence forte, le résultat est donc vrai pour tout $n \in \mathbb{N}$.

\done

\bigskip

De fonctions Haskell donnant l'écriture d'un entier dans une base quelconque ou convertissant cette écriture en décimal sont données enappendice~\label{app:Haskell_baseb}.

