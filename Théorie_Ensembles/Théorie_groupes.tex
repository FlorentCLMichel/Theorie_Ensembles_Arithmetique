\subsection{Éléments de théorie des groupes}

Dans cette section, nous donnons quelques concepts de base de théorie des groupes. 

\subsubsection{Définitions}

\noindent\textbf{Définition (magma) :} Un \textit{magma} $\mathcal{M}$ est un couple formé par un ensemble $M$ et une \textit{loi de composition interne} $\cdot$ sur $M$ (parfois appelée \textit{opération}), c'est-à-dire une fonction de $M \times M$ vers $M$. 
    Si $a$ et $b$ sont deux éléments de $M$, on note $a \cdot b$ l'image de $(a,b)$ par $\cdot$.

\medskip

\noindent\textbf{Définition (élément neutre) :} Soit $(M,\cdot)$ un magma.
    Un élément $e$ de $M$ est dit \textit{élément neutre} si 
    \begin{equation*}
        \forall m \in M, \, e \cdot m = m \wedge m \cdot e = m.
    \end{equation*}
    Un magma admettant un élément neutre est dit \textit{unifère}.

\medskip

\noindent\textbf{Lemme :} Un magma admet au plus un élément neutre.

\medskip

\noindent\textbf{Démonstration :} Soit $(M,\cdot)$ un magma et $e$ et $f$ deux éléments identités pour ce magma.
    Puisque $e$ est un élément identité, $e \cdot f = f$.
    Puisque $f$ est un élément identité, $e \cdot f = e$.
    Par commutativité et transitivité de l'égalité, on en déduit $e = f$.

    \hfill \square

\medskip

\noindent\textbf{Définition (morphisme de magmas) :} Soit $\left(M, \cdot \right)$ et $\left(N, \ast \right)$ deux magmas.
    Une fonction $f$ de $M$ vers $N$ est dite \textit{morphisme de magmas} de $\left(M, \cdot \right)$ vers $\left(N, \ast \right)$ si elle satisfait : 
    \begin{equation*}
        \forall (a,b) \in M^2, \, f (a \cdot b) = f(a) \ast f(b) .
    \end{equation*}

\medskip

\noindent\textbf{Définition (isomorpisme de magmas) :} Soit $\mathcal{M}$ et $\mathcal{N}$ deux magmas.
    Un morphise de magmas $f$ de $\mathcal{M}$ vers $\mathcal{N}$ est dit \textit{isomorphisme de magmas} s'il est également une bijection.

\medskip

\noindent\textbf{Lemme :} L'image d'un élément neutre par un morphisme de magmas surjectif (et donc, en particulier, par un isomorphisme) est un élément neutre.

\medskip

\noindent\textbf{Démonstration :} Soit $\left(M, \cdot \right)$ et $\left(N, \ast \right)$ deux magmas et soit $f$ un morphisme surjectif du premier vers le second. 
    Soit $e$ un élément neutre de $\left(M, \cdot \right)$. 
    On a $f(e) \in N$. 
    Soit $y$ un élément de $N$. 
    Il s'agit de montrer que $f(e) \ast y = y$ et $y \ast f(e) = y$.

    Puisque $f$ est surjectif, on peut choisir un élément $x$ de $M$ tel que $f(x) = y$. 
    Donc, $f(e) \ast y = f(e) \ast f(x)$. 
    Puisque $f$ est un morphisme de magmas, $f(e) \ast f(x) = f(e \cdot x)$. 
    Puisque $e$ est un élément neutre pour $\cdot$, $e \cdot x = x$. 
    Donc, $f(e) \ast f(x) = f(x)$.
    Donc, $f(e) \ast y = y$. 

    De même, puisque $f$ est un morphisme de magmas, $f(x) \ast f(e) = f(x \cdot e)$. 
    Puisque $e$ est un élément neutre pour $\cdot$, $x \cdot e = x$. 
    Donc, $f(x) \ast f(e) = f(x)$.
    Donc, $y \ast f(x) = y$. 

    \hfill \square

\medskip

\noindent\textbf{Corrolaire :} Soit $\mathcal{M}$ et $\mathcal{N}$ deux magmas. 
    On suppose qu'il existe un morphisme de magmas surjectif de $\mathcal{M}$ vers $\mathcal{N}$. 
    Si $\mathcal{M}$ est unifère, alors $\mathcal{N}$ l'est aussi. 
    (Car l'image par le morphisme de l'élément neutre de $\mathcal{M}$ est un élément neutre pour $\mathcal{N}$.)

\medskip

\noindent\textbf{Lemme :} L'inverse d'un isomorphisme de magmas est un isomorphisme de magmas.

\medskip

\noindent\textbf{Démonstration :} Soit $\left(M, \cdot \right)$ et $\left(N, \ast \right)$ deux magmas et soit $f$ un isomorphisme du premier vers le second. 
    Soit $g$ l'inverse de $f$ (qui existe et est une bijection puisque $f$ est une bijection). 
    Montrons que $g$ est un morphisme de magmas. 
    Puisque $g$ est également une bijection, il s'agira alors d'un isomorphisme. 

    Soit $a$ et $b$ deux éléments de $N$. 
    Puisque $f$ est une bijection, elle est surjective, donc on peut choisir deux éléments $c$ et $d$ de $M$ tels que $a = f(c)$ et $b = f(d)$. 
    On a alors : $g(a) \cdot g(b) = g(f(c)) \cdot g(f(d))$.
    Puisque $g$ est l'inverse de $f$, cela donne : $g(a) \cdot g(b) = c \cdot d$. 

    Par ailleurs, $g(a \ast b) = g(f(c) \ast f(d))$. 
    Puisque $f$ est un morphisme de magmas on a $f(c) \ast f(d) = f (c \cdot d)$, donc cela donne $g(a \ast b) = g(f(c \cdot d))$.
    Puisque $g$ est l'inverse de $f$, cela donne : $g(a \ast b) = c \cdot d$. 
    Donc, $g(a \ast b) = g(a) \cdot g(b)$. 

    Cela étant vrai pour tous éléments $a$ et $b$ de $M$, on en déduit que $g$ est un morphisme de magmas, et donc un isomorphisme, de $\left(M, \cdot \right)$ vers $\left(N, \ast \right)$.

    \hfill \square

\medskip

\noindent\textbf{Définition (associativité) :} Soit $(M,\cdot)$ un magma.
    La loi de composition interne $\cdot$ est dite \textit{associative} si
    \begin{equation*}
        \forall a \in M, \, \forall b \in M, \, \forall c \in M , \,  
        (a \cdot b) \cdot c = a \cdot (b \cdot c).
    \end{equation*}
    Le magma $(M, \cdot)$ est alors dit \textit{associatif}. 
    Un magma associatif est aussi appelé \textit{demi-groupe}.

\medskip

\noindent\textbf{Lemme :} Soit $\mathcal{M}_1$ et $\mathcal{M}_2$ deux magmas.
    On suppose que $\mathcal{M}_1$ est associatif et qu'il existe un morphisme de magmas surjectif de $\mathcal{M}_1$ vers $\mathcal{M}_2$.
    Alors $\mathcal{M}_2$ est associatif.

\medskip

\noindent\textbf{Démonstration :} Soit $(M,\cdot)$ et $(N,\ast)$ deux magmas. 
    On suppose que le premier est associatif et qu'il existe un morphisme de magmas surjectif, $f$, du premier vers le second. 
    Soit $a$, $b$ et $c$ trois éléments de $N$. 
    Puisque $f$ est surjectif, on peut choisir trois éléments $d$, $e$ et $g$ de $M$ tels que $f(d) = a$, $f(e) = b$ et $f(g) = c$.
    On a alors : $(a \ast b) \ast c = (f(d) \ast f(e)) \ast f(g)$.
    Puisque $f$ est un morphisme de magmas, $(f(d) \ast f(e)) \ast f(g) = f(d \cdot e) \ast f(g) = f((d \cdot e) \cdot g)$.
    Puisque $\cdot$ est associative, $f((d \cdot e) \cdot g) = f(d \cdot (e \cdot g))$. 
    En utilisant à nouveau le fait que $f$ est un morphisme de magmas, on obtient : $f(d \cdot (e \cdot g)) = f(d) \ast f(e \cdot g)$ et $f(d) \ast f(e \cdot g)) = f(d) \ast (f(e) \ast f(g))$. 
    Enfin, puisque $d$, $e$ et $g$ sont des antécédents respectifs de $a$, $b$ et $c$ par $f$, on a $f(d) \ast (f(e) \ast f(g)) = a \ast (b \ast c)$. 
    En combinant ces formules et en utilisant la transitivité de l'égalité, il vient : $(a \ast b) \ast c = a \ast (b \ast c)$.

    \hfill \square

\medskip

\noindent\textbf{Définition (commutativité) :} Soit $(M,\cdot)$ un magma.
    La loi de composition interne $\cdot$ est dite \textit{commutative} si
    \begin{equation*}
        \forall a \in M, \, \forall b \in M,  
        a \cdot b = b \cdot a.
    \end{equation*}
    Le magma $(M, \cdot)$ est alors dit \textit{commutatif} ou \textit{abélien}.

\medskip

\noindent\textbf{Lemme :} Soit $\mathcal{M}_1$ et $\mathcal{M}_2$ deux magmas.
    On suppose que $\mathcal{M}_1$ est commutatif et qu'il existe un morphisme de magmas surjectif de $\mathcal{M}_1$ vers $\mathcal{M}_2$.
    Alors $\mathcal{M}_2$ est commutatif.

\medskip

\noindent\textbf{Démonstration :} Soit $(M,\cdot)$ et $(N,\ast)$ deux magmas. 
    On suppose que le premier est commutatif et qu'il existe un morphise de magmas surjectif, $f$, du premier vers le second. 
    Soit $a$ et $b$ deux éléments de $N$. 
    Puisque $f$ est surjectif, on peut choisir deux éléments $d$ et $e$ de $M$ tels que $f(d) = a$ et $f(e) = b$.
    On a alors : $a \ast b = f(d) \ast f(e)$.
    Puisque $f$ est un morphisme de magmas, $f(d) \ast f(e) = f(d \cdot e)$.
    Puisque $\cdot$ est commutative, $d \cdot e = e \cdot d$, donc $f(d \cdot e) = f(e \cdot d)$. 
    En utilisant à nouveau le fait que $f$ est un morphisme de magmas, on obtient : $f(e \cdot d) = f(e) \ast f(d)$. 
    Enfin, puisque $d$ et $e$ sont des antécédents respectifs de $a$ et $b$ par $f$, on a $f(e) \ast f(d) = b \ast a$. 
    En combinant ces formules et en utilisant la transitivité de l'égalité, il vient : $a \ast b = b \ast a$.

    \hfill \square

\medskip

\noindent\textbf{Définition (monoïde) :} Un magma unifère et associatif est appelé \textit{monoïde}.

\medskip

\noindent\textbf{Définition (monoïde abélien) :} Un magma unifère, associatif et abélien est dit \textit{monoïde abélien}.

\medskip

\noindent\textbf{Définition (morpisme de monoïdes) :} Un morphisme de magmas d'un monoïde vers un autre est dit \textit{morphisme de monoïdes}.

\medskip

\noindent\textbf{Définition (isomorpisme de monoïdes) :} Un isomorphisme de magmas d'un monoïde vers un autre est dit \textit{isomorphisme de monoïdes}.

\medskip

\noindent\textbf{Lemme :} L'inverse d'un isomorphisme de monoïdes est un isomorphisme de monoïdes.

\medskip

\noindent\textbf{Démonstration :} Conséquence directe du même résultat pour un isomorphisle de magmas.

\medskip

\noindent\textbf{Définition (inverse) :} Soit $(M,\cdot)$ un monoïde et $e$ son élément neutre. 
    Soit $m$ un élément de $M$. 
    Un élément $n$ de $M$ est dit \textit{inverse} de $m$ (pour $\cdot$) si $m \cdot n = e \wedge  n \cdot m = e$.
    
\medskip

\noindent\textbf{Lemme :} Soit $(M,\cdot)$ un monoïde et $m$ un élément de $M$.
    Alors, $m$ admet au plus un seul inverse pour $\cdot$.

\medskip

\noindent\textbf{Démonstration :} Soit $(M,\cdot)$ un monoïde et $e$ son élément neutre. 
    Soit $m$ un élément de $M$. 
    Soit $n$ et $o$ deux inverse de $m$ pour $\cdot$. 
    Alors, $(n \cdot m) \cdot o = e \cdot o = o$.
    Par ailleurs, $n \cdot (m \cdot o) = n \cdot e = n$.
    Puisque $\cdot$ est associative, $(n \cdot m) \cdot o = n \cdot (m \cdot o)$. 
    Donc, $o = n$.

    \hfill \square

\medskip

\noindent\textbf{Définition (groupe) :} Soit $(M,\cdot)$ un monoïde. 
    Si chaque élément de $M$ admet un inverse pour $\cdot$, alors $(M,\cdot)$ est appelé \textit{groupe}.

\medskip

\noindent\textbf{Définition (groupe abélien) :} Soit $(M,\cdot)$ un monoïde abélien. 
    Si chaque élément de $M$ admet un inverse pour $\cdot$, alors $(M,\cdot)$ est appelé \textit{groupe abélien}.

\medskip

\noindent\textbf{Définition (morphisme de groupes) :} Un morphisme de magmas d'un groupe vers un autre est dit \textit{morphisme de groupes}.

\medskip

\noindent\textbf{Définition (isomorphisme de groupes) :} Un isomorphisme de magmas d'un groupe vers un autre est dit \textit{isomorphisme de groupes}.

\medskip

\noindent\textbf{Lemme :} L'inverse d'un isomorphisme de groupes est un isomorphisme de groupes.

\medskip

\noindent\textbf{Démonstration :} Conséquence direct du même résultat pour un isomorphisme de magmas.

\medskip

\noindent\textbf{Définition (puissance) :} Soit $(M,\cdot)$ un monoïde abélien et $e$ son élément neutre. 
    On définit la suite de fonctions $\left( f_m \right)_{m \in \mathbb{N}}$ de $M$ vers $M$ par récurrence de la manière suivante : 
    \begin{itemize}[nosep]
        \item Pour tout élément $m$ de $M$, $f_0(m) = e$.
        \item Pour tout entier naturel $n$, pour tout élément $m$ de $M$, $f_{n+1}(m) = m \cdot f_n(m)$.
    \end{itemize}
    Soit $n$ un entier naturel et $m$ un élément de $M$. 
    On notera $m^n$ l'élément $f_n(m)$.

\medskip

\noindent\textbf{Définition (cyclicité) :} Un groupe anélien $(G, \cdot)$ est dit $\textit{cyclique}$ s'il existe un élément $g$ de $G$ tel que : 
    \begin{equation*}
        \forall x \in G, \, \exists n \in \mathbb{N}, \, g^n = x.
    \end{equation*}
    Un tel élément $g$ est dit \textit{générateur} du groupe.

\medskip

\noindent\textbf{Définition (sous-groupe) :} Soit $(G, \cdot)$ un groupe et $H$ un sous-ensemble non-vide de $G$.
    Si $(H, \cdot)$ est un groupe, alors il est dit \textit{sous-groupe} de $(G, \cdot)$.

\medskip

\noindent\textbf{Lemme :} Soit $(G, \cdot)$ un groupe et $H$ un sous-ensemble non-vide de $G$ tel que $(H, \cdot)$ est un groupe.
    Soit $e$ lélément neutre de $(G, \cdot)$. 
    Alors $e \in H$.

\medskip

\noindent\textbf{Démonstration :} Puisque $(H, \cdot)$ est un groupe, donc unifere, $H$ est non vide. 
    Soit $h$ un élément de $H$. 
    Puisque $(H, \cdot)$ est un groupe, l'inverse $l$ de $h$ appartient aussi à $H$, et donc $l \cdot h$ également. 
    Puisque $l$ est l'inverse de $H$, $l \cdot h = e$, ce aui conclut la preuve.

    \done

\subsubsection{Quelques résultats}

\noindent\textbf{Lemme :} Parmis les ensembles construits précédemment, 
    \begin{itemize}[nosep]
        \item $(\mathbb{N},+)$, $(\mathbb{N},\times)$ et $(\mathbb{Z},\times)$ sont des monoïdes abéliens,
        \item $(\mathbb{Z},+)$ est un groupe abélien.
    \end{itemize}

\medskip

\noindent\textbf{Démonstration :} Nous avons déjà démontré tous les éléments nécessaires. 
    En effet, 
    \begin{itemize}[nosep]
        \item Les operations $+$ et $\times$ sont des lois de compositions internes sur $\mathbb{N}$ et $\mathbb{Z}$, donc $(\mathbb{N},+)$, $(\mathbb{N},\times)$, $(\mathbb{Z},+)$ et $(\mathbb{Z},\times)$ sont des magmas.
        \item Les operations $+$ et $\times$ sont associatives et admettent chacune un élément neutre ($0$ pour la première et $1$ pour la seconde) dans $\mathbb{N}$ et $\mathbb{Z}$, donc $(\mathbb{N},+)$, $(\mathbb{N},\times)$, $(\mathbb{Z},+)$ et $(\mathbb{Z},\times)$ sont des monoïdes.
        \item Les operations $+$ et $\times$ sont commutatives, donc $(\mathbb{N},+)$, $(\mathbb{N},\times)$, $(\mathbb{Z},+)$ et $(\mathbb{Z},\times)$ sont des monoïdes abéliens.
        \item Pour tout élément $z$ de $\mathbb{Z}$, on a $z + (-z) = (-z) + z = 0$, donc $-z$ est un inverse de $z$ pour $+$. 
            $(\mathbb{Z},+)$ est donc un groupe abélien.
    \end{itemize}

\subsubsection{Groupe quotient}

\noindent\textbf{Définition :} Soit $(G, \cdot)$ un groupe abélien et $(H, \cdot)$ un sous-groupe de $(G, \cdot)$.
    On définit l'ensemble $G \divslash H$ comme l'ensemble des classes d'équivalences de $G$ pour la relation $R$ définie par : $\forall g \in G \, \forall g' \in G \, g R g' \Leftrightarrow g^{-1} \cdot g' \in H$, oú un exposant $-1$ indique l'inverse. 
    Pour tout élément $g$ de $G$, on note $\bar{g}$ la classe d'équivalence de $g$ pour $R$.
    On définit la loi de composition interne $\cdot$ sur $G \divslash H$ de la manière suivante : soit $c_1$ et $c_2$ deux éléments de $G \divslash H$, on peut choisir un éléments $g_1$ de $c_1$ et un élément $g_2$ de $c_2$ ; on pose alors $c_1 \cdot c_2 = \overline{g_1 \cdot g_2}$. 
    Alors, 
    \begin{itemize}[nosep]
        \item La loi de composition interne $\cdot$ sur $G \divslash H$ est bien définie.
        \item $(G \divslash H, \cdot)$ est un groupe abélien, appelé \textit{groupe quotient} de $(G, \cdot)$ et $(H, \cdot)$.
    \end{itemize}

\medskip

\noindent\textbf{Démonstration :} Pour le premier point, il s'agit de montrer que le résultat ne dépends pas du choix de $g_1$ et $g_2$.
    Soit $c_1$ et $c_2$ deux éléments de $G \divslash H$. 
    Soit $g_1$ et $g_3$ deux éléments de $c_1$ et $g_2$ et $g_4$ deux éléments de $c_2$.
    On a alors $g_1 R g_3$ et $g_2 R g_4$.
    On peut donc choisir deux éléments $h_1$ et $h_2$ de $H$ tels que $g_1^{-1} \cdot g_3 = h_1$ et $g_2^{-1} \cdot g_4 = h_2$.
    On a alors : $(g_1 \cdot g_2)^{-1} \cdot (g_3 \cdot g_4) = (g_2^{-1} \cdot g_1^{-1}) \cdot (g_3 \cdot g_4) = (g_1^{-1} \cdot g_3) \cdot \cdot (g_2^{-1} \cdot g_4) = h_1 \cdot h_2$. 
    Puisque $(H, \cdot)$ est un groupe, on en déduit que $(g_1 \cdot g_2)^{-1} \cdot (g_3 \cdot g_4) \in H$, donc $(g_1 \cdot g_2) R (g_3 \cdot g_4)$, et donc $\overline{g_1 \cdot g_2} = \overline{g_3 \cdot g_4}$.

    Montrons maintenant qu'il s'agit d'un groupe abélien : 
    \begin{itemize}[nosep]
        \item Par définition, $\cdot$ est une loi de composition interne sur $G \divslash H$.
        \item Commutativité : Soit $c_1$ et $c_2$ deux éléments de $G \divslash H$.
            Soit $g_1$ un élément de $c_1$ et $g_2$ un élément de $c_2$. 
            On a : $c_1 \cdot c_2 = \overline{g_1 \cdot g_2}$ et $c_2 \cdot c_1 = \overline{g_2 \cdot g_1}$.
            Puisque $(G, \cdot)$ est abélien, $g_1 \cdot g_2 = g_2 \cdot g_1$, donc $c_1 \cdot c_2 = c_2 \cdot c_1$.
        \item Soit $e$ l'élément neutre de $(G, \cdot)$. 
            Soit $c$ un élément de $G \divslash H$ et $g$ un élément de $c$. 
            On a : $c \cdot \bar{e} = \overline{g \cdot e} = \bar{g} = c$.
            Par commutativité, cela implique également $\bar{e} \times c = c$.
            Donc, $\bar{e}$ est un élément neutre de $G \divslash H$ pour $\cdot$.
        \item Associativité : Soit $c_1$, $c_2$ et $c_3$ trois éléments de $G \divslash H$.
            Soit $g_1$ un élément de $c_1$, $g_2$ un élément de $c_2$ et $g_3$ un élément de $c_3$.
            On a : $c_1 \cdot (c_2 \cdot c_3) = c_1 \cdot \overline{g_2 \cdot g_3} = \overline{g_1 \cdot (g_2 \cdot g_3)} = \overline{(g_1 \cdot g_2) \cdot g_3} = \overline{g_1 \cdot g_2} \cdot c_3 = (c_1 \cdot c_2) \cdot c_3$.
        \item Soit $c$ un élément de $G \divslash H$ et $g$ un élément de $c$. 
            Notons $g^{-1}$ l'inverse de $g$ et $e$ l'élément neutre de $(G,\cdot)$. 
            Alors, $c \cdot \bar{g^{-1}} = \overline{g \cdot g^{-1}} = \bar{e}$.
            Par commutativité, cela implique également $\bar{g^{-1}} \times c = \bar{e}$.
            Puisque $\bar{e}$ est un élément neutre pour $\cdot$, on conclut que $\bar{g^{-1}}$ est un inverse de $c$.
    \end{itemize}

    \done

\medskip
