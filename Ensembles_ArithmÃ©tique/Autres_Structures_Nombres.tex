\section{Autres ensembles remarquables}

\localtoc

\noindent Dans cette section, nous présentons brièvement d'autres ensembles de nombres remarquables ainsi que leur structure.

\subsection{Nombres rationnels}

\subsubsection{L'ensemble des nombres rationnels \texorpdfstring{$\mathbb{Q}$}{Q}}

Définissons (dans cette section seulement) la relation binaire $\mathcal{R}$ sur $\mathbb{Z} \times \mathbb{Z}^*$ de la manière suivante : si $x$ et $y$ sont deux éléments de cet ensemble, soit $a$, $b$, $c$ et $d$ les quatre (uniques) entiers relatifs (avec $b \neq 0$ et $d \neq 0$) tels que $x = (a, b)$ et $y = (c, d)$, alors $r \, \mathcal{R} \, y$ si et seulement si $a d = b c$. 
Autrement dit, 
\begin{equation*}
    \forall (a, b, c, d) \in \mathbb{Z} \times \mathbb{Z}^* \times \mathbb{Z} \times \mathbb{Z}^* \;
    (a, b) \, \mathcal{R} \, (c, d)
    \Leftrightarrow a d = b c .
\end{equation*}

\medskip

\noindent\textbf{Lemme :} La relation $\mathcal{R}$ est une relation d'équivalence. 

\medskip

\noindent\textbf{Démonstration :} Montrons que $\mathcal{R}$ satisfait les trois propriétés d'une relation d'équivalence.
    Dans toute cette démonstration, $x$, $y$ et $z$ sont trois éléments de $\mathbb{Z} \times \mathbb{Z}^*$ et $a$, $b$, $c$, $d$, $e$, $f$ sont six entiers tels que $b \neq 0$, $d \neq 0$, $f \neq 0$, $x = (a, b)$, $y = (c, d)$ et $z = (e, f)$.
    \begin{itemize}[nosep]
        \item \textit{Réflexivité :} Puisque la multiplication sur $\mathbb{Z}$ est commutative, $a b = b a$, donc $x \, \mathcal{R} \, x$.
        \item \textit{Symétrie :} Supposons $x \, \mathcal{R} \, y$. 
            Alors, $a d = b c$.
            Donc, $c b = d a$.
            Donc, $y \, \mathcal{R} \, x$.
        \item \textit{Transitivité :} Supposons $x \, \mathcal{R} \, y$ et $y \, \mathcal{R} \, z$.
            Alors, $a d = b c$ et $c f = d e$.
            Multiplions les deux membres de la première égalité par $f$ : il vient $a d f = b c f$, donc (en utilisant la seconde égalité) $a d f = b d e$, donc $d a f = d b e$.
            Puisque $d \neq 0$, on a donc $a f = b e$, et donc $x \, \mathcal{R} \, z$.
    \end{itemize}
    
    \done

\medskip

\noindent\textbf{Lemme :} Soit $a$ un entier et $b$ et $\lambda$ deux entiers non nuls. 
    Alors $(\lambda \times a, \lambda \times b) \mathrel{\mathcal{R}} (a, b)$.

\medskip

\noindent\textbf{Démonstration :} Puisque $\lambda$ et $b$ sont non nuls, $\lambda \times b$ l'est également.
    En outre, $a \times (\lambda \times b) = (a \times \lambda) \times b = (\lambda \times a) \times b$.

    \done

\medskip

On appelle \emph{ensemble des nombres rationnels}, noté $\mathbb{Q}$, l'ensemble des classes d'équivalences de $\mathcal{R}$, appelées \emph{nombres rationnels}, ou simplement \emph{rationnels}. 
\index{Nombre rationnel} \index{Rationnel} \sindex[isy]{$\mathbb{Q}$}
Autrement dit, $\mathbb{Q}$ est l'ensemble défini par : 
\begin{equation*}
    \mathbb{Q} = (\mathbb{Z} \times \mathbb{Z}^*) \divslash \mathcal{R} .
\end{equation*}
Soit $q$ un nombre rationnel et $a$ et $b$ deux entiers tels que $(a, b) \in q$. 
Alors, $a$ est appelé le \emph{numérateur} de la représentation $(a, b)$ et $b$ son \emph{dénominateur}.
\index{Numérateur} \index{Dénominateur}
Elle sera notée $a \divslash b$, ou $\frac{a}{b}$.
\sindex[isy]{$\divslash$}

Dans cette section, si $x$ est un élément de $\mathbb{Z} \times \mathbb{Z}^*$, on note $\bar{x}$ la classe d'équivalence de $x$ pour la relation $\mathcal{R}$, \emph{i.e.}
\begin{equation*}
    \bar{x} = \left\lbrace y \in \mathbb{Z} \times \mathbb{Z}^* \middle\vert x \mathrel{\mathcal{R}} y \right\rbrace .
\end{equation*}

\medskip

\noindent\textbf{Lemme :} Soit $q$ un nombre rationnel. 
Si $q$ admet une représentation dont le numérateur est nul, alors toutes ses représentations ont un numérateur nul et tous les éléments de $\mathbb{Z} \times \mathbb{Z}^*$ de première composante nulle sont des représentants de $q$. 

\medskip

\noindent\textbf{Démonstration :} On suppose que $q$ admet une représentation de numérateur nul. 
    On peut alors choisir un entier $b$ non nul tel que $(0, b) \in q$. 
    Soit $r$ une représentation de $q$.
    Soit $c$ et $d$ deux entiers tels que $(c, d)  = r$.
    Alors, $0 \times d = b \times c$.
    Donc, $b c = 0$.
    Puisque $b \neq 0$, on en déduit que $c = 0$.
    Donc, la représentation $r$ est de numérateur nul.

    Soit $r$ un élément de $\mathbb{Z} \times \mathbb{Z}^*$ de première composante nulle.
    On peut choisir un entier non nul $d$ tel que $r = (0, d)$.
    Puisque $0 \times d = 0 = 0 \times b$, on a $r \mathrel{\mathcal{R}} (0, b)$, et donc $r \in q$.

    \done

\medskip

\noindent\textbf{Corrolaire :} Il existe donc une seule classe d'équivalence contenant un représentant de numérateur nul. 
Ce nombre rationnel est aussi dit \emph{nul}, et parfois noté $0$ quand il n'y a pas d'ambiguité.

\medskip

\noindent\textbf{Notation :} On note $\mathbb{Q}^*$ l'ensemble $\mathbb{Q} \setminus \big\lbrace \overline{(0,1)} \big\rbrace$.

\medskip

\noindent\textbf{Remarque :} D'après le lemme précédent, tout représentant de tout élément de $\mathbb{Q}^*$ a un numérateur non nul.

\medskip

\noindent\textbf{Définition (représentation irréductible) :} 
    Soit $q$ un élément de $\mathbb{Q}$. 
    Soit $a$ un élément de $\mathbb{Z}$ et $b$ un élément de $\mathbb{Z}^*$ tels que $(a, b) \in q$.
    Alors, $(a, b)$ est dit \textit{représentation irreductible} de $q$ si et seulement si une des deux propriétés suivantes est satisfaite : 
    \begin{itemize}[nosep]
        \item $a = 0$ et $b = 1$.
        \item $a \neq 0$, $b > 0$ et $\abs{a}$ et $\abs{b}$ sont premiers entre eux.
    \end{itemize}
\index{Représentation irréductible}

\medskip

\noindent\textbf{Lemme :} Tout nombre rationnel a une unique représentation irréductible. 
    En outre, soit $r$ cette représentation irréductible, $a$ son numerateur et $b$ son dénominateur, un élément $(c, d)$ appartient à $q$ si et seulement si il existe un entier non nul $k$ tel que $c = k a$ et $d = k b$.

\medskip

\noindent\textbf{Démonstration :} Soit $x$ un nombre rationnel. 
\begin{itemize}[nosep]
    \item Si $x$ admet une représentation de numérateur nul, alors toutes ses représentations le sont.
        Donc, $x$ ne peut avoir de représentation de la forme $(a, b)$ avec $a \neq 0$.
        Il ne peut donc avoir que $(0, 1)$ comme représentation irréductible.
        Montrons que c'en est bien une. 
        Soit $r$ une représentation de $x$, $a$ son numérateur et $b$ son dénominateur. 
        Puisque $a = 0$, on a : $a \times 1 = 0 = b \times 0$. 
        Donc, $(0, 1)$ est bien une représentation de $x$, et donc sa représentation irréductible. 
        Par ailleurs, ses représentations sont toutes de la forme $(0, k)$, égal à $(0 \times k, 1 \times k)$, pour un entier non nul $k$, et tous les éléments de $\mathbb{Z} \times \mathbb{Z}^*$ de cette forme sont des représentations de $x$ puisque $1 \times 0 = 0 = 0 \times k$.
    \item Sinon, toutes les représentations de $x$ ont un numérateur non nul. 
        Soit $r$ une de ses représentations, $a$ son numerateur et $b$ son dénominateur. 
        Soit $l$ le pgcd de $\abs{a}$ et $\abs{b}$.
        On peut choisir deux entiers naturels $p$ et $x$ tels que $\abs{a} = p l$ et $\abs{b} = q l$.
        En outre, puisque $\abs{b}$ est non nul, $q$ est non nul, donc $(\mathrm{sgn}(a) \, \mathrm{sgn}(b) \, p, q)$ est un élément de $\mathbb{Z} \times \mathbb{Z}^*$. 
        En outre, $p \neq 0$ (puisque $a \neq 0$), donc $\mathrm{sgn}(a) \, \mathrm{sgn}(b) \, p \neq 0$, et $q > 0$. 
        Puisque les valeurs absolues du numérateur et du dénominateur (égales à $p$ et $q$) sont premières entre elles, il s'agit d'une représentation irreductible.
        En outre, $a q = \mathrm{sgn}(a) \abs{a} q = \mathrm{sgn}(a) p q l = \abs{b} \mathrm{sgn}(a) p = b \, \mathrm{sgn}(a) \, \mathrm{sgn}(b) p$, donc $(\mathrm{sgn}(a) \, \mathrm{sgn}(b) \, p, q) \mathrel{\mathcal{R}} (a, b)$, donc $(\mathrm{sgn}(a) \, \mathrm{sgn}(b) \, p, q)$ est une représentation de $x$.
        Montrons que toutes ses seprésentations sont de la forme $(\lambda \mathrm{sgn}(a) \mathrm{sgn}(b) p, \lambda q)$ pour un entier non nul $\lambda$ et que tous les éléments de $\mathbb{Z} \times \mathbb{Z}^*$ de cette forme sont des représentations de $x$.
        Cela montrera que $x$ n'admet aucune autre représentation irréductible (avec $\lambda \neq 1$) car le numérateur est négatif si $\lambda < 0$ ou les valeurs absolues des numérateur et dénominateur (égales à $\abs{\lambda} p$ et $\abs{\lambda} q$) ne pourraient être premières entre elles pour $\lambda > 1$ (ayant un diviseur commun $\lambda$ strictement supérieur à $1$).
        \begin{itemize}[nosep]
            \item \emph{Toutes les représentations de $x$ sont de cette forme :} Soit $r$ une représentation de $x$ et $c$ et $d$ deux entiers relatifs non nuls tels que $(c, d) = r$. 
                Alors, $d \, \mathrm{sgn}(a) \, \mathrm{sgn}(b) p = c q$.
                En prenant la valeur absolue, il vient : $\abs{d} p = \abs{c} q$. 
                Donc, $q$ divise $\abs{d} p$. 
                Puisque $q$ est premier avec $p$, on en déduit que $q$ divise $\abs{d}$.
                On peut donc choisir un entier naturel $k$ tel que $\abs{d} = k q$.
                Soit $\lambda$ l'entier relatif définit par $\lambda = \mathrm{sgn}(d) k$. 
                Alors, $d = \lambda q$.
                En outre, $d$ est non nul, donc $\lambda$ est non nul. 
                Puisque $d \, \mathrm{sgn}(a) \mathrm{sgn}(b) p = c q$, on a : $\lambda q \, \mathrm{sgn}(a) \, \mathrm{sgn}(b) p = c q$.
                Puisque $q$ est non nul, cela donne : $\lambda \, \mathrm{sgn}(a) \, \mathrm{sgn}(b) p = c$.
                Donc, $r = (\lambda \, \mathrm{sgn}(a) \, \mathrm{sgn}(b) p, \lambda q)$.
            \item \emph{Tous les éléments de cette forme sont des représentations de $x$ :} Soit $\lambda$ un entier non nul.
                Alors, $a \lambda q = \mathrm{sgn}(a) \abs{a} \lambda q = \mathrm{sgn}(a) \lambda p q l = \abs{b} \lambda \mathrm{sgn}(a) p = b \lambda \, \mathrm{sgn}(a) \, \mathrm{sgn}(b) p$.
                Donc, $(\lambda \, \mathrm{sgn}(a) \, \mathrm{sgn}(b) p, \lambda q)$ est une représentation de $x$.
        \end{itemize}
\end{itemize}

\done

\medskip

\noindent\textbf{Notation :} Pour tout entier $n$, le nombre rationnel $\overline{(n, 1)}$ est parfois simplement noté $n$ quand il n'y a pas d'ambiguité.

\subsubsection{Structure de corps}

\noindent\textbf{Définition :} On définit les trois opérations $+$ (\emph{addition}), $-$ (\emph{soustraction}) et $\times$ (\emph{multiplication}) de $\mathbb{Q} \times \mathbb{Q}$ vers $\mathbb{Q}$ et l'opération $\divslash$ (\emph{division}) de $\mathbb{Q} \times \mathbb{Q}^*$ vers $\mathbb{Q}$ de la manière suivante. \index{Addition}\index{Soustraction}\index{Multiplication}\index{Division} 
Soit $\bar{x}$ et $\bar{y}$ deux éléments de $\mathbb{Q}$. 
On peut choisir un représentant $x$ de $\bar{x}$ et un représentant $y$ de $\bar{y}$, puis quatre entiers $a$, $b$, $c$ et $d$ tels que $b$ et $d$ sont non nuls, $x = (a, b)$ et $y = (c, d)$.
Alors, 
\begin{itemize}[nosep]
    \item \sindex[isy]{$+$} $\bar{x} + \bar{y}$ est la classe d'équivalence de $(a \times d + b \times c, b \times d)$,
    \item \sindex[isy]{$-$} $\bar{x} - \bar{y}$ est la classe d'équivalence de $(a \times d - b \times c, b \times d)$,
    \item \sindex[isy]{$\times$} $\bar{x} \times \bar{y}$ est la classe d'équivalence de $(a \times c, b \times d)$,
    \item \sindex[isy]{$\div$}\sindex[isy]{$\divslash$} si $\bar{y} \in \mathbb{Q}^*$ (et donc $c \neq 0$), $\bar{x} \div \bar{y}$ (aussi noté $\bar{x} \divslash \bar{y}$) est la classe d'équivalence de $(a \times d, b \times c)$.%
        \footnote{Puisque $b \neq 0$ et $c \neq 0$, $b \times c \neq 0$, donc $(a \times d, b \times c)$ est bien un élément de $\mathbb{Z} \times \mathbb{Z}^*$.}
\end{itemize}
Quand il n'y a pas d'ambiguité, on pourra omettre le symbole $\times$. 
On notera parfois simplement par un signe $-$ (à gauche) la multiplication par $(-1) \divslash 1$.
Notons que 
\begin{itemize}[nosep]
    \item Le nombre rationnel $\bar{x} - \bar{y}$ est égal à $\bar{x} + \bar{z}$, où $\bar{z}$ est la lasse d'équivalence de $(-c, d)$.
    \item Si $c \neq 0$, $\bar{x} \div \bar{y}$ est égal à $\bar{x} \times \bar{z}$, où $\bar{z}$ est la lasse d'équivalence de $(d, c)$.
\end{itemize}

\medskip

Nous montrons ci-après que ces opérations sont bien définies, \emph{i.e.}, qu'elles ne dépendent pas des représentants choisis pour $\bar{x}$ et $\bar{y}$.
Notons d'abord quelques propriétés découlant directement de leurs définitions : avec les mêmes notations, 
\begin{itemize}[nosep]
    \item $(\bar{x} + \bar{y}) - \bar{y} = \bar{x}$,
    \item $(\bar{x} - \bar{y}) + \bar{y} = \bar{x}$,
    \item si $\bar{y} \in \mathbb{Q}^*$, $(\bar{x} \times \bar{y}) \divslash \bar{y} = \bar{x}$,
    \item si $\bar{y} \in \mathbb{Q}^*$, $(\bar{x} \divslash \bar{y}) \times \bar{y} = \bar{x}$,
\end{itemize}

\medskip

\noindent\textbf{Démonstration :} Avec les mêmes notations, 
\begin{equation*}
    (\bar{x} + \bar{y}) - \bar{y} 
        = \frac{a \times d + b \times c}{b \times d} - \frac{c}{d}
        = \frac{(a \times d + b \times c) \times d - c \times (b \times d)}{(b \times d) \times d}
        = \frac{a \times (d \times d)}{b \times (d \times d)}
        = \frac{a}{b} 
        = \bar{x}.
\end{equation*}
(Où on a utilisé le fait que $d \times d \neq 0$ puisque $d \neq 0$.)
De même, 
\begin{equation*}
    (\bar{x} - \bar{y}) + \bar{y} 
        = \frac{a \times d - b \times c}{b \times d} + \frac{c}{d}
        = \frac{(a \times d - b \times c) \times d + c \times (b \times d)}{(b \times d) \times d}
        = \frac{a \times (d \times d)}{b \times (d \times d)}
        = \frac{a}{b} 
        = \bar{x} .
\end{equation*}
Si $\bar{y}$, est non nul, on a aussi : 
\begin{equation*}
    (\bar{x} \times \bar{y}) \divslash \bar{y} 
        = \frac{a \times c}{b \times d} \divslash \frac{c}{d}
        = \frac{(a \times c) \times d}{(b \times d) \times c} 
        = \frac{a \times (c \times d)}{b \times (c \times d)} 
        = \frac{a}{b}
        = \bar{x}.
\end{equation*}
(où on a utilisé le fait que $c \times d$ est non nul puisque ni $c$ ni $d$ n'est nul)
et
\begin{equation*}
    (\bar{x} \divslash \bar{y}) \times \bar{y} 
        = \frac{a \times d}{b \times c} \times \frac{c}{d}
        = \frac{(a \times d) \times c}{(b \times c) \times d}
        = \frac{a \times (c \times d)}{b \times (c \times d)} 
        = \frac{a}{b}
        = \bar{x}.
\end{equation*}

\done

\medskip

\noindent\textbf{Lemme :} Avec les notations ci-dessus, ces quatre définitions sont indépendantes des choix des représentants de $\bar{x}$ et $\bar{y}$.

\medskip

\noindent\textbf{Démonstration :} Avec les mêmes notations, soit $x$ et $x'$ deux représentants de $\bar{x}$, $y$ et $y'$ deux représentants de $\bar{y}$, et $a$, $a'$, $b$, $b'$, $c$, $c'$, $d$ et $d'$ huit entiers tels que $b \neq 0$, $b' \neq 0$, $d \neq 0$, $d' \neq 0$, $x = (a, b)$, $x' = (a', b')$, $y = (c, d)$ et $y' = (c', d')$. 
Il s'agit de montrer que les nombres rationnels obtenus en calculant $\bar{x} + \bar{y}$, $\bar{x} - \bar{y}$, $\bar{x} \times \bar{y}$ et, si $\bar{y}$ est non nul, $\bar{x} \divslash \bar{y}$ en prenant soit les représentants $x$ et $y$ soit $x'$ et $y'$ sont identiques.
Pour ce faire, on utilise les égalités : $a \times b' = a' \times b$ et $c \times d' = c' \times d$ (reflétant les prédicats $(a, b) \mathrel{\mathcal{R}} (a', b')$ et $(c, d) \mathrel{\mathcal{R}} (c', d')$). 

\noindent\emph{Addition :}
    \begin{equation*}
        \begin{aligned}
            (a \times d + b \times c) \times (b' \times d')
            & = (a \times d \times b' \times d' + b \times c \times b' \times d')
            = (a \times b' \times d \times d' + c \times d' \times b \times b') \\
            & {} = (a' \times b \times d \times d' + c' \times d \times b \times b') 
            = (a' \times d' \times b \times d + b' \times c' \times b \times d) \\
            & {} = (a' \times d' + b' \times c') \times (b \times d) .
        \end{aligned}
    \end{equation*}
    Donc, $(a \times d + b \times c, b \times d) \mathrel{\mathcal{R}} (a' \times d' + b' \times c', b' \times d')$. 
    Cela montre que $\overline{(a \times d + b \times c, b \times d)} = \overline{(a' \times d' + b' \times c', b' \times d')}$, et donc que le résultat de l'addition est le même que l'on choisisse les représentants $x$ et $y$ ou les représentants $x'$ et $y'$.

\noindent\emph{Soustraction :}
    \begin{equation*}
        \begin{aligned}
            (a \times d - b \times c) \times (b' \times d')
            & = (a \times d \times b' \times d' - b \times c \times b' \times d')
            = (a \times b' \times d \times d' - c \times d' \times b \times b') \\
            & {} = (a' \times b \times d \times d' - c' \times d \times b \times b') 
            = (a' \times d' \times b \times d - b' \times c' \times b \times d) \\
            & {} = (a' \times d' - b' \times c') \times (b \times d) .
        \end{aligned}
    \end{equation*}
    Donc, $(a \times d - b \times c, b \times d) \mathrel{\mathcal{R}} (a' \times d' - b' \times c', b' \times d')$. 
    Cela montre que $\overline{(a \times d - b \times c, b \times d)} = \overline{(a' \times d' - b' \times c', b' \times d')}$, et donc que le résultat de la soustraction est le même que l'on choisisse les représentants $x$ et $y$ ou les représentants $x'$ et $y'$.

\noindent\emph{Multiplication :}
    \begin{equation*}
        (a \times c) \times (b' \times d')
        = (a \times b') \times (c \times d')
        = (a' \times b) \times (c' \times d)
        = (a' \times c') \times (b \times d) .
    \end{equation*}
    Donc, $(a \times c, b \times d) \mathrel{\mathcal{R}} (a' \times c', b' \times d')$. 
    Cela montre que $\overline{(a \times c, b \times d)} = \overline{(a' \times c', b' \times d')}$, et donc que le résultat de la multiplication est le même que l'on choisisse les représentants $x$ et $y$ ou les représentants $x'$ et $y'$.

\noindent\emph{Division :} Supposons $\bar{y}$ non nul. Alors, $c \neq 0$ et $c' \neq 0$.
    On a :
    \begin{equation*}
        (a \times d) \times (b' \times c')
        = (a \times b') \times (c' \times d)
        = (a' \times b) \times (c \times d')
        = (a' \times d') \times (b \times c) .
    \end{equation*}
    Donc, $(a \times d, b \times c) \mathrel{\mathcal{R}} (a' \times d', b' \times c')$. 
    Cela montre que $\overline{(a \times d, b \times c)} = \overline{(a' \times d', b' \times c')}$, et donc que le résultat de la division est le même que l'on choisisse les représentants $x$ et $y$ ou les représentants $x'$ et $y'$.

\done

\medskip

\noindent\textbf{Lemme :} Soit $p$ et $q$ deux nombres rationnels tels que $p - q = 0$.
    ALors, $p = q$.

\medskip

\noindent\textbf{Démonstration :} Soit $a$, $b$, $c$ et $d$ quatre entiers relatifs tels que $b \neq 0$, $d \neq 0$, $p = a \divslash b$ et $q = c \divslash d$.
    Si $p - q = 0$, alors $(a d - b c) \divslash (b d) = 0$, donc $a d = b c$, donc $p = q$.

    \done

\medskip

\noindent\textbf{Lemme :} Soit $a$ et $b$ deux entiers et $c$ un entier naturel non nul.
    Alors,
    \begin{equation*}
        \frac{a}{c} + \frac{b}{c} = \frac{a + b}{c}.
    \end{equation*}

\medskip

\noindent\textbf{Démonstration :} On a : 
    \begin{equation*}
        \frac{a}{c} + \frac{b}{c} 
        = \frac{(a \times c) + (b \times c)}{c \times c}
        = \frac{(a + b) \times c}{c \times c}.
    \end{equation*}
    Puisque $c \neq 0$ on a $((a + b) \times cn c \times c) \mathrel{\mathcal{R}} (a \times b, c)$, donc cela donne 
    \begin{equation*}
        \frac{a}{c} + \frac{b}{c} 
        = \frac{a + b}{c}.
    \end{equation*}

    \done

\medskip

\noindent\textbf{Lemme :} Les opérations $+$ et $\times$ sont commutatives.

\medskip

\noindent\textbf{Démonstration :} Découle directement de la commutativité des opérations $+$ et $\times$ sur $\mathbb{Z}$. 
    Montrons cependant cela explicitement par soucis de complétude. 
    Soit $\bar{x}$ et $\bar{y}$ deux éléments de $\mathbb{Q}$. 
    Soit $x$ un représentant de $\bar{x}$, $y$ un représentant de $\bar{y}$, et $a$, $b$, $c$, $d$ quatre entiers tels que $b \neq 0$, $d \neq 0$, $x = (a, b)$ et $y = (c, d)$.
    On a : 
    \begin{equation*}
        \bar{x} + \bar{y} 
            = \frac{a}{b} + \frac{c}{d} 
            = \frac{(a \times d) + (b \times c)}{b \times d}
            = \frac{(b \times c) + (a \times d)}{d \times b}
            = \frac{(c \times b) + (d \times a)}{d \times b}
            = \frac{c}{d} + \frac{a}{b}
            = \bar{y} + \bar{x}
    \end{equation*}
    et 
    \begin{equation*}
        \bar{x} \times \bar{y} 
            = \frac{a}{b} \times \frac{c}{d} 
            = \frac{a \times c}{b \times d}
            = \frac{c \times a}{d \times b}
            = \frac{c}{d} \times \frac{a}{b}
            = \bar{y} \times \bar{x} .
    \end{equation*}

\done

\medskip

\noindent\textbf{Lemme :} Le triplet $(\mathbb{Q}, +, \times)$ est un corps, d'éléments neutres $0$ pour $+$ et $1$ pour $\times$.

\medskip

\noindent\textbf{Démonstration :} Montrons qu'il satisfait toutes les propriétés requises.
\begin{itemize}[nosep]
    \item Montrons déjà que $(\mathbb{Q}, +)$ est un groupe abélien d'élément neutre $0$.
        \begin{itemize}[nosep]
            \item Puisque $+$ est une loi de composition interne sur $\mathbb{Q}$, $(\mathbb{Q}, +)$ est un magma.
            \item \emph{L'opération $+$ est associative :} 
                Soit $\bar{x}$, $\bar{y}$ et $\bar{z}$ trois éléments de $\mathbb{Q}$. 
                On peut choisir six entiers $a$, $b$, $c$, $d$, $e$, $f$ tels que $b \neq 0$, $d \neq 0$, $f \neq 0$, $\bar{x} = a \divslash b$, $\bar{y} = c \divslash d$ et $\bar{z} = e \divslash f$.
                On a alors : 
                \begin{equation*} \begin{aligned}
                    (\bar{x} + \bar{y}) + \bar{z} 
                    & = \Big( \frac{a}{b} + \frac{c}{d} \Big) + \frac{e}{f}
                    = \frac{(a \times d) + (c \times b)}{b \times d} + \frac{e}{f}
                    = \frac{((a \times d) + (c \times b)) \times f + e \times (b \times d)}{(b \times d) \times f} \\
                    & = \frac{a \times (d \times f) + b \times (c \times f) + b \times (e \times d)}{b \times (d \times f)} 
                    = \frac{a \times (d \times f) + b \times ((c \times f) + (e \times d))}{b \times (d \times f)} \\
                    & = \frac{a}{b} + \frac{(c \times f) + (e \times d)}{d \times f} 
                    = \frac{a}{b} + \Big( \frac{c}{d} + \frac{e}{f} \Big)
                    = \bar{x} + (\bar{y} + \bar{z}) .
                \end{aligned} \end{equation*}
            \item L'opération $+$ est commutative (voir ci-dessus).
            \item \emph{Le nombre rationnel $0$ est neutre pour $+$ :} 
                Soit $\bar{x}$ un nombre rationnel, $x$ un de ses représentants et $a$ et $b$ deux entiers tels que $b \neq 0$ et $x = (a, b)$. 
                On a : 
                \begin{equation*}
                    0 + \bar{x} = \frac{0}{1} + \frac{a}{b}
                        = \frac{(0 \times b) + (1 \times a)}{1 \times b}
                        = \frac{0 + a}{b}
                        = \frac{a}{b}
                        = \bar{x}.
                \end{equation*}
                Puisque $+$ est commutative, cela implique également $\bar{x} + 0 = \bar{x}$.
            \item \emph{Tout élément de $\mathbb{Q}$ admet un inverse pour $+$ :} 
                Soit $\bar{x}$ un nombre rationnel et $x$ un de ses représentants. 
                On peut choisir deux entiers $a$ et $b$ tels que $b \neq 0$ et $x = (a, b)$.
                Soit $\bar{y}$ le nombre rationnel admettant $(-a, b)$ pour représentant.
                \begin{equation*}
                    \bar{x} + \bar{y} = \frac{a}{b} + \frac{-a}{b}
                        = \frac{(a \times b) + (b \times (-a))}{b \times b}
                        = \frac{(a + (-a)) \times b}{b \times b}
                        = \frac{0 \times b}{b \times b}
                        = \frac{0}{b \times b}
                        = 0 .
                \end{equation*}
                Puisque l'opération $+$ est commutative, cela montre également que $\bar{y} + \bar{x} = 0$, et donc que $\bar{y}$ est un inverse de $\bar{x}$ pour $+$.
                \end{itemize}
    \item Montrons que $(\mathbb{Q}, \times)$ est un magma associatif :
        \begin{itemize}[nosep]
            \item Puisque $\times$ est une loi de composition interne sur $\mathbb{Q}$, $(\mathbb{Q}, \times)$ est un magma.
            \item \emph{L'opération $\times$ est associative :} 
                Cette propriété découle directement de l'associativité de la multiplication d'entiers. 
                Montrons cela explicitement. 
                Soit $\bar{x}$, $\bar{y}$ et $\bar{z}$ trois éléments de $\mathbb{Q}$. 
                On peut choisir six entiers $a$, $b$, $c$, $d$, $e$, $f$ tels que $b \neq 0$, $d \neq 0$, $f \neq 0$, $\bar{x} = a \divslash b$, $\bar{y} = c \divslash d$ et $\bar{z} = e \divslash f$.
                On a alors : 
                \begin{equation*} \begin{aligned}
                    (\bar{x} \times \bar{y}) \times \bar{z} 
                    & = \Big( \frac{a}{b} \times \frac{c}{d} \Big) \times \frac{e}{f}
                    = \frac{a \times c}{b \times d} \times \frac{e}{f}
                    = \frac{(a \times c) \times e}{(b \times d) \times f} \\
                    & = \frac{a \times (c \times e)}{b \times (d \times f)} 
                    = \frac{a}{b} \times \frac{c \times e}{d \times f} 
                    = \frac{a}{b} \times \Big( \frac{c}{d} \times \frac{e}{f} \Big)
                    = \bar{x} \times (\bar{y} \times \bar{z}) .
                \end{aligned} \end{equation*}
        \end{itemize}
    \item L'opération $\times$ est commutative (voir ci-dessus). 
    \item \emph{L'opération $\times$ est distributive sur $+$ :} 
        Soit $\bar{x}$, $\bar{y}$ et $\bar{z}$ trois éléments de $\mathbb{Q}$. 
        On peut choisir six entiers $a$, $b$, $c$, $d$, $e$, $f$ tels que $b \neq 0$, $d \neq 0$, $f \neq 0$, $\bar{x} = a \divslash b$, $\bar{y} = c \divslash d$ et $\bar{z} = e \divslash f$.
        On a alors : 
        \begin{equation*} \begin{aligned}
            \bar{x} \times (\bar{y} + \bar{z})
            & = \frac{a}{b} \times \Big( \frac{c}{d} + \frac{e}{f} \Big)
            = \frac{a}{b} \times \frac{(c \times f) + (e \times d)}{d \times f}
            = \frac{a \times ((c \times f) + (e \times d))}{b \times (d \times f)}
            = \frac{(a \times (c \times f)) + (a \times (e \times d))}{b \times (d \times f)} \\
            & = \frac{a \times (c \times f)}{b \times (d \times f)} + 
                \frac{a \times (e \times d)}{b \times (d \times f)}
            = \frac{(a \times c) \times f}{(b \times d) \times f} + 
                \frac{(a \times e) \times d}{(b \times f) \times d}
            = \frac{a \times c}{b \times d} + 
                \frac{a \times e}{b \times f}
            = \Big( \frac{a}{b} \times \frac{c}{d} \Big)
                + \Big( \frac{a}{b} \times \frac{e}{f} \Big) \\
            & = (\bar{x} \times \bar{y}) + (\bar{x} \times \bar{z}) .
        \end{aligned} \end{equation*}
        Puisque $\times$ est commutative, cela montre également (en échangeant les rôles de $\bar{x}$ et $\bar{z}$) que $(\bar{x} + \bar{y}) \times \bar{z} = (\bar{x} \times \bar{z}) + (\bar{y} \times \bar{z})$ pour tous rationnels $\bar{x}$, $\bar{y}$, $\bar{z}$.%
            \footnote{En effet, puisque l'opération $\times$ est commutative, on a : $(\bar{x} + \bar{y}) \times \bar{z} = \bar{z} \times (\bar{x} + \bar{y}) = (\bar{z} \times \bar{x}) + (\bar{z} \times \bar{y}) = (\bar{x} \times \bar{z}) + (\bar{y} \times \bar{z})$.}
    \item \emph{Le nombre rationnel $1$ est neutre pour $\times$ :} 
        Soit $\bar{x}$ un nombre rationnel, $x$ un de ses représentants et $a$ et $b$ deux entiers tels que $b \neq 0$ et $x = (a, b)$. 
        On a : 
        \begin{equation*}
            1 \times \bar{x} = \frac{1}{1} \times \frac{a}{b}
                = \frac{1 \times a}{1 \times b}
                = \frac{a}{b}
                = \bar{x}.
        \end{equation*}
        Puisque $\times$ est commutative, cela implique également $\bar{x} \times 1 = \bar{x}$.
    \item L'anneau $(Q, +, \times)$ est non nul puisque $0 \neq 1$ (puisque le nombre rationnel $1$ a pour représentant $(1,1)$, qui n'est pas un représentant de $0$ puisque son numérateur n'est pas nul).
    \item \emph{Tout élément de $\mathbb{Q}$ distinct de $0$ admet un inverse pour $\times$ :} 
        Soit $\bar{x}$ un nombre rationnel non nul. 
        Soit $x$ un de ses représentants. 
        On peut choisir deux entiers $a$ et $b$ tels que $b \neq 0$ et $x = (a, b)$.
        Puisque $\bar{x}$ est non nul, $a \neq 0$. 
        Soit $\bar{y}$ le nombre rationnel admettant $(b, a)$ pour représentant.
        On a : $\bar{x} \times \bar{y} = \overline{(a, b)} \times \overline{(b, a)} = \overline{(a \times b, b \times a)} = \overline{(a \times b, a \times b)} = \overline{(1 \times (a \times b), 1 \times (a \times b))}$.
        Puisque $a$ et $b$ sont non nuls, $a \times b$ l'est également.
        Donc, $(1 \times (a \times b), 1 \times (a \times b)) \mathrel{\mathcal{R}} (1, 1)$.
        Donc, $\bar{x} \times \bar{y} = 1$.
        Puisque l'opération $\times$ est commutative, cela montre également que $\bar{y} \times \bar{x} = 1$, et donc que $\bar{y}$ est un inverse de $\bar{x}$ pour $\times$.
\end{itemize}

\medskip

\noindent\textbf{Remarque :} Sauf mention contraire, on parle généralement d'inverse pour désigner l'inverse pour la multiplication. 

\medskip

\noindent\textbf{Remarque :} Soit $x$ un élément de $\mathbb{Q}^*$. 
    Alors l'inverse de $x$ est $1 \divslash x$, aussi noté $x^{-1}$. 

\done

\subsubsection{Puissances}

\noindent\textbf{Définition :} On définit par récurrence les \emph{puissances entières positives} d'un nombre rationnel $x$ par :
    \begin{itemize}[nosep]
        \item $x^0 = 1$,
        \item pour tout entier naturel $n$, $x^{n+1} = x \times x^n$.
    \end{itemize}
    Notons que $x^1 = x$ et $x^n \neq 0$ pour tout entier naturel $n$ (se montre facilement par récurrence).
    \index{Puissance}
    Pour tout entier naturel non nul $n$, on définit la \emph{puissance entière négative} $x^{-n}$ comme égale à l'inverse de $x^n$.
    \index{Puissance négative}

\medskip

\noindent\textbf{Lemme :} Soit $q$ un nombre rationnel et $n$ un entier naturel.
    Si $q \neq 0$, alors $q^n \neq 0$.

\medskip

\noindent\textbf{Démonstration :} On procède par récurrence sur $n$.
    Pour $n = 0$, on a $q^n = 1$, donc $q^n \neq 0$.
    
    Supposons le résultat vrai au rang $n$. 
    Alors, $q^{n+1} = q^n \times q$.
    Puisque $q^n \neq 0$ et $q \neq 0$ (et puisque $(\mathbb{Q}, +, \times)$ est un corps), on en déduit que $q^{n+1} \neq 0$.

    Par récurrence, le résultat attendu est vrai pour tout entier naturel $n$

\done

\medskip

\noindent\textbf{Lemme :} Soit $q$ un nombre rationnel et $n$ et $m$ deux entiers. 
    On suppose $q \neq 0$ ou $n \geq 0$ et $m \geq 0$.
    Alors, 
    \begin{itemize}[nosep]
        \item $q^n \times q^m = q^{n+m}$,
        \item $(q^n)^m = q^{n m}$.
    \end{itemize}

\medskip

\noindent\textbf{Démonstration :}\label{demo:rel_puissances_q} 
Notons que, sous les hypothèses de l'énoncé, les nombres $q^n$, $q^m$, $q^{n+m}$ et $q^{n m}$ sont bien définis.

On propose d'abord de montrer la première propriété pour toutes valeurs positives de $m$ par récurrence. 
Pour $m = 0$, on a : $q^n \times q^m = q^n \times 1 = q^n = q^{n + 0} = q^{n+m}$. 
Supposons le résultat attendu vrai au rang $m$.
Alors, $q^n \times q^{m+1} = q^n \times (q^m \times q) = (q^n \times q^m) \times q = q^{n+m} \times q = q^{(n+m) + 1} = q^{n + (m+1)}$.
Le résultat attendu est donc vrai au rang $m+1$. 
Par récurrence, il l'est pour tout entier naturel $m$.

Supposons maintenant $m < 0$. 
Alors, $q \neq 0$, donc, $q^{-m} \neq 0$.
Or, $q^{n+m} \times q^{-m} = q^{(n+m)-m} = q^n = q^n \times (q^m \times q^{-m}) = (q^n \times q^m) \times q^{-m}$. 
Donc, $q^{n+m} = q^n \times q^m$.

Montrons maintenant la second propriété, d'abord pour les valeurs positives de $m$.
Pour $m = 0$, on a : $(q^n)^m = (q^n)^0 = 1 = q^0 = q^{n \times 0} = q^{n m}$. 
Supposons le résultat attendu vrai au rang $m$.
Alors, $(q^n)^{m+1} = (q^n)^m \times q^n = q^{n m} \times q^n = q^{(n m) + n} = q^{(n+1) m}$.
Le résultat attendu est donc vrai au rang $m+1$. 
Par récurrence, il l'est pour tout entier naturel $m$.

Supposons maintenant que $m$ est strictement négatif.
Alors, $-m$ est strictement positif. 
Donc, $(q^n)^{-m} = q^{n \times (-m)} = q^{- n m}$.
Or, par définition, $(q^n)^m = 1 \divslash ((q^n)^{-m})$. 
Donc, $(q^n)^m = 1 \divslash q^{- n m} = q^{n m}$. 

\done

\subsubsection{Relation d'ordre}

\noindent\textbf{Définition :} On définit la relation $\leq$ sur $\mathbb{Q}$ de la manière suivante. 
    Soit $x$ et $y$ deux éléments de $\mathbb{Q}$ et $a$, $b$, $c$, $d$ quatre entiers tels que $b \neq 0$, $d \neq 0$, $x = a \divslash b$ et $y = c \divslash d$.
    On pose alors $x \leq y$ si et seulement si $a \, b \, d^2 \leq c \, d \, b^2$.
    Un nombre rationnel $q$ est dit \emph{positif} si $0 \leq q$ et \emph{négatif} si $q \leq 0$.

\medskip

\noindent\textbf{Lemme :} Cette définition ne dépend pas du choix des entiers $a$, $b$, $c$ et $d$.

\medskip

\noindent\textbf{Démonstration :} 
    Montrons que, avec les mêmes notations, l'inégalité est satisfaite si et seulement si elle l'est en choisissant les représentatios irréductibles de $x$ et $y$.
    Notons $a_0$ le numérateur de la représentation irréductible de $x$, $b_0$ son dénominateur, $c_0$ le numérateur de la représentation irréductible de $y$ et $d_0$ son dénominateur.
    On peut choisir deux entiers relatifs non nuls $\lambda$ et $\eta$ tels que $a = \lambda \, a_0$, $b = \lambda \, b_0$, $c = \eta \, c_0$, et $d = \eta \, d_0$.
    Donc, $a \, b \, d^2 = \lambda^2 \, \eta^2 \, a_0 \, b_0 \, d_0^2$ et $c \, d \, b^2 = \lambda^2 \, \eta^2 \, c_0 \, d_0 \, b_0^2$.
    Puisque $\lambda$ et $\eta$ sont non nuls, $\lambda^2$ et $\eta^2$ sont strictement positifs, donc $\lambda^2 \, \eta^2$ également.
    Donc, $a \, b \, d^2 \leq c \, d \, b^2$ si et seulement si $a_0 \, b_0 \, d_0^2 \leq c_0 \, d_0 \, b_0^2$.
    
    \done

\medskip

\noindent\textbf{Remarque :} Avec les notations ci-dessus, 
\begin{itemize}[nosep]
    \item $a \divslash b$ est positif si et seulement si $a = 0$ ou $a$ et $b$ sont de même signe ;
    \item $a \divslash b$ est négatif si et seulement si $a = 0$ ou $a$ et $b$ sont de signes opposés.
\end{itemize}

\medskip

\noindent\textbf{Lemme :} La relation ainsi définie est une relation d'ordre total sur $\mathbb{Q}$.

\medskip

\noindent\textbf{Démonstration :} Ce résultat découle essentiellement du fait que $\leq$ est une relation d'ordre total sur $\mathbb{Z}$.
    Montrons cela explicitement.
    Dans cette preuve, $x$, $y$ et $z$ sont trois éléments de $\mathbb{Q}$ arbitraires et $a$, $b$, $c$, $d$, $e$, $f$ sont six entiers tels que $b \neq 0$, $d \neq 0$, $f \neq 0$, $x = a \divslash b$, $y = c \divslash d$ et $z = e \divslash f$.
    \begin{itemize}[nosep]
        \item On a $a \, b \, d^2 = a \, b \, d^2$, donc $a \, b \, d^2 \leq a \, b \, d^2$, donc $x \leq x$.
        \item Si $x \leq y$ et $y \leq x$, alors $a \, b \, d^2 \leq c \, d \, b^2$ et $c \, d \, b^2 \leq a \, b \, d^2$, donc $a \, b \, d^2 = c \, d \, b^2$.
            Donc, $(a \times d) \times (b \times d) = (c \times b) \times (b \times d)$.
            Puisque $b$ et $d$ sont non nuls, $b \times d \neq 0$, donc cela implique $a \times d = c \times b$, donc $(a, b) \mathrel{\mathcal{R}} (c, d)$, et donc $x = y$.
        \item Si $x \leq y$ et $y \leq z$, alors $a \, b \, d^2 \leq c \, d \, b^2$ et $c \, d \, f^2 \leq e \, f \, d^2$.
            Donc, $a \, b \, d^2 \, f^2 \leq c \, d \, b^2 \, f^2 \leq e \, f \, b^2 \, d^2$.
            Puisque $d$ est non nul, $d^2$ est strictement positif.
            Cela montre donc que $a \, b \, f^2 \leq e \, f \, b^2$, et donc $x \leq z$.
        \item Puisque $\leq$ est une relation d'ordre total sur $\mathbb{Z}$, on a $a \, b \, d^2 \leq c \, d \, b^2$ ou $c \, d \, b^2 \leq a \, b \, d^2$, donc $x \leq y$ ou $y \leq x$.
    \end{itemize}

    \done

\medskip

\noindent\textbf{Définition :} On définit la relation d'ordre $\geq$ et les relations d'ordre strict $<$ et $>$ sur $Q$ de la manière suivante : pour tous éléments $x$ et $y$ de $\mathbb{Q}$, 
\begin{itemize}[nosep]
    \item $x \geq y$ si et seulement si $y \leq x$,
    \item $x < y$ si et seulement si $(x \leq y) \wedge (x \neq y)$,
    \item $x > y$ si et seulement si $x < y$.
\end{itemize}
Un nombre rationnel $q$ est dit \emph{strictement positif} si $0 < q$ et \emph{strictement négatif} si $q < 0$.

\medskip

\noindent\textbf{Remarque :}
    Soit $x$ et $y$ deux éléments de $\mathbb{Q}$ et $a$, $b$, $c$, $d$ quatre entiers tels que $b \neq 0$, $d \neq 0$, $x = a \divslash b$ et $y = c \divslash d$.
    Si $x = y$, alors $a \, d = c \, b$, donc (en multipliant les deux côtés par $b \, d$), $a \, b \, d^2 = c \, d \, b^2$.
    Réciproquement, si $a \, b \, d^2 = c \, d \, b^2$, alors $(a \, b) (b \, d) = (c \, d) \, (b \, d)$.
    Puisque $b$ et $d$ sont non nuls, $b \, d$ l'est également, donc cela implique $a \, b = c \, d$, et donc $x = y$.
    Ainsi, $x = y \Leftrightarrow a \, b \, d^2 = c \, d \, b^2$.
    Le prédicat $x < y$ est donc équivalent à $a \, b \, d^2 < c \, d \, b^2$, et $x > y$ à $a \, b \, d^2 > c \, d \, b^2$.

\medskip

\noindent\textbf{Lemme :} Soit $x$ et $y$ deux éléments de $\mathbb{Q}$. 
    On suppose $x < y$. 
    Alors, il existe un élément $z$ de $\mathbb{Q}$ satisfaisant $x < z < y$. 

\medskip

\noindent\textbf{Démonstration :} Soit $a$ et $b$ les numérateur et dénominateur d'un représentant de $x$ et $c$ et $d$ ceux d'un représentant de $y$. 
    Soit $e$ et $f$ les entiers naturels donnés par : $e = a \, d + b \, c$ et $f = 2 \, b \, d$.
    Puisque $b$, $d$ et l'entier $2$ sont non nuls, $f$ l'est également.
    Soit $z$ le nombre rationnel défini par : $z = e \divslash f$.
    Montrons que $x < z < y$. 
    Pour cela, on utilise le fait que, puisque $x < y$, $a \, b \, d^2 < c \, b^2 \, d$.

    Tout d'abord, 
    \begin{equation*}
        a \, b \, f^2 
        = 4 \, a \, b^3 \, d^2
        = 2 \, a \, b^3 \, d^2 + 2 \, a \, b^3 \, d^2
        < 2 \, a \, b^3 \, d^2 + 2 \, b^4 \, c \, d 
    \end{equation*}
    et 
    \begin{equation*}
        e \, f \, b^2
        = (a \, d + b \, c) \, 2 \, b^3 \, d
        = 2 \, a \, b^3 \, d^2 + 2 \, b^4 \, c \, d. 
    \end{equation*}
    Donc, $a \, b \, f^2 < e \, f \, b^2$, donc $x < z$.

    En outre, 
    \begin{equation*}
        c \, d \, f^2 
        = 4 \, c \, b^2 \, d^3
        = 2 \, c \, b^2 \, d^3 + 2 \, c \, b^2 \, d^3
        > 2 \, a \, b \, d^4 + 2 \, b^2 \, c \, d^3 
    \end{equation*}
    et 
    \begin{equation*}
        e \, f \, d^2
        = (a \, d + b \, c) \, 2 \, b \, d^3
        = 2 \, a \, b \, d^4 + 2 \, b^2 \, c \, d^3. 
    \end{equation*}
    Donc, $e \, f \, d^2 < c \, d \, f^2$, donc $z < y$.

    \done

\medskip

\noindent\textbf{Définition (rappel) :} Un nombre rationnel $q$ est dit
    \begin{itemize}[nosep]
        \item \emph{positif} si $q \geq 0$,
        \item \emph{négatif} si $q \leq 0$,
        \item \emph{strictement positif} si $q > 0$,
        \item \emph{strictement négatif} si $q < 0$.
    \end{itemize}

\medskip

\noindent\textbf{Lemme :} Soit $a$, $b$ et $c$ trois nombres entiers.
    On suppose que $c > 0$.
    Alors, $a \divslash c \leq b \divslash c \Leftrightarrow a \leq b$.

\medskip

\noindent\textbf{Démonstration :} Par définition, on a : $a \divslash c \leq b \divslash c \Leftrightarrow a \, c^3 \leq b \, c^3$.
    Puisque $c > 0$, $c^2 > 0$ et $c^3 > 0$, donc $a \, c^3 \leq b \, c^3 \Leftrightarrow a \leq c$.
    On a donc bien $a \divslash c \leq b \divslash c \Leftrightarrow a \leq b$.

    \done

\medskip

\noindent\textbf{Lemme :} Soit $p$ et $q$ deux nombres rationnels, et $r$ un nombre rationnel positif. 
    Alors, $p \leq q \Rightarrow r \times p \leq r \times q$.

\medskip

\noindent\textbf{Démonstration :}
    Soit $a$, $b$, $c$ et $d$ quatre entiers relatifs tels que $b \neq 0$, $d \neq 0$, $p = a \divslash b$ et $q = c \divslash d$.
    Supposons $p \leq q$.
    Alors, $b^2 \, c \, d \leq a \, b \, d^2$.
    Soit $e$ et $f$ deux entiers relatifs tels que $f \neq 0$ et $r = e \divslash f$.
    Puisque $r$ est positif, $e$ et $f$ ont le même signe si $e$ est non nul. 
    Donc, $e \, f \geq 0$, et donc $e \, f^3 \geq 0$.
    Donc, $b^2 \, c \, d \, e \, f^3 \leq a \, b \, d^2 \, e \, f^3$. 
    Donc, $(c \, e) \divslash (d \, f) \leq (a \, e) \divslash (b \, f)$. 
    Donc, $r \times p \leq r \times q$.

    \done

\medskip

\noindent\textbf{Lemme :} Soit $p$ et $q$ deux nombres rationnels, et $r$ un nombre rationnel négatif. 
    Alors, $p \leq q \Rightarrow r \times p \geq r \times q$.

\medskip

\noindent\textbf{Démonstration :}
    Soit $a$, $b$, $c$ et $d$ quatre entiers relatifs tels que $b \neq 0$, $d \neq 0$, $p = a \divslash b$ et $q = c \divslash d$.
    Supposons $p \leq q$.
    Alors, $b^2 \, c \, d \leq a \, b \, d^2$.
    Soit $e$ et $f$ deux entiers relatifs tels que $f \neq 0$ et $r = e \divslash f$.
    Puisque $r$ est négatif, $e$ et $f$ ont des ignes opposés si $e$ est non nul. 
    Donc, $e \, f \leq 0$, et donc $e \, f^3 \leq 0$.
    Donc, $b^2 \, c \, d \, e \, f^3 \geq a \, b \, d^2 \, e \, f^3$. 
    Donc, $(c \, e) \divslash (d \, f) \geq (a \, e) \divslash (b \, f)$. 
    Donc, $r \times p \geq r \times q$.

    \done

\medskip

\noindent\textbf{Lemme :} Soit $p$ et $q$ deux nombres rationnels strictement positifs. 
    Alors, $p \leq q \Rightarrow p^{-1} \geq q^{-1}$.

\medskip

\noindent\textbf{Démonstration :} 
    Soit $a$, $b$, $c$ et $d$ quatre entiers tels que $b \neq 0$, $c \neq 0$, $p = a \divslash b$ et $q = c \divslash d$.
    Puisque $p$ et $q$ sont non nuls, on a $a \neq 0$ et $c \neq 0$.
    
    Supposons $p \leq q$.
    Alors, $a \, b \, d^2 \leq c \, d \, b^2$.
    Puisque $p$ est strictement positif, $a$ et $b$ sont su même signe, donc $a \, b > 0$.
    De même, puisque $q$ est strictement positif, $c$ et $d$ sont su même signe, donc $c \, d > 0$.
    Donc, $a^2 \, b^2 \, c \, d^3 \leq a \, b^3 \, c^2 \, d^2$.
    Donc, $c \, d \, a^2 \, b^2 \, d^2 \leq a \, b \, c^2 \, b^2 \, d^2$.
    Puisque $b \neq 0$ et $d \neq 0$, cela implique
    Donc, $c \, d \, a^2 \leq a \, b \, c^2$, donc $d \divslash c \leq b \divslash a$, et donc $q^{-1} \leq p^{-1}$.

    \done

\medskip

\noindent\textbf{Corrolaire 1 :} Soit $p$ et $q$ deux nombres rationnels strictement positifs. 
    Si $p < q$, alors $p^{-1} > q^{-1}$. 
    (En effet, on a $p^{-1} \geq q^{-1}$ et $p^{-1} \neq q^{-1}$, sans quoi on aurait $p = q$.)

\medskip

\noindent\textbf{Corrolaire 2 :} Soit $p$ et $q$ deux nombres rationnels strictement négatifs. 
    Alors, $-p$ et $-q$ sont strictement positif, $p^{-1} = -((-p)^{-1})$ et $q^{-1} = -(-q)^{-1}$.
    Donc, si $p \leq q$, alors $p^{-1} \geq q^{-1}$ et, si $p < q$, alors $p^{-1} > q^{-1}$. 

\medskip

\noindent\textbf{Lemme :} Soit $p$, $q$ et $r$ trois nombres rationnels.
    Si $p \leq q$, alors $p + r \leq q + r$.

\medskip

\noindent\textbf{Démonstration :} 
    Soit $a$, $b$, $c$, $d$, $e$ et $f$ quatre entiers tels que $b \neq 0$, $d \neq 0$, $f \neq 0$, $p = a \divslash b$, $q = c \divslash d$, et $r = e \divslash f$.
    Par définition de l'addition de rationnels, on a : $p + r = (a \, f + e \, b) \divslash (b \, f)$ et $q + r = (c \, f + e \, d) \divslash (d \, f)$.
    Notons $A$, $B$, $C$ et $D$ les entiers $a \, f + e \, b$, $b \, f$, $c \, f + e \, d$ et $d \, f$. 
    Alors, $p + r = A \divslash B$ et $q + r = C \divslash D$.
    On a aussi : 
    \begin{equation*}
        A \, B \, D^2 = (a \, f + e \, b) \, b \, d^2 \, f^3
    \end{equation*}
    et 
    \begin{equation*}
        C \, B^2 \, D = (c \, f + e \, d) \, b^2 \, d \, f^3.
    \end{equation*}

    Supposons $p \leq q$.
    Alors, par définition de la relation d'ordre $\leq$, on a : $a \, b \, d^2 \leq c \, d \, b^2$. 
    Puisque $f^4$ est positif (car égal à $(f^2)^2$), on a donc : $a \, b \, d^2 \, f^4 \leq c \, d \, b^2 \, f^4$.
    Donc, $a \, b \, d^2 \, f^4 + e \, b^2 \, d^2 \, f^3 \leq c \, d \, b^2 \, f^4 + e \, b^2 \, d^2 \, f^3$.
    Le membre de gauche de cette inégalité n'est autre que $A \, B ,\ D^2$, et celui de droite que $C \, B^2 \, D$.
    Donc, $A \, B \, D^2 \leq C \, B^2 \, D$.
    Donc, $p + r \leq q + r$.

    \done

\medskip

\noindent\textbf{Corrolaire :} Soit $p$, $q$ et $r$ trois nombres rationnels.
    \begin{itemize}[nosep]
        \item Si $p \geq q$, alors $p + r \geq q + r$.
        \item Si $p < q$, alors $p + r < q + r$.
        \item Si $p > q$, alors $p + r > q + r$.
    \end{itemize}

\medskip

\noindent\textbf{Démonstration :} 
    Pour montrer le premier point, notons que si $p \geq q$, alors $q \leq p$, donc $q + r \leq p + r$, donc $p + r \geq q + r$.
    Pour les deux autres, on note que si $p \neq q$, alors $p + r \neq q + r$ (sans quoi soustraire $r$ donnerait $p = q$). 
    Les deux derniers points découlent directement de ceci et des deux résultats précédents.

    \done

\medskip

\noindent\textbf{Définition :} On définit les deux fonctions $\mathrm{min}$ et $\mathrm{max}$ sur $\mathbb{Q}^2$ de la manière suivante : si $a$ et $b$ sont deux nombres rationnels, alors 
\begin{equation*}
    \mathrm{min}(a,b) = \left\lbrace \begin{aligned}
        a & & \text{si} & & a \leq b \\
        b & & \text{si} & & a > b \\
    \end{aligned} \right. 
    , \qquad
    \mathrm{max}(a,b) = \left\lbrace \begin{aligned}
        a & & \text{si} & & a \geq b \\
        b & & \text{si} & & a < b \\
    \end{aligned} \right. 
    .
\end{equation*}
\index{Min}\index{Max}

\subsubsection{Valeur absolue}

\noindent\textbf{Définition :} \index{Valeur absolue} \sindex[isy]{$\abs{\cdot}$}
    Soit $q$ un nombre rationnel. 
    On définit sa \emph{valeur absolue}, notée $\abs{q}$, de la manière suivante : 
    \begin{itemize}[nosep]
        \item si $q \geq 0$, alors $\abs{q} = q$ ; 
        \item si $q < 0$, alors $\abs{q} = -q$.
    \end{itemize}

\medskip

\noindent\textbf{Lemme :} Soit $q$ un nombre rationnel.
    Alors, $- \abs{q} \leq q \leq \abs{q}$.

\medskip

\noindent\textbf{Démonstration :} 
    \begin{itemize}
        \item Si $q \geq 0$, alors $\abs{q} = q$, donc $q \leq \abs{q}$, et $-\abs{q} \leq 0 \leq q$, donc $- \abs{q} \leq q$.
        \item Si $q \leq 0$, alors $\abs{q} = -q$, donc $-\abs{q} = q$, donc $q \geq -\abs{q}$, et $\abs{q} \geq 0 \geq q$, donc $\abs{q} \geq q$.
    \end{itemize}

    \done

\medskip

\noindent\textbf{Corrolaire :} Soit $p$ et $q$ deux un nombres rationnel.
    On suppose $\abs{p} \leq q$.
    Alors, $-q \leq p \leq q$.

\medskip

\noindent\textbf{Démonstration :} 
    Puisque $\abs{p} \leq q$, $-\abs{p} \geq -q$, donc $-q \leq -\abs{p} \leq p \leq \abs{p} \leq q$. 

    \done

\medskip

\noindent\textbf{Lemme :} Soit $q$ un nombre rationnel et $a$ et $b$ deux entiers relatifs tels que $b \neq 0$ et $q = a \divslash b$.
    Alors, $\abs{q} = \abs{a} / \abs{b}$.

\medskip

\noindent\textbf{Démonstration :} On distingue quatre cas selon les signes de $a$ et de $b$.
    \begin{itemize}[nosep]
        \item Si $a \geq 0$ et $b \geq 0$, alors $q \geq 0$, donc $\abs{q} = q = a \divslash b = \abs{a} \divslash \abs{b}$.
        \item Si $a < 0$ et $b \geq 0$, alors $q < 0$, donc $\abs{q} = - q = (-a) \divslash b = \abs{a} \divslash \abs{b}$.
        \item Si $a \geq 0$ et $b < 0$, alors $q < 0$, donc $\abs{q} = - q = a \divslash (-b) = \abs{a} \divslash \abs{b}$.
        \item Si $a < 0$ et $b < 0$, alors $q \geq 0$, donc $\abs{q} = q = (-a) \divslash (-b) = \abs{a} \divslash \abs{b}$.
    \end{itemize}
    Dans tous les cas, on a bien $\abs{q} = \abs{a} / \abs{b}$.

    \done

\medskip

\noindent\textbf{Corrolaire :} Avec les mêmes notations, $\abs{q} = 0$ est équivalent à $\abs{a} = 0$, donc à $a = 0$, et donc à $q = 0$.

\medskip

\noindent\textbf{Corrolaire :} Avec les mêmes notations, $\abs{-q} = \abs{q}$ (puisqu'on peut obtenir $-q$ à partir de $q$ en remplaçant $a$ par $-a$, ce qui ne change pas sa valeur absolue).

\medskip

\noindent\textbf{Lemme :} Soit $p$ et $q$ deux nombres rationnels.
    Alors, $\abs{p \times q} = \abs{p} \times \abs{q}$ et, si $q \neq 0$, $\abs{p \divslash q} = \abs{p} \divslash \abs{q}$.

\medskip

\noindent\textbf{Remarque :} Ce lemme offre une démontstration alternative du corrolaire précédent, en notant que $-q = (-1) \times q$ et $\abs{-1} = 1$.

\medskip

\noindent\textbf{Démonstration :} Relativement évidente en utilisant le lemme précédent. 
    Écrivons-là exlicitement.
    Soit $a$, $b$, $c$ et $d$ quatre entiers relatifs tels que $b \neq 0$, $d \neq 0$, $p = a \divslash b$ et $q = c \divslash d$.
    On a : 
    \begin{equation*}
        \abs{p \times q} 
        = \abs{\frac{a}{b} \times \frac{c}{d}}
        = \abs{\frac{a c}{b d}}
        = \frac{\abs{a c}}{\abs{b d}}
        = \frac{\abs{a} \abs{c}}{\abs{b} \abs{d}}
        = \frac{\abs{a}}{\abs{b}} \times \frac{\abs{c}}{\abs{d}}
        = \abs{p} \times \abs{q} .
    \end{equation*}

    Si $q \neq 0$, alors $a \neq 0$. 
    Dans ce cas, on a :
    \begin{equation*}
        \abs{\frac{p}{q}} 
        = \abs{\frac{c}{d} \divslash \frac{a}{b}}
        = \abs{\frac{b c}{a d}}
        = \frac{\abs{b c}}{\abs{a d}}
        = \frac{\abs{b} \abs{c}}{\abs{a} \abs{d}}
        = \frac{\abs{b}}{\abs{a}} \divslash \frac{\abs{c}}{\abs{d}}
        = \abs{p} \divslash \abs{q} .
    \end{equation*}
    
    \done

\medskip

\noindent\textbf{Lemme (inégalité triangulaire) :} Soit $p$ et $q$ deux nombres rationnels. 
    Alors, $\abs{p} - \abs{q} \leq \abs{p + q} \leq \abs{p} + \abs{q}$.
    \index{Inégalité triangulaire}

\medskip

\noindent\textbf{Démonstration :}
    Montrons d'abord la première inégalité. 
    Si $p$ et $q$ sont de même signe, alors 
    \begin{itemize}[nosep]
        \item Si $p \geq 0$, $\abs{p + q} = p + q = \abs{p} + \abs{q}$.
        \item Si $p < 0$, $\abs{p + q} = -(p + q) = - p + (- q) = \abs{p} + \abs{q}$.
    \end{itemize}
    Si $p$ et $q$ sont de signes opposés, 
    \begin{itemize}[nosep]
        \item Si $\abs{p} \geq \abs{q}$, $p + q$ est su même signe que $p$.
        \begin{itemize}[nosep]
            \item Si $p \geq 0$, $\abs{p + q} = p + q = \abs{p} - \abs{q}$.
            \item Si $p < 0$, $\abs{p + q} = -(p + q) = -p + (-q) = \abs{p} - \abs{q}$.
        \end{itemize}
        \item Si $\abs{p} < \abs{q}$, $p + q$ est su même signe que $q$.
        \begin{itemize}[nosep]
            \item Si $q \geq 0$, $\abs{p + q} = p + q = \abs{q} - \abs{p} > 0 > \abs{p} - \abs{q}$.
            \item Si $q < 0$, $\abs{p + q} = -(p + q) = -p + (-q) = \abs{q} - \abs{p} > 0 > \abs{p} - \abs{q}$.
        \end{itemize}
    \end{itemize}

    Montrons maintenant la seconde inégalité. 
    Soit $a$, $b$, $c$ et $d$ quatre entiers relatifs tels que $b \neq 0$, $d \neq 0$, $p = a \divslash b$ et $q = c \divslash d$.
    On a : 
    \begin{equation*}
        \abs{p + q} = \frac{\abs{a d + b c}}{\abs{b c}}
    \end{equation*}
    et 
    \begin{equation*}
        \abs{p} + \abs{q} = \frac{\abs{a d} + \abs{b c}}{\abs{b c}} .
    \end{equation*}
    Le résultat attendu est donc équivalent à $\abs{b d}^3 \abs{a d + b c} \leq \abs{b d}^3 (\abs{a d} + \abs{b c})$, ou, puisque $b d \neq 0$ (et donc $\abs{b d}^3 \neq 0$), à $ \abs{a d + b c} \leq \abs{a d} + \abs{b c}$. 
    Puisque $a d$ et $b c$ sont deux entiers, c'est bien le cas.

    \done

\medskip

\noindent\textbf{Corroaire :} Soit $p$ et $q$ deux nombres rationnels. 
    Puisque $\abs{-q} = \abs{q}$, on a : $\abs{p} - \abs{q} \leq \abs{p - q} \leq \abs{p} + \abs{q}$.

\medskip

\noindent\textbf{Lemme :} Soit $q$ un nombre rationnel et $p$ un nombre rationnel positif. 
    Alors, $\abs{q} \leq p \Leftrightarrow -p \leq q \leq p$.

\medskip

\noindent\textbf{Démontration :} 
\begin{itemize}[nosep]
    \item Supposons $\abs{q} \leq p$. 
        Alors, 
        \begin{itemize}[nosep]
            \item Si $q$ est positif, alors $0 \leq q$ (par définition d'un nombre positif), donc $-p \leq q$~\footnote{On a $-p \leq 0$ puisque $p$ est positif.}, et $q \leq p$ (car $\abs{q} \leq p$), donc $-p \leq q \leq p$.
            \item Sinon, $q < 0$, donc $q \leq p$~\footnote{On a $0 \leq p$ puisque $p$ est positif.}, et $-q \leq p$ (car $\abs{q} \leq p$), donc $-p \leq q$, donc $-p \leq q \leq p$.
        \end{itemize}
    \item Supposons $-p \leq q \leq p$.
        Alors, 
        \begin{itemize}[nosep]
            \item Si $q$ est positif, la seconde inégalité donne $\abs{q} \leq p$.
            \item Sinon $q$, la première inégalité, qui implique $p \geq -q$, donne $p \geq \abs{q}$, donc $\abs{q} \leq p$.
        \end{itemize}
\end{itemize}

\done

\subsubsection{Limite d'une suite de rationnels}

\noindent\textbf{Définition :} Soit $u$ une suite de nombres rationnels, \emph{i.e.}, un élément de $\mathbb{Q}^{\mathbb{N}}$.
    Soit $l$ un nombre rationnel.
    On dit que \emph{$u$ converge vers $l$}, que \emph{$u$ tend vers $l$} ou que \emph{$u$ admet $l$ pour limite} si le prédicat suivant est vrai : 
    \begin{equation*}
        \forall \epsilon \in \mathbb{Q} \, 
        \epsilon > 0 \Rightarrow
            \exists n \in \mathbb{N} \, 
            \forall m \in \mathbb{N} \, 
            m \geq n \Rightarrow
                \abs{u_m - l} \leq \epsilon .
    \end{equation*}
    Cela peut être noté $u \rightarrow l$\sindex[isy]{$\rightarrow$}.
    Une suite admettant une limite est dite \emph{convergente}. 
    Une suite n'admettant aucune limite est dite \emph{divergente}. 

\medskip

\noindent\textbf{Lemme :} Une suite de nombres rationnels admet au plus une seule limite.

\medskip

\noindent\textbf{Démonstration :} Soit $u$ une suite de nombres rationnels. 
    Soit $l_1$ et $l_2$ deux limites de $u$. 
    Montrons que $l_1 = l_2$. 
    On procède par l'absurde. 
    Supposons donc que $l_1 \neq l_2$.
    Alors, $l_1 - l_2 \neq 0$. 
    Posons $\epsilon = \abs{l_1 - l_2} \divslash 4$. 
    Alors, $\epsilon > 0$.
    
    Par définition de la limite, on peut choisir deux entiers naturels $n_1$ et $n_2$ tels que, pour tout entier naturel $m$,
    \begin{itemize}[nosep]
        \item si $m \geq n_1$, $\abs{u_m - l_1} \leq \epsilon$,
        \item si $m \geq n_2$, $\abs{u_m - l_2} \leq \epsilon$.
    \end{itemize}
    Définissons $m$ par $m = \mathrm{max}(n_1, n_2)$.
    Alors, $m \geq n_1$ et $m \geq n_2$, donc $\abs{u_m - l_1} \leq \epsilon$ et $\abs{u_m - l_2} \leq \epsilon$.
    Puisque $l_1 - l_2 = (l_1 - u_m) + (u_m - l_2)$, on a : $\abs{l_1 - l_2} \leq \abs{l_1 - u_m} + \abs{u_m - l_2}$ (inégalité triangulaire), donc $\abs{l_1 - l_2} \leq \abs{u_m - l_1} + \abs{u_m - l_2}$, donc $\abs{l_1 - l_2} \leq 2 \epsilon$.
    Donc, $4 \epsilon \leq 2 \epsilon$, donc $2 \epsilon < 0$, donc $\epsilon < 0$.
    On obtient une contradiction. 
    On en déduit que l'hypothèse de départ ne peut qu'être fausse, et donc que $l_1 = l_2$.

    \done

\medskip

\medskip

\noindent\textbf{Lemme :} Soit $u$ une suite de nombres rationnels. 
    On suppose que $u$ admet une limite $l$. 
    Soit $n$ un entier naturel. 
    Soit $v$ la suite de nombres rationnels définie par : pour tout entier naturel $m$, $v_m = u_{m+n}$.
    Alors, $v$ tend vers $l$.

\medskip

\noindent\textbf{Démonstration :} Soit $\epsilon$ un nombre rationnel strictement positif.
    Puisque $u$ tend vers $l$, on peut choisir un entier naturel $m$ tel que, pour tout entier naturel $k$ supérieur ou égal à $m$, $\abs{u_k - l} \leq \epsilon$.
    Pour tout entier naturel $k$ supérieur ou égal à $m$, $k + n$ l'est également, donc $\abs{u_{k+n} - l} \leq \epsilon$, et donc $\abs{v_k - l} \leq \epsilon$.

    \done

\medskip

\noindent\textbf{Lemme :} Soit $q$ un nombre rationnel, $n_0$ un entier naturel et $u$ et $v$ deux suites de nombres rationnels. 
    On suppose que $u$ admet $q$ pour limite et que $v_n = u_n$ pour tout entier naturel $n$ supérieur ou égal à $n_0$.
    Alors $v$ admet $q$ pour limite.

\medskip

\noindent\textbf{Démonstration :}
    Soit $\epsilon$ un nombre rationnel strictement positif.
    Puisque $u$ admet $q$ pour limite, on peut choisir un entier naturel $n_\epsilon$ tel que $\abs{u_m - q} \leq \epsilon$ pour tout entier naturel $m$ supérieur ou égal à $n_\epsilon$. 
    Soit $n_\epsilon'$ l'entier naturel défini par : $n_\epsilon' = \mathrm{max} \left( n_0, n_\epsilon \right)$.
    
    Soit $m$ un entier naturel supérieur ou égal à $n_\epsilon'$.
    On a $m \geq n_\epsilon$, donc $\abs{u_m - q} \leq \epsilon$.
    En outre, $m \geq n_0$, donc $v_m = u_m$.
    Donc, $\abs{v_m - q} \leq \epsilon$.
    
    Cela montre que $\abs{v_m - q} \leq \epsilon$ pour tout entier naturel $m$ suérieur ou égal à $n_\epsilon'$. 
    On en déduit que $v$ admet $q$ pour limite.

    \done

\medskip

\noindent\textbf{Lemme :} Soit $u$ une suite de nombre rationnels convergente. 
    Alors, pour tout nombre rationnel $\epsilon$ strictement positif, il existe un entier naturel $n$ tel que, pour tous entiers naturels $m$ et $k$, $((m \geq n) \wedge (k \geq n)) \Rightarrow \abs{u_m - u_k} \leq \epsilon$.
    (Une suite satisfaisant cette propriété est dite \emph{de Cauchy}, voir Section~\ref{subsub:Cauchy}\index{Suite de Cauchy}.)

\medskip

\noindent\textbf{Démonstration :} Soit $u$ une suite de nombres rationnels convergente et $l$ sa limite.
    Soit $\epsilon$ un nombre rationnel strictement positif. 
    Alors, $\epsilon \divslash 2$ est aussi strictement positif.
    On peut donc choisir un entier naturel $n$ tel que, pour tout entier naturel $m$ supérieur ou égal à $n$, $\abs{u_m - l} \geq \epsilon \divslash 2$.
    Soit $m$ et $k$ deux entiers naturels supérieurs ou égaux à $n$. 
    On a (en utilisant l'inégalité triangulaire) : $\abs{u_m - u_k} = \abs{(u_m - l) + (l - u_k)} \leq \abs{u_m - l} + \abs{l - u_k} \leq \epsilon \divslash 2 + \epsilon \divslash 2$, donc $\abs{u_m - u_k} \leq \epsilon$. 

    \done

\medskip

\noindent\textbf{Exemples :} 
\begin{itemize}[nosep]
    \item Soit $q$ un nombre rationnel. 
        Soit $u$ la suite de nombres rationnels définie par : pour tout entier naturel $n$, $u_n = q$. 
        (Cette suite est parfois dite \emph{suite constante égale à $q$}\index{Suite constante}.)
        Alors, $u$ admet $q$ pour limite.
        En effet, pour tout rationnel $\epsilon$ tel que $\epsilon > 0$ et pour tout entier naturel $m$ tel que $m \geq 0$, on a $\abs{u_m - q} = 0$, donc $\abs{u_m - q} \leq \epsilon$.
    \item Soit $u$ la suite de nombres rationnels définie par : pour tout entier naturel $n$, $u_n = 1 \divslash (n+1)$.
        Montrons que cette suite a pour limite $0$. 
        Soit $\epsilon$ un nombre rationnel strictement positif et $a$ et $b$ deux entiers relatifs tels que $(a, b)$ est la représentation irréductible de $\epsilon$.
        Alors, $a$ et $b$ sont tous deux positifs (car $\epsilon \geq 0$) et $a \neq 0$ (car $\epsilon \neq 0$), donc $a \geq 1$.
        Pour tout entier naturel $n$ tel que $n \geq b$, on a (puisque $n \, b$ est positif) : $n \, b^2 \leq n^2 \, b$, donc $n \, b^2 \leq a \, n^2 \, b$, donc $1 \divslash n \leq a \divslash b$, donc (puisque $n+1 \geq n$) $1 \divslash (n+1) \leq a \divslash b$, donc $u_n \leq \epsilon$, et donc (puisque $\abs{u_n} = u_n$) $\abs{u_n - 0} \leq \epsilon$.
    \item La suite $u$ définie par $u_n = (-1)^n$ pour tout entier naturel $n$ n'admet pas de limite.
        En effet, elle prend successivement les valeurs $1$ pour $n$ pair et $-1$ pour $n$ impair (cela se montre facilement par récurrence sur $n$~\footnote{%
    Pour $n = 0$, on a $u_0 = (-1)^0 = 1$.
    Soit $n$ un entier naturel et supposons le résultat attendu vrai au rang $n$.
    Si $n$ est pair, alors $n+1$ est impair et $u_n = 1$, donc $(-1)^n = 1$, donc $(-1)^{n+1} = -1$, donc $u_{n+1} = -1$.
    Sinon, $n$ est impair, donc $n+1$ est pair et $u_n = -1$, donc $(-1)^n = -1$, donc $(-1)^{n+1} = 1$, donc $u_{n+1} = 1$.
    Le résultat attendu est donc vrai au rang $n+1$.
    Par récurrence, il l'est donc pour tout entier naturel.
}), distants de $2$, et n'est donc pas de Cauchy. 
\end{itemize}

\medskip

\noindent\textbf{Corrolaire :} Soit $u$ une suite de nombres rationnels convergente.
    Alors, on peut choisir deux nombres rationnels $a$ et $b$ tels que, pour tout entier naturel $n$, $a \leq u_n \leq b$.
    (\emph{Rappel :} Une suite satisfaisant cette propriété est dite \emph{bornée}.)
    \index{Suite bornée}

\medskip

\noindent\textbf{Démonstration :} Notons $l$ la limite de $u$.
    Par définition de la limite, on peut choisir un entier naturel $n$ tel que, pour tout entier naturel $m$ supérieur ou égal à $n$, $\abs{u_n-l} \leq 1$, donc $-1 \leq u_n-l \leq 1$, et donc $l-1 \leq u_n \leq l+1$. 
    Prenons $n$ strictement positif.
    (Si le $n$ défini précédemment est nul, il suffit de le remplacer par $n+1$, qui satisfait également la propriété le définissant.)
    Soit $a$ le minimum de $\lbrace u_0, u_1, \dots, u_{n-1}, l-1 \rbrace$ et $b$ le maximum de $\lbrace u_0, u_1, \dots, u_{n-1}, l+1 \rbrace$.
    Alors, pour tout entier naturel $m$, 
    \begin{itemize}[nosep]
        \item Si $m < n$, alors $u_n \geq a$ par définition du maximum et $u_n \leq b$ par définition du minimum.
        \item Sinon, $l-1 \leq u_n$, donc $a \leq u_n$, et $u_n \leq l+1$, donc $u_n \leq b$.
    \end{itemize}
    Dans les deux cas, on a bien $a \leq u_n \leq b$.

    \done

\medskip

\noindent\textbf{Lemme :} Soit $u$ et $v$ deux suites bornées et $q$ un nombre rationnel.
    \begin{itemize}[nosep]
        \item Définissons la suite $a$ par : pour tout entier naturel $n$, $a_n = u_n + q$.
            Alors, $a$ est bornée.
        \item Définissons la suite $b$ par : pour tout entier naturel $n$, $b_n = u_n \times q$.
            Alors, $b$ est bornée.
        \item Définissons la suite $c$ par : pour tout entier naturel $n$, $c_n = u_n + v_n$.
            Alors, $c$ est bornée.
        \item Définissons la suite $d$ par : pour tout entier naturel $n$, $d_n = u_n \times v_n$.
            Alors, $d$ est bornée.
    \end{itemize}

\medskip

\noindent\textbf{Démonstration :}
    Soit $A_u$, $A_v$, $B_u$ et $B_v$ quatre nombres rationnels tels que~: pour tout entier naturel $n$, $A_u \leq u_n \leq B_u$ et $A_v \leq v_n \leq B_v$.
    \begin{itemize}[nosep]
        \item Pour tout entier naturel $n$, $A_u + q \leq u_n + q \leq B_u + q$, donc $A_u + q \leq a_n \leq B_u + q$.
            Donc, $a$ est bornée.
        \item Soit $C_u$ le nombre rationnel défini par : $C_u = \mathrm{max}\left(\abs{A_u}, \abs{B_u}\right)$.
            Alors, $C_u \geq B_u$ et $-C_u \leq A_u$.%
            \begin{foot}
                En effet, 
                \begin{itemize}[nosep]
                    \item On a~: $C_u \geq \abs{B_u} \geq B_u$, donc $C_u \geq B_u$.
                    \item On a~: $-C_u \leq -\abs{A_u} \leq A_u$, donc $-C_u \leq A_u$.
                \end{itemize}
                \vspace*{-3ex}
            \end{foot} 
            Donc, pour tout entier naturel $n$, on a $-C_u \leq u_n \leq C_u$, et donc $- \abs{q} \, C_u \leq b_n \leq \abs{q} \, C_u$.%
            \begin{foot}
                En effet, 
                \begin{itemize}[nosep]
                    \item Si $q$ est positif, alors $-C_u \leq u_n \leq C_u$, implique $- q \, C_u \leq b_n \leq q \, C_u$, et donc $- \abs{q} \, C_u \leq b_n \leq \abs{q} \, C_u$. 
                    \item Si $q$ est négatif, alors $-C_u \leq u_n \leq C_u$, implique $- q \, C_u \geq b_n \geq q \, C_u$, donc $q \, C_u \leq b_n \leq -q \, C_u$, et donc $- \abs{q} \, C_u \leq b_n \leq \abs{q} \, C_u$. 
                \end{itemize}
                \vspace*{-3ex}
            \end{foot}
            Donc, $b$ est bornée.
        \item Pour tout entier naturel $n$, $u_n + v_n \leq u_n + B_v \leq B_u + B_v$ et $u_n + v_n \geq u_n + A_v \geq A_u + A_v$, donc $A_u + A_v \leq c_n \leq B_u + B_v$. 
            Donc, $c$ est bornée.
        \item Soit $C_u$ et $C_v$ les nombres rationnels définis par : $C_u = \mathrm{max}\left(\abs{A_u}, \abs{B_u}\right)$ et $C_v = \mathrm{max}\left(\abs{A_v}, \abs{B_v}\right)$.
            Alors, pour tout entier naturel $n$, on a $\abs{u_n} \leq C_u$ et $\abs{v_n} \leq C_v$.%
            \begin{foot}
                En effet, on a $-C_u \leq u_n \leq C_u$, donc,
                \begin{itemize}[nosep]
                    \item si $u_n$ est positif, $\abs{u_n} = u_n$, donc $\abs{u_n} \leq C_u$~;
                    \item si $u_n$ est négatif, $\abs{u_n} = -u_n$, donc $\abs{u_n} \leq -(-C_u)$, donc $\abs{u_n} \leq C_u$.
                \end{itemize}
                Et de même avec $u$ remplacé par $v$.
            \end{foot}
            Donc, $\abs{u_n} \, \abs{v_n} \leq C_u \, \abs{v_n} \leq C_u \, C_v$.
            Donc, $\abs{u_n \, v_n} \leq C_u \, C_v$.
            Donc, $-C_u \, C_v \leq u_n \, v_n \leq C_u \, C_v$.
            Donc, $-C_u \, C_v \leq d_n \leq C_u \, C_v$.
            Donc, $d$ est bornée.
    \end{itemize}

    \done

\medskip

\noindent\textbf{Notation :} Soit $u$ et $v$ deux suites de nombres rationnels et $q$ un nombre rationnel.
    On définit les suites $u + q$, $q + u$, $u - q$, $q - u$, $u \times q$, $q \times u$, $u + v$, $u - v$ et $u \times v$ par : pour tout entier naturel $n$, 
    \begin{itemize}[nosep]
        \item $(q + u)_n = q + u_n$,
        \item $(u + q)_n = u_n + q$,
        \item $(u - q)_n = u_n - q$,
        \item $(q - u)_n = q - u_n$,
        \item $(u \times q)_n = u_n \times q$,
        \item $(q \times u)_n = q \times u_n$,
        \item $(u + v)_n = u_n + v_n$,
        \item $(u - v)_n = u_n - v_n$,
        \item $(u \times v)_n = u_n \times v_n$. 
    \end{itemize}
    Avec les mêmes notations, on peut noter $-u$ la suite $(-1) \times u$.
    Avec les mêmes notations, si $q \neq 0$, on définit la suite $u \divslash q$ par : pour tout entier naturel $n$, $(u \divslash q)_n - u_n \divslash q$ et.
    Si $v_n \neq 0$ pour tout entier naturel $n$, on définit également la suite $u \divslash v$, aussi notée $u \div v$, par $\left( u \divslash v \right)_n = u_n \divslash v_n$ pour tout entier naturel $n$.

\medskip

\noindent\textbf{Lemme :} Soit $q$ un nombre rationnel et $u$ une suite de nombres rationnels admettant $q$ pour limite.
    Soit $p$ un nombre rationnel et $n$ un entier naturel.
    \begin{itemize}[nosep]
        \item Si $u_m \leq p$ pour tout entier naturel $m$ supérieur ou égal à $n$, alors $q \leq p$.
        \item Si $u_m \geq p$ pour tout entier naturel $m$ supérieur ou égal à $n$, alors $q \geq p$.
    \end{itemize}

\medskip

\noindent\textbf{Démonstration :} On procède par l'absurde.
    
    Traitons d'abord le cas où $u_m \leq p$ pour tout entier naturel $m$ supérieur ou égal à $n$.
    Supposons que $q > p$. 
    Alors, $q - p > 0$, donc $(q - p) \div 2 > 0$. 
    On peut donc choisir un entier naturel $m$ tel que, pour entier naturel $k$, $k \geq m \Rightarrow \abs{u_n - q} \leq (q - p) \div 2$.
    Notons $k$ le maximum de $n$ et $m$. 
    Puisque $k \geq n$, $u_k \leq p$. 
    Puisque $q - u_k = (q - p) + (p - u_k)$, on a donc : $q - u_k \geq q - p$, donc $q - u_k \geq 0$ et $\abs{q - u_k} \geq q - p$.
    Puisque $q - p > (q - p) \div 2$, on a donc $\abs{q - u_k} > (q - p) \div 2$.
    Cela est une contradiction puisque $k \geq m$.
    On en déduit que l'hypothèse de départ est fausse, et donc que $q \leq p$

    Le cas où $u_m \geq p$ pour tout entier naturel $m$ supérieur ou égal à $n$ est similaire. 
    Supposons par l'absurde que $p > q$. 
    Alors, $p - q > 0$, donc $(p - q) \div 2 > 0$. 
    On peut donc choisir un entier naturel $m$ tel que, pour entier naturel $k$, $k \geq m \Rightarrow \abs{u_n - q} \leq (p - q) \div 2$.
    Notons $k$ le maximum de $n$ et $m$. 
    Puisque $k \geq n$, $u_k \geq p$. 
    Puisque $u_k - q = (u_k - p) + (p - q)$, on a donc : $u_k - q \geq p - q$, donc $u_k - q \geq 0$ et $\abs{q - u_k} \geq p - q$.
    Puisque $p - q > (p - q) \div 2$, on a donc $\abs{q - u_k} > (p - q) \div 2$.
    Cela est une contradiction puisque $k \geq m$.
    On en déduit que l'hypothèse de départ est fausse, et donc que $p \leq q$

    \done

\medskip

\noindent\textbf{Lemme :} Soit $p$ et $q$ deux nombres rationnels et $u$ et $v$ deux suites de nombres rationnels admettant respectivement $p$ et $q$ pour limites. 
    Alors, 
    \begin{itemize}[nosep]
        \item Définissons la suite $w$ par : $w = u + v$. 
            Alors, $w$ admet $p + q$ pour limite.
        \item Définissons la suite $x$ par : $x = u - v$. 
            Alors, $w$ admet $p - q$ pour limite.
        \item Définissons la suite $y$ par : $y = u \times v$. 
            Alors, $w$ admet $p \times q$ pour limite.
        \item Supposons $q \neq 0$. 
            Alors, on peut choisir un entier naturel $n_0$ (voir démonstration ci-dessous) tel que $v_n \neq 0$ pour tout entier naturel $n$ tel que $n \geq n_0$.
            Définissons la suite $z$ par : pour tout entier naturel $n$, $z_n = 0$ si $n < n_0$ et $z_n = u_n \divslash v_n$ si $n \geq n_0$.
            Alors, $z$ admet $p \divslash q$ pour limite.
    \end{itemize}

\medskip

\noindent\textbf{Démonstration :}
    Dans toute cette démonstration, $\epsilon$ désigne un nombre rationnel strictement positif arbitraire.

    Soit $n_u$ et $n_v$ deux entiers naturels tels que, pour tout entier naturel $n$, 
    \begin{itemize}[nosep]
        \item si $n \geq n_u$, alors $\abs{u_n - p} \leq \ep \divslash 2$, 
        \item si $n \geq n_v$, alors $\abs{v_n - q} \leq \ep \divslash 2$. 
    \end{itemize}
    (Ces deux entiers existent par définition de la limite puisque $\ep \divslash 2 > 0$.)
    Soit $n_w$ l'entier naturel défini par : $n_w = \mathrm{max}(n_u, n_v)$. 
    Soit $n$ un entier naturel supérieur ou égal à $n_w$. 
    Alors $n \geq n_u$ et $n \geq n_v$. 
    On a alors : $\abs{w_n - (p+q)} = \abs{(u_n-p) + (v_n-q)} \leq \abs{u_n-p} + \abs{v_n-q} \leq \epsilon \divslash 2 + \epsilon \divslash 2 = \epsilon$.
    (Où l'on a utilisé l'inégalité triangulaire)
    Donc, $\abs{w_n - (p+q)} \leq \epsilon$.
    Cela montre que la suite $w$ admet $p+q$ comme limite.

    Avec les mêmes notations, on a : $\abs{x_n - (p-q)} = \abs{(u_n-p) - (v_n-q)} \leq \abs{u_n-p} + \abs{v_n-q} \leq \epsilon \divslash 2 + \epsilon \divslash 2 = \epsilon$.
    Donc, $\abs{x_n - (p-q)} \leq \epsilon$.
    Cela montre que la suite $x$ admet $p-q$ comme limite.

    Puisque les deux suites $u$ et $v$ sont convergentes, elles sont également bornées.
    On peut donc choisir quatre nombres rationnels $a$, $b$, $c$ et $d$ tels que, pour tout entier naturel $n$, $a \leq u_n \leq b$ et $c \leq v_n \leq d$.
    Soit $m$ le maximum de $\lbrace \abs{a}, \abs{b}, \abs{c}, \abs{d}, 1 \rbrace$.
    On montre facilement que, pour tout entier naturel $n$, $-m \leq u_n \leq m$ et $-m \leq v_n \leq m$.%
    \footnote{
        En effet, pour tout entier naturel $n$, 
        \begin{itemize}[nosep]
            \item $-m \leq -\abs{a} \leq a \leq u_n$,
            \item $m \geq \abs{b} \geq b \geq u_n$,
            \item $-m \leq -\abs{c} \leq c \leq v_n$,
            \item $m \geq \abs{d} \geq d \geq v_n$.
        \end{itemize}
    }
    Puisque $m \geq 1$, $m > 0$, donc $\epsilon \div (2 m) > 0$. 
    On peut donc choisir deux entiers naturels $n_1$ et $n_2$ tels que, pour tout entier naturel $n$ : 
    \begin{itemize}[nosep]
        \item si $n \geq n_1$, $\abs{u_n - p} \leq \epsilon \div (2 m)$,
        \item si $n \geq n_2$, $\abs{v_n - q} \leq \epsilon \div (2 m)$.
    \end{itemize}
    Soit $n_0$ le maximum de $n_1$ et $n_2$. 
    Pour tout entier naturel $n$ supérieur ou égal à $n_0$, on a : $\abs{u_n v_n - p q} = \abs{u_n \, (v_n - q) + (u_n - p) \, q} \leq \abs{u_n \, (v_n - q)} + \abs{(u_n - p) \, q} \leq \abs{u_n} \abs{v_n - q} + \abs{u_n - p} \abs{q} \leq m \, (\epsilon \divslash (2 m)) + (\epsilon \divslash (2 m)) \, m = \epsilon$.
    (On a ulilisé le fait que $c \leq q \leq d$%
    ~\footnote{En effet, si $q < c$, on aurait $u_n < c$ pour $n$ suffisamment grand pour que $\abs{u_n - q} \leq (c - q) \divslash 2$, ce qui est impossible.
        De même, si $q > d$, on aurait $u_n > d$ pour $n$ suffisamment grand pour que $\abs{u_n - q} \leq (q - d) \divslash 2$, ce qui est impossible.
        (Voir démonstration plus détaillée ci-dessus.)}%
    , donc $\abs{q} \leq m$.)
    Cela montre que $w$ tend vers $p \, q$.

    Pour le dernier point, supposons que $q \neq 0$. 
    Alors, $\abs{q - 0} > 0$, donc $\abs{q - 0} \divslash 2 > 0$. 
    Donc, on peut choisir un entier naturel $n_0$ tel que, pour tout entier naturel $n$ tel que $n \geq n_0$, $\abs{u_n - q} \leq \abs{q - 0} \divslash 2$.
    Or, $\abs{0 - q} = \abs{q - 0}$, donc $\abs{0 - q} > \abs{q - 0} \divslash 2$. 
    On en déduit que, pour tout entier naturel $n$ tel que $n \geq n_0$, $u_n \neq 0$, ce qui permet de définir la suite $z$ comme dans l'énoncé.

    Définissons les suites $\tilde{v}$ et $\tilde{z}$ par : pour tout entier naturel $n$, 
    \begin{itemize}[nosep]
        \item Si $n < n_0$, $\tilde{v}_n = 0$.
        \item Si $n \geq n_0$, $\tilde{v}_n = 1 \divslash v_n$.
        \item $\tilde{z}_n = u_n \times \tilde{v}_n$.
    \end{itemize}
    Alors, $\tilde{z}_n = z_n$ pour tout entier naturel $n$ supérieur ou égal à $n_0$. 
    Il suffit donc de montrer que $\tilde{z}$ tend vers $p \divslash q$.
    Pour ce faire, il est suffisant (d'après le résultat sur la multiplication) de montrer que $\tilde{v}$ tend vers $1 \divslash q$.

    Afin de simplifier encore le problème, définissons la suite $v'$ par : pour tout entier naturel $n$, $v_n' = \tilde{v}_n \times q$.
    Soit $\tilde{q}$ la suite définie par $\tilde{q}_n = 1 \divslash q$ pour tout entier naturel $n$. 
    On a $\tilde{v}_n = v_n' \times \tilde{q}_n$ pour tout entier naturel $n$ et $\tilde{q}$ tend vers $1 \divslash q$. 
    Si l'on peut montrer que $v'$ tend vers $1$, on en déduira que $\tilde{v}$ tend vers $1 \divslash q$.

    Puisque $\abs{q} \divslash 2 > 0$, on peut choisir un entier naturel $n_1$ tel que, pour tout entier naturel $n$, 
    \begin{equation*}
        n \geq n_1 \Rightarrow \abs{v_n - q} \leq \frac{\abs{q}}{2} .
    \end{equation*}
    Pour tout entier naturel $n$ supérieur ou égal à $n_1$, on a donc $\fabs{q - v_n} \leq \fabs{q} \divslash 2$, donc, d'après l'inégalité triangulaire, $\fabs{q} - \fabs{v_n} \leq \fabs{q} \divslash 2$, et donc $\fabs{v_n} \geq \fabs{q} \divslash 2$. 

    Soit $\epsilon$ un nombre rationnel strictement positif quelconque. 
    Alors, $\abs{q} \, \epsilon \divslash 2$ est strictement positif.
    On peut donc choisir un entier naturel $n^{(\epsilon)}$ tel que, pour tout entier naturel $n$ supérieur ou égal à $n^{(\epsilon)}$, $\abs{v_n - q} \leq \abs{q} \, \epsilon \divslash 2$. 
    Soit $m^{(\epsilon)}$ l'entier naturel défini par : $m^{(\epsilon)} = \mathrm{max} \left( n^{(\epsilon)}, \mathrm{max} \left( n_0, n_1 \right) \right)$. 
    Alors, pour tout entier naturel $n$ supérieur ou égal à $m^{(\epsilon)}$, 
    \begin{equation*}
        \abs{v'_n - 1} 
            = \abs{\tilde{v}_n \, q - 1}
            = \abs{\frac{q}{v_n} - 1}
            = \abs{\frac{q - v_n}{v_n}}
            = \frac{\abs{q - v_n}}{\abs{v_n}} .
    \end{equation*}
    Puisque $\fabs{v_n} \geq \fabs{q} \divslash 2$ et $\fabs{q - v_n} = \fabs{v_n - q} \leq \fabs{q} \, \epsilon \divslash 2$, on obtient : $\fabs{v'_n - 1} \leq \epsilon$.
    Cela montre que $v'$ tend vers $1$.
    
    \done

\medskip

\noindent\textbf{Définition :} Soit $u$ une suite de nombres rationnels. 
    On dit que \emph{$u$ tend vers $+\infty$} (ou \emph{tend vers $\infty$}), et on note $u \rightarrow + \infty$ (ou $u \rightarrow \infty$), si
    \begin{equation*}
        \forall q \in \mathbb{Q} \, 
        \exists n \in \mathbb{N} \, 
        \forall m \in \mathbb{N} \, 
        m \geq n \Rightarrow
            u_m \geq q.
    \end{equation*}
    On dit que \emph{$u$ tend vers $-\infty$}, et on note $u \rightarrow - \infty$, si
    \begin{equation*}
        \forall q \in \mathbb{Q} \, 
        \exists n \in \mathbb{N} \, 
        \forall m \in \mathbb{N} \, 
        m \geq n \Rightarrow
            u_m \leq q.
    \end{equation*}

\medskip

\noindent\textbf{Remarque :} Par définition, une suite tendant vers $+\infty$ ne peut être majorée et une suite tendant vers $-\infty$ ne peut être minorée.

\medskip

\noindent\textbf{Lemme :} Une suite tendant vers $+\infty$ ou vers $-\infty$ est divergente. 
    De manière équivalente, une suite convergente ne peut tendre vers $+\infty$ ni vers $-\infty$.

\medskip

\noindent\textbf{Démonstration :} Une suite tendant vers $+\infty$ ou vers $-\infty$ ne peut être bornée par définition, donc elle ne peut être convergente.

\done

\medskip

\noindent\textbf{Lemme :} Une suite ne peut tendre à la fois vers $+\infty$ et vers $-\infty$. 

\medskip

\noindent\textbf{Démonstration :} Supposons par l'absurde que l'on puisse choisir une suite $u$ tendant vers $+\infty$ et vers $-\infty$.
    Puisque $u$ tend vers $+\infty$, on peut choisir un entier naturel $n_+$ tel que $u_n \geq 1$ pour tout entier naturel $n$ supérieur ou égal à $n_+$.
    Puisque $u$ tend vers $-\infty$, on peut choisir un entier naturel $n_-$ tel que $u_n \leq -1$ pour tout entier naturel $n$ supérieur ou égal à $n_-$.
    Puisque $\mathrm{max}(n_+, n_-) \geq n_+$ et  $\mathrm{max}(n_+, n_-) \geq n_-$, on a alors $u_{ \mathrm{max}(n_+, n_-) } \geq 1$ et $u_{ \mathrm{max}(n_+, n_-) } \leq -1$, donc $-1 \geq 1$, ce qui est faux.
    On en déduit que l'hypothèse de départ est fausse.

    \done

\medskip

\noindent\textbf{Lemme :} Soit $u$ et $v$ deux suites de rationnels.
    \begin{itemize}[nosep]
        \item Si $u$ tend vers $+\infty$ et s'il existe un entier naturel $n$ tel que $v_m \geq u_m$ pour tout entier naturel $m$ supérieur ou égal à $n$, alors $v$ tend vers $+\infty$.
        \item Si $u$ tend vers $-\infty$ et s'il existe un entier naturel $n$ tel que $v_m \leq u_m$ pour tout entier naturel $m$ supérieur ou égal à $n$, alors $v$ tend vers $-\infty$.
    \end{itemize}

\medskip

\noindent\textbf{Démonstration :} 
    Dans cette démonstration, $n$ est un entier naturel.
    \begin{itemize}[nosep]
        \item Supposons que $u$ tend vers $+\infty$ et $v_m \geq u_m$ pour tout entier naturel $m$ supérieur ou égal à $n$.
            Puisque $u$ tend vers $+\infty$, on peut choisir un entier naturel $l$ tel que $u_m \geq q$ pour tout entier naturel $m$ supérieur ou égal à $l$.
            Pour tout entier naturel $m$ supérieur ou égal à $\mathrm{max}(n,l)$, on a donc $v_m \geq u_m \geq q$, et donc $v_m \geq q$.
            Cela montre que $v$ tend vers $+\infty$.
        \item Supposons que $u$ tend vers $-\infty$ et $v_m \leq u_m$ pour tout entier naturel $m$ supérieur ou égal à $n$.
            Soit $q$ un nombre rationnel.
            Puisque $u$ tend vers $-\infty$, on peut choisir un entier naturel $l$ tel que $u_m \leq q$ pour tout entier naturel $m$ supérieur ou égal à $l$.
            Pour tout entier naturel $m$ supérieur ou égal à $\mathrm{max}(n,l)$, on a donc $v_m \leq u_m \leq q$, et donc $v_m \geq q$.
            Cela montre que $v$ tend vers $-\infty$.
    \end{itemize}

    \done

\medskip

\noindent\textbf{Lemme :} La suite de rationnels $u$ définie par : pour tout entier naturel $n$, $u_n = (n, 1)$ tend vers $+\infty$. 

\medskip

\noindent\textbf{Démonstration :} Soit $q$ un nombre rationnel.
    Mntrons qu'il existe un entier naturel $n$ tel que, pour tout entier naturel $m$ supérieur ou égal à $n$, $u_n \geq \abs{q}$, et donc (puisque $\abs{q} \geq q$) $u_n \geq q$.

    Soit $a$ et $b$ les deux entiers naturels tels que $(a, b)$ est la représentation irréductible de $\abs{q}$.
    Alors, $b \geq 1$.
    Puisque $\abs{q}$ est positif, $a \geq 0$.
    Pour tout entier naturel $m$ supérieur ou égal à $a$, on a donc : $b \times \u_m = (b \times m, 1) \geq (b \times a, 1) \geq (a, 1) = \abs{q} \times b$, donc $u_m \geq \abs{q}$.
    Prendre $n = a$ convient donc.

    On en déduit que $u$ tend vers $+\infty$.
    
    \done

\medskip

\noindent\textbf{Lemme :} Soit $u$ et $v$ deux suites tendant vers $+\infty$ et $q$ un nombre rationnel.
    \begin{itemize}[nosep]
        \item Définissons la suite $a$ définie par : pour tout entier naturel $n$, $a_n = u_n + q$.
            Alors, $a$ tend vers $+\infty$.
        \item Définissons la suite $b$ définie par : pour tout entier naturel $n$, $b_n = u_n \times q$.
            Alors, 
            \begin{itemize}[nosep]
                \item Si $q > 0$, alors $b$ tend vers $+\infty$.
                \item Si $q = 0$, alors $b$ tend vers $0$.
                \item Si $q < 0$, alors $b$ tend vers $-\infty$.
            \end{itemize}
        \item Définissons la suite $c$ définie par : pour tout entier naturel $n$, $c_n = u_n + v_n$.
            Alors, $c$ tend vers $+\infty$.
        \item Définissons la suite $d$ définie par : pour tout entier naturel $n$, $d_n = u_n \times v_n$.
            Alors, $d$ tend vers $+\infty$.
    \end{itemize}

\medskip

\noindent\textbf{Démonstration :} 
\begin{itemize}[nosep]
    \item Soit $p$ un nombre rationnel.
        Alors, $p - q$ est aussi rationnel.
        On peut donc choisir un entier naturel $n$ tel que $u_m \geq p - q$ pour tout entier naturel $m$ supérieur ou égal à $n$. 
        Pour tout tel entier $m$, on a alors $a_m \geq p$.
        Cela montre que la suite $a$ tend vers $+\infty$.
    \item Traitons séparément les trois cas : 
        \begin{itemize}[nosep]
            \item \emph{Cas $q > 0$ :} Soit $p$ un nombre rationnel.
                Puisque $u$ tend vers $+\infty$, on peut choisir un entier naturel $n$ tel que, pour tout entier naturel $m$ satisfaisant $m \geq n$, on a $u_m \geq p \divslash q$.
                Puisque $q$ est strictement positif, cela implique $u_m \times q \geq p$, et donc $b_m \geq p$, pour tout entier naturel $m$ supérieur ou égal à $n$.
                Cela étant valable pour tout nombre rationnel $p$, on en déduit que $b$ tend vers $+\infty$.
            \item \emph{Cas $q = 0$ :} Soit $\epsilon$ un nombre rationnel strictement positif. Pour tout entier naturel $n$, on a $b_n = 0$, donc $\abs{b_n - 0} = \abs{0} = 0$, et donc $\abs{b_n - 0} \leq \epsilon$.
                Cela montre que $b$ tend vers $0$.
            \item \emph{Cas $q < 0$ :} Soit $p$ un nombre rationnel.
                Puisque $u$ tend vers $+\infty$, on peut choisir un entier naturel $n$ tel que, pour tout entier naturel $m$ satisfaisant $m \geq n$, on a $u_m \geq p \divslash q$.
                Puisque $q$ est strictement négatif, cela implique $u_m \times q \leq p$, et donc $b_m \leq p$, pour tout entier naturel $m$ supérieur ou égal à $n$.
                Cela étant valable pour tout nombre rationnel $p$, on en déduit que $b$ tend vers $-\infty$.
        \end{itemize}
    \item Soit $p$ un nombre rationnel.
        Puisque $u \rightarrow +\infty$, on peut choisir un entier naturel $n_1$ tel que $u_m \geq p$ pour tout entier naturel $m$ supérieur ou égal à $n_1$.
        En outre, puisque $v \rightarrow +\infty$, on peut choisir un entier naturel $n_2$ tel que $v_m \geq 0$ pour tout entier naturel $m$ supérieur ou égal à $n_2$.
        Notons $n$ le maximum de $n_1$ et $n_2$.
        Pour tout entier naturel $m$ tel que $m \geq n$, on a $m \geq n_1$, donc $u_m \geq p$, et $m \geq n_2$, donc $v_m \geq 0$, et donc $c_m \geq p$.
        Cela montre que $c \rightarrow +\infty$.
    \item Soit $p$ un nombre rationnel.
        Puisque $u \rightarrow +\infty$, on peut choisir un entier naturel $n_1$ tel que $u_m \geq \abs{p}$ pour tout entier naturel $m$ supérieur ou égal à $n_1$.
        En outre, puisque $v \rightarrow +\infty$, on peut choisir un entier naturel $n_2$ tel que $v_m \geq 1$ pour tout entier naturel $m$ supérieur ou égal à $n_2$.
        Notons $n$ le maximum de $n_1$ et $n_2$.
        Pour tout entier naturel $m$ tel que $m \geq n$, on a $m \geq n_1$, donc $u_m \geq p$, et $m \geq n_2$, donc $v_m \geq 1$, et donc $d_m \geq \abs{p}$, et donc $d_m \geq p$.
        Cela montre que $d \rightarrow +\infty$.
\end{itemize}

\done

\medskip

\noindent Les deux derniers points peuvent être généralisés comme suit.

\medskip

\noindent\textbf{Lemme :} Soit $u$ et $v$ deux suites de rationnels.
    On suppose due $u$ tend vers $+\infty$.
    \begin{itemize}[nosep]
        \item S'il existe un rationnel $q$ tel que $v_n \geq q$ pour tout entier naturel $n$, alors $u + v$ tend vers $+\infty$.
        \item S'il existe un rationnel strictement positif $q$ et un entier naturel $n$ tel que $v_m \geq q$ pour tout entier naturel $m$ supérieur ou égal à $m$, alors $u \times v$ tend vers $+\infty$.
    \end{itemize}

\medskip

\noindent\textbf{Démonstration :} 
\begin{itemize}[nosep]
    \item Supposons que la condition sur $v$ du premier point est satisfaite, et définissons $q$ de la même manière.
        Soit $p$ un nombre rationnel.
        Puisque $u \rightarrow +\infty$, on peut choisir un entier naturel $n$ tel que $u_m \geq p - q$ pour tout entier naturel $m$ supérieur ou égal à $n_1$.
        Pour tout entier naturel $m$ tel que $m \geq n$, on a $u_m \geq p - q$ et $v_m \geq p$, donc $(u + v)_m \geq p$.
        Cela montre que $u + v \rightarrow +\infty$.
    \item Supposons que la condition sur $v$ du second point est satisfaite, et définissons $n$ et $q$ de la même manière.
        Soit $p$ un nombre rationnel. 
        Puisque $u \rightarrow +\infty$, on peut choisir un entier naturel $m$ tel que $u_l \geq \abs{p} \divslash q$ pour tout entier naturel $l$ supérieur ou égal à $m$.
        Notons $k$ le maximum de $\lbrace n, m \rbrace$.
        Pour tout entier naturel $l$ supérieur ou égal à $k$, on a $l \geq n$, donc $v_l \geq q$, et $l \geq m$, donc $u_l \geq \abs{p} \divslash q$, et donc (puisque $q > 0$) $(u \times v)_l \geq \abs{p} \geq p$.
        Cela montre que $u \times v \rightarrow +\infty$.
\end{itemize}

\done

\medskip

\noindent\textbf{Lemme :} Soit $u$ et $v$ deux suites tendant vers $-\infty$ et $q$ un nombre rationnel.
    \begin{itemize}[nosep]
        \item Définissons la suite $a$ définie par : pour tout entier naturel $n$, $a_n = u_n + q$.
            Alors, $a$ tend vers $-\infty$.
        \item Définissons la suite $b$ définie par : pour tout entier naturel $n$, $b_n = u_n \times q$.
            Alors, 
            \begin{itemize}[nosep]
                \item Si $q > 0$, alors $b$ tend vers $-\infty$.
                \item Si $q = 0$, alors $b$ tend vers $0$.
                \item Si $q < 0$, alors $b$ tend vers $+\infty$.
            \end{itemize}
        \item Définissons la suite $c$ définie par : pour tout entier naturel $n$, $c_n = u_n + v_n$.
            Alors, $c$ tend vers $-\infty$.
        \item Définissons la suite $d$ définie par : pour tout entier naturel $n$, $d_n = u_n \times v_n$.
            Alors, $d$ tend vers $+\infty$.
    \end{itemize}

\medskip

\noindent\textbf{Démonstration :} Définissons les suite $w$ et $x$ par : pour tout entier naturel $n$, $w_n = -u_n$ et $x_n = -v_n$.
    Montrons que $w$ et $x$ tendent vers $+\infty$. 
    Cela nous permettra de nous ramener aisément aux cas précédents.

    Soit $p$ un nombre rationnel quelconque.
    Puisque $u$ tend vers $-\infty$, on peut choisir un entier naturel $n$ tel que, pour tout entier naturel $m$ supérieur ou égal à $n$, $u_m \leq -p$. 
    Donc, pour tout entier naturel $m$ supérieur ou égal à $n$, $- u_m \geq p$, et donc $w_n \geq p$.
    Cela montre que $w$ tend vers $+\infty$.
    Même démonstration pour $x$ en remplaçant $u$ par $v$ et $w$ par $x$.

    \begin{itemize}[nosep]
        \item Définissons la suite $\bar{a}$ par : pour tout entier naturel $n$, $\bar{a}_n = - a_n$.
            Alors, pour tout entier naturel $n$, $\bar{a}_n = w_n - q = w_n + (-q)$.
            Puisque la suite $w$ tend vers $+\infty$, on en déduit que $\bar{a}$ tend également vers $+\infty$.
            En outre, pour tout entier naturel $n$, on a $a_n = - \bar{a}_n = -1 \times \bar{a}_n$.
            Puisque $-1$ est strictement négatif et puisque $\bar{a}$ tend vers $+\infty$, on en déduit qur $a$ tend vers $-\infty$.
        \item Notons que, pour tout entier naturel $n$, $b_n = (-q) \times (-u_n) = (-q) \times \bar{u}_n$. 
            \begin{itemize}[nosep]
                \item \emph{Cas $q > 0$ :} Dans ce cas, $-q < 0$, donc, puisque $\bar{u}$ tend vers $+\infty$, $b$ tend vers $-\infty$.
                \item \emph{Cas $q = 0$ :} Dans ce cas, $b_n = 0$ pour tout entier naturel $n$, donc $b$ tend vers $0$.
                \item \emph{Cas $q < 0$ :} Dans ce cas, $-q > 0$, donc, puisque $\bar{u}$ tend vers $+\infty$, $b$ tend vers $+\infty$.
            \end{itemize}
        \item Définissons la suite $\bar{c}$ par : pour tout entier naturel $n$, $\bar{c}_n = - c_n$.
            Pour tout entier naturel $n$, on a $\bar{c}_n = -u_n - v_n = w_n + x_n$.
            Puisque $w$ et $x$ tendent vers $+\infty$, on en déduit que $\bar{c}$ tend vers $+\infty$.
            En outre, pour tout entier naturel $n$, on a $c_n = - \bar{c}_n = -1 \times \bar{c}_n$.
            Puisque $-1$ est strictement négatif et puisque $\bar{c}$ tend vers $+\infty$, on en déduit qur $c$ tend vers $-\infty$.
        \item Pour tout entier naturel $n$, on a : $d_n = u_n \times v_n = (- u_n) \times (- v_n) = w_n \times x_n$.
            Puisque $w$ et $x$ tendent vers $+\infty$, on en déduit que $d$ tend vers $+\infty$.
    \end{itemize}

    \done

\medskip

\noindent\textbf{Lemme :} Soit $u^+$ une suite tendant vers $+\infty$ et $u^-$ une suite tendant vers $-\infty$.
    Soit $v$ une suite bornée.
    On définit les deux suites $w^+$ et $w^-$ par : pour tout entier naturel $n$, $w^+_n = u^+_n + v_n$ et $w^-_n = u^ -_n + v_n$.
    Alors, $w^+$ tend vers $+\infty$ et $w^-$ tend vers $-\infty$.

\medskip

\noindent\textbf{Démonstration :} 
    Soit $a$ et $b$ deux nombres rationnels tels que : $\forall n \, \in \mathbb{N} \, a \leq v_n \leq b$. 
    (De tels nombres existent puisque $v$ est bornée.)
    Soit $c$ un nombre rationnel quelconque.

    Puisque $u^+$ tend vers $+\infty$, on peut choisir un entier naturel $n$ tel que, pour tout entier naturel $m$ supérieur ou égal à $n$, $u^+_m \geq c - a$.
    Alors, pour tout entier naturel $m$ supérieur ou égal à $n$, on a $u^+_m + v_m \geq (c - a) + v_m \geq (c - a) + a \geq c$, donc $w^+_m \geq c$.
    Cela montre que $w^+$ tend vers $+\infty$.
    
    Puisque $u^-$ tend vers $-\infty$, on peut choisir un entier naturel $n$ tel que, pour tout entier naturel $m$ supérieur ou égal à $n$, $u^-_m \leq c - b$.
    Alors, pour tout entier naturel $m$ supérieur ou égal à $n$, on a $u^-_m + v_n \leq (c - b) + v_m \geq (c - b) + b \geq c$, donc $w^-_m \leq c$.
    Cela montre que $w^-$ tend vers $-\infty$.

    \done

\medskip

\noindent\textbf{Lemme :} Soit $u^+$ une suite tendant vers $+\infty$ et $u^-$ une suite tendant vers $-\infty$.
    Soit $v$ une suite convergente et $l$ sa limite.
    On suppose $l \neq 0$.
    On définit les deux suites $w^+$ et $w^-$ par : $w^+ = u^+ \times v$ et $w^- = u^- \times v$.
    Alors, 
    \begin{itemize}[nosep]
        \item Si $l > 0$, alors $w^+$ tend vers $+\infty$ et $w^-$ tend vers $-\infty$.
        \item Si $l < 0$, alors $w^+$ tend vers $-\infty$ et $w^-$ tend vers $+\infty$.
    \end{itemize}

\medskip

\noindent\textbf{Démonstration :} 
\begin{itemize}[nosep]
    \item Supposons $l > 0$.
        Alors, $l \divslash 2 > 0$.
        Soit $p$ un nombre rationnel quelconque. 
        Puisque $u^+$ et $u^-$ tendent respectivement vers $+\infty$ et $-\infty$, on peut choisir deux entiers naturels $n^+$ et $n^-$ tels que
        \begin{itemize}[nosep]
            \item pour tout entier naturel $m$ supérieur ou égal à $n^+$, $u^+_m \geq \abs{p} \divslash (l \divslash 2)$,
            \item pour tout entier naturel $m$ supérieur ou égal à $n^-$, $u^-_m \leq -\abs{p} \divslash (l \divslash 2)$.
        \end{itemize}
        Puisque $v$ tend vers $l$, on peux choisir un entier naturel $n^v$ tel que, pour tour entier naturel $m$ supérieur ou égal à $n^v$, $\abs{v_m - l} \leq l \divslash 2$, donc $-l \divslash 2 \leq v_m - l \leq l \divslash 2$, donc $v_m \geq l \divslash 2$.
        
        Soit $m^+$ le maximum de $n^+$ et $n^v$. 
        Alors, $m^+ \geq n^+$ et $m^+ \geq n^v$.
        Pour tout entier naturel $r$ supérieur ou égal à $m^+$, on a alors $u^+_r \geq \abs{p} \divslash (l \divslash 2)$ et $v_r \geq l \divslash 2$, donc $w^+_r \geq \left( \abs{p} \divslash (l \divslash 2)\right) \times v_r \geq \abs{p} \geq p$. 
        Cela étant vrai pour tout nombre rationnel $p$, on en déduit que $w^+$ tend vers $+\infty$.

        Soit $m^-$ le maximum de $n^-$ et $n^v$. 
        Alors, $m^- \geq n^-$ et $m^- \geq n^v$.
        Pour tout entier naturel $r$ supérieur ou égal à $m^-$, on a alors $u^-_r \leq -\abs{p} \divslash (l \divslash 2)$ et $v_r \geq l \divslash 2$, donc $w^-_r \leq \left( - \abs{p} \divslash (l \divslash 2)\right) \times v_r \leq p$. 
        Cela étant vrai pour tout nombre rationnel $p$, on en déduit que $w^-$ tend vers $-\infty$.

    \item \emph{Demonstration alternative du premier point:} 
    \begin{itemize}[nosep]
        \item Avec les mêmes notations, puisque $v_r \geq l \divslash 2$ pour tout entier naturel $r$ supérieur ou égal à $n^v$ et puisque $l \divslash 2 > 0$, $u^+ \times v$ tend vers $+\infty$ d'après un lemme précédent, donc $w^+$ tend vers $+\infty$.
        \item On a : $w^- = - ((-u^-) \times v)$.
            Puisque $u^- \to -\infty$, $-u^- \to +\infty$, donc $(-u^-) \times v \to +\infty$, et donc $w^- \to -\infty$.
    \end{itemize}

    \item Définissons la suite $x$ par : $x = -v$.
        Alors, $x$ tend vers $- l \divslash 2$, qui est strictement positif.
        Définissons les suites $x^+$ et $x^-$ par : $x^+ = u^+ \times x$ et $x^- = u^- \times x$.
        Alors, d'après le cas précédent, $x^+ \to +\infty$ et $x^- \to -\infty$.
        Puisque $v^+ = - x^+$ et $v^- = - x^-$, on en déduit que $v^+ \to -\infty$ et que $v^- \to +\infty$.
\end{itemize}

\done

%\subsection{Le tore \texorpdfstring{$\mathbb{T}$}{T}}
%
%\noindent\textbf{Définition :} On définit dans cette section le \emph{tore}\index{Tore} $\mathbb{T}$ comme l'ensemble des suites d'éléments de $\lbrace 0, 1 \rbrace$ qui ne sont pas constantes égales à $1$ à partir d'un certain rang : 
%\begin{equation*}
%    \mathbb{T} = \left\lbrace
%        x \in \lbrace 0, 1 \rbrace^{\mathbb{N}} 
%        \middle\vert
%        \forall n \in \mathbb{N} \, \exists m \in \mathbb{N} \, (m \geq n) \wedge (x_n = 0)
%    \right\rbrace .
%\end{equation*}
%Quand il n'y a pas d'ambiguité, on note $0$ l'élément de $\mathbb{T}$ dont tous les éléments sont égaux à $0$.
%
%À tout élément $x$ de $\mathbb{T}$, on associe un élément $\mathsf{Seq(x)}$.
%
% WHAT FOLLOWS IS WRONG (the definition of the addition does not work, for instance, for (in binary) 101 + 011)
%
%\medskip
%
%\noindent\textbf{Addition :} On définit l'opération $+$ de $\mathbb{T} \times \mathbb{T}$ vers $\mathbb{T}$ comme suit. 
%    Soit $x$ et $y$ deux éléments de $\mathbb{T}$.
%    Soit $n$ un entier naturel. 
%    L'ensemble des éléments $m$ de $\mathbb{N}$ tels que $m \geq n$ et $x_m = 0$ est non vide (par définition de $\mathbb{T}$) et est un sous-ensemble de $\mathbb{N}$, donc il admet un minimum, noté $m_x$. 
%    De même, l'ensemble des éléments $m$ de $\mathbb{N}$ tels que $m \geq n$ et $y_m = 0$ admet un minimum, noté $m_y$. 
%    Soit $m$ le minimum de $m_x$ et $m_y$. 
%    On définit par récurrence finie la suite $(a_k, b_k)_{k \in [\![0, m-n]\!]}$ de la manière suivante :%
%    \footnote{On peut se ramener à une récurrence usuelle en posant cette définition si $k + 1 \leq m-n$ ou $a_{k+1} = b_{k+1} = 0$ sinon dans le second point.}
%    \begin{itemize}[nosep]
%        \item $b_0 = 1$ si $x_m = y_m = 1$ ou $b_0 = 0$ sinon ; $a_0 = 0$ si $x_m = y_m$ ou $a_0 = 1$ sinon.
%        \item Pour tout élément $k$ de $[\![0, m-n-1]\!]$, $a_{k+1}$ est égal à 
%            \begin{itemize}[nosep]
%                \item $0$ si $0$ ou $2$ des trois entiers $x_{m-k}$, $y_{m-k}$, $b_k$ est égal à $1$,
%                \item $1$ si $1$ ou $3$ des trois entiers $x_{m-k}$, $y_{m-k}$, $b_k$ est égal à $1$
%            \end{itemize}
%            et $b_{k+1}$ et égal à 
%            \begin{itemize}[nosep]
%                \item $0$ si $0$ ou $1$ des trois entiers $x_{m-k}$, $y_{m-k}$, $b_k$ est égal à $1$,
%                \item $1$ si $2$ ou $3$ des trois entiers $x_{m-k}$, $y_{m-k}$, $b_k$ est égal à $1$.
%            \end{itemize}
%    \end{itemize}
%    On pose alors $z_n = a_{m-n}$.
%    On définit ensuite $x + y$ comme suit : 
%    \begin{itemize}[nosep]
%        \item Si $z_n = 1$ pour tout entier naturel $n$, alors $x + y = 0$.
%        \item Sinon, s'il existe un entier naturel $m$ non nul tel que $z_k = 1$ pour tout entier naturel $k$ supérieur ou égal à $m$, on définit $m_0$ comme le minimum de l'ensemble des entiers naturels satisfaisant cette propriété. 
%            Notons que $m_0 > 0$ (sans quoi $z$ satisferait la propriété du point précédent).
%            On pose alors, pour tout entier naturel $n$ : 
%            \begin{itemize}[nosep]
%                \item Si $n < m_0 - 1$, $(x+y)_n = z_n$.
%                \item Si $n = m_0 - 1$, $(x+y)_n = 1$.
%                \item Si $n \geq m_0$, $(x+y)_n = 0$.
%            \end{itemize}
%        \item Sinon, $x + y = z$.
%    \end{itemize}
%    Notons que $x + y$ est alors bien un élément de $\mathbb{T}$.
%
%\medskip
%
%\noindent\textbf{Lemme :} L'addition ainsi définie est commutative.
%
%\medskip
%
%\noindent\textbf{Démonstration :} Évident car, avec les notations ci-dessus, $x$ et $y$ jouent des rôles interchangeables.
%
%\done
%
%\medskip
%
%\noindent\textbf{Lemme :} Pour tout élément $x$ de $\mathbb{T}$, $0 + x = x$.
%
%\medskip
%
%\noindent\textbf{Démonstration :} Soit $x$ un élément de $\mathbb{T}$ et $n$ un entier naturel.
%    En prenant les notations de la définition avec $y = 0$, on a $m_y = n$. 
%    (En effet, $0_n = 0$, $n \geq n$ et, pour tout entier naturel $m$ satisfaisant les deux propriétés, $m \geq n$ par définition.)
%    Puisque $m_x \geq n$ par définition, le minimum de $m_x$ et $m_y$ est $n$, donc $m = n$.
%    Donc, $m-n = 0$.
%    Donc, $a_{m-n} = a_0$.
%    Si $x_n = 0$, $x_n = y_n$, donc $a_0 = 0$.
%    Sinon, $x_n \neq y_n$, donc $a_0 = 1$.
%    Dans les deux cas, on a $a_0 = x_n$, et donc $z_n = x_n$.
%    Cela montre que $z = x$, donc que $z \in \mathbb{T}$, donc $0 + x = z$, et donc $0 + x = x$.
%
%    \done
%
%%\medskip
%%
%%\noindent\textbf{Lemme :} Dans la définition de l'addition, remplacer $m_0$ par un entier relatif $l$ tel que $l \geq m_0$ ne change pas le résultat.
%%
%%\medskip
%%
%%\noindent\textbf{Démonstration :} ***
%
%\medskip
%
%\noindent\textbf{Lemme :} Soit $x$ un élément de $\mathbb{T}$ ayant au moins un élément égal à $1$.
%    Soit $i$ un entier naturel tel que $x_i = 1$.
%    Soit $y$ et $z$ les éléments de $\mathbb{T}$ défini par : pour tout entier naturel $j$, 
%    \begin{itemize}[nosep]
%        \item si $j = i$, $y_j = 0$ et $z_j = 1$,
%        \item sinon, $y_j = x_j$ et $z_j = 0$.
%    \end{itemize}
%    Alors, $y + z = x$.
%
%\medskip
%
%\noindent\textbf{Démonstration :} ***
%
%%\medskip
%%
%%\noindent\textbf{Lemme :} On définit la suite de fonctions $\mathrm{pow2}$ de $\mathbb{T}$ vers $\mathbb{N}$ de la manière suivante : pour tout entier naturel $n$ et tout élément $x$ de $\mathbb{T}$, $\mathrm{pow2}_n(x) = \sum_{k=0}^n 2^{n-k} \times x_k$.
%%Soit $x$ et $y$ deux éléments de $\mathbb{T}$ et $n$ un entier naturel tel que $x_{n+1} = 0$.
%%Alors, 
%%\begin{equation*}
%%    \mathrm{pow2}_n(x+y) = (\mathrm{pow2}_n(x) + \mathrm{pow2}_n(y)) \mathrel{\%} 2^{n+1}. 
%%\end{equation*}
%%où $\%$ désigne le reste de la division Euclidienne.
%%
%%\medskip
%%
%%\noindent\textbf{Démonstration :} ***
%%
%%\medskip
%%
%%\noindent\textbf{Lemme :} Soit $x$ et $y$ deux éléments de $\mathbb{T}$. 
%%    On suppose que, pour tout entier naturel $n$, il existe un entier naturel $m$ supérieur ou égal à $n$ tel que $\mathrm{pow2}_m(x) = \mathrm{pow2}_m(y)$.
%%    Alors $x = y$. 
%%
%%\medskip
%%
%%\noindent\textbf{Démonstration :} Supposons par l'absurde que $x \neq y$. 
%%    Soit $n$ le plue petit  entier naturel tel que $x_n \neq y_n$, donc $(x_n = 0 \wedge y_n = 1) \vee (x_n = 1 \vee y_n = 0)$. 
%%    (Cet entier existe puisque l'ensemble des entiers satisfaisant cette propriété est un sous-ensemble non vide (puisque $x \neq y$ de $\mathbb{N}$.))
%%    Pour fixer les idées, supposons $x_n = 0$ et $y_n = 1$. 
%%    (La démonstration est identique dans l'autre cas en échangeant les rôles de $x$ et $y$.)
%%    Soit $m$ un entier naturel supérieur ou égal à $n$. 
%%    Montrons que $\mathrm{pow2}_m(x) < \mathrm{pow2}_m(y)$, et donc $\mathrm{pow2}_m(x) \neq \mathrm{pow2}_m(y)$, ce qui constituera une contradiction.
%%
%%    ***
%
%\medskip
%
%\noindent\textbf{Lemme :} L'addition est associative.
%
%\medskip
%
%\noindent\textbf{Démonstration :} ***
%
%\medskip
%
%\noindent\textbf{Multiplication par un entier :} On définit l'opération $\times$ de $\mathbb{N} \times \mathbb{T}$ vers $\mathbb{T}$ comme suit. 
%    Définissons d'abord par récurrence la suite $f$ de fonctions de $\mathbb{T}$ vers $\mathbb{T}$ de la manière suivante: 
%    \begin{itemize}[nosep]
%        \item Pour tout élément $x$ de $\mathbb{T}$, $f_0(x) = 0$.
%        \item Pour tout entier naturel $n$ et tout élément $x$ de $\mathbb{T}$, $f_{n+1}(x) = f_n(x) + x$.
%    \end{itemize}
%    Pour tout entier naturel $n$ et tout élément $x$ de $\mathbb{T}$, on pose alors $n \times x = f_n(x)$.
%
%\medskip
%
%\noindent\textbf{Remarque :} Avec cette définition, pour tout élément $x$ de $\mathbb{T}$, $0 \times x = 0$ et $1 \times x = x$.
%
%\medskip
%
%\noindent\textbf{Lemme :} La multiplication ainsi définie est distributive sur l'addition.
%
%\medskip
%
%\noindent\textbf{Démonstration :} Soit $x$ et $y$ deux éléments de $\mathbb{T}$.
%    Montrons par récurrence sur $n$ que, pour tout entier naturel $n$, $n \times (x + y) = (n \times x) + (n \times y)$.
%    
%    Pour $n = 0$, on a : $n \times (x + y) = 0 \times (x + y) = 0$ et $(n \times x) + (n \times y) = (0 \times x) + (0 \times y) = 0 + 0 = 0$.
%    Donc, $n \times (x + y) = (n \times x) + (n \times y)$. 
%    Soit $n$ un entier naturel tel que $n \times (x + y) = (n \times x) + (n \times y)$.
%    Alors, $(n + 1) \times (x + y) = (n \times (x + y)) + (x + y) = ((n \times x) + (n \times y)) + (x + y)$. 
%    En utilisant l'associativité et la commutativité de l'addition, cela donne : $(n + 1) \times (x + y) = ((n \times x) + x) + ((n \times y) + y) = ((n + 1) \times x) + ((n + 1) \times y)$. 
%
%    Par récurrence, la propriété attendue est donc vraie pour tout entier naturel $n$.
%
%    \done
%
%\medskip
%
%\noindent\textbf{Lemme :} La multiplication ainsi définie est distributive sur l'addition d'entiers.
%
%\medskip
%
%\noindent\textbf{Démonstration :} Soit $x$ un élément de $\mathbb{T}$.
%    Montrons par récurrence sur $n$ que, pour tout entier naturel $n$, pour tout entier naturel $m$, $(m + n) \times = (m \times x) + (n \times x)$.
%
%    Pour $n = 0$, on a pour tout entier naturel $m$ : $(m + n) \times x = m \times x$ et $(m \times x) + (n \times x) = (m \times x) + (0 \times x) = (m \times x) + 0 = m \times x$. 
%    Donc, $(m + n) \times x = (m \times x) + (n \times x)$.
%
%    Soit $n$ un entier naturel tel que le résultat attendu est vrai.
%    Soit $m$ un entier naturel.
%    On a : $(m + (n+1)) \times x = ((m+1) + n) \times x = ((m+1) \times x) + (n \times x) = ((m \times x) + x) + (n \times x) = (m \times x) = (x + (n \times x)) = (m \times x) + ((n + 1) \times n)$.
%    Le résultat est donc vrai au rang $n + 1$.
%
%    Par récurrence, il l'est pour tout entier naturel $n$. 
%
%    \done

\subsection{Nombres réels}

\subsubsection{Suites de Cauchy}
\label{subsub:Cauchy}

\noindent\textbf{Définition :} Soit $u$ une suite de nombres rationnels. 
    Elle est dite \textit{de Cauchy}\index{Suite de Cauchy} si la propriété suivante est satisfaite : 
    \begin{equation*}
        \forall \epsilon \in \mathbb{Q} \, 
        \exists n \in \mathbb{N} \,
        \forall m \in \mathbb{N} \,
        \forall k \in \mathbb{N} \,
        (m \geq n) \wedge (k \geq n)
        \Rightarrow \abs{u_n - u_k} \leq \epsilon
        .
    \end{equation*}

\medskip

Intuitivement, une suite est de Cauchy si la distance maximale entre un de ses éléments et les suivants tend vers $0$. 
On pourrait penser qu'une telle suite doit être convergente. 
Ce n'est en fait pas le cas sur $\mathbb{Q}$, mais c'est là l'intuition menant aux nombres réels, qui peuvent être formellement définis comme limites des suites de Cauchy de rationnels. 
(On montrera aussi que toute suite de Cauchy de nombres réels converge.)

\medskip

\noindent\textbf{Lemme :} Toute suite de nombres rationnels majorée et croissante est de Cauchy. 

\medskip

\noindent\textbf{Démonstration :} Soit $u$ une suite de nombres rationnels majorée et croissante. 
    Soit $M$ un majorant de $u$.
    On suppose par l'absurde que la suite $u$ n'est pas de Cauchy. 
    On peut donc choisir un nombre rationnel $\epsilon$ strictement positif tel que, pour tout entier naturel $n$, il existe deux entiers naturels $m$ et $k$ supérieurs ou égaux à $n$ tels que $\abs{u_m - u_k} > \epsilon$. 
    Montrons par récurrence sur $l$ que, pour tout entier naturel $l$, on peut choisir un entier naturel $n$ tel que $u_n \geq u_0 + l \times \epsilon$. 
    Cela montrera que $u$ tend vers $+\infty$. 
    En effet, pour tout nombre rationnel $q$, on peut choisir un entier naturel $l$ tel que $u_0 + l \times \epsilon \geq q$~\footnote{
        On peut montrer cela de la manière suivante. 
        Soit $v$, $w$ et $x$ les trois suites définies par : pour tout entier naturel $l$, $v_l = l$, $w_l = v_l \times \epsilon$, et $x_l = u_0 + w_l$.
        Alors, $v$ tend vers $+\infty$ (voir ci-dessus), $w = \epsilon \times v$, et $x = u_0 + w$.
        Puisque $\epsilon > 0$, on en déduit que $w$ tend vers $+\infty$, et donc que $x$ tend vers $+\infty$. 
        On peut donc choisir un entier naturel $l$ tel que $x_k \geq q$ pour tout entier naturel $k$ supérieur ou égal à $l$, et donc, en particulier, $x_l \geq q$, et donc $u_0 + l \times \epsilon \geq q$.
    }, donc $u_n \geq q$ avec le $n$ définit ci-dessus, et donc (puisque $u$ est croissante) $u_m \geq q$ pour tout ntier naturel $m$ supérieur ou égal à $n$.
    Ainsi, $u$ tendra vers $+\infty$, ce qui est impossible puisque $u$ est majorée.
    On en déduira que $u$ est de Cauchy.

    Pour $l = 0$, le résultat attendu est évident puisque $u_0 = u_0$, donc $u_0 \geq u_0 + 0 \times \epsilon$, donc $u_n \geq u_0 + l \times \epsilon$ pour $n = 0$.

    Soit $l$ un entuer naturel. 
    On suppose avoir un netier naturel $n$ tel que $u_n \geq u_0 + l \times \epsilon$.
    Par hypothèse, on peut choisir deux entiers naturels $a$ et $b$ supérieurs ou égaux à $n$ tels que $\abs{u_a - u_b} > \epsilon$.
    Sans perte de généralité, on suppose $a \leq b$ (si ce n'est pas le cas, on s'y ramène en échangeant les rôles de $a$ et $b$, ce qui ne change pas le phrase précédente puisque $\abs{u_b - u_a} = \abs{u_a - u_b}$).
    Puisque $u$ est croissante, $u_a \leq u_b$, donc $\abs{u_a - u_b} = u_b - u_a$.
    Donc, $u_b - u_a \geq \epsilon$, et donc $u_b \geq u_a + \epsilon$.
    Puisque $a \geq n$ et puisque $u$ est croissante, $u_a \geq u_n$, donc $u_b \geq u_n + \epsilon$, donc $u_b \geq (l+1) \times \epsilon$.
    Le résultat attendu est donc vrai au rang $l+1$. 

    Par récurrence, il est vrai pour tout entier naturel $l$; ce qui conclut la preuve.

    \done

\medskip

\noindent\textbf{Lemme :} Toute suite de nombres rationnels convergente st de Cauchy. 

\medskip

\noindent\textbf{Démonstration :} Soit $u$ uns suite de nombres rationnels.
    On suppose que $u$ est convergente. 
    Soit $l$ sa limite. 
    Soit $\epsilon$ un nombre rationnel strictement positif. 
    Alors, $\epsilon \divslash 2$ est strictement positif.
    Puisque $u$ tend vers $l$, on peut choisir un entier naturel $n$ tel que, pour tout entier naturel $m$ supérieur ou égal à $n$, $\abs{u_m - l} \leq \epsilon \divslash 2$. 
    Soit $m$ et $p$ deux entiers naturels supérieurs ou égaux à $n$. 
    On a : $\abs{u_n - u_m} = \abs{(u_n - l) + (l - u_m)} \leq \abs{u_n - l} + \abs{l - u_m} = \epsilon \divslash 2 + \epsilon \divslash 2 = \epsilon$. 
    La suite $u$ satisfait donc la définition d'une suite de Cauchy. 

    \done

\medskip

\begin{tcolorbox}[breakable, enhanced jigsaw]
\textbf{Remarque :} La réciproque est fausse.
    Pour montrer cela, considérons la suite $u$ définie par récurrence de la manière suivante : 
    \begin{itemize}[nosep]
        \item $u_0 = 1$
        \item pour tout entier naturel $n$, $u_{n+1} = u_n + \left(2 - u_n^2 \right) \divslash 4$ .
    \end{itemize}
   
    Supposons que la suite $u$ admette une limite $l$. 
    Définissons les suites $v$ et $w$ par : pour tout entier naturel $n$, $v_n = u_{n+1}$ et $w_n = u_n + \left(2 - u_n^2 \right) \divslash 2$.
    Alors, $v$ tend vers $l$ et $w$ tend vers $l + (2 - l \times l) \divslash 4$. 
    Puisque $v_n = u_n$ pour tout entier naturel $n$, on a donc $l = l + (2 - l \times l) \divslash 4$, donc $(2 - l \times l) \divslash 4 = 0$, donc $2 - l \times l = 0$, et donc $l \times l = 2$.

    Montrons que cela est impossible. 
    On procède par l'absurde. 
    Supposons qu'il existe un nombre rationnel $q$ tel que $q^2 = 2$. 
    Sans perte de généralité, on peut supposer $q \geq 0$ (si ce n'est pas le cas, il suffit de remplacer $q$ par $-q$, qui est alors positif et de carré égal à $2$.).
    Soit $a$ et $b$ deux entiers tels que $(a, b)$ est la représentation irréductible de $q$. 
    Puisque $q$ est positif, $a$ et $b$ sont tous deux positifs.
    En outre, $q \neq 0$ (sas quoi on aurait $q^2 = 0$, et donc $q^2 \neq 2$). 
    Donc, $q \neq 0$. 
    Donc, $a$ et $b$ sont premiers entre eux.
    
    Par ailleurs, on a : $b^2 \times q^2 = a^2$, donc $2 \, b^2 = a^2$. 
    Cela montre que $2$ divise $a^2$. 
    Puisque $2$ est premier, on en conclut que $2$ divise $a$. 
    On peut donc choisir un entier naturel $c$ tel que $a = 2 \, c$.
    Donc, $a^2 = 4 \, c^2$, et donc $b^2 = 2 \, c$.
    Donc, $2$ divise $b^2$, et donc $b$.
    Ainsi, $2$ est un diviseur comun à $a$ et $b$, ce qui est impossible puisque $a$ et $b$ sont premiers entre eux.
    On en déduit qu'un tel nombre rationnel $q$ ne peut exister.

    \medskip

    Montrons maintenant que la suite $u$ est de Cauchy. 
    Pour ce faire, on montre d'abord par récurrence que $u_n^2 \leq 2$ pour tout entier naturel $n$. 
    Cette suite est donc majorée—en effet, pour tout entier naturel $n$, on aura $u_n \leq 2$ (sans quoi on aurait $u_n^2 > 4 > 2$). 
    Pour tout entier naturel $n$, on aura alors $\left(2 - u_n^2 \right) \divslash 4 \geq 0$, montrant que la suite $u$ est croissante, et donc (car majorée et croissante) de Cauchy.

    Plus préciément, on montre par récurrence sur $n$ que, pour tout entier naturel $n$, $0 \leq u_n$ et $u_n^2 \leq 2$.
    
    \medskip

    Pour $n = 0$, on a $u_n = 1$, donc $u_n \geq 0$ et $u_n^2 = 1$, donc $u_n^2 \leq 2$. 

    Soit $n$ un entier naturel et supposons que $u_n \geq 0$ et $u_n^2 \leq 2$. 
    Alors, $2 - u_n^2 \geq 0$, donc $u_{n+1} \geq u_n$, donc $u_{n+1} \geq 0$.
    
    Par ailleurs, on a : 
    \begin{equation*} \begin{aligned}
        u_{n+1} \times u_{n+1} 
         & = u_n^2 + \left(2 - u_n^2 \right) \times u_n \divslash 2 + \left(2 - u_n^2 \right)^2 \divslash 16 \\
         & = \frac{1}{4} + u_n - \frac{u_n^2}{2} \times u_n + \frac{u_n^4}{16} .
    \end{aligned}\end{equation*}
    Puisque $u_n^2 \leq 2$, $u_n^4 \leq 4$.
    En outre, $u_n \leq 3 \divslash 2$ (sans quoi on aurait $u_n^2 > 9 \divslash 4 = 2 + \divslash 4 > 2$).
    Donc, 
    \begin{equation*}
        u_{n+1} \times u_{n+1} \leq 2 - \frac{u_n^2}{2} \times u_n.
    \end{equation*}
    Enfin, $u_n \geq 0$, cela donne $u_{n+1} \leq 2$.
    Le résultat attendu est donc vrai au rang $n+1$.
    Par récurrence, il l'est pour tout entier naturel $n$, ce qui conclut la preuve.
\end{tcolorbox}

\medskip

\noindent\textbf{Lemme :} Toute suite de nombres rationnels de Cauchy est bornée. 

\medskip

\noindent\textbf{Démonstration :} 
    Soit $u$ une suite de nombres rationnels. 
    On suppose que $u$ est de Cauchy. 
    On peut choisir un entier naturel $n_0$ tel que $\abs{u_n - u_m} \leq 1$ pour tous entiers naturels $n$ et $m$ supérieurs ou égaux à $n_0$. 
    Soit $E$ l'ensemble $\lbrace \abs{u_0}, \abs{u_1}, \dots, \abs{u_{n_0 - 1}}, \abs{u_{n_0}} + 1 \rbrace$. 
    L'ensemble $E$ est fini et non vide (son cardinal est $n_0$). Donc (puisque $\leq$ est une relation d'ordre total sur $\mathbb{Q}$), il admet un maximum, noté $M$ dans la suite de cette démonstration.
    Soit $n$ un entier naturel quelconque. 
    Si $n < n_0$, alors $\abs{u_n} \in E$, donc $\abs{u_n} \leq M$.
    Sinon, on a $\abs{u_n - u_{n_0}} \leq 1$, donc, d'après l'inégalité triangulaire, $\abs{u_n} - \abs{u_{n_0}} \leq 1$, donc $\abs{u_n} \leq \abs{u_{n_0}} + 1$.
    Puisque $\abs{u_{n_0}} + 1 \in E$, $\abs{u_{n_0}} + 1 \leq M$, donc $u_n \leq M$. 

    Pour tout entier naturel $n$, on a donc $\abs{u_n} \leq M$, et donc $-M \leq u_n \leq M$.
    Cela montre que la suite $u$ est bornée. 

    \done

\medskip

\emph{À écrire...}

\subsubsection{Les nombres réels comme limites de suites de Cauchy}
\sindex[isy]{$\mathbb{R}$}

\noindent\textbf{Définition :} Soit $\mathcal{U}$ l'ensemble des suites de Cauchy de nombres rationnels.
    On définit la relation d'équivalence $R$ sur $\mathcal{U}$ par : soit deux élément $u$ et $v$ de $\mathcal{U}$, $u \mathrel{R} v$ si et seulement si $u - v$ tend vers $0$.
    Un \emph{nombre réel} est une classe d'équivalence de $R$.
    L'\emph{ensemble des nombres réels}, noté $\mathbb{R}$, est l'ensemble des classes d'équivalences de $R$. 

\medskip

\noindent\textbf{Preuve qu'il s'agit bien d'une relation d'équivalence :}
\begin{itemize}[nosep]
    \item \emph{Réflexivité :} Soit $u$ une suite de Cauchy.
        Alors, $u - u$ est la suite constante égale à $0$, donc tend vers $0$.
        Donc, $u \mathrel{R} u$.
    \item \emph{Symétrie :} Soit $u$ et $v$ deux suites de Cauchy telles que $u \mathrel{R} v$.
        Alors, $u - v$ tend vers $0$.
        Donc, $v - u$ tend vers $0$.
        Donc, $v \mathrel{R} u$.
    \item \emph{Transitivité :} Soit $u$, $v$. et $w$ trois suites de Cauchy telles que $u \mathrel{R} v$ et $v \mathrel{R} w$.
        Alors, $u - v$ tend vers $0$ et $v - w$ tend vers $0$.
        Puisque $u - w = (u - v) + (v - w)$~\footnote{
            En effet, pour tout entier naturel $n$, on a : $((u - v) + (v - w))_n = (u - v)_n + (v - w)_n = (u_n - v_n) + (v_n - w_n) = (u_n + (-v_n)) + (v_n - w_n) = u_n + ((-v_n) + (v_n - w_n)) = u_n + (- w_n) = u_n - w_n = (u - w)_n$.
        }, on en déduit que $u - w$ tend vers $0$, et donc que $u \mathrel{R} w$.
\end{itemize}

\done

\medskip

\emph{À écrire...}

\subsubsection{Relation d'ordre}

\emph{À écrire...}

\subsubsection{Structure de corps}

\emph{À écrire...}

\subsubsection{Puissances}

\noindent\textbf{Définition :} On définit par récurrence les \emph{puissances entières positives} d'un nombre réel $x$ par :
    \begin{itemize}[nosep]
        \item $x^0 = 1$,
        \item pour tout entier naturel $n$, $x^{n+1} = x \times x^n$.
    \end{itemize}
    Notons que $x^1 = x$ et $x^n \neq 0$ pour tout entier naturel $n$ (se montre facilement par récurrence).
    \index{Puissance}
    Pour tout entier naturel non nul $n$, on définit la \emph{puissance entière négative} $x^{-n}$ comme égale à $1 \divslash z^x$.
    \index{Puissance négative}

\medskip

\noindent\textbf{Définition :} Soit $x$ et $y$ deux nombres réels et $n$ un entier naturel non nul.
    On dit que $y$ est une \emph{racine $n$e} de $x$ si $x^n = y$.
    \index{Racine}

\medskip

\noindent\textbf{Lemme :} Soit $x$ un nombre réel et $n$ et $m$ deux entiers. 
    On suppose $q > 0$ ou $n \geq 0$ et $m \geq 0$.
    Alors, 
    \begin{itemize}[nosep]
        \item $x^n \times x^m = x^{n+m}$,
        \item $(x^n)^m = x^{n m}$.
    \end{itemize}

\medskip

\noindent\textbf{Démonstration :} Voir démonstration du théorème équivalent pour les nombres rationnels page \pageref{demo:rel_puissances_q} en remplaçant $q$ par $x$.

\medskip

\noindent\textbf{Définition :} Soit $n$ un entier naturel non nul. 
    Tout nombre réel positif $x$ a exactement une racine $n$e positive, notée $x^{1 \divslash n}$ ou $\sqrt[n]{x}$.
    (Pour $n = 2$, le nombre $2$ pourra être omis dans cette seconde notation.)

\medskip

\noindent\textbf{Démonstration :} ***

\medskip

\emph{À écrire...}

\subsubsection{Nombres entiers et rationnels comme sous-ensembles des nombres réels}
\index{Nombre rationnel} \index{Rationnel}
\index{Nombre entier} \index{Entier}

On dira, quand il n'y a pas d'ambiguité, qu'un nombre réel $x$ est \emph{rationnel} si au moins un de ses éléments converge dans $\mathbb{Q}$.
On dira que $x$ est \emph{entier} si la limite de cet élément est entière, \emph{i.e.} si le dénominateur de sa représentation irréductible est $1$.

\medskip

\noindent\textbf{Lemme :} Soit $x$ un nombre réel. Si $x$ est rationnel selon la définition ci-dessus, alors tous ses éléments convergent dans $\mathbb{Q}$ et leurs limites sont identiques.

\medskip

\noindent\textbf{Démonstration :} ***

\subsubsection{Le tore \texorpdfstring{$\mathbb{T}$}{T}}

On définit le tore $\mathbb{T}$ par l'ensemble des classe d'équivalences pour la relation $R$ sur $\mathbb{R}$ définie par : pour tous nombres réels $x$ et $y$, $a \mathrel{R} y$ si et seulement si $x - y$ est entier. 

\medskip

\noindent\textbf{Preuve qu'il s'agit bien d'une relation d'équivalence :}
\begin{itemize}[nosep]
    \item \emph{Réflexivité :} Soit $x$ un nombre réel.
        Par définition, $x - x$ a pour représentant la suite constante égale à $0$, qui a pour limite $0$.
        Donc, $x - x$ est entier, donc $x \mathrel{R} x$.
    \item \emph{Symétrie :} Soit $x$ et $y$ deux ombres réels.
        On suppose $x \mathrel{R} y$. 
        Soit $u$ un représentant de $x$ et $v$ un représentant de $y$.
        Alors, $u - v$ conerge et sa limite est un entier $n$.
        Donc, $v - u$ tend vers $-n$, qui est aussi entier.
        Donc, $y \mathrel{R} x$.
    \item \emph{Transitivité :} Soit $x$, $y$ et $z$ tels que $x \mathrel{R} y$ et $y \mathrel{R} z$. 
        Soit $u$ un représentant de $x$, $v$ un représentant de $y$ et $w$ un représentant de $z$.
        Alors, $u - v$ tend vers un entier $n$ et $v - w$ tend vers un entier $m$.
        Donc, $u - w$ (égale à $(u - v) + (v - w)$) tend vers $n + m$, qui est entier.
        Donc, $x \mathrel{R} z$.
\end{itemize}

\done
 
\subsection{Nombres complexes}

\noindent\textbf{Définition :} 
    \sindex[isy]{$\mathbb{C}$}
    On définit l'ensemble des \emph{nombres complexes}, noté $\mathbb{C}$, par : $\mathbb{C} = \mathbb{R}^2$.
    Si $a$ et $b$ sont deux nombres résls, le nombre complexe $(a, b)$ sera aussi noté $a + b \, \mathrm{i}$, $a + \mathrm{i} \, b$, $b \, \mathrm{i} + a$, ou $\mathrm{i} \, b + a$.
    AVec les mêmes notations, $a + 0 \, \mathrm{i}$ pourra être noté $a$ et $0 + a \, \mathrm{i}$ pourra être noté $a \, \mathrm{i}$ quand il n'y a pas d'ambiguïté.
    On définit l'ensemble $\mathbb{C}^*$ par : $\mathbb{C}^* = \mathbb{C} \setminus \lbrace (0, 0) \rbrace$.

\medskip

\noindent\textbf{Définition :} \index{Partie réelle}\index{Partie imaginaire} \sindex[isy]{$\mathrm{Re}$} \sindex[isy]{$\mathrm{Im}$}
    Soit $z$ un nombre complexe et $x$ et $y$ les deux nombres réels tels que $z = (x, y)$.
    On appelle $x$ \emph{partie réelle} de $z$, notée $\mathrm{Re}(z)$, et $y$ \emph{partie imaginaire} de $z$, notée $\mathrm{Im}(z)$.

\medskip

\noindent\textbf{Définition :} \index{Conjugaison} \sindex[isy]{$*$}
    Soit $z$ un nombre complexe.
    On appelle \emph{conjugué}, ou \emph{conjugué complexe}, de $z$, noté $z^*$ le nombre complexe $\left( \mathrm{Re}(z), - \mathrm{Im}(z) \right)$.

\medskip

\noindent\textbf{Définition :} \index{Module} \sindex[isy]{$\abs{\cdot}$}
    Soit $z$ un nombre complexe. 
    Le \emph{module} de $z$, noté $\abs{z}$, est le nombre réel (positif) $\sqrt{\mathrm{Re}(z)^2 + \mathrm{Im}(z)^2}$.

\subsubsection{Structure de corps}

\noindent\textbf{Définition :} 
    \sindex[isy]{$+$} \sindex[isy]{$-$} \sindex[isy]{$\times$} 
    On définit les trois opérations $+$, $-$ et $\times$ de $\mathbb{C}^2$ vers $\mathbb{C}$ comme suit :
    \begin{itemize}[nosep]
        \item $(a + b \, \mathrm{i}) + (c + d \, \mathrm{i}) = (a + c, (b + d) \, \mathrm{i})$,
        \item $(a + b \, \mathrm{i}) - (c + d \, \mathrm{i}) = (a - c, (b - d) \, \mathrm{i})$,
        \item $(a + b \, \mathrm{i}) \times (c + d \, \mathrm{i}) = (a \, c -  b \, d, (a \, d + b \, c) \, \mathrm{i})$.
    \end{itemize}
    \sindex[isy]{$\div$} \sindex[isy]{$\divslash$}
    On définit aussi l'opération $\div$ de $\mathbb{C} \times \mathbb{C}^*$ vers $\mathbb{C}$ comme suit : pour tous nombres réels $a$, $b$, $c$ et $d$ tels que $(c \neq 0) \vee (d \neq 0)$, 
    \begin{equation*}
        (a + b \, \mathrm{i}) \div (c + d \, \mathrm{i}) = 
        \left( 
            \left( a \, c + b \, d \right) \divslash \left( c^2 + d^2 \right)
            \left( b \, c - a \, d \right) \divslash \left( c^2 + d^2 \right)
        \right) .
    \end{equation*}
    Comme pour les ensemble précédents, le symbole $\times$ est parfois élidé et $\div$ remplacé par $\divslash$, et (en l'absence de parenthèses), $\times$ et $\divslash$ sont prioritaires sur $+$ et $-$.

\medskip

\emph{À écrire...}

\subsubsection{Puissances}

\noindent\textbf{Définition :} On définit par récurrence les \emph{puissances entières positives} d'un nombre complexe $z$ par :
    \begin{itemize}[nosep]
        \item $z^0 = 1$,
        \item pour tout entier naturel $n$, $z^{n+1} = z \times z^n$.
    \end{itemize}
    Notons que $z^1 = z$ et $z^n \neq 0$ pour tout entier naturel $n$ (se montre facilement par récurrence).
    \index{Puissance}
    Pour tout entier naturel non nul $n$, on définit la \emph{puissance entière négative} $z^{-n}$ comme égale à $1 \divslash z^n$.
    \index{Puissance négative}

\medskip

\noindent\textbf{Lemme :} Soit $z$ un nombre complexe et $n$ et $m$ deux entiers. 
    On suppose $q > 0$ ou $n \geq 0$ et $m \geq 0$.
    Alors, 
    \begin{itemize}[nosep]
        \item $z^n \times z^m = z^{n+m}$,
        \item $(z^n)^m = z^{n m}$.
    \end{itemize}

\medskip

\noindent\textbf{Démonstration :} Voir démonstration du théorème équivalent pour les nombres rationnels page \pageref{demo:rel_puissances_q} en remplaçant $q$ par $z$.

\subsection{Éléments de Topologie Générale}

\noindent\textbf{Définition :} Soit $E$ un ensemble.
    Un ensemble $T$ est une \emph{topologie sur $E$} si les conditions suivantes sont satisfaites ; le couple $(E, T)$ est alors dit \emph{espace topologique} et les éléments de $T$ sont ses \emph{ouverts} :
    \begin{itemize}[nosep]
        \item $T$ est un ensemble de parties de $E$ : $\forall x \in T \, x \subset E$,
        \item l'ensemble vide appartient à $T$ : $\emptyset \in T$,
        \item $E$ appartient à $T$ : $E \in T$,
        \item toute réunion d'ouverts est un ouvert : $\forall I \, I \subset T \Rightarrow \cup I \in T$,
        \item toute intersection finie d'ouverts est un ouvert : pour tout entier naturel strictement positif $n$, $\forall O_1 \in T \, \forall O_2 \in T \cdots \forall O_n \in T \, O_1 \cap O_2 \cap \cdots \cap O_n \in T$.
    \end{itemize}

\medskip

\noindent\textbf{Définition :} Soit $(E, T)$ un espace topologique.
    Un sous-ensemble $F$ de $E$ est dit \emph{fermé} si $E \setminus F$ et un ouvert, \emph{i.e.}, si $E \setminus F \in T$.

\medskip

\noindent\textbf{Définition :} Soit $(E, T)$ et $(F, U)$ deux espaces topologiques et $f$ une fonction de $E$ vers $F$.
    On dit que $f$ est \emph{continue} (pour les topologies $T$ et $U$) si, pour tout sous-ensemble $E'$ de $E$, l'image de $E'$ par $f$ (\emph{i.e.} l'ensemble des $f(e)$ pour $e \in E'$) est un élément de $U$.
