\section{Autres ensembles remarquables}

\localtoc

\noindent Dans cette section, nous présentons brièvement d'autres ensembles de nombres remarquables ainsi que leur structure.

\subsection{Nombres rationnels}

\subsubsection{L'ensemble des nombres rationnels \texorpdfstring{$\mathbb{Q}$}{Q}}

Définissons (dans cette section seulement) la relation binaire $\mathcal{R}$ sur $\mathbb{Z} \times \mathbb{Z}^*$ de la manière suivante : si $x$ et $y$ sont deux éléments de cet ensemble, soit $a$, $b$, $c$ et $d$ les quatre (uniques) entiers relatifs (avec $b \neq 0$ et $d \neq 0$) tels que $x = (a, b)$ et $y = (c, d)$, alors $r \, \mathcal{R} \, y$ si et seulement si $a d = b c$. 
Autrement dit, 
\begin{equation*}
    \forall (a, b, c, d) \in \mathbb{Z} \times \mathbb{Z}^* \times \mathbb{Z} \times \mathbb{Z}^* \;
    (a, b) \, \mathcal{R} \, (c, d)
    \Leftrightarrow a d = b c .
\end{equation*}

\medskip

\noindent\textbf{Lemme :} La relation $\mathcal{R}$ est une relation d'équivalence. 

\medskip

\noindent\textbf{Démonstration :} Montrons que $\mathcal{R}$ satisfait les trois propriétés d'une relation d'équivalence.
    Dans toute cette démonstration, $x$, $y$ et $z$ sont trois éléments de $\mathbb{Z} \times \mathbb{Z}^*$ et $a$, $b$, $c$, $d$, $e$, $f$ sont six entiers tels que $b \neq 0$, $d \neq 0$, $f \neq 0$, $x = (a, b)$, $y = (c, d)$ et $z = (e, f)$.
    \begin{itemize}[nosep]
        \item \textit{Réflexivité :} Puisque la multiplication sur $\mathbb{Z}$ est commutative, $a b = b a$, donc $x \, \mathcal{R} \, x$.
        \item \textit{Symétrie :} Supposons $x \, \mathcal{R} \, y$. 
            Alors, $a d = b c$.
            Donc, $c b = d a$.
            Donc, $y \, \mathcal{R} \, x$.
        \item \textit{Transitivité :} Supposons $x \, \mathcal{R} \, y$ et $y \, \mathcal{R} \, z$.
            Alors, $a d = b c$ et $c f = d e$.
            Multiplions les deux membres de la première égalité par $f$ : il vient $a d f = b c f$, donc (en utilisant la seconde égalité) $a d f = b d e$, donc $d a f = d b e$.
            Puisque $d \neq 0$, on a donc $a f = b e$, et donc $x \, \mathcal{R} \, z$.
    \end{itemize}
    
    \done

\medskip

\noindent\textbf{Lemme :} Soit $a$ un entier et $b$ et $\lambda$ deux entiers non nuls. 
    Alors $(\lambda \times a, \lambda \times b) \mathrel{\mathcal{R}} (a, b)$.

\medskip

\noindent\textbf{Démonstration :} Puisque $\lambda$ et $b$ sont non nuls, $\lambda \times b$ l'est également.
    En outre, $a \times (\lambda \times b) = (a \times \lambda) \times b = (\lambda \times a) \times b$.

    \done

\medskip

On appelle \emph{ensemble des nombres rationnels}, noté $\mathbb{Q}$, l'ensemble des classes d'équivalences de $\mathcal{R}$, appelées \emph{nombres rationnels}. 
\aindex{Nombre rationnel} \sindex[isy]{$\mathbb{Q}$}
Autrement dit, $\mathbb{Q}$ est l'ensemble défini par : 
\begin{equation*}
    \mathbb{Q} = (\mathbb{Z} \times \mathbb{Z}^*) \divslash \mathcal{R} .
\end{equation*}
Soit $q$ un nombre rationnel et $a$ et $b$ deux entiers tels que $(a, b) \in q$. 
Alors, $a$ est appelé le \emph{numérateur} de la représentation $(a, b)$ et $b$ son \emph{dénominateur}.
\index{Numérateur} \index{Dénominateur}
Elle sera notée $a \divslash b$, ou $\frac{a}{b}$.
\sindex[isy]{$\divslash$}

Dans cette section, si $x$ est un élément de $\mathbb{Z} \times \mathbb{Z}^*$, on note $\bar{x}$ la classe d'équivalence de $x$ pour la relation $\mathcal{R}$, \emph{i.e.}
\begin{equation*}
    \bar{x} = \left\lbrace y \in \mathbb{Z} \times \mathbb{Z}^* \middle\vert x \mathrel{\mathcal{R}} y \right\rbrace .
\end{equation*}

\medskip

\noindent\textbf{Lemme :} Soit $q$ un nombre rationnel. 
Si $q$ admet une représentation dont le numérateur est nul, alors toutes ses représentations ont un numérateur nul et tous les éléments de $\mathbb{Z} \times \mathbb{Z}^*$ de première composante nulle sont des représentants de $q$. 

\medskip

\noindent\textbf{Démonstration :} On suppose que $q$ admet une représentation de numérateur nul. 
    On peut alors choisir un entier $b$ non nul tel que $(0, b) \in q$. 
    Soit $r$ une représentation de $q$.
    Soit $c$ et $d$ deux entiers tels que $(c, d)  = r$.
    Alors, $0 \times d = b \times c$.
    Donc, $b c = 0$.
    Puisque $b \neq 0$, on en déduit que $c = 0$.
    Donc, la représentation $r$ est de numérateur nul.

    Soit $r$ un élément de $\mathbb{Z} \times \mathbb{Z}^*$ de première composante nulle.
    On peut choisir un entier non nul $d$ tel que $r = (0, d)$.
    Puisque $0 \times d = 0 = 0 \times b$, on a $r \mathrel{\mathcal{R}} (0, b)$, et donc $r \in q$.

    \done

\medskip

\noindent\textbf{Corrolaire :} Il existe donc une seule classe d'équivalence contenant un représentant de numérateur nul. 
Ce nombre rationnel est aussi dit \emph{nul}.

\medskip

\noindent\textbf{Notation :} On note $\mathbb{Q}^*$ l'ensemble $\mathbb{Q} \setminus \big\lbrace \overline{(0,1)} \big\rbrace$.

\medskip

\noindent\textbf{Remarque :} D'après le lemme précédent, tout représentant de tout élément de $\mathbb{Q}^*$ a un numérateur non nul.

\medskip

\noindent\textbf{Définition (représentation irréductible) :} 
    Soit $q$ un élément de $\mathbb{Q}$. 
    Soit $a$ un élément de $\mathbb{Z}$ et $b$ un élément de $\mathbb{Z}^*$ tels que $(a, b) \in q$.
    Alors, $(a, b)$ est dit \textit{représentation irreductible} de $q$ si et seulement si une des deux propriétés suivantes est satisfaite : 
    \begin{itemize}[nosep]
        \item $a = 0$ et $b = 1$.
        \item $a \neq 0$, $b > 0$ et $\abs{a}$ et $\abs{b}$ sont premiers entre eux.
    \end{itemize}
\index{Représentation irréductible}

\medskip

\noindent\textbf{Lemme :} Tout nombre rationnel a une unique représentation irréductible. 
    En outre, soit $r$ cette représentation irréductible, $a$ son numerateur et $b$ son dénominateur, un élément $(c, d)$ appartient à $q$ si et seulement si il existe un entier non nul $k$ tel que $c = k a$ et $d = k b$.

\medskip

\noindent\textbf{Démonstration :} Soit $x$ un nombre rationnel. 
\begin{itemize}[nosep]
    \item Si $x$ admet une représentation de numérateur nul, alors toutes ses représentations le sont.
        Donc, $x$ ne peut avoir de représentation de la forme $(a, b)$ avec $a \neq 0$.
        Il ne peut donc avoir que $(0, 1)$ comme représentation irréductible.
        Montrons que c'en est bien une. 
        Soit $r$ une représentation de $x$, $a$ son numérateur et $b$ son dénominateur. 
        Puisque $a = 0$, on a : $a \times 1 = 0 = b \times 0$. 
        Donc, $(0, 1)$ est bien une représentation de $x$, et donc sa représentation irréductible. 
        Par ailleurs, ses représentations sont toutes de la forme $(0, k)$, égal à $(0 \times k, 1 \times k)$, pour un entier non nul $k$, et tous les éléments de $\mathbb{Z} \times \mathbb{Z}^*$ de cette forme sont des représentations de $x$ puisque $1 \times 0 = 0 = 0 \times k$.
    \item Sinon, toutes les représentations de $x$ ont un numérateur non nul. 
        Soit $r$ une de ses représentations, $a$ son numerateur et $b$ son dénominateur. 
        Soit $l$ le pgcd de $\abs{a}$ et $\abs{b}$.
        On peut choisir deux entiers naturels $p$ et $x$ tels que $\abs{a} = p l$ et $\abs{b} = q l$.
        En outre, puisque $\abs{b}$ est non nul, $q$ est non nul, donc $(\mathrm{sgn}(a) \, \mathrm{sgn}(b) \, p, q)$ est un élément de $\mathbb{Z} \times \mathbb{Z}^*$. 
        En outre, $p \neq 0$ (puisque $a \neq 0$), donc $\mathrm{sgn}(a) \, \mathrm{sgn}(b) \, p \neq 0$, et $q > 0$. 
        Puisque les valeurs absolues du numérateur et du dénominateur (égales à $p$ et $q$) sont premières entre elles, il s'agit d'une représentation irreductible.
        En outre, $a q = \mathrm{sgn}(a) \abs{a} q = \mathrm{sgn}(a) p q l = \abs{b} \mathrm{sgn}(a) p = b \, \mathrm{sgn}(a) \, \mathrm{sgn}(b) p$, donc $(\mathrm{sgn}(a) \, \mathrm{sgn}(b) \, p, q) \mathrel{\mathcal{R}} (a, b)$, donc $(\mathrm{sgn}(a) \, \mathrm{sgn}(b) \, p, q)$ est une représentation de $x$.
        Montrons que toutes ses seprésentations sont de la forme $(\lambda \mathrm{sgn}(a) \mathrm{sgn}(b) p, \lambda q)$ pour un entier non nul $\lambda$ et que tous les éléments de $\mathbb{Z} \times \mathbb{Z}^*$ de cette forme sont des représentations de $x$.
        Cela montrera que $x$ n'admet aucune autre représentation irréductible (avec $\lambda \neq 1$) car le numérateur est négatif si $\lambda < 0$ ou les valeurs absolues des numérateur et dénominateur (égales à $\abs{\lambda} p$ et $\abs{\lambda} q$) ne pourraient être premières entre elles pour $\lambda > 1$ (ayant un diviseur commun $\lambda$ strictement supérieur à $1$).
        \begin{itemize}[nosep]
            \item \emph{Toutes les représentations de $x$ sont de cette forme :} Soit $r$ une représentation de $x$ et $c$ et $d$ deux entiers relatifs non nuls tels que $(c, d) = r$. 
                Alors, $d \, \mathrm{sgn}(a) \, \mathrm{sgn}(b) p = c q$.
                En prenant la valeur absolue, il vient : $\abs{d} p = \abs{c} q$. 
                Donc, $q$ divise $\abs{d} p$. 
                Puisque $q$ est premier avec $p$, on en déduit que $q$ divise $\abs{d}$.
                On peut donc choisir un entier naturel $k$ tel que $\abs{d} = k q$.
                Soit $\lambda$ l'entier relatif définit par $\lambda = \mathrm{sgn}(d) k$. 
                Alors, $d = \lambda q$.
                En outre, $d$ est non nul, donc $\lambda$ est non nul. 
                Puisque $d \, \mathrm{sgn}(a) \mathrm{sgn}(b) p = c q$, on a : $\lambda q \, \mathrm{sgn}(a) \, \mathrm{sgn}(b) p = c q$.
                Puisque $q$ est non nul, cela donne : $\lambda \, \mathrm{sgn}(a) \, \mathrm{sgn}(b) p = c$.
                Donc, $r = (\lambda \, \mathrm{sgn}(a) \, \mathrm{sgn}(b) p, \lambda q)$.
            \item \emph{Tous les éléments de cette forme sont des représentations de $x$ :} Soit $\lambda$ un entier non nul.
                Alors, $a \lambda q = \mathrm{sgn}(a) \abs{a} \lambda q = \mathrm{sgn}(a) \lambda p q l = \abs{b} \lambda \mathrm{sgn}(a) p = b \lambda \, \mathrm{sgn}(a) \, \mathrm{sgn}(b) p$.
                Donc, $(\lambda \, \mathrm{sgn}(a) \, \mathrm{sgn}(b) p, \lambda q)$ est une représentation de $x$.
        \end{itemize}
\end{itemize}

\done

\medskip

\noindent\textbf{Notation :} Pour tout entier $n$, le nombre rationnel $\overline{(n, 1)}$ est parfois simplement noté $n$ quand il n'y a pas d'ambiguité.

\subsubsection{Structure de corps}

\noindent\textbf{Définition :} On définit les trois opérations $+$ (\emph{addition}), $-$ (\emph{soustraction}) et $\times$ (\emph{multiplication}) de $\mathbb{Q} \times \mathbb{Q}$ vers $\mathbb{Q}$ et l'opération $\divslash$ (\emph{division}) de $\mathbb{Q} \times \mathbb{Q}^*$ vers $\mathbb{Q}$ de la manière suivante. \index{Addition}\index{Soustraction}\index{Multiplication}\index{Division} 
\sindex[isy]{$+$}\sindex[isy]{$-$}\sindex[isy]{$\times$}\sindex[isy]{$\divslash$}
Soit $\bar{x}$ et $\bar{y}$ deux éléments de $\mathbb{Q}$. 
On peut choisir un représentant $x$ de $\bar{x}$ et un représentant $y$ de $\bar{y}$, puis quatre entiers $a$, $b$, $c$ et $d$ tels que $b$ et $d$ sont non nuls, $x = (a, b)$ et $y = (c, d)$.
Alors, 
\begin{itemize}[nosep]
    \item $\bar{x} + \bar{y}$ est la classe d'équivalence de $(a \times d + b \times c, b \times d)$,
    \item $\bar{x} - \bar{y}$ est la classe d'équivalence de $(a \times d - b \times c, b \times d)$,
    \item $\bar{x} \times \bar{y}$ est la classe d'équivalence de $(a \times c, b \times d)$,
    \item si $\bar{y} \in \mathbb{Q}^*$ (et donc $c \neq 0$), $\bar{x} \divslash \bar{y}$ est la classe d'équivalence de $(a \times d, b \times c)$.%
        \footnote{Puisque $b \neq 0$ et $c \neq 0$, $b \times c \neq 0$, donc $(a \times d, b \times c)$ est bien un élément de $\mathbb{Z} \times \mathbb{Z}^*$.}
\end{itemize}
Quand il n'y a pas d'ambiguité, on pourra omettre le symbole $\times$. 

\medskip

Nous montrons ci-après que ces opérations sont bien définies, \emph{i.e.}, qu'elles ne dépendent pas des représentants choisis pour $\bar{x}$ et $\bar{y}$.
Notons d'abord quelques propriétés découlant directement de leurs définitions : avec les mêmes notations, 
\begin{itemize}[nosep]
    \item $(\bar{x} + \bar{y}) - \bar{y} = \bar{x}$,
    \item $(\bar{x} - \bar{y}) + \bar{y} = \bar{x}$,
    \item si $\bar{y} \in \mathbb{Q}^*$, $(\bar{x} \times \bar{y}) \divslash \bar{y} = \bar{x}$,
    \item si $\bar{y} \in \mathbb{Q}^*$, $(\bar{x} \divslash \bar{y}) \times \bar{y} = \bar{x}$,
\end{itemize}

\medskip

\noindent\textbf{Démonstration :} Avec les mêmes notations, 
\begin{equation*}
    (\bar{x} + \bar{y}) - \bar{y} 
        = \frac{a \times d + b \times c}{b \times d} - \frac{c}{d}
        = \frac{(a \times d + b \times c) \times d - c \times (b \times d)}{(b \times d) \times d}
        = \frac{a \times (d \times d)}{b \times (d \times d)}
        = \frac{a}{b} 
        = \bar{x}.
\end{equation*}
(Où on a utilisé le fait que $d \times d \neq 0$ puisque $d \neq 0$.)
De même, 
\begin{equation*}
    (\bar{x} - \bar{y}) + \bar{y} 
        = \frac{a \times d - b \times c}{b \times d} + \frac{c}{d}
        = \frac{(a \times d - b \times c) \times d + c \times (b \times d)}{(b \times d) \times d}
        = \frac{a \times (d \times d)}{b \times (d \times d)}
        = \frac{a}{b} 
        = \bar{x} .
\end{equation*}
Si $\bar{y}$, est non nul, on a aussi : 
\begin{equation*}
    (\bar{x} \times \bar{y}) \divslash \bar{y} 
        = \frac{a \times c}{b \times d} \divslash \frac{c}{d}
        = \frac{(a \times c) \times d}{(b \times d) \times c} 
        = \frac{a \times (c \times d)}{b \times (c \times d)} 
        = \frac{a}{b}
        = \bar{x}.
\end{equation*}
(où on a utilisé le fait que $c \times d$ est non nul puisque ni $c$ ni $d$ n'est nul)
et
\begin{equation*}
    (\bar{x} \divslash \bar{y}) \times \bar{y} 
        = \frac{a \times d}{b \times c} \times \frac{c}{d}
        = \frac{(a \times d) \times c}{(b \times c) \times d}
        = \frac{a \times (c \times d)}{b \times (c \times d)} 
        = \frac{a}{b}
        = \bar{x}.
\end{equation*}

\done

\medskip

\noindent\textbf{Lemme :} Avec les notations ci-dessus, ces quatre définitions sont indépendantes des choix des représentants de $\bar{x}$ et $\bar{y}$.

\medskip

\noindent\textbf{Démonstration :} Avec les mêmes notations, soit $x$ et $x'$ deux représentants de $\bar{x}$, $y$ et $y'$ deux représentants de $\bar{y}$, et $a$, $a'$, $b$, $b'$, $c$, $c'$, $d$ et $d'$ huit entiers tels que $b \neq 0$, $b' \neq 0$, $d \neq 0$, $d' \neq 0$, $x = (a, b)$, $x' = (a', b')$, $y = (c, d)$ et $y' = (c', d')$. 
Il s'agit de montrer que les nombres rationnels obtenus en calculant $\bar{x} + \bar{y}$, $\bar{x} - \bar{y}$, $\bar{x} \times \bar{y}$ et, si $\bar{y}$ est non nul, $\bar{x} \divslash \bar{y}$ en prenant soit les représentants $x$ et $y$ soit $x'$ et $y'$ sont identiques.
Pour ce faire, on utilise les égalités : $a \times b' = a' \times b$ et $c \times d' = c' \times d$ (reflétant les prédicats $(a, b) \mathrel{\mathcal{R}} (a', b')$ et $(c, d) \mathrel{\mathcal{R}} (c', d')$). 

\noindent\emph{Addition :}
    \begin{equation*}
        \begin{aligned}
            (a \times d + b \times c) \times (b' \times d')
            & = (a \times d \times b' \times d' + b \times c \times b' \times d')
            = (a \times b' \times d \times d' + c \times d' \times b \times b') \\
            & {} = (a' \times b \times d \times d' + c' \times d \times b \times b') 
            = (a' \times d' \times b \times d + b' \times c' \times b \times d) \\
            & {} = (a' \times d' + b' \times c') \times (b \times d) .
        \end{aligned}
    \end{equation*}
    Donc, $(a \times d + b \times c, b \times d) \mathrel{\mathcal{R}} (a' \times d' + b' \times c', b' \times d')$. 
    Cela montre que $\overline{(a \times d + b \times c, b \times d)} = \overline{(a' \times d' + b' \times c', b' \times d')}$, et donc que le résultat de l'addition est le même que l'on choisisse les représentants $x$ et $y$ ou les représentants $x'$ et $y'$.

\noindent\emph{Soustraction :}
    \begin{equation*}
        \begin{aligned}
            (a \times d - b \times c) \times (b' \times d')
            & = (a \times d \times b' \times d' - b \times c \times b' \times d')
            = (a \times b' \times d \times d' - c \times d' \times b \times b') \\
            & {} = (a' \times b \times d \times d' - c' \times d \times b \times b') 
            = (a' \times d' \times b \times d - b' \times c' \times b \times d) \\
            & {} = (a' \times d' - b' \times c') \times (b \times d) .
        \end{aligned}
    \end{equation*}
    Donc, $(a \times d - b \times c, b \times d) \mathrel{\mathcal{R}} (a' \times d' - b' \times c', b' \times d')$. 
    Cela montre que $\overline{(a \times d - b \times c, b \times d)} = \overline{(a' \times d' - b' \times c', b' \times d')}$, et donc que le résultat de la soustraction est le même que l'on choisisse les représentants $x$ et $y$ ou les représentants $x'$ et $y'$.

\noindent\emph{Multiplication :}
    \begin{equation*}
        (a \times c) \times (b' \times d')
        = (a \times b') \times (c \times d')
        = (a' \times b) \times (c' \times d)
        = (a' \times c') \times (b \times d) .
    \end{equation*}
    Donc, $(a \times c, b \times d) \mathrel{\mathcal{R}} (a' \times c', b' \times d')$. 
    Cela montre que $\overline{(a \times c, b \times d)} = \overline{(a' \times c', b' \times d')}$, et donc que le résultat de la multiplication est le même que l'on choisisse les représentants $x$ et $y$ ou les représentants $x'$ et $y'$.

\noindent\emph{Division :} Supposons $\bar{y}$ non nul. Alors, $c \neq 0$ et $c' \neq 0$.
    On a :
    \begin{equation*}
        (a \times d) \times (b' \times c')
        = (a \times b') \times (c' \times d)
        = (a' \times b) \times (c \times d')
        = (a' \times d') \times (b \times c) .
    \end{equation*}
    Donc, $(a \times d, b \times c) \mathrel{\mathcal{R}} (a' \times d', b' \times c')$. 
    Cela montre que $\overline{(a \times d, b \times c)} = \overline{(a' \times d', b' \times c')}$, et donc que le résultat de la division est le même que l'on choisisse les représentants $x$ et $y$ ou les représentants $x'$ et $y'$.

\done

\medskip

\noindent\textbf{Lemme :} Soit $a$ et $b$ deux entiers et $c$ un entier naturel non nul.
    Alors,
    \begin{equation*}
        \frac{a}{c} + \frac{b}{c} = \frac{a + b}{c}.
    \end{equation*}

\medskip

\noindent\textbf{Démonstration :} On a : 
    \begin{equation*}
        \frac{a}{c} + \frac{b}{c} 
        = \frac{(a \times c) + (b \times c)}{c \times c}
        = \frac{(a + b) \times c}{c \times c}.
    \end{equation*}
    Puisque $c \neq 0$ on a $((a + b) \times cn c \times c) \mathrel{\mathcal{R}} (a \times b, c)$, donc cela donne 
    \begin{equation*}
        \frac{a}{c} + \frac{b}{c} 
        = \frac{a + b}{c}.
    \end{equation*}

    \done

\medskip

\noindent\textbf{Lemme :} Les opérations $+$ et $\times$ sont commutatives.

\medskip

\noindent\textbf{Démonstration :} Découle directement de la commutativité des opérations $+$ et $\times$ sur $\mathbb{Z}$. 
    Montrons cependant cela explicitement par soucis de complétude. 
    Soit $\bar{x}$ et $\bar{y}$ deux éléments de $\mathbb{Q}$. 
    Soit $x$ un représentant de $\bar{x}$, $y$ un représentant de $\bar{y}$, et $a$, $b$, $c$, $d$ quatre entiers tels que $b \neq 0$, $d \neq 0$, $x = (a, b)$ et $y = (c, d)$.
    On a : 
    \begin{equation*}
        \bar{x} + \bar{y} 
            = \frac{a}{b} + \frac{c}{d} 
            = \frac{(a \times d) + (b \times c)}{b \times d}
            = \frac{(b \times c) + (a \times d)}{d \times b}
            = \frac{(c \times b) + (d \times a)}{d \times b}
            = \frac{c}{d} + \frac{a}{b}
            = \bar{y} + \bar{x}
    \end{equation*}
    et 
    \begin{equation*}
        \bar{x} \times \bar{y} 
            = \frac{a}{b} \times \frac{c}{d} 
            = \frac{a \times c}{b \times d}
            = \frac{c \times a}{d \times b}
            = \frac{c}{d} \times \frac{a}{b}
            = \bar{y} \times \bar{x} .
    \end{equation*}

\done

\medskip

\noindent\textbf{Lemme :} Le triplet $(\mathbb{Q}, +, \times)$ est un corps, d'éléments neutres $0$ pour $+$ et $1$ pour $\times$.

\medskip

\noindent\textbf{Démonstration :} Montrons qu'il satisfait toutes les propriétés requises.
\begin{itemize}[nosep]
    \item Montrons déjà que $(\mathbb{Q}, +)$ est un groupe abélien d'élément neutre $0$.
        \begin{itemize}[nosep]
            \item Puisque $+$ est une loi de composition interne sur $\mathbb{Q}$, $(\mathbb{Q}, +)$ est un magma.
            \item \emph{L'opération $+$ est associative :} 
                Soit $\bar{x}$, $\bar{y}$ et $\bar{z}$ trois éléments de $\mathbb{Q}$. 
                On peut choisir six entiers $a$, $b$, $c$, $d$, $e$, $f$ tels que $b \neq 0$, $d \neq 0$, $f \neq 0$, $\bar{x} = a \divslash b$, $\bar{y} = c \divslash d$ et $\bar{z} = e \divslash f$.
                On a alors : 
                \begin{equation*} \begin{aligned}
                    (\bar{x} + \bar{y}) + \bar{z} 
                    & = \Big( \frac{a}{b} + \frac{c}{d} \Big) + \frac{e}{f}
                    = \frac{(a \times d) + (c \times b)}{b \times d} + \frac{e}{f}
                    = \frac{((a \times d) + (c \times b)) \times f + e \times (b \times d)}{(b \times d) \times f} \\
                    & = \frac{a \times (d \times f) + b \times (c \times f) + b \times (e \times d)}{b \times (d \times f)} 
                    = \frac{a \times (d \times f) + b \times ((c \times f) + (e \times d))}{b \times (d \times f)} \\
                    & = \frac{a}{b} + \frac{(c \times f) + (e \times d)}{d \times f} 
                    = \frac{a}{b} + \Big( \frac{c}{d} + \frac{e}{f} \Big)
                    = \bar{x} + (\bar{y} + \bar{z}) .
                \end{aligned} \end{equation*}
            \item L'opération $+$ est commutative (voir ci-dessus).
            \item \emph{Le nombre rationnel $0$ est neutre pour $+$ :} 
                Soit $\bar{x}$ un nombre rationnel, $x$ un de ses représentants et $a$ et $b$ deux entiers tels que $b \neq 0$ et $x = (a, b)$. 
                On a : 
                \begin{equation*}
                    0 + \bar{x} = \frac{0}{1} + \frac{a}{b}
                        = \frac{(0 \times b) + (1 \times a)}{1 \times b}
                        = \frac{0 + a}{b}
                        = \frac{a}{b}
                        = \bar{x}.
                \end{equation*}
                Puisque $+$ est commutative, cela implique également $\bar{x} + 0 = \bar{x}$.
            \item \emph{Tout élément de $\mathbb{Q}$ admet un inverse pour $+$ :} 
                Soit $\bar{x}$ un nombre rationnel et $x$ un de ses représentants. 
                On peut choisir deux entiers $a$ et $b$ tels que $b \neq 0$ et $x = (a, b)$.
                Soit $\bar{y}$ le nombre rationnel admettant $(-a, b)$ pour représentant.
                \begin{equation*}
                    \bar{x} + \bar{y} = \frac{a}{b} + \frac{-a}{b}
                        = \frac{(a \times b) + (b \times (-a))}{b \times b}
                        = \frac{(a + (-a)) \times b}{b \times b}
                        = \frac{0 \times b}{b \times b}
                        = \frac{0}{b \times b}
                        = 0 .
                \end{equation*}
                Puisque l'opération $+$ est commutative, cela montre également que $\bar{y} + \bar{x} = 0$, et donc que $\bar{y}$ est un inverse de $\bar{x}$ pour $+$.
                \end{itemize}
    \item Montrons que $(\mathbb{Q}, \times)$ est un magma associatif :
        \begin{itemize}[nosep]
            \item Puisque $\times$ est une loi de composition interne sur $\mathbb{Q}$, $(\mathbb{Q}, \times)$ est un magma.
            \item \emph{L'opération $\times$ est associative :} 
                Cette propriété découle directement de l'associativité de la multiplication d'entiers. 
                Montrons cela explicitement. 
                Soit $\bar{x}$, $\bar{y}$ et $\bar{z}$ trois éléments de $\mathbb{Q}$. 
                On peut choisir six entiers $a$, $b$, $c$, $d$, $e$, $f$ tels que $b \neq 0$, $d \neq 0$, $f \neq 0$, $\bar{x} = a \divslash b$, $\bar{y} = c \divslash d$ et $\bar{z} = e \divslash f$.
                On a alors : 
                \begin{equation*} \begin{aligned}
                    (\bar{x} \times \bar{y}) \times \bar{z} 
                    & = \Big( \frac{a}{b} \times \frac{c}{d} \Big) \times \frac{e}{f}
                    = \frac{a \times c}{b \times d} \times \frac{e}{f}
                    = \frac{(a \times c) \times e}{(b \times d) \times f} \\
                    & = \frac{a \times (c \times e)}{b \times (d \times f)} 
                    = \frac{a}{b} \times \frac{c \times e}{d \times f} 
                    = \frac{a}{b} \times \Big( \frac{c}{d} \times \frac{e}{f} \Big)
                    = \bar{x} \times (\bar{y} \times \bar{z}) .
                \end{aligned} \end{equation*}
        \end{itemize}
    \item L'opération $\times$ est commutative (voir ci-dessus). 
    \item \emph{L'opération $\times$ est distributive sur $+$ :} 
        Soit $\bar{x}$, $\bar{y}$ et $\bar{z}$ trois éléments de $\mathbb{Q}$. 
        On peut choisir six entiers $a$, $b$, $c$, $d$, $e$, $f$ tels que $b \neq 0$, $d \neq 0$, $f \neq 0$, $\bar{x} = a \divslash b$, $\bar{y} = c \divslash d$ et $\bar{z} = e \divslash f$.
        On a alors : 
        \begin{equation*} \begin{aligned}
            \bar{x} \times (\bar{y} + \bar{z})
            & = \frac{a}{b} \times \Big( \frac{c}{d} + \frac{e}{f} \Big)
            = \frac{a}{b} \times \frac{(c \times f) + (e \times d)}{d \times f}
            = \frac{a \times ((c \times f) + (e \times d))}{b \times (d \times f)}
            = \frac{(a \times (c \times f)) + (a \times (e \times d))}{b \times (d \times f)} \\
            & = \frac{a \times (c \times f)}{b \times (d \times f)} + 
                \frac{a \times (e \times d)}{b \times (d \times f)}
            = \frac{(a \times c) \times f}{(b \times d) \times f} + 
                \frac{(a \times e) \times d}{(b \times f) \times d}
            = \frac{a \times c}{b \times d} + 
                \frac{a \times e}{b \times f}
            = \Big( \frac{a}{b} \times \frac{c}{d} \Big)
                + \Big( \frac{a}{b} \times \frac{e}{f} \Big) \\
            & = (\bar{x} \times \bar{y}) + (\bar{x} \times \bar{z}) .
        \end{aligned} \end{equation*}
        Puisque $\times$ est commutative, cela montre également (en échangeant les rôles de $\bar{x}$ et $\bar{z}$) que $(\bar{x} + \bar{y}) \times \bar{z} = (\bar{x} \times \bar{z}) + (\bar{y} \times \bar{z})$ pour tous rationnels $\bar{x}$, $\bar{y}$, $\bar{z}$.%
            \footnote{En effet, puisque l'opération $\times$ est commutative, on a : $(\bar{x} + \bar{y}) \times \bar{z} = \bar{z} \times (\bar{x} + \bar{y}) = (\bar{z} \times \bar{x}) + (\bar{z} \times \bar{y}) = (\bar{x} \times \bar{z}) + (\bar{y} \times \bar{z})$.}
    \item \emph{Le nombre rationnel $1$ est neutre pour $\times$ :} 
        Soit $\bar{x}$ un nombre rationnel, $x$ un de ses représentants et $a$ et $b$ deux entiers tels que $b \neq 0$ et $x = (a, b)$. 
        On a : 
        \begin{equation*}
            1 \times \bar{x} = \frac{1}{1} \times \frac{a}{b}
                = \frac{1 \times a}{1 \times b}
                = \frac{a}{b}
                = \bar{x}.
        \end{equation*}
        Puisque $\times$ est commutative, cela implique également $\bar{x} \times 1 = \bar{x}$.
    \item L'anneau $(Q, +, \times)$ est non nul puisque $0 \neq 1$ (puisque le nombre rationnel $1$ a pour représentant $(1,1)$, qui n'est pas un représentant de $0$ puisque son numérateur n'est pas nul).
    \item \emph{Tout élément de $\mathbb{Q}$ distinct de $0$ admet un inverse pour $\times$ :} 
        Soit $\bar{x}$ un nombre rationnel non nul. 
        Soit $x$ un de ses représentants. 
        On peut choisir deux entiers $a$ et $b$ tels que $b \neq 0$ et $x = (a, b)$.
        Puisque $\bar{x}$ est non nul, $a \neq 0$. 
        Soit $\bar{y}$ le nombre rationnel admettant $(b, a)$ pour représentant.
        On a : $\bar{x} \times \bar{y} = \overline{(a, b)} \times \overline{(b, a)} = \overline{(a \times b, b \times a)} = \overline{(a \times b, a \times b)} = \overline{(1 \times (a \times b), 1 \times (a \times b))}$.
        Puisque $a$ et $b$ sont non nuls, $a \times b$ l'est également.
        Donc, $(1 \times (a \times b), 1 \times (a \times b)) \mathrel{\mathcal{R}} (1, 1)$.
        Donc, $\bar{x} \times \bar{y} = 1$.
        Puisque l'opération $\times$ est commutative, cela montre également que $\bar{y} \times \bar{x} = 1$, et donc que $\bar{y}$ est un inverse de $\bar{x}$ pour $\times$.
\end{itemize}

\medskip

\noindent\textbf{Remarque :} Soit $x$ un élément de $\mathbb{Q}^*$. 
    Alors l'inverse de $x$ est $1 \divslash x$. 

\done

\subsubsection{Relation d'ordre}

\noindent\textbf{Définition :} On définit la relation $\leq$ sur $Q$ de la manière suivante. 
    Soit $x$ et $y$ deux éléments de $\mathbb{Q}$ et $a$, $b$, $c$, $d$ quatre entiers tels que $b \neq 0$, $d \neq 0$, $x = a \divslash b$ et $y = c \divslash d$.
    On pose alors $x \leq y$ si et seulement si $a \, b \, d^2 \leq c \, d \, b^2$.

\medskip

\noindent\textbf{Lemme :} Cette définition ne dépend pas du choix des entiers $a$, $b$, $c$ et $d$.

\medskip

\noindent\textbf{Démonstration :} 
    Montrons que, avec les mêmes notations, l'inégalité est satisfaite si et seulement si elle l'est en choisissant les représentatios irréductibles de $x$ et $y$.
    Notons $a_0$ le numérateur de la représentation irréductible de $x$, $b_0$ son dénominateur, $c_0$ le numérateur de la représentation irréductible de $y$ et $d_0$ son dénominateur.
    On peut choisir deux entiers relatifs non nuls $\lambda$ et $\eta$ tels que $a = \lambda \, a_0$, $b = \lambda \, b_0$, $c = \eta \, c_0$, et $d = \eta \, d_0$.
    Donc, $a \, b \, d^2 = \lambda^2 \, \eta^2 \, a_0 \, b_0 \, d_0^2$ et $c \, d \, b^2 = \lambda^2 \, \eta^2 \, c_0 \, d_0 \, b_0^2$.
    Puisque $\lambda$ et $\eta$ sont non nuls, $\lambda^2$ et $\eta^2$ sont strictement positifs, donc $\lambda^2 \, \eta^2$ également.
    Donc, $a \, b \, d^2 \leq c \, d \, b^2$ si et seulement si $a_0 \, b_0 \, d_0^2 \leq c_0 \, d_0 \, b_0^2$.
    
    \done

\medskip

\noindent\textbf{Lemme :} La relation ainsi définie est une relation d'ordre total sur $\mathbb{Q}$.

\medskip

\noindent\textbf{Démonstration :} Ce résultat découle essentiellement du fait que $\leq$ est une relation d'ordre total sur $\mathbb{Z}$.
    Montrons cela explicitement.
    Dans cette preuve, $x$, $y$ et $z$ sont trois éléments de $\mathbb{Q}$ arbitraires et $a$, $b$, $c$, $d$, $e$, $f$ sont six entiers tels que $b \neq 0$, $d \neq 0$, $f \neq 0$, $x = a \divslash b$, $y = c \divslash d$ et $z = e \divslash f$.
    \begin{itemize}[nosep]
        \item On a $a \, b \, d^2 = a \, b \, d^2$, donc $a \, b \, d^2 \leq a \, b \, d^2$, donc $x \leq x$.
        \item Si $x \leq y$ et $y \leq x$, alors $a \, b \, d^2 \leq c \, d \, b^2$ et $c \, d \, b^2 \leq a \, b \, d^2$, donc $a \, b \, d^2 = c \, d \, b^2$.
            Donc, $(a \times d) \times (b \times d) = (c \times b) \times (b \times d)$.
            Puisque $b$ et $d$ sont non nuls, $b \times d \neq 0$, donc cela implique $a \times d = c \times b$, donc $(a, b) \mathrel{\mathcal{R}} (c, d)$, et donc $x = y$.
        \item Si $x \leq y$ et $y \leq z$, alors $a \, b \, d^2 \leq c \, d \, b^2$ et $c \, d \, f^2 \leq e \, f \, d^2$.
            Donc, $a \, b \, d^2 \, f^2 \leq c \, d \, b^2 \, f^2 \leq e \, f \, b^2 \, d^2$.
            Puisque $d$ est non nul, $d^2$ est strictement positif.
            Cela montre donc que $a \, b \, f^2 \leq e \, f \, b^2$, et donc $x \leq z$.
        \item Puisque $\leq$ est une relation d'ordre total sur $\mathbb{Z}$, on a $a \, b \, d^2 \leq c \, d \, b^2$ ou $c \, d \, b^2 \leq a \, b \, d^2$, donc $x \leq y$ ou $y \leq x$.
    \end{itemize}

    \done

\medskip

\noindent\textbf{Définition :} On définit la relation d'ordre $\geq$ et les relations d'ordre strict $<$ et $>$ sur $Q$ de la manière suivante : pour tous éléments $x$ et $y$ de $\mathbb{Q}$, 
\begin{itemize}[nosep]
    \item $x \geq y$ si et seulement si $y \leq x$,
    \item $x < y$ si et seulement si $(x \leq y) \wedge (x \neq y)$,
    \item $x > y$ si et seulement si $x < y$.
\end{itemize}

\medskip

\noindent\textbf{Remarque :}
    Soit $x$ et $y$ deux éléments de $\mathbb{Q}$ et $a$, $b$, $c$, $d$ quatre entiers tels que $b \neq 0$, $d \neq 0$, $x = a \divslash b$ et $y = c \divslash d$.
    Si $x = y$, alors $a \, d = c \, b$, donc (en multipliant les deux côtés par $b \, d$), $a \, b \, d^2 = c \, d \, b^2$.
    Réciproquement, si $a \, b \, d^2 = c \, d \, b^2$, alors $(a \, b) (b \, d) = (c \, d) \, (b \, d)$.
    Puisque $b$ et $d$ sont non nuls, $b \, d$ l'est également, donc cela implique $a \, b = c \, d$, et donc $x = y$.
    Ainsi, $x = y \Leftrightarrow a \, b \, d^2 = c \, d \, b^2$.
    Le prédicat $x < y$ est donc équivalent à $a \, b \, d^2 < c \, d \, b^2$, et $x > y$ à $a \, b \, d^2 > c \, d \, b^2$.

\medskip

\noindent\textbf{Lemme :} Soit $x$ et $y$ deux éléments de $\mathbb{Q}$. 
    On suppose $x < y$. 
    Alors, il existe un élément $z$ de $\mathbb{Q}$ satisfaisant $x < z < y$. 

\medskip

\noindent\textbf{Démonstration :} Soit $a$ et $b$ les numérateur et dénominateur d'un représentant de $x$ et $c$ et $d$ ceux d'un représentant de $y$. 
    Soit $e$ et $f$ les entiers naturels donnés par : $e = a \, d + b \, c$ et $f = 2 \, b \, d$.
    Puisque $b$, $d$ et l'entier $2$ sont non nuls, $f$ l'est également.
    Soit $z$ le nombre rationnel défini par : $z = e \divslash f$.
    Montrons que $x < z < y$. 
    Pour cela, on utilise le fait que, puisque $x < y$, $a \, b \, d^2 < c \, b^2 \, d$.

    Tout d'abord, 
    \begin{equation*}
        a \, b \, f^2 
        = 4 \, a \, b^3 \, d^2
        = 2 \, a \, b^3 \, d^2 + 2 \, a \, b^3 \, d^2
        < 2 \, a \, b^3 \, d^2 + 2 \, b^4 \, c \, d 
    \end{equation*}
    et 
    \begin{equation*}
        e \, f \, b^2
        = (a \, d + b \, c) \, 2 \, b^3 \, d
        = 2 \, a \, b^3 \, d^2 + 2 \, b^4 \, c \, d. 
    \end{equation*}
    Donc, $a \, b \, f^2 < e \, f \, b^2$, donc $x < z$.

    En outre, 
    \begin{equation*}
        c \, d \, f^2 
        = 4 \, c \, b^2 \, d^3
        = 2 \, c \, b^2 \, d^3 + 2 \, c \, b^2 \, d^3
        > 2 \, a \, b \, d^4 + 2 \, b^2 \, c \, d^3 
    \end{equation*}
    et 
    \begin{equation*}
        e \, f \, d^2
        = (a \, d + b \, c) \, 2 \, b \, d^3
        = 2 \, a \, b \, d^4 + 2 \, b^2 \, c \, d^3. 
    \end{equation*}
    Donc, $e \, f \, d^2 < c \, d \, f^2$, donc $z < y$.

    \done

\subsection{Le tore \texorpdfstring{$\mathbb{T}$}{T}}

\noindent\textbf{Définition :} On définit dans cette section le \emph{tore}\index{Tore} $\mathbb{T}$ comme l'ensemble des suites d'éléments de $\lbrace 0, 1 \rbrace$ non constantes égale à $1$ à partir d'un certain rang : 
\begin{equation*}
    \mathbb{T} = \left\lbrace
        x \in \lbrace 0, 1 \rbrace^{\mathbb{N}} 
        \middle\vert
        \forall n \in \mathbb{N} \, \exists m \in \mathbb{N} \, (m \geq n) \wedge (x_n = 0)
    \right\rbrace .
\end{equation*}
Quand il n'y a pas d'ambiguité, on note $0$ l'élément de $\mathbb{T}$ dont tous les éléments sont égaux à $0$.

\medskip

\noindent\textbf{Addition :} On définit l'opération $+$ de $\mathbb{T} \times \mathbb{T}$ vers $\mathbb{T}$ comme suit. 
    Soit $x$ et $y$ deux éléments de $\mathbb{T}$.
    Soit $n$ un entier naturel. 
    L'ensemble des éléments $m$ de $\mathbb{N}$ tels que $m \geq n$ et $x_m = 0$ est non vide (par définition de $\mathbb{T}$) et est un sous-ensemble de $\mathbb{N}$, donc il admet un minimum, noté $m_x$. 
    De même, l'ensemble des éléments $m$ de $\mathbb{N}$ tels que $m \geq n$ et $y_m = 0$ admet un minimum, noté $m_y$. 
    Soit $m$ le minimum de $m_x$ et $m_y$. 
    On définit par récurrence finie la suite $(a_k, b_k)_{k \in [\![0, m-n]\!]}$ de la manière suivante :%
    \footnote{On peut se ramener à une récurrence usuelle en posant cette définition si $k + 1 \leq m-n$ ou $a_{k+1} = b_{k+1} = 0$ sinon dans le second point.}
    \begin{itemize}[nosep]
        \item $b_0 = 1$ si $x_m = y_m = 1$ ou $b_0 = 0$ sinon ; $a_0 = 0$ si $x_m = y_m$ ou $a_0 = 1$ sinon.
        \item Pour tout élément $k$ de $[\![0, m-n-1]\!]$, $a_{k+1}$ est égal à 
            \begin{itemize}[nosep]
                \item $0$ si $0$ ou $2$ des trois entiers $x_{m-k}$, $y_{m-k}$, $b_k$ est égal à $1$,
                \item $1$ si $1$ ou $3$ des trois entiers $x_{m-k}$, $y_{m-k}$, $b_k$ est égal à $1$
            \end{itemize}
            et $b_{k+1}$ et égal à 
            \begin{itemize}[nosep]
                \item $0$ si $0$ ou $1$ des trois entiers $x_{m-k}$, $y_{m-k}$, $b_k$ est égal à $1$,
                \item $1$ si $2$ ou $3$ des trois entiers $x_{m-k}$, $y_{m-k}$, $b_k$ est égal à $1$.
            \end{itemize}
    \end{itemize}
    On pose alors $z_n = a_{m-n}$.
    On définit ensuite $x + y$ comme suit : 
    \begin{itemize}[nosep]
        \item Si $z_n = 1$ pour tout entier naturel $n$, alors $x + y = 0$.
        \item Sinon, s'il existe un entier naturel $m$ non nul tel que $z_k = 1$ pour tout entier naturel $k$ supérieur ou égal à $m$, on définit $m_0$ comme le minimum de l'ensemble des entiers naturels satisfaisant cette propriété. 
            Notons que $m_0 > 0$ (sans quoi $z$ satisferait la propriété du point précédent).
            On pose alors, pour tout entier naturel $n$ : 
            \begin{itemize}[nosep]
                \item Si $n < m_0 - 1$, $(x+y)_n = z_n$.
                \item Si $n = m_0 - 1$, $(x+y)_n = 1$.
                \item Si $n \geq m_0$, $(x+y)_n = 0$.
            \end{itemize}
        \item Sinon, $x + y = z$.
    \end{itemize}
    Notons que $x + y$ est alors bien un élément de $\mathbb{T}$.

\medskip

\noindent\textbf{Lemme :} L'addition ainsi définie est commutative.

\medskip

\noindent\textbf{Démonstration :} Évident car, avec les notations ci-dessus, $x$ et $y$ jouent des rôles interchangeables.

\done

\medskip

\noindent\textbf{Lemme :} Pour tout élément $x$ de $\mathbb{T}$, $0 + x = x$.

\medskip

\noindent\textbf{Démonstration :} Soit $x$ un élément de $\mathbb{T}$ et $n$ un entier naturel.
    En prenant les notations de la définition avec $y = 0$, on a $m_y = n$. 
    (En effet, $0_n = 0$, $n \geq n$ et, pour tout entier naturel $m$ satisfaisant les deux propriétés, $m \geq n$ par définition.)
    Puisque $m_x \geq n$ par définition, le minimum de $m_x$ et $m_y$ est $n$, donc $m = n$.
    Donc, $m-n = 0$.
    Donc, $a_{m-n} = a_0$.
    Si $x_n = 0$, $x_n = y_n$, donc $a_0 = 0$.
    Sinon, $x_n \neq y_n$, donc $a_0 = 1$.
    Dans les deux cas, on a $a_0 = x_n$, et donc $z_n = x_n$.
    Cela montre que $z = x$, donc que $z \in \mathbb{T}$, donc $0 + x = z$, et donc $0 + x = x$.

    \done

%\medskip
%
%\noindent\textbf{Lemme :} Dans la définition de l'addition, remplacer $m_0$ par un entier relatif $l$ tel que $l \geq m_0$ ne change pas le résultat.
%
%\medskip
%
%\noindent\textbf{Démonstration :} ***

\medskip

\noindent\textbf{Lemme :} Soit $x$ un élément de $\mathbb{T}$ ayant au moins un élément égal à $1$.
    Soit $i$ un entier naturel tel que $x_i = 1$.
    Soit $y$ et $z$ les éléments de $\mathbb{T}$ défini par : pour tout entier naturel $j$, 
    \begin{itemize}[nosep]
        \item si $j = i$, $y_j = 0$ et $z_j = 1$,
        \item sinon, $y_j = x_j$ et $z_j = 0$.
    \end{itemize}
    Alors, $y + z = x$.

\medskip

\noindent\textbf{Démonstration :} ***

%\medskip
%
%\noindent\textbf{Lemme :} On définit la suite de fonctions $\mathrm{pow2}$ de $\mathbb{T}$ vers $\mathbb{N}$ de la manière suivante : pour tout entier naturel $n$ et tout élément $x$ de $\mathbb{T}$, $\mathrm{pow2}_n(x) = \sum_{k=0}^n 2^{n-k} \times x_k$.
%Soit $x$ et $y$ deux éléments de $\mathbb{T}$ et $n$ un entier naturel tel que $x_{n+1} = 0$.
%Alors, 
%\begin{equation*}
%    \mathrm{pow2}_n(x+y) = (\mathrm{pow2}_n(x) + \mathrm{pow2}_n(y)) \mathrel{\%} 2^{n+1}. 
%\end{equation*}
%où $\%$ désigne le reste de la division Euclidienne.
%
%\medskip
%
%\noindent\textbf{Démonstration :} ***
%
%\medskip
%
%\noindent\textbf{Lemme :} Soit $x$ et $y$ deux éléments de $\mathbb{T}$. 
%    On suppose que, pour tout entier naturel $n$, il existe un entier naturel $m$ supérieur ou égal à $n$ tel que $\mathrm{pow2}_m(x) = \mathrm{pow2}_m(y)$.
%    Alors $x = y$. 
%
%\medskip
%
%\noindent\textbf{Démonstration :} Supposons par l'absurde que $x \neq y$. 
%    Soit $n$ le plue petit  entier naturel tel que $x_n \neq y_n$, donc $(x_n = 0 \wedge y_n = 1) \vee (x_n = 1 \vee y_n = 0)$. 
%    (Cet entier existe puisque l'ensemble des entiers satisfaisant cette propriété est un sous-ensemble non vide (puisque $x \neq y$ de $\mathbb{N}$.))
%    Pour fixer les idées, supposons $x_n = 0$ et $y_n = 1$. 
%    (La démonstration est identique dans l'autre cas en échangeant les rôles de $x$ et $y$.)
%    Soit $m$ un entier naturel supérieur ou égal à $n$. 
%    Montrons que $\mathrm{pow2}_m(x) < \mathrm{pow2}_m(y)$, et donc $\mathrm{pow2}_m(x) \neq \mathrm{pow2}_m(y)$, ce qui constituera une contradiction.
%
%    ***

\medskip

\noindent\textbf{Lemme :} L'addition est associative.

\medskip

\noindent\textbf{Démonstration :} ***

\medskip

\noindent\textbf{Multiplication par un entier :} On définit l'opération $\times$ de $\mathbb{N} \times \mathbb{T}$ vers $\mathbb{T}$ comme suit. 
    Définissons d'abord par récurrence la suite $f$ de fonctions de $\mathbb{T}$ vers $\mathbb{T}$ de la manière suivante: 
    \begin{itemize}[nosep]
        \item Pour tout élément $x$ de $\mathbb{T}$, $f_0(x) = 0$.
        \item Pour tout entier naturel $n$ et tout élément $x$ de $\mathbb{T}$, $f_{n+1}(x) = f_n(x) + x$.
    \end{itemize}
    Pour tout entier naturel $n$ et tout élément $x$ de $\mathbb{T}$, on pose alors $n \times x = f_n(x)$.

\medskip

\noindent\textbf{Remarque :} Avec cette définition, pour tout élément $x$ de $\mathbb{T}$, $0 \times x = 0$ et $1 \times x = x$.

\medskip

\noindent\textbf{Lemme :} La multiplication ainsi définie est distributive sur l'addition.

\medskip

\noindent\textbf{Démonstration :} Soit $x$ et $y$ deux éléments de $\mathbb{T}$.
    Montrons par récurrence sur $n$ que, pour tout entier naturel $n$, $n \times (x + y) = (n \times x) + (n \times y)$.
    
    Pour $n = 0$, on a : $n \times (x + y) = 0 \times (x + y) = 0$ et $(n \times x) + (n \times y) = (0 \times x) + (0 \times y) = 0 + 0 = 0$.
    Donc, $n \times (x + y) = (n \times x) + (n \times y)$. 
    Soit $n$ un entier naturel tel que $n \times (x + y) = (n \times x) + (n \times y)$.
    Alors, $(n + 1) \times (x + y) = (n \times (x + y)) + (x + y) = ((n \times x) + (n \times y)) + (x + y)$. 
    En utilisant l'associativité et la commutativité de l'addition, cela donne : $(n + 1) \times (x + y) = ((n \times x) + x) + ((n \times y) + y) = ((n + 1) \times x) + ((n + 1) \times y)$. 

    Par récurrence, la propriété attendue est donc vraie pour tout entier naturel $n$.

    \done

\medskip

\noindent\textbf{Lemme :} La multiplication ainsi définie est distributive sur l'addition d'entiers.

\medskip

\noindent\textbf{Démonstration :} Soit $x$ un élément de $\mathbb{T}$.
    Montrons par récurrence sur $n$ que, pour tout entier naturel $n$, pour tout entier naturel $m$, $(m + n) \times = (m \times x) + (n \times x)$.

    Pour $n = 0$, on a pour tout entier naturel $m$ : $(m + n) \times x = m \times x$ et $(m \times x) + (n \times x) = (m \times x) + (0 \times x) = (m \times x) + 0 = m \times x$. 
    Donc, $(m + n) \times x = (m \times x) + (n \times x)$.

    Soit $n$ un entier naturel tel que le résultat attendu est vrai.
    Soit $m$ un entier naturel.
    On a : $(m + (n+1)) \times x = ((m+1) + n) \times x = ((m+1) \times x) + (n \times x) = ((m \times x) + x) + (n \times x) = (m \times x) = (x + (n \times x)) = (m \times x) + ((n + 1) \times n)$.
    Le résultat est donc vrai au rang $n + 1$.

    Par récurrence, il l'est pour tout entier naturel $n$. 

    \done

\subsection{Nombres réels}

\subsubsection{Notion de limite}

\emph{À écrire...}

\subsubsection{Les nombres réels comme limites de suites de Cauchy}

\emph{À écrire...}

\subsubsection{Relation d'ordre}

\emph{À écrire...}

\subsubsection{Structure de corps}

\emph{À écrire...}

\subsection{Nombres complexes}

\emph{À écrire...}
