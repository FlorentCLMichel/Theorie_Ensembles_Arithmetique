\section{Autres ensembles remarquables}

Dans cette section, nous présentons brièvement d'autres ensembles de nombres remarquables ainsi que leur structure.

\subsection{Nombres rationnels}

\subsubsection{L'ensemble des nombres rationnels \texorpdfstring{$\mathbb{Q}$}{Q}}

Définissons la relation binaire $\mathcal{R}$ sur $\mathbb{Z} \times \mathbb{Z}^*$ de la manière suivante : si $x$ et $y$ sont deux éléments de cet ensemble, soit $a$, $b$, $c$ et $d$ les quatre (unique) entiers relatifs (avec $b \neq 0$ et $d \neq 0$) tels que $x = (a, b)$ et $y = (b, c)$, alors $r \mathcal{R} y$ si et seulement si $a d = b c$. 
Autrement dit, 
\begin{equation*}
    \forall (a, b, c, d) \in \mathbb{Z} \times \mathbb{Z}^* \times \mathbb{Z} \times \mathbb{Z}^*
    (a, b) \mathop{\mathcal{R}} (c, d)
    \Leftrightarrow a d = b c .
\end{equation*}

\medskip

\noindent\textbf{Lemme :} La relation $\mathcal{R}$ est une relation d'équivalence. 

\medskip

\noindent\textbf{Démonstration :} ***

\medskip

On appelle \emph{ensemble des nombres rationnels}, noté $\mathbb{Q}$, l'ensemble des classes d'équivalences de $\mathcal{R}$, appelées \emph{nombres rationnels}. 
\index{Nombre rationnel} \sindex[isy]{$\mathbb{Q}$}
Soit $q$ un nombre rationnel et $a$ et $b$ dens entiers tels que $(a, b)$ est un élément de $q$. 
Alors, $a$ est appelé le \emph{numérateur} de la repsésentation $(a, b)$ et $b$ son \emph{dénominateur}.
\index{Numérateur} \index{Dénominateur}
Elle sera notée $a \divslash b$.

\medskip

\noindent\textbf{Lemme :} Soit $q$ un nombre rationnel. 
Si $q$ admet une représentation dont le numérateur est nul, alors toutes ses représentations ont un numérateur nul.

\medskip

\noindent\textbf{Démonstration :} On suppose que $q$ admet une représentation de numérateur nul. 
    On peut donc choisir un entier $b$ non nul tel que $(0, b) \in q$. 
    Soit $r$ une représentation de $q$.
    Soit $c$ et $d$ deux entiers tels que $(c, d)  = r$.
    Alors, $0 \times d = b \times c$.
    Donc, $b c = 0$.
    Puisque $b \neq 0$, on en déduit que $c = 0$.
    Donc, la représentation $r$ est de numérateur nul.

    \done

\medskip

\noindent\textbf{Définition (représentation irréductible) :} 
    Soit $q$ un élément de $\mathbb{Q}$. 
    Soit $a$ un élément de $\mathbb{Z}$ et $b$ un élément de $\mathbb{Z}^*$ tels que $(a, b) \in q$.
    Alors, $(a, b)$ est dit \textit{représentation irreductible} de $q$ si et seulement si une des deux propriétés suivantes est satisfaite : 
    \begin{itemize}[nosep]
        \item $a = 0$ et $b = 1$.
        \item $a \neq 0$, $b > 0$ et $\abs{a}$ et $\abs{b}$ sont premiers entre eux.
    \end{itemize}
\index{Représentation irréductible}

\medskip

\noindent\textbf{Lemme :} Tout nombre rationnel a une unique représentation irréductible. 
    En outre, soit $r$ cette représentation irréductible, $a$ son numerateur et $b$ son dénominateur, un élément $(c, d)$ appartient à $q$ si et seulement si il existe un entier non nul $k$ tel que $c = k a$ et $d = k b$.

\medskip

\noindent\textbf{Démonstration :} Soit $x$ un nombre rationnel. 
\begin{itemize}[nosep]
    \item Si $x$ admet une représentation de numérateur nul, alors toutes ses représentations le sont.
        Donc, $x$ ne peut avoir de représentation de la forme $(a, b)$ avec $a \neq 0$.
        Il ne peut donc avoir que $(0, 1)$ commme représentation irréductible.
        Montrons que c'en est bien une. 
        Soit $r$ une représentation de $x$, $a$ son numérateur et $b$ son dénominateur. 
        Puisque $a = 0$, on a : $a \times 1 = 0 = b \times 0$. 
        Donc, $(0, 1)$ est bien une représentation de $x$, et donc sa représentation irréductible. 
        Par ailleurs, ses représentations sont toutes de la forme $(0, k)$, égal à $(0 \times k, 1 \times k)$, pour un entier non nu $k$, et tous les éléments de $\mathbb{Z} \times \mathbb{Z}^*$ de cette forme sont des représentations de $x$ puisque $1 \times 0 = 0 = 0 \times k$.
    \item Sinon, toutes les représentations de $x$ ont un numérateur non nul. 
        Soit $r$ une de ses repeésentations, $a$ son numerateur et $b$ son dénominateur. 
        Soit $l$ le pgcd de $\abs{a}$ et $\abs{b}$.
        On peut choisir deux entiers naturels $p$ et $x$ tels que $\abs{a} = p l$ et $\abs{b} = q l$.
        En outre, puisque $\abs{b}$ est non nul, $q$ est non nul, donc $(\mathrm{sgn}(a) \mathrm{sgn}(b) p, q)$ est un élément de $\mathbb{Z} \times \mathbb{Z}^*$. 
        Puisque $p$ et $q$ sont premiers entre eux, $p \neq 0$ (puisque $a \neq 0$), donc $\mathrm{sgn}(a) \mathrm{sgn}(b) p \neq 0$, et $q > 0$, il s'agit d'une représentation irreductible.
        En outre, $a q = \mathrm{sgn}(a) \abs{a} q = \mathrm{sgn}(a) p q l = \abs{b} \mathrm{sgn}(a) p = b \mathrm{sgn}(a) \mathrm{sgn}(b) p$, donc $(\mathrm{sgn}(a) \mathrm{sgn}(b) p, q)$ est une représentation de $x$.
        Montrons que toutes ses seprésentations sont de la forme $(\lambda \mathrm{sgn}(a) \mathrm{sgn}(b) p, \lambda q)$ pour un entier non nul $\lambda$ et que tous les éléments de $\mathbb{Z} \times \mathbb{Z}^*$ de cette forme sont des représentations de $x$.
        Cela montrera que $x$ n'admet aucune autre représentation irréductible (avec $\lambda \neq 1$) car le numérateur est négatif si $\lambda < 0$ ou les valeurs absolues des numérateur et dénominateur (égales à $\abs{\lambda} p$ et $\abs{\lambda} q$) ne pourraient être premiers entre eux pour $\lambda > 1$ (ayant un diviseur commun $\lambda$ strictement supérieur à $1$).
        \begin{itemize}[nosep]
            \item \emph{Toutes les représentations de $x$ sont de cette forme :} Soit $r$ une représentation de $x$ et $c$ et $d$ deux entiers relatifs non nuls tels que $(c, d) = r$. 
                Alors, $d \, \mathrm{sgn}(a) \mathrm{sgn}(b) p = c q$.
                En prenant la valeur absolue, il vient : $\abs{d} p = \abs{c} q$. 
                Donc, $q$ divise $\abs{d} p$. 
                Puisque $q$ est premier avec $p$, on en déduit que $q$ divise $\abs{d}$.
                On peut donc choisir un entier naturel $k$ tel que $\abs{d} = k q$.
                Soit $\lambda$ l'entier relatif définit par $\lambda = \mathrm{sgn}(d) k$. 
                Alors, $d = \lambda q$.
                En outre, $d$ est non nul, donc $\lambda$ est non nul. 
                Puisque $d \, \mathrm{sgn}(a) \mathrm{sgn}(b) p = c q$, on a : $\lambda q \, \mathrm{sgn}(a) \mathrm{sgn}(b) p = c q$.
                Puisque $q$ est non nul, cela donne : $\lambda \, \mathrm{sgn}(a) \mathrm{sgn}(b) p = c$.
                Donc, $r = (\lambda \, \mathrm{sgn}(a) \mathrm{sgn}(b) p, \lambda q)$.
            \item \emph{Tous les éléments de cette forme sont des représentations de $x$ :} Soit $\lambda$ un entier non nul.
                Alors, $a \lambda q = \mathrm{sgn}(a) \abs{a} \lambda q = \mathrm{sgn}(a) \lambda p q l = \abs{b} \lambda \mathrm{sgn}(a) p = b \lambda \, \mathrm{sgn}(a) \mathrm{sgn}(b) p$.
                Donc, $(\lambda \, \mathrm{sgn}(a) \mathrm{sgn}(b) p, \lambda q)$ est une représentation de $x$.
        \end{itemize}
\end{itemize}

\done

\subsubsection{Structure de corps}

\emph{À écrire...}

\subsubsection{Relation d'ordre}

\emph{À écrire...}

\subsection{Nombres réels}

\subsubsection{Notion de limite}

\emph{À écrire...}

\subsubsection{Les nombres réels comme limites de suites de Cauchy}

\emph{À écrire...}

\subsubsection{Relation d'ordre}

\emph{À écrire...}

\subsubsection{Structure de corps}

\emph{À écrire...}

\subsection{Nombres complexes}

\emph{À écrire...}
