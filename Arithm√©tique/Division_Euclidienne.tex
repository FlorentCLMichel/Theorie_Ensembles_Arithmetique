\subsection{Division euclidienne}

Soit $a$ un entier naturel et $b$ un entier naturel non nul. 
Soit $E$ l'ensemble défini par : $E = \lbrace a - k b \vert k \in \mathbb{N} \, \wedge \, a \geq k b \rbrace$%
~\footnote{Plus rigoureusement, on peut définir cet ensemble par : 
\begin{equation*}
    E = \lbrace
        n \in \mathbb{N} 
        \vert
        \exists k \in \mathbb{N} \,  
        a \geq k b \wedge a - k b = n
    \rbrace .
\end{equation*}}%
. 
L'ensemble $E$ est un sous-ensemble non vide (il contient au moins $a$, obtenu pour $k=0$) de $\mathbb{N}$, donc il admet un unique élément minimal $r$. 
Cet élément est appelé \textit{reste de la division euclidienne de $a$ par $b$}. 
L'unique entier naturel $q$ tel que $r = a - q b$%
~\footnote{Un tel entier existe car $r$ est un élément de $E$. 
    Montrons qu'il est bien unique. 
    Soit $q$ et $q'$ deux entiers naturels tels que $a \geq q b$, $a \geq q' b$, $r = a - q b$ et $r = a - q' b$.
    Alors, $a - q b = a - q' b$.
    Donc, $q b = q' b$.
    Puisque $b$ est non nul, on en déduit $q = q'$.
}  
est appelé \textit{quotient de la division euclidienne de $a$ par $b$}. 
Notons que l'on a : $a = q b + r$. 

En outre, $0 \leq r < b$. 
En effet, $r$ est un entier naturel par définition de $E$, donc $r \geq 0$ et, si $r$ était supérieur ou égal à $b$, on aurait $r - b = (a - q b) - b = a - (q b + b) = a - (q+1) b$, et donc $r - b \in E$. 
Puisque $b > 0$, $r - b < r$ ; cela contredirait donc la définition de $r$ comme minimum de $E$. 

Notons que $r$ est le seul élément de $E$ compris entre $0$ et $b-1$ (inclus). 
En effet, si $s$ est un tel élément, alors les autres éléments de $E$ sont tous de la forme $s + k b$ avec $k \in \mathbb{Z}$. 
Si $k > 0$, $s + k b >= b$ car $k \geq 1$ et $s \geq 0$.
Si $k < 0$, $s + k b < 0$ car $k b \leq -b$ et $s < b$, donc $s + k b \notin E$.
On en déduit que $S$ est l'élément minimal de $E$, et donc que $s=r$. 

Si le reste de la division euclidienne de $a$ par $b$ est $0$, on dit que \textit{$b$ divise $a$}, ou encore que \textit{$b$ est un diviseur de $a$}, que \textit{$a$ admet $b$ pour diviseur} ou que \textit{$a$ est un multiple de $b$}. 
Notons que, si $a$ est non nul $b$ ne peut alors être strictement supérieur à $a$. 
(En effet, soit $q$ le quotient de la division Euclidienne de $a$ par $b$, on a $a = q \times b$, donc soit $q = 0$ et $a = 0$, soit $q \geq 1$ et $a \geq b$.)
Notons aussi que tout entier naturel divise $0$.

Notons aussi que tout entier naturel non nul se divise lui-même. 
En effet, soit $a$ un entier naturel non nul, on a $a - a = 0$, donc le minimal de l'ensemble $E$ définit ci-dessus pour $b=a$ est $0$.

Un entier naturel est dit \textit{pair} s'il est un multiple de $2$ et \textit{impair} sinon. 

\bigskip

\noindent\textbf{Définition :} On définit les deux opérations $\divslash\!\divslash$ et $\%$ de $\mathbb{N} \times \mathbb{N}^*$ vers $\mathbb{N}$ de la manire suivante : pour tout entier naturel $a$ et tout entier naturel $b$ non nul $a \mathrel{\divslash\!\divslash} b$ est le quotient de la division Euclidienne de $a$ par $b$ et $a \mathrel{\%} b$ est son reste. 

\bigskip

\noindent\textbf{Définition :} Soit $a$ et $b$ deux entiers naturels non nuls. L'ensemble de leurs diviseurs communs est un sous-ensemble de $\mathbb{N}$ non vide (il contient au moins $1$) et borné supérieurement par le minimum de $a$ et $b$. Il admet donc un unique élément maximal, appelé \textit{plus grand diviseur commun}, ou \textit{pgcd}, de $a$ et $b$. 
    Notons que cet entier est toujours supérieur ou égal à $1$. 
    Si $n$ est un entier naturel, on considère que le pgcd de $n$ et $0$ (ou de $0$ et $n$) est $n$.
    (Ainsi, le pgcd de $0$ et $0$ est $0$.)

\bigskip

\noindent\textbf{Définition :} Deux entiers naturels $a$ et $b$ sont dits \textit{premiers entre eux} si leur pgcd est $1$.

\bigskip

\noindent\textbf{Lemme :} Soit $a$ et $b$ deux entiers naturels non nuls et $c$ leur pgcd.
    On note $d$ et $e$ les entiers $a \mathrel{\divslash\!\divslash} c$ et $b \mathrel{\divslash\!\divslash} c$. 
    Alors, $a = d \times c$, $b = e \times c$, et $d$ et $e$ sont premiers entre eux.

\medskip

\noindent\textbf{Démonstration :} 
    Puisque $c$ est un diviseur de $a$, le reste de la divirion Euclidienne de $a$ par $c$ est $0$, donc $d \times c = a$.
    De même, puisque $c$ est un diviseur de $b$, le reste de la divirion Euclidienne de $b$ par $c$ est $0$, donc $e \times c = a$.

    Supposons par l'absurde que $d$ et $e$ ne soient pas premiers entre eux. 
    Alors, $d$ et $e$ admettent un diviseur commun $f$ tel que $f > 1$.
    On peut donc choisir deux entiers naturels non nuls $g$ et $h$ tels que $d = g \times f$ et $e = h \times f$.
    Donc, $a = g \times f \times c$ et $b = h \times f \times c$. 
    Donc, $f \times c$ est un diviseur commun à $a$ et $b$. 
    Puisque $f > 0$ et $c > 0$, $f \times c > c$, ce qui contredit la définition du pgcd. 
    On en déduit que l'hypothèse de départ est fausse et que $d$ et $e$ sont premiers entre eux. 

   \done 

\bigskip

Une fonction Haskell calculant le pgcd de deux entiers naturels non nuls est donnée en appendice~\ref{app:Haskell_pgcd}. 

\subsubsection{Modulo}

Soit $p$, $q$ et $r$ trois entiers relatifs. 
On écrit $p \equiv r \, [q]$, ou $p \equiv r \, \mathrm{mod} \, q$, ou encore $p \equiv r \, ( \mathrm{mod} \, q)$ s'il existe un entier relatif $k$ tel que $p = r + k q$.  
On dit alors que \textit{$p$ est égal à $r$ modulo $q$}.
Notons que, pour tout entier relatif $s$, on a alors aussi $p \equiv (r + s q) \, [q]$.
Notons aussi que l'on a toujours $p \equiv p \, [q]$ (puisque $p = p + 0 q$). 

\medskip 

\noindent\textbf{Lemme :} Soit $q$, $p_1$, $p_2$, $r_1$ et $r_2$ cinq entiers relatifs tels que $p_1 \equiv r_1 \, [q]$ et $p_2 \equiv r_2 \, [q]$.
Alors, 
\begin{itemize}[nosep]
    \item $p_1 + p_2 \equiv (r_1 + r_2) \, [q]$, 
    \item $p_1 - p_2 \equiv (r_1 - r_2) \, [q]$, 
    \item $p_1 p_2 \equiv (r_1 r_2) \, [q]$.
\end{itemize}

\medskip

\noindent\textbf{Démonstration :} Choisissons deux entiers $k_1$ et $k_2$ tels que $p_1 = r_1 + k_1 q$ et $p_2 = r_2 + k_2 q$. 
On a :
\begin{itemize}[nosep]
    \item $p_1 + p_2 = (r_1 + r_2) + (k_1 + k_2) \, q$, 
    \item $p_1 - p_2 = (r_1 - r_2) + (k_1 - k_2) \, q$, 
    \item $p_1 p_2 = (r_1 r_2) + (r_1 k_2 + r_2 k_1 + k_1 k_2) \, q$.
\end{itemize}

\done

\bigskip

Un entier naturel $n$ est pair si et seulement si $n \equiv 0 \, [2]$ et impair si et seulement si $n \equiv 1 \, [2]$. 
On montre ainsi facilement que la somme de deux nombres pairs est paire, la somme de deux impairs est paire, et la somme d'un pair et d'un impair est impaire.

Plus généralement, si $p$ et $q$ sont deux entiers naturels non nuls et si $r$ est le reste de la division euclidienne de $p$ par $q$, alors $p \equiv r [q]$. 
En outre, $r$ est le seul entier vérifiant les conditions $p \equiv r [q]$ et $0 \leq r < q-1$. 
(En effet, les autres entiers satisfaisant la première relation difèrent d'un multiple de $q$, et sont donc soit strictement négatifs soit supérieurs ou égaux à $q$ ; voir l'argument plus haut dans cette sous-section.)

